\documentclass[]{article}
\usepackage{lmodern}
\usepackage{amssymb,amsmath}
\usepackage{ifxetex,ifluatex}
\usepackage{fixltx2e} % provides \textsubscript
\ifnum 0\ifxetex 1\fi\ifluatex 1\fi=0 % if pdftex
  \usepackage[T1]{fontenc}
  \usepackage[utf8]{inputenc}
\else % if luatex or xelatex
  \ifxetex
    \usepackage{mathspec}
    \usepackage{xltxtra,xunicode}
  \else
    \usepackage{fontspec}
  \fi
  \defaultfontfeatures{Mapping=tex-text,Scale=MatchLowercase}
  \newcommand{\euro}{€}
\fi
% use upquote if available, for straight quotes in verbatim environments
\IfFileExists{upquote.sty}{\usepackage{upquote}}{}
% use microtype if available
\IfFileExists{microtype.sty}{%
\usepackage{microtype}
\UseMicrotypeSet[protrusion]{basicmath} % disable protrusion for tt fonts
}{}
\ifxetex
  \usepackage[setpagesize=false, % page size defined by xetex
              unicode=false, % unicode breaks when used with xetex
              xetex]{hyperref}
\else
  \usepackage[unicode=true]{hyperref}
\fi
\hypersetup{breaklinks=true,
            bookmarks=true,
            pdfauthor={},
            pdftitle={},
            colorlinks=true,
            citecolor=blue,
            urlcolor=blue,
            linkcolor=magenta,
            pdfborder={0 0 0}}
\urlstyle{same}  % don't use monospace font for urls
\setlength{\parindent}{0pt}
\setlength{\parskip}{6pt plus 2pt minus 1pt}
\setlength{\emergencystretch}{3em}  % prevent overfull lines
\setcounter{secnumdepth}{0}

\date{}

\begin{document}

\section{La via del calcio}\label{la-via-del-calcio}

Il Ca di cui parleremo è la presenza dello ione all'interno delle
cellule.

Come tutti gli ioni che troviamo negli esseri viventi ce n'è una quota
dentro le cellule e una fuori. Noi ci occuperemo soprattutto del calcio
intracellulare. Il calcio intracellulare partecipa alle trasduzioni del
segnale. Siamo sempre nell'ambito della comunicazione cellulare.

Nella cellula sono stato descritti molti sistemi capaci di trasportare
lo ione da una parte all'altra, chiamati \emph{sistemi di mobilizzazione
attiva o passiva dello ione calcio}.

Il Ca è presente nella cellula sia come ione libero che come ione legato
ad altre molecole, soprattutto proteine (legano preferenzialmente ioni
bivalenti). Anche gli ioni monovalenti possono legarsi alle proteine ma
lo fanno con una minore affinità. Poi c'è una sistuaizone particolare
che è quella dei pori ionici.

In parte il Ca\(^2\)\(^+\) è uno ione libero nel citosol e in parte è
presente negli organelli cellulari (libero o legato a proteine).

La quantità di Ca\(^2\)\(^+\) presente come ione libero è molto piccola,
ma in realtà nella cellula non è che ci sia poco calcio (questo è quasi
tutto legato a proteine o sequestrato negli organuli, e anche qui è
prevalentemente legato a proteine).

La concentrazione del calcio citosolico si dice che è intorno a 100 nM
ma può essere anche più bassa (è una concentraizone molto bassa =
10\(^-\)\(^7\) M), mentre nel mezzo extracellulare lo ione calcio ha una
concentrazione di qualche mM (intorno a 10\(^-\)\(^3\) M).

la membrana cellulare divide dunque due soluzioni in cui il calcio ha
concentrazioni diverse: all'esterno circa da 2 a 5 10\(^-\)\(^3\) M
mentre nel citosol è attorno a 10\(^-\)\(^7\) M.

La quantità di calcio per volume all'intenro della cellula è quasi
uguale a quella all'esterno, ma è legato a proteine.

Il calcio può presentare degli aumenti transitori, di durata variabile
(da qualche mSec ad alcuni minuti) dove la concetrazione può salire per
poi ridiscendere. Concentraizoni molto elevate (oltre 1000 nM) diventano
tossiche per la cellula fino a mandare la cellula in apoptosi o
addiritura in necrosi.

Dentro la cellula il calcio è mantenuto a livelli molto bassi in
soluzione perchè ci sono dei sistemi tampone, detti \emph{buffer del
calcio}, e perchè questo è quasi tutto legato a proteine.

Il calcio conservato negli organuli ha indotto i fisiologi a parlare di
\emph{riserva intracellulare di calcio} (o store, cioè magazzini del
calcio) anche perchè il Ca\(^2\)\(^+\) può essere rilasicato da queste
riserve.

Le riserve che mobilizzano più facilemtne il calcio sono rapppresentate
dal reticolo endoplasmatico (si possono individuare 3 tipologie
morfologiche di RE = liscio, ruvido e membrana nucleare).

In questi organuli la concentrazione del Ca\(^2\)\(^+\) è intorno al
microM.

Nel RE ci sono varie proteine che legano il Ca\(^2\)\(^+\) e che dunque
favoriscono l'accumulo dello stesso. Proteine di questo tipo sono
presenti anche nel citosol.

Lo ione calcio presenta un forte gradiente (uno dei più forti che si
conoscono negli esseri viventi, ma non il più forte). La membrana
cellulare è pochissimo permeabile al Ca\(^2\)\(^+\) e il potenziale di
equilibrio del calcio è molto lontano da quello di membrana. C'è un
forte gradiente elettrochimico che spinge lo ione ad entrare nella
cellula, ma non può farlo perchè la membrana è impermeabile.
Occasionalmente, tuttavia, la membrana cellulare può diventare
temporaneamente permeabile al calcio grazie all'apertura dei
\emph{canali del calcio}. A questo punto abbiamo degli afflussi di
Ca\(^2\)\(^+\) dall'esterno all'interno della cellula, con aumento della
concentrazione interna dello ione.

Quando si attiva una corrente del sodio o del potassio superiore a
quello basale si ha un flusso e un deflusso di ioni. Queste correnti
però non sono tali da modificare le concentrazioni intra- ed
extracellulari delgi ioni perchè ne movimentano troppi pochi.

Nel caso del calcio le concentrazioni però sono molto più basse e qundi
è sufficiente un ingresso dello ione dovuto all'apertura dei canali per
provocare una modficazione della concentrazione citosolica dello ione
(cosa che non accade per Na\(^+\), K\(^+\) e Cl\(^-\)). Questi aumenti
sono transitori.

Fenomeni di questo genere sono stati chiamati \emph{``segnali del
calcio''}.

Il \emph{potenziale di equilibiro del Ca\(^2\)\(^+\)} è di \textbf{130
mV}.

Lo stesso fenomeno può verificarsi anche dagli organuli al citosol.
Sugli organuli ci sono dei canali che possono aprirsi originando un
deflusso degli ioni dagli stessi al citosol.

I canali del calcio sono \emph{canali ionici}: sono proteine di membrana
che presentano un poro attaverso il quale può essere condotto uno ione
attraverso la membrana. Lo ione passa sempre \emph{o dall'esterno al
citosol o dalle riserve intracellulari al citosol}. Non è mai stato
osservato un trasporto nel senso opposto.

Nella plasmamembrana esistono diversi tipi di canali del calcio che sono
alla base della sensibilità, del movimento, ecc.

Questi canali possono rispondere ad una variazione di voltaggio, oppure
essere regolati da recettori (cioè rispondono ad un ligando, i più
tipici sono glutammato, serotonina e acetilcolina, sono tutti
neurotrasmettitori).

Questi canali li troviamo nel tessuto muscolare e nel tessuto nervoso
(ma non solo).

I \textbf{canali del calcio voltaggio-dipendenti} generalmente non
operano isolatamente, ma lavorano in sinergia con altri tipi di canali
ionici. I canali voltaggio-dipendenti del calcio sono attivato dalla
depolarizzazione, per cui la corrente I\(_C\)\(_a\), evocata da uno
stimolo depolarizzante, assume facilmente carattere autorigenerativo ed
è in grado di generare potenziali d'azione.

Oltre che nella genesi dle potenziale d'azione, i canali
voltaggio-dipendenti del calcio svolgono un ruolo cruciale nel regolare
l'ingresso nel citosol, attraverso la membrana plasmatica, dello ione
Ca\(^2\)\(^+\), un fattore attivo in molti processi essenziali nelle
funzioni delle cellule (es. promotore del rilascio del
neurotrasmettitore alle giunzioni sinaptiche, attivatore dell'apparato
contrattile nelle fibre muscolari, ecc).

Misure elettrofisiologiche delle correnti di calcio in cellule native
hanno rivelato l'esistenza di due gruppi di canali del calcio
voltaggio-dipendenti, sulla base della voltaggio-dipendenza della loro
attivazione:

\begin{itemize}
\itemsep1pt\parskip0pt\parsep0pt
\item
  i \textbf{canali del calcio a bassa soglia} (\emph{Low Voltage
  Activated, LVA}) si attivano in risposta a modeste depolarizzazioni
  della membrana plasmatica e mostrano una inattivazione
  voltaggio-dipendente e rapida;
\item
  i \textbf{canali del calcio ad alta soglia} (\emph{High Voltage
  Activated, HVA}) richiedono depolarizzazioni più ampie per aprirsi e
  si attivano molto più lentamente.
\end{itemize}

I canli voltaggio-dipendenti hanno un funzionamento descritto da 3
stati: chiuso, aperto e inattivo. I canali del calcio si aprono per
depolarizzazione della membrana, ciò significa che se la membrana è al
suo potenziale di riposo il canale è chiuso, mentre se la cellula si
depolarizza il canale si apre.

Una \textbf{depolarizzazione}, in biologia, è la \emph{diminuzione del
valore assoluto del potenziale di membrana di una cellula}. Così, quando
il potenziale di membrana di una cellula si avvicina allo zero, si ha
una depolarizzazione.

Il canale può presentarsi anche in \emph{forma inattiva} e in questo
caso non reagisce alla depolarizzazione (dunque rimane sempre chiuso).

La proteina costitutiva dei canali voltaggio-dipendenti del calcio,
detta subunità \(/alpha\)\(_1\) è attraversata assialmente dal ``poro''
che viene percorso dagli ioni Ca\(^2\)\(^+\) che vi fluiscono quando il
canale si trova nello stato aperto. La subunità \(/alpha\)\(_1\) è
solitamente associata a 4 subunità accessorie di minor dimensione
(\(/alpha\)\(_2\), \(/beta\) e \(/delta\)). La presenza in alcune delle
loro anse citosoliche di siti fosforilabili fa pensare che esse possano
svolgere una funzione regolatrice delle caratteristiche funzionali della
subunità \(/alpha\)\(_1\) cui sono associate.

I vari tipi di canali del calcio sono accomunati da una sensibilità più
o meno spiccata ai cationi bi- e tri- valenti Cd\(^2\)\(^+\),
Ni\(^2\)\(^+\), Co\(^2\)\(^+\) e la La\(^3\)\(^+\), che esercitano
un'azione bloccante.

Tra i \textbf{canali attivati da ligandi} troviamo il \emph{recettore
del glutammato}, l'\emph{N-metil-D-aspartato} (detto anche
\textbf{NMDA}), che ha azione diretta sul recettore causando l'apertura
del poro, mentre il manganese stimola il canale.

Il \emph{glutammato} è un neurotrasmettitore delle cellule nervose.

Nelle sinapsi, quando arriva uno stimolo, si ha l'apertura di più canali
del calcio nella cellula presinaptica. Uno degli effetti del segnale del
calcio è l'esocitosi, che permette il rilascio delle vescicole che
contengono il neurotrasmettitore (nella cellula presinaptica sono canali
voltaggio-dipedenti). Ponendo che le vescicole contengano glutammato,
questo invade la camera sinaptica legandosi ai recettori sull'altra
cellula.

Altri tipi di canali che reagisocno a vari stimoli sono i \textbf{TRP
(Transient Receptor Potential)}. Questi canali costituiscono un'ampia
superfamiglia di canali versatili che sono espressi praticamente in
tutti i tipi cellulari nei vertebrati e negli invertebrati. Questi
canali sono cruciali nell'innesco di meccanismi che permettono
l'ingresso di Ca\(^2\)\(^+\), Mg\(^2\)\(^+\) e tracce di altri ioni
metallici in risposta a stimoli intra- ed extracellulari in un ampio
spettro di processi fisiologici.

Questi sono canali generalmente \emph{non selettivi}.

I canali TRP sono proteine di membrana intrinseche con sei domini
transmembranari e una regione che ne costituisce il poro, permeabile ai
cationi.

Uno dei più studiati è il gruppo dei \textbf{vanilloidi (TRPV)} perchè
coinvolti, ad esempio, nella sensibilità olfattiva e termica (anche le
risposte al peperoncino o al mentolo).

\subsection{\texorpdfstring{Il rilascio intracellulare del
Ca\(^2\)\(^+\)}{Il rilascio intracellulare del Ca\^{}2\^{}+}}\label{il-rilascio-intracellulare-del-ca2}

Esistono diversi meccanismi tramite cui le riserve interne di calcio
possono essere liberate.

I \textbf{canali IP\(_3\)R dipendenti} (recettori dell'\emph{inositolo
trifosfato}, IP3) sono enormi proteine presenti sul reticolo
endoplasmatico formate da \emph{4 subunità} (ognuna da 313 kDa) con un
grosso \emph{dominio citosolico} e un più piccolo \emph{dominio
intrareticolo}. Le 4 subunità formano una struttura circolare con un
poro centrale.

La transizione del canale dalla conformazione chiusa a quella aperta
viene stimolata dall'inositolo trifosfato (innesco) e dal calcio. L'IP3
è un polialcol (i polialcoli sono molecole simili agli zuccheri ma
presentanti tutte funzioni alcoliche, mentre gli zuccheri hanno una
funzione aldeidica e le altre sono funzioni alcoliche).

L'IP3 ha \emph{3 funzioni alcoliche esterificate a 3 gruppi fosfato}.

In particolare la molecola che ci interessa è l'\emph{1,4,5 trifosfato}.

Questa molecola deriva da un \emph{fosfolipide di membrana}, il
\textbf{fosfatidilinositolo}, che può essere fosforilato diventando
\textbf{fosfatidilinositolo} 4,5-bisfosfato. Successivamente si forma
l'\textbf{inositolo 1,4,5-trifosfato}; il terzo gruppo fosfato deriva
dal fostato che teneva la molecola attaccata al fosfolipide, lasciando
così una moleocla di \emph{diacilglierolo} nella membrana. L'inositolo
si stacca dalla membrana per mezzo dell'enzima \textbf{fosfolipasi C},
diffonde nel citosol e si lega al canale.

Il recettore è formato da 4 subunità e può legare fino a 4 molecole di
IP3. Maggiore è la quantità di IP3 rilasciato, maggiore è la quantità di
calcio rilasciato. Tutto questo è messo in moto da una fosfolipasi che
viene attivata da una proteina G, la quale a sua volta viene attivata da
un ormone.

Un altro tipo di recettore è l'\textbf{RYR}. Questo è un complesso
proteico transmembrana del reticolo sarco/endoplasmatico formato da 4
subunità. Questo canale non ha un agonista fisiologico preciso: il
principale è il Ca, quando la sua concentrazione citosolica aumenta il
canale si apre.

Questo tipo di canale risponde ad un moderato aumento del calcio
citosolico (tra 0,1 e 1 \(/mu\)M). Nel muscolo striato tuttavia, le
concentrazioni che lo attivano vanno anche oltre il \(/mu\)M, ma passato
questo valore il Ca diventa inattivante. Per questo si pensa che siano
presenti due siti nel canale: uno a bassa affinità e uno ad alta
affinità che inibiscono o stimolano il canale.

Per i canali RYR sono stati scoperti vari agonisti farmacologici quali
eparina, caffeina e rianodina (quest'ultima è la molecola agonista più
nota). La rianodina a concentrazioni basse stimola l'apertura del canale
mentre a concentrazioni alte ne stimola la chiusura (sempre rispetto
alla concentrazione basale).

\textbf{Lessico:}

\begin{itemize}
\itemsep1pt\parskip0pt\parsep0pt
\item
  si dice \emph{``ingresso''} quando uno ione entra nella cellula
  dall'esterno;
\item
  si dice \emph{``rilascio''} quando uno ione arriva da una riserva
  intracellulare;
\item
  si dice \emph{``espulsione''} o \emph{``estrusione''} quando il calcio
  viene trasferito dal citosol all'esterno;
\item
  si dice \emph{``sequestro''} quando lo ione viene trasferito dal
  citosol agli organuli.
\end{itemize}

Con questi due recettori abbiamo appena visto fenomeni di rilascio del
calcio. Anche questi generano segnali del calcio.

Sono state anche descritte anche mobilitazioni dello ione che \emph{non}
generano segnali del calcio; queste sono dovute a correnti di trasporto
passivo o facilitato. Questo fenomeno viene definito come *``ingresso
capacitativo del Ca\(^2\)\(^+\).

Le attività di sequestro del calcio portano lo ione dentro le riserve
mentre le attività di rilascio lo portano al di fuori di esse.

Se in una cellula prevalgono le attività di rilascio gli ``store''
intracellulari tendono a scaricarsi e siccome il calcio citosolico poi
viene anche espulso, tutta la cellula si scarica.

Quanto più intenso è il segnale del calcio proveniente dalle riserve,
tanto più la cellula si impoverisce. L'ingresso capacitativo del
Ca\(^2\)\(^+\) si oppone a questo fenomeno ricaricando gli store senza
tuttavia indurre un segnale del calcio. Sono stati identificati diversi
meccanismi detti \textbf{SOC} (\emph{Store-Operated Channels}, è un
canale), \textbf{CRAC} (Calcium Release Activated Channels, fenomeno) e
SOCE. \textbf{CONTROLLARE}

È stato poi individuato un fenomeno secondo il quale nel RE sono
presenti delle proteine chiamate STIM che avvertono la diminuzione delle
riserve del calcio all'interno del reticolo. Si è postulato che il
processo di \textbf{CCE} (\emph{Capacitative Calcium Entry}) necessiti
dell'interazione di due proteine, \textbf{Orai} e \textbf{STIM}
(\emph{Stromal Interaction Molecule}); la prima è il canale vero e
proprio localizzato nella membrana plasmatica, l'altra il modulatore di
Orai localizzato nella membrana del RE.

Le proteine STIM si aggregano e si legano a canali ORAI attivandoli. A
questo punto il Ca\(^2\)\(^+\) entra nella cellula e viene sequestrato
nel reticolo ricaricandolo, passando probabilmente dal citosol.

\end{document}
