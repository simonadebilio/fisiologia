\documentclass[]{article}
\usepackage{lmodern}
\usepackage{amssymb,amsmath}
\usepackage{ifxetex,ifluatex}
\usepackage{fixltx2e} % provides \textsubscript
\ifnum 0\ifxetex 1\fi\ifluatex 1\fi=0 % if pdftex
  \usepackage[T1]{fontenc}
  \usepackage[utf8]{inputenc}
\else % if luatex or xelatex
  \ifxetex
    \usepackage{mathspec}
    \usepackage{xltxtra,xunicode}
  \else
    \usepackage{fontspec}
  \fi
  \defaultfontfeatures{Mapping=tex-text,Scale=MatchLowercase}
  \newcommand{\euro}{€}
\fi
% use upquote if available, for straight quotes in verbatim environments
\IfFileExists{upquote.sty}{\usepackage{upquote}}{}
% use microtype if available
\IfFileExists{microtype.sty}{%
\usepackage{microtype}
\UseMicrotypeSet[protrusion]{basicmath} % disable protrusion for tt fonts
}{}
\ifxetex
  \usepackage[setpagesize=false, % page size defined by xetex
              unicode=false, % unicode breaks when used with xetex
              xetex]{hyperref}
\else
  \usepackage[unicode=true]{hyperref}
\fi
\hypersetup{breaklinks=true,
            bookmarks=true,
            pdfauthor={},
            pdftitle={},
            colorlinks=true,
            citecolor=blue,
            urlcolor=blue,
            linkcolor=magenta,
            pdfborder={0 0 0}}
\urlstyle{same}  % don't use monospace font for urls
\setlength{\parindent}{0pt}
\setlength{\parskip}{6pt plus 2pt minus 1pt}
\setlength{\emergencystretch}{3em}  % prevent overfull lines
\setcounter{secnumdepth}{0}

\date{}

\begin{document}

Il potenziale di membrana è più vicino al potenziale di equilibrio degli
ioni maggiormente permeabili.

Se varia la permeabilità di uno ione varierà il potenziale di membrana.

\begin{center}\rule{0.5\linewidth}{\linethickness}\end{center}

\textbf{(CONTROLLARE REGISTRAZIONE)}

Siccome il p di ec \textbf{(chiedere Jess)} dello ione varia a seconda
delle sue concentrazioni, se varia la concentrazione dello ione varia
anche il potenziale di equilibrio dello ione.

Le concentrazioni ioniche cellulari si mantengono perlopiù costanti e
non influenzano il potenziale di membrana.

Si può usare il fattore sperimentalmente. Per questi studi si possono
indurre depolarizzazioni cellulari. Si può aumentare la concentrazione
extracellulare del potassio.
\_\_\_\_\_\_\_\_\_\_\_\_\_\_\_\_\_\_\_\_\_\_\_\_\_\_\_\_\_\_\_\_\_\_\_\_\_\_\_\_\_\_\_\_\_\_\_\_\_\_\_\_\_\_\_\_\_\_\_\_\_\_\_\_\_\_\_\_\_\_\_\_\_\_\_\_\_\_\_\_\_\_\_\_\_\_\_\_\_\_\_\_\_\_\_\_\_\_\_\_\_\_\_\_\_\_\_\_\_\_\_\_\_\_\_\_\_\_\_\_\_\_\_\_\_\_\_\_\_\_\_\_\_\_\_\_\_\_\_\_\_\_\_\_

Con il termine \textbf{depolarizzazione} si indica una \emph{diminuzione
in valore assoluto} del potenziale di membrana (il valore si sposta
verso lo 0 da valori negativi). Il modo più semplice per indurre una
depolarizzazione è aumentare la concentrazione esterna del potassio. Se
si abbassa il potenziale di equilibrio dello ione potassio si abbassa
anche il potenziale di membrana.

Se la concentrazione esterna del potassio aumenta, il potenziale di
membrana aumenta come valore numerico ma si depolarizza andando verso un
potenziale nullo.

Il potenziale di membrana di una cellula è un potenziale di diffusione e
non un potenziale di equilibrio.

Fintanto che il potenziale è stabile le correnti sono globalmente nulle,
ovvero si ha un bilancio in pareggio tra le correnti in ingresso e
quelle in uscita.

Le concentrazioni ioniche nei fluidi extracellulari e nel citosol
cellulare si mantengono praticamente sempre costanti; questo fattore può
essere usato sperimentalmente per studi sull'attività elettrica tra le
cellule o nella cellula, inducendo depolarizzazione.

Si prendono in considerazione correnti del sodio e del potassio, il
cloro non ha affetto perché ha potenziale quasi identico a quello di
membrana.

Le correnti di diffusione del cloro sono pressoché nulle, mentre le
correnti di diffusione del potassio e del sodio non sono nulle perché
hanno potenziale diverso da quello di membrana.

Una corrente negativa indica una corrente in ingresso nella cellula
(Na\(^+\)), mentre una corrente positiva indica una corrente in uscita
(K\(^+\)).

Una cellula deve trovarsi in una situazione in cui la corrente in
ingresso è data quasi totalmente da ioni sodio mentre quella in uscita è
data quasi totalmente da ioni potassio. La corrente del potassio deve
essere uguale all'opposto della corrente del sodio (I\(_K\) =
-I\(_N\)\(_a\)). Si ottiene che il rapporto delle conduttanze vale circa
11. Questo ci fa capire che la permeabilità al potassio è poco più di 10
volte quella al sodio nella membrana. Lo ione sodio è sottoposto ad una
forza maggiore del potassio. Una elevata forza elettromotrice muove gli
ioni sodio.

Gli ioni sodio sono esposti ad un forte campo elettrico, mentre quelli
potassio ad uno debole. La forte forza che agisce sugli ioni sodio va
moltiplicata per una piccola conduttanza (la \textbf{conduttanza} è
l'\emph{espressione quantitativa di un conduttore ad essere percorso da
corrente elettrica}), viceversa quella piccola degli ioni potassio viene
moltiplicata per una elevata conduttanza. Questi due prodotti forniscono
la stessa corrente.

Le correnti sono in pareggio e garantiscono l'elettroneutralità ma gli
ioni si muovono.

Entrano ioni sodio ed escono ioni potassio.

Questa è la condizione cellulare a riposo.

Finchè le correnti sono costanti, uguali e opposte come direzione, si
mantiene il potenziale di membrana.

Le concentrazioni intracellulari di sodio e potassio non rimangono
costanti a causa dei continui scambi.

Si ha dissipamento del potenziale di diffusione perché le concentrazioni
vanno ad avvicinarsi così come il potenziale di equilibrio. La
situazione è mantenuta da meccanismi che riportano gli ioni indietro.

Questo sistema è la \textbf{pompa sodio-potassio} la quale pompa ioni
sodio e potassio alla stessa velocità con cui compiono il passaggio
inverso ed impedisce che vengano dissipati quei differenziali. Questa
pompa mantiene la cellula in riposo con un potenziale di diffusione.
Inizialmente non era ben chiaro come gli ioni attraversassero la
membrana.

L'equazione di Goldman parte dal presupposto che la membrana sia un
conduttore permeato da una corrente ionica in tutta la sua matrice
fisica. In realtà gli ioni attraversano la membrana solo in punti
precisi ossia pori realizzati in particolari proteine chiamate
\textbf{canali ionici}.

I canali ionici sono formati da proteine collocate nella membrana che
sporgono da entrambi i lati. Alcuni canali ionici sono sempre aperti
(es. canali del cloro e del potassio, questi sono gli ioni maggiormente
permeabili), altri sono canali \textbf{``gate''} che possono essere
aperti e chiusi da operatori che possono essere di natura diversa.

Si distinguono tre casi:

\begin{itemize}
\itemsep1pt\parskip0pt\parsep0pt
\item
  \emph{canali operati da ligando} (si comportano come i recettori) in
  cui una molecola segnale interagisce con il canale e lo fa aprire;
\item
  \emph{canali operati da forza meccanica} come stiramenti e estensioni
  applicate alla membrana cellulare;
\item
  \emph{canali operati da voltaggio e da potenziale di membrana}, si
  aprono e si chiudono a seconda del potenziale di membrana in
  particolari momenti in cui questo effettua delle variazioni.
\end{itemize}

I \textbf{canali del sodio voltaggio-dipendenti}. Possono essere pensati
come una sorta di tubo realizzato da tanti passaggi trasmembrana
organizzati a moduli (o domini) costituiti da 6 passaggi transmembrana.

In questi domini ci sono sporgenze extracellulari, motivi peptidici che
sporgono, e che possono occludere in maniera efficace o non il poro
centrale.

Uno dei passaggi transmembrana di ciascun dominio è un sensore del
voltaggio. Questa è una porzione della molecola in grado di effettuare
una modificazione conformazionale se cambia il potenziale di membrana. È
una porzione ricca di cariche che risentono della variaziano del
potenziale di membrana producendo un riorientamento della parte della
molecola che chiude o riapre il poro.

Il canale è chiuso o aperto a seconda del valore del potenziale di
membrana. I canali si aprono quando la cellula si depolarizza passando
da valori negativi a valori che vanno verso lo zero.

I \textbf{canali del potassio} sono un po' più complessi e ne esistono
di vari tipi.

Alcuni canali del potassio sono formati da \emph{tetrameri} ossia canali
costiutiti da 4 subunità che formano un poro centrale. Canali di questo
tipo sono voltaggio o calcio dipendenti.

I \textbf{canali leak} sono canali formati da dimeri e sono tipici della
cellula a riposo.

I \textbf{canali del potassio rettificanti anomali (o inward
rectifiers)} sono canali del potassio a 2 \(/alpha\) -eliche
transmembrana (2 STM) che permangono aperti in condizioni di
iperpolarizzazione della membrana e si chiudono quanto questa si
depolarizza. Sono tipici delle cellule muscolari cardiache, stabilizzano
il potenziale quando la membrana è nello stato di riposo, ma quando uno
stimolo sovrasoglia induce un potenziale d'azione, i canali si chiudono
e permettono allo stimolo una durata maggiore. Fanno parte di questa
famiglia i canali del potassio attivati da proteine G.

In tutte queste situazioni so realizza un poro centrale che può essere
chiuso o aperto.

La cellula riesce ad avere proprietà elettriche mantenendo allo stesso
tempo un equilibrio osmotico. L'acqua può infiltrarsi attraverso lo
strato lipidico ma la permeabilità dell'acqua può essere maggiore se
nella membrana si trovano le aquaporine.

\subsection{Equilibrio di Donnan}\label{equilibrio-di-donnan}

Frederick Donnan ha dimostrato che tra due soluzioni acquose separate da
una membrana che sia \emph{impermeabile ad uno solo dei soluti} si
stabilisce un equilibrio garantito da una \emph{differenza di
potenziale} transmebranaria.

Una conseguenza dell'equilibrio di Donnan è che tra i due compartimenti
si stabilisce una \emph{differenza di pressione osmotica}, maggiore nel
compartimento contenente lo ione non diffusibile.

All'equilibrio la membrana assume potenziale di equilibrio.

Il prodotto degli ioni diffusibili ad un lato della membrana è uguale al
prodotto degli ioni diffusibili dall'altro lato della stessa.

All'equilibrio le soluzioni sono elettroneutre. La somma delle cariche
positive è uguale alla somma delle cariche negative. La somma delle
concentrazioni delle sostanze diffusibili da una parte è maggiore di
quella delle stesse dall'altra. Un settore è quello che sta dentro la
cellula, mentre l'altro è quello che sta fuori.

Se si applica alla cellula il modello precedente, si ottiene un
equilibrio con potenziale di membrana e una concentrazione osmotica
intracellulare maggiore di quella del mezzo esterno. In questo modo la
cellula raggiunge l'equilibrio di Donnan e va in equilibrio
elettrochimico, ma la non in equilibrio osmotico. Di conseguenza, per
raggiungere una situazione di equilibrio osmotico, la cellula tenderebbe
a rigonfiare, perdendo però l'equilibrio di Donnan. Questo modello
porterebbe a lisi cellulare.

Abbiamo dunque una continua oscillazione fra i due equilibri.
Introducendo il sodio (Na) extracellulare la cellula raggiunge
l'equilibrio di Donnan. Così ci sono ioni non diffusibili intra e ioni
non diffusibili extracellulari. Si parla di doppio equilibrio di Donnan.
Ci sono cariche diffusibili e non.

L'aggiunta di Na rappresenta un controbilanciamento degli anioni
intracellulari non diffusibili. Il doppio equilibrio non è ancora la
situazione reale della cellula perché non prende in considerazione il
fatto che si tratta di un \emph{potenziale di diffusione} (visti i
continui spostamenti).

Si parla di modello \textbf{``pump and leak''}. Questo è il modello
fisiologico della cellula vicino a quello di Donnan ed è un potenziale
di diffusione mediato da movimenti ionici attivi dovuti alla pompa Na-K
e mediato da correnti.

Essendo gli ioni diversamente permeabili esiste una separazione di
cariche strettamente prossima alla membrana. L'interno presenta una
maggiore quantità di cariche negative mentre l'esterno presenta una
maggiore quantità di cariche positive.

\subsubsection{Regolazione del volume
cellulare}\label{regolazione-del-volume-cellulare}

Le cellule regolano il volume cellulare grazie a trasportatori di vario
tipo. Se il mezzo è ipotonico aumentano il volume rigonfiando, mentre se
il mezzo è ipertonico raggrinziscono.

Dopo una prima fase di aggiustamento isosmotico tornano al volume
iniziale.

Questo è un fenomeno fondamentale: la cellula risponde alla variazione
osmotica mettendo in atto degli accorgimenti che le permettono di
tornare al volume iniziale.

Questi accorgimenti vengono chiamati \textbf{RVD} (Regulatory Volume
Decrease), è un fenomeno che segue all'esposizione della cellula ad una
soluzione ipotonica, e \textbf{RVI} (Regulatory Volume Increase), è un
fenomeno che segue all'esposizione della cellula ad una soluzione
ipertonica.

Con l'RVD la cellula espelle osmoliti e perde acqua riducendo il proprio
volume (sfrutta la permeabilità ai canali potassio e cloro, gli ioni più
permeabili). L'RVI invece, porta all'acquisizione di osmoliti per
acquisire acqua tramite l'aumento della permeabilità al sodio (facendoli
entrare all'interno della cellula), ma anche altri ioni per mezzo di
cotrasportatori sodio-dipendenti.

Gli ioni entrano e la pompa sodio-potassio porta fuori Na\(^+\) e dentro
K\(^+\) per mantenere l'equilibrio. Nei trasportatori transepiteliali le
cellule lavorano sul lato basolaterale in RVD, sul lato apicale con RVI
sfruttando il movimento di ioni da un lato all'altro dell'epitelio.

Nell'elettrofisiologia si tende a considerare una cellula come un
circuito elettrico. Si considerano potenziale, corrente e conduttanza
insieme alla capacità. La \textbf{capacità} è una caratteristica che
impegna cariche in alternativa alla resistenza, ed è una caratteristica
della membrana.

La membrana è assimilabile ad un \emph{conduttore con una}
\textbf{resistenza} ma anche ad un condensatore con una capacità. Il
condensatore accumula cariche su una superficie, proprio come è in grado
di fare la membrana cellulare. Queste cariche non generano corrente ma
sono ``catturate'' dalla membrana.

La \textbf{resistenza} è la matrice attraverso cui passano le cariche
che generano corrente (sono i canali ionici della cellula).

Le proprietà elettriche delle cellule sono studiate in
elettrofisiologia. L'elettrofisiologia stimola le cellule
elettricamente, effettua delle misurazioni e ne ricava dei dati sulle
proprietà elettriche delle cellule.

Le modalità operative utilizzate sono due:

\begin{itemize}
\itemsep1pt\parskip0pt\parsep0pt
\item
  \textbf{current clamp}, significa \emph{``a corrente bloccata''}. In
  questo caso si lavora con cellule vive (in genere con buone correnti
  come i neuroni e le cellule muscolari) normalmente isolate, mantenute
  in coltura e montate su un vetrino in una camera posta su un
  microscopio collegato a \emph{micromanipolatori}. All'interno delle
  cellule in esame si inseriscono degli elettrodi (tubicini di vetro
  pieni di una soluzione che conduce elettricità), e la cellula viene
  posta in un circuito. L'elettrodo presenta una punta inserita nella
  cellula e una punta inserita nel bagno di perfusione che contiene la
  cellula. Al momento dell'immersione dell'elettrodo nel bagno
  contenente le cellule, vengono inviati dei gradini rettangolari di
  potenziale, di solito di 5 mV di ampiezza e qualche ms di durata. Per
  misurare il potenziale si utilizza un voltmetro che misura potenziali
  dell'ordine dei millivolt.
\end{itemize}

In questa modalità di lavoro si applicano alle cellule correnti di
intensità note che vengono attivati istantaneamente e disattivati
istantaneamente (rettangolari). La durata degli impulsi è dell'ordine
dei millisecondi altrimenti la membrana cellulare si rovinerebbe.
L'applicazione della corrente produce una registrazione di potenziale.
Se cambiano le correnti che passano attraverso la membrana cambia anche
il potenziale. Questo esperimento consente di misurare resistenza e
capacità.

Il potenziale di membrana in ogni istante del processo è uguale al
potenziale \textbf{(???)}

(immagine p191)

V\(_m\) = V\(_0\) + (V\(_f\)-V\(_0\)) e\^{}(t/Rm + m)

(immagine1)

Non si ha una relazione lineare, la curva è la reazione effettiva.

Con correnti crescenti si hanno potenziali più depolarizzati.

La resistenza specifica è molto elevata rispetto a quella dei componenti
elettrici degli elettrodomestici mentre la capacità è molto bassa.

La resistenza della membrana non è costante. Quando gli sperimentatori
hanno cercato di assimilare la membrana cellulare ad un circuito
elettrico speravano fosse comodo applicare le leggi della fisica che
regolano l'andamento dei parametri elettrici in un circuito elettrico.

Interessante è la relazione tra intensità di corrente, potenziale,
resistenza o conduttanza. La legge di Ohm mette in relazione potenziale
e conduttanza.

In maniera grafica è rappresentata da una retta di equazione I = gV e I
= 1/R * V.

Su una stessa cellula è possibile applicare correnti sempre più forti e
registrare i potenziali conseguenti. Se la cellula si comporta come un
conduttore ohmico ci deve essere una relazione lineare tra corrente
applicata e potenziale registrato.

Metto poi in relazione facendo il grafico con i valori di corrente e il
potenziale registrato per ognuno di essi. Verifico se portano ad una
relazione lineare. La linearità ci dice che la conduttanza è costante.

Gli sperimentatori si accorsero che ciò nella realtà non si verifica.
Infatti la conduttanza della membrana, man mano che il potenziale
aumenta e la cellula si depolarizza, aumenta.

La conduttanza aumenta con la depolarizzazione. Vedi schema. Si è capito
il perché studiando l'attività dei canali.

Per fare questo è stato necessario adottare un'altra modalità di lavoro
che è il \textbf{voltage clamp} (letteralmente ``blocco del
voltaggio''). Questa metodica sperimentale è atta a \emph{separare la
componente resistiva e capacitiva} e a misuratre nello specifico solo
quella resistiva al variare del potenziale di membrana.

Per fare questo viene utilizzato un \textbf{amplificatore
differenziale}. Questo è uno strumento che riceve ai suoi ingressi (+ e
-) due segnali di potenziale elettrico, che moltiplica per un fattore
arbitrario; esso restituisce quindi in uscita un segnale di potenziale
che è il risultato della sottrazione (o comunque combinazione lineare)
dei segnali in ingresso. Il potenziale in uscita può poi essere
trasformato in un segnale di corrente tramite un convertitore
voltaggio-corrente.

Un microelettrodo inserito in una cellula ne ``legge'' il potenziale di
membrava \textbf{V\(_m\)} (rispetto a un elettrodo di riferimento posto
nella camera portacampione) e lo invia a uno degli ingressi
dell'amplificatore differenziale. Lo sperimentatore invia quindi
all'altro ingresso un valore di potenziale elettrico
\textbf{V\(_c\)\(_l\)} (\emph{potenziale di clamp o di comando}) che
vuole imporre alla membrana. L'amplificatore opera il confronto tra i
due segnali ed eroga una corrente di segno e ampiezza tali da
minimizzare la differenza tra V\(_m\) e V\(_c\)\(_l\). Il tutto funziona
con la logica di un feedback negativo e la corrente iniettata
dall'amplificatore è pertanto detta \emph{corrente di feedbacK}.

Nel metodo del current clamp si lavora a corrente bloccata, mentre in
quello di voltage clamp si misura la corrente tenendo fermo il
potenziale (l'opposto).

L'attività dei canali risponde alla variazione di potenziale, perciò man
mano che la cellula si depolarizza ci sono canali ionici che si aprono
in risposta alla variazione. L'apertura dei canali porta ad un aumento
della conduttanza perché si aprono nuove vie alla corrente ionica.

La cellula \emph{non} è un conduttore ohmico; nella cellula minime
perturbazioni di corrente portano ad ampie variazioni di resistenza e
conduttanza.

Nel voltage clamp si può creare un tracciato di corrente per vedere sia
la componente capacitiva che quella resistiva. Quando viene applicato il
potenziale la corrente schizza subito ad un livello molto alto per poi
abbassarsi ed equilibrarsi. La prima è la corrente capacitiva. Sono una
quota di cariche che vanno a caricare il condensatore. La funzione
resistiva si sviluppa nel tempo fino ad assumere un valore massimo e
costante. La relazione tra le componenti non è lineare.

Man mano che si accumulavano dati sul ruolo dei vari ioni è diventato
importante studiare l'attività dei singoli canali specifici per
determinati ioni. Una tecnica per selezionare i canali è chiamata
\textbf{patch clamp}. Strappa frammenti di membrana facendoli aderire
alla punta di un elettrodo. Si utilizzano elettrodi con punte più larghe
che succhiano pezzi di membrana strappandola via. L'elettrodo non viene
inserito a perforare la membrana. L'adesione della membrana
all'elettrodo deve essere perfetta per impedire fughe di correnti che
non vengono registrate. La corrente che entra nell'elettrodo deve
passare attraverso la membrana e se l'adesione dell'elettrodo non è
perfetta l'esperimento non è valido. Si parla di \textbf{seal} come
esperimento. Con un buon seal si registra solo che correnti passano
attraverso l'elettrodo considerando solo i canali che si trovano in
prossimità della superficie dell'elettrodo. Si registrano anche correnti
di singolo canale (picoampère).

Le correnti ci danno anche la visualizzazione di come funziona un canale
e l'idea di canale chiuso e aperto. Un canale è un sistema che funziona
come un tutto o un nulla: se aperto passa corrente, se chiuso non passa.
Se il canale è inattivo non passa corrente anche se in realtà abbiamo
brevi momenti di apertura del canale che causa correnti di alcuni
picoampère. Se il canale è attivato da voltaggio prevalgono i momenti di
passaggio di corrente a quelli di assenza di corrente. La corrente è
anche più forte. Per studiare un canale alcuni fanno esprimere il gene
di quel canale (che è una proteina o più proteine) in un modello
cellulare opportuno tra cellule che esprimono pochi canali come oociti
di rana o rospo (poche correnti basse e dimensioni grandi che si
ottengono facilmente e sono facilmente manipolabili) altre cellule sono
le HEK (cellule di rene di embrione umano che esprimono poche correnti e
sono facilmente transfettabili e quindi è facile fargli esprimere un
gene alieno). La cellula esprime il DNA del gene per il canale e nelle
cellule che esprimono queste canale effettuare misure di patch clamp.
Saranno prevalenti i canali indotti tramite manipolazione genetica e si
può registrare l'attività di questi canali. In questo modo si possono
caratterizzare le attività dei canali ionici. Queste si caratterizzano
realizzando due tipi di dati (dato di corrente e relazione tra corrente
e potenziale). Si usa il grafico I-V oppure le tracce di corrente.

\end{document}
