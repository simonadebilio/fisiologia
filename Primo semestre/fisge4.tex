\documentclass[]{article}
\usepackage{lmodern}
\usepackage{amssymb,amsmath}
\usepackage{ifxetex,ifluatex}
\usepackage{fixltx2e} % provides \textsubscript
\ifnum 0\ifxetex 1\fi\ifluatex 1\fi=0 % if pdftex
  \usepackage[T1]{fontenc}
  \usepackage[utf8]{inputenc}
\else % if luatex or xelatex
  \ifxetex
    \usepackage{mathspec}
    \usepackage{xltxtra,xunicode}
  \else
    \usepackage{fontspec}
  \fi
  \defaultfontfeatures{Mapping=tex-text,Scale=MatchLowercase}
  \newcommand{\euro}{€}
\fi
% use upquote if available, for straight quotes in verbatim environments
\IfFileExists{upquote.sty}{\usepackage{upquote}}{}
% use microtype if available
\IfFileExists{microtype.sty}{%
\usepackage{microtype}
\UseMicrotypeSet[protrusion]{basicmath} % disable protrusion for tt fonts
}{}
\ifxetex
  \usepackage[setpagesize=false, % page size defined by xetex
              unicode=false, % unicode breaks when used with xetex
              xetex]{hyperref}
\else
  \usepackage[unicode=true]{hyperref}
\fi
\hypersetup{breaklinks=true,
            bookmarks=true,
            pdfauthor={},
            pdftitle={},
            colorlinks=true,
            citecolor=blue,
            urlcolor=blue,
            linkcolor=magenta,
            pdfborder={0 0 0}}
\urlstyle{same}  % don't use monospace font for urls
\setlength{\parindent}{0pt}
\setlength{\parskip}{6pt plus 2pt minus 1pt}
\setlength{\emergencystretch}{3em}  % prevent overfull lines
\setcounter{secnumdepth}{0}

\date{}

\begin{document}

\section{\texorpdfstring{Meccanismi Off del segnale del
Ca\(^2\)\(^+\)}{Meccanismi Off del segnale del Ca\^{}2\^{}+}}\label{meccanismi-off-del-segnale-del-ca2}

Questo ione si trova ad una concentrazione citosolica molto bassa,
mentre all'esterno della cellula si trova in soluzione ad una
concentrazione comparabile a quella del K\(^+\) (ricade tra gli ioni più
abbondanti).

La differente concentrazione del calcio tra l'esterno e l'interno della
cellula permette dei temporanei aumenti della concentrazione dello ione.
Questo fenomeno non avviene per altri ioni che rimangono praticamente
sempre costanti in soluzione nel citosol e all'esterno.

Il Ca\(^2\)\(^+\) è lo ione che presenta le più ampie variazioni a
livello citosolico (altri ioni sono lo zinco, il rame, ma sono fenomeni
molto piu contenuti).

Le variaizoni del calcio hanno un forte impatto sul metabolismo
cellulare.

Com'è possibile che la cellula mantenga la concentrazione di questo ione
nel citosol molto bassa? Questo meccanismo viene detto ``omeostasi del
calcio''.

\subsection{L'omeostasi del calcio}\label{lomeostasi-del-calcio}

Il Ca\(^2\)\(^+\) lega facilmente molte proteine e questo è già un
motivo per cui la sua concentrazione tende a rimanere molto bassa.
Questo ione si lega a delle molecole che a loro volta si legano alla
cellula e viene accumulato internamente alla cellula come riserva.

Un'elevata concentrazione del calcio (alcune centinaia di nM) non viene
tollerata dalla cellula per tempi lunghi, ovvero tempi superiori a
qualche decina di sec.~Superata una concentrazione di 1 \(\mu\)M la
cellula inizia a subire danni. Tuttavia questa è una situazione che
fisiologicamente non si verifica mai.

In seguito al verificarsi di aumenti transitori del calcio, questo viene
rapidamente riportato alle concentrazioni basali, tra 50 e 100 nM (non
c'è un valore fisso), da una serie di sistemi.

I principali sistemi sono due pompe del calcio: \textbf{SERCA}, una
calcio ATPai del reticolo sarco/endoplasmatico e \textbf{PMCA}, una
pompa presente sulla membrana plasmatica.

Questi sono trasportatori attivi primari, ossia creano il gradiente del
calcio. Consumano ATP per trasportare lo ione contro un gradiente molto
accentuato.

Sulla membrana esiste poi un altro trasportatore. Questo è un
\textbf{antiporto sodio-calcio}, ovvero un cotrasportatore che trasporta
sodio all'interno della cellula e calcio all'esterno di essa. In mote
cellule il suo ruolo è fondamentale, tanto che se inibito causa la
perdita dell'omeostasi del calcio nella cellula.

Questo antiporto è estremamente dipendente dalla \emph{pompa
sodio-potassio} la quale mantiene il gradiente del sodio; se questa
viene inibita salta l'omeostasi del calcio perché viene meno la funzione
dell'antiporto.

LA \textbf{PMCA} è una pompa del calcio presente sulla membrana
plasmatica e in grado di trasportare lo ione dal citosol all'esterno (lo
espelle dalla cellula).

Esistono 4 isoforme di questa pompa, di cui due specifiche del tessuto
nervoso. I tessuti con le maggiori dinamiche del calcio sono quello
nervoso e quello muscolare (sono le cellule che presentano le più forti
correnti del calcio e dunque anche i più potenti sistemi di omeostasi).
Le correnti del calcio si studiano molto meglio in queste cellule
piuttosto che in altri tipi cellulari dove è piu difficile metterle in
evidenza.

Altri tessuti presentano altre isoforme.

Questa pompa è presente in tutti i tipi cellulari perchè è il principale
fattore di omeostasi del calcio in quasi tutti i tipi cellulari.

Nel sistema nervoso ne troviamo due isoforme specifiche.

Questa pompa presenta 10 segmenti transmembrana e grosse parti
citosoliche. Può essere fosforilata e legare la calmodulina.

Il suo funzionamente può essere regolato da vari eventi metabolici
cellulari.

Dal punto di vista biochimico-fisico il trasportatore ha un'elevata
affinità e bassa capacità.

Questo vuol dire che la porma riesce a legare lo ione anche a
concentraizoni bassissime, ma ha un tasso di trasporto lento (trasporta
moli al sec). Entrambe queste caratteristiche, ovvero la K\(_m\) e la
V\(_m\)\(_a\)\(_x\), possono modificarsi se la pompa viene modificata o
se lega la calmodulina.

Un sistema analogo è presente sul reticolo endoplasmatico ed è chiamato
\textbf{SERCA}.

Il reticolo endoplasmatico della cellula muscolare striata è chiamato
\emph{sarco-endoplasmatico} perchè ha una conformazione e una
funzionalità particolare.

Questa pompa presenta \emph{3 isoforme}: una del muscolo scheletrico,
una del muscolo cardiaco e una presente negli altri tessuti.

Questa pompa ha un'attività importante nel tessuto muscolare striato
dove è espressa ad altissimi livelli e funziona maggiormente quanto più
il calcio citosolico aumenta.

L'\textbf{antiporto} invece lavora con bassa affinità ed elevata
capacità: ha un'affinità per lo ione \emph{inferiore} a quella della
PMCA, ma ha una capacità (ovvero una velocità) di trasporto maggiore di
quello della PMCA (trasporta circa 2000 ioni al secondo per ogni singolo
antiporto).

La stechiometria in alcuni tessuti è di \emph{3 Na\(^+\) : 1
Ca\(^2\)\(^+\)} (ma non sempre, ad esempio nei bastoncelli della retina
è \emph{4 Na\(^+\) : 1 Ca\(^2\)\(^+\) + 1 K\(^+\)}.

Anche questo tipo di trasportatore espelle lo ione dalla cellula (come
la PMCA) mentre la SERCA sequestra lo ione all'intenro del reticolo
sarcoendoplasmatico.

In ogni caso tutte queste strutture sottraggono ioni Ca\(^2\)\(^+\) al
citosol quando la concentrazione è abbondante.

L'antiporto lavora solo a livelli molto alti di calcio e con resa molto
grande.

A livelli bassi di calcio si lavora molto più lentamente, agisce in
questo modo la PMCA. Agiscono anche quando la concentrazione dello ione
è estremamente bassa. I sistemi a bassa affinità possono lavorare solo
oltre ad una certa concentrazione dello ione poichè al di sotto di essa
non riescono più a legarlo a causa di un'affinità troppa bassa.

I sistemi con un'elevata affinità invece, riescono a legare il calcio
anche quando ha concentrazioni bassissime e lo rilasciano dove le
concentrazioni sono elevate.

Dal punto di vista chimico-fisico l'attività di questi trasportatori è
controintuitiva perchè possono legare lo ione dove la concentrazione è
bassa e liberarlo dove è alta. La cellula adotta vari sistemi con vari
gradi di affinità: quando il Ca\(^2\)\(^+\) è alto intervengono sistemi
con bassa affinità ma elevata capacità perchè il Ca\(^2\)\(^+\) deve
essere portato via in fretta.

Ci sono poi dei sistemi che regolano dinamiche ancora più lente del
calcio. Tra questi troviamo i mitocondri, che possono sequestrare il
calcio dal citosol e accumularlo al loro interno per poi rilasciarlo al
momento opportuno.

I mitocondri dispongono di vari trasportatori del calcio sulla membrana
interna.

La membrana esterna è poco selettiva mentre quella interna è molto
selettiva.

Inoltre la membrana interna del mitocondrio è carica elettricamente e ha
un potenziale di membrana molto forte a causa del trasporto di ioni
idrogeno al suo esterno. Il mitocondrio è un potentissimo trasportatore
di ioni idrogeno.

In condizioni basali il Ca\(^2\)\(^+\) tende ad entrare nel mitocondrio.
Nel mitocondrio la concentrazione del calcio è molto più elevata che nel
citosol, eppure esso è in grado di incorporare passivamente lo ione.

La concentrazione del calcio nel mitocondrio è circa doppia rispetto a
quella del citosol.

La forza elettrochimica trainante che agisce sullo ione è data dalla
differenza tra il potenziale elettrochimico dello ione meno il
potenziale di membrana del mitocondrio. Il gradiente elettrochimico
spinge lo ione all'interno del mitocondrio, ovvero nel compartimento in
cui esso è più concentrato per effetto di un potenziale mitocondriale
elevato.

Tuttavia questo non è l'unico movimento che lo ione può fare attraverso
la membrana mitocondriale.

Non sono stati descritti veri e propri canali del calcio mitocondriali.
I principali sistemi utilizzati dal mitocondrio sono:

\begin{itemize}
\itemsep1pt\parskip0pt\parsep0pt
\item
  un \emph{uniporto}, trasferisce lo ione dal citosol all'interno del
  mitocondrio che è negativo rispetto al citosol per via del forte
  gradiente protonico della membrana interna;
\item
  un \emph{antiporto Na/Ca} che veicola il calcio all'esterno;
\item
  un \emph{antiporto H/Ca} (calcio all'esterno);
\item
  un trasportatore capace di sequestrare il calcio rapidamente quando lo
  ione aumenta a livello citosolico.
\end{itemize}

Fisiologicamente il mitocondrio è impegnato soprattutto in attività di
sequestro del calcio. Questo avviene generalemnte quando lo ione
raggiunge concentrazioni elevate nel citosol o quando le concentrazioni
sono elevate per lunghi periodi.

Si pensa che l'accumulo dello ione nel mitocondrio diventi significativo
quando questo supera la concentrazione citosolica di 1 \(\mu\)M.

A questi livelli di Ca\(^2\)\(^+\) nel citosol il mitocondrio diventa il
principale sistema di abbasamento della concentrazione del
Ca\(^2\)\(^+\) nel citosol, dopodichè intervengono anche gli altri
sistemi visti.

Nei mitocondri si trova anche il \textbf{PTP (Permeability Transition
Pore)}, un sistema legato all'apoptosi e attraverso cui il calcio esce
massicciamente dall'organello.

Esiste dunque un'interazione tra calcio e apoptosi nella quale il
mitocondrio è profondamente coinvolto.

Il gradiente del Ca\(^2\)\(^+\) è un gradiente molto forte che richiede
molta forza per essere creato e mantenuto. Questa forza non è tutta a
carico di un solo sistema ma ce n'è più di uno:

\begin{itemize}
\itemsep1pt\parskip0pt\parsep0pt
\item
  la \textbf{Ca\(^2\)\(^+\) ATPasi}, che mantiene un livello bassissimo
  di calcio nel citosol;
\item
  l'\textbf{antiporto Na\(^+\)/Ca\(^2\)\(^+\)}, che ha una funzione
  essenziale nel picco del calcio;
\item
  il \textbf{mitocondrio}, che ha una funzione molto importante quando
  il calcio tende a stazionare a livelli elevati. Un esempio è quando la
  cellula è stimolata a lungo da un ormone che fa aumentare il
  Ca\(^2\)\(^+\). In questo caso lo ione tenderebbe a salire fino a
  valori non fisiologici e dannosi per la cellula. Quando lo stimolo
  cessa il mitocondrio aiuta la cellula a ristabilire le concentrazioni
  fisiologiche.
\end{itemize}

Il calcio è l'unico ione a dare manifestazioni molto evidenti poichè
praticamente non è presente nel citosol; basta un minimo scompenso
nell'omeostasi perchè la sua concentrazione aumenti, mentre per farla
ritornare ai livelli adeguati (bassi) devono intervenire sistemi con una
potenza notevole rispetto all'omeostasi di tutti gli altri ioni.

Tramite lo studio di cellule in coltura cresciute in una situazione dove
le concentrazioni del Ca\(^2\)\(^+\) sono lontane dall'omeostasi, si è
notato che la cellula perde il citoscheletro diventando sferica,
successivamente inizia a rilasciare delle vescicole (\emph{blebbing}) e
alla fine la membrana degenera (la cellula perde l'impermeabilità e va
in rapida degenerazione).

Questo è un disordine del calcio rapido e brutale che porta la cellula a
morte per disfacimento, una morte non descrivibile con dei processi
biologici. In questo caso si parla di \emph{necrosi}.

Anche un disordine di minore intensità ma prolungato nel tempo (il
calcio aumenta più gradualmente ma perdura nel tempo) può portare la
cellula alla morte tramite un processo chiamato \emph{apoptosi}. In
questo tipo di fenomeno si generano un ingresso e un rilascio di calcio
dal RE.

Quando le cellule vanno in apoptosi presentano delle modificazioni
morfologiche che possono essere descritte perchè ripetibili (si
osservano sempre in corrispondenza di questo fenomeno).

In una cellula che si predispone all'apoptosi abbiamo delle
modificazioni mitocondriali che portano alla formazione di un \emph{poro
di transizione}; questo è un complesso molecolare che si forma sulla
membrana mitocondriale interna e che la rende permeabile a molti soluti
tra cui il calcio. Un altro evento è il \emph{rilascio del citocromo C}
sempre attraverso il poro di transizione. Il citocromo C è abbondante
nella matrice e induce la cascata di apoptosi nel citosol tramite
l'attivazione delle \textbf{capsasi} (proteasi). In questa serie di
eventi si ha ingresso di calcio prima della formazione del poro, mentre
dopo la sua formazione si ha rilascio di calcio dal mitocondrio.

Il forte rilasico di calcio durante questi fenomeni aumenta il fenomeno
stesso.

Perchè il calcio esce dal mitocondrio quando si forma il poro di
transizione? Tendenzialmente il mitocondrio non rilascia Ca\(^2\)\(^+\)
ma lo sequestra.

Anche il mitocondrio presenta una sua omeostasi del calcio, ma con la
formazione del poro si rompe l'equilibrio delle pompe e si ha di
conseguenza una caduta del potenziale di membrana mitocondriale
(\emph{psi}). La caduta del potenziale di membrana è uno dei segnali
principali dell'inizio del processo di apoptosi.

La caduta del potenziale inverte il gradiente elettrochimico del
Ca\(^2\)\(^+\) così che il potenziale di membrana tenda ad andare verso
il potenziale di equilibrio, facendo sì che il Ca\(^2\)\(^+\) esca dal
mitocondrio (ne possiede estremamente di più rispetto al citosol).

Il poro di transizione si origina dall'aggregazione di molte molecole,
alcune della membrana esterna e alcune della membrana interna (si forma
un ponte tra le due membrane). Le moleocle coinvolte sono molte.

Questo fenomeno causa una caduta del potenziale di membrana, una caduta
della concentrazione dell'ATP e un aumento delle ROS.

\subsection{\texorpdfstring{La decodificazione del segnale del
Ca\(^2\)\(^+\)}{La decodificazione del segnale del Ca\^{}2\^{}+}}\label{la-decodificazione-del-segnale-del-ca2}

Il sistema cellulare utilizza lo ione come segnale facendo avvenire
delle variazioni di concentrazione dello ione stesso (per fare questo
deve lavorare a bassi livelli di concentrazione). Lo ione oscilla tra
concentrazioni molto basse (tra 100 nM e 1 \(\mu\)M) che gli permettono
di variare in tempi molto brevi (si parla di millisecondi), perchè
smuovere concentrazioni maggiori richiederebbe tempi troppo lunghi.

Ma cosa avviene quando il calcio aumenta a livello citosolico? Questo
segnale viene decodificato metabolicamente dalla cellula.

Lo stimolo indotto dallo ione può tradursi o nell'entità dell'aumento o
nella frequenza delle oscillazioni. La frequenza delle oscillazioni
riflette l'intensità dello stimolo extracellulare.

Queste sono situazioni che si verificano dopo l'applicazione dell'ormone
vasopressina a concentrazioni crescenti. La potenza dell'azione ormonale
si traduce in una maggior frequenza di oscillazione.

La conversione di queste oscillazioni in variazioni metaboliche è dovuta
a proteine tra cui la \emph{calmodulina} e la \emph{troponina c}.

La clamodulina è una piccola proteina citosolica specializzata nel
legame con il Ca\(^2\)\(^+\). Questa proteina presenta due zone
globulari e una lineare. Nelle due zone globulari sono presenti dei siti
di legame del calcio. Questi siti di legame sono motivi tipici di
proteine che legano il calcio. Nel mondo delle proteine esistono dei
motivi che si ripetono.

Il motivo delle proteine che lega il calcio è chiamato \textbf{``mano
EF''}. Questa regione ha una struttura elica-ansa-elica (ricordano due
dita di una mano come pollice ed indice).

Per ogni sito di legame del calcio la calmodulina ha un sito
\emph{mano-EF} (in totale 4).

La calmodulina ha un'affinità per il calcio tarata appena al di sopra
del livello di base del calcio, quindi una costante di affinità intorno
al 100 nM, o poco sopra. La costante di affinità di una molecola per
un'altra ci dice quale sarà l'intervallo o la zona di livelli di
concentrazione nei quali l'interazione avverrà preferenzialmente.

La calmodulina lega il calcio non appena questo sale sopra il livello
basale. Questa proteina ``sente'' il calcio non appena si verifica un
segnale del calcio, dopodichè lega lo ione. A questo punto la
\textbf{CAM (calcio-calmodulina)} può interagire con altre proteine
(questo è il passaggio che conduce dall'azione dello ione al
metabolismo).

Le altre proteine con cui può interagire la CAM sono prevalentemente
enzimi e lo ione condiziona altri enzimi grazie al suo legame con la
calmodulina.

Lo ione può poi attivare direttamente altre proteine. Uno dei fattori
principali attivati dal calcio è la \textbf{protein-chinasi C} (\emph{c}
sta per calcio).

Ne esistono molte isoforme con azioni differenti. L'attivazione della
protein chinasi C, come la A, ha una ripercussione notevole sul
metabolismo cellulare.

Le chinasi sono enzimi multifunzione perchè attivano varie cose nella
cellula. Il calcio e l'AMP-ciclico sono detti ``secondi messaggeri''
perchè aumentano nella cellula in risposta a vari ormoni.

\section{Il potenziale d'azione}\label{il-potenziale-dazione}

Il potenziale di azione cellulare è un potenziale di diffusione che si
mantiene costante fintanto che le correnti sono costanti (queste sono
costanti finchè la permeabilità della membrana ai relativi ioni è
costante).

Il potenziale d'azione può essere pensato come una ``fuga''
dall'omeostasi con ritorno.

Osserviamo il potenziLe d'azione in cellule ben dotate di canali ionici.
Il potenziale d'azione consiste in una variazione repentina del
potenziale di membrana e richiede dunque forti correnti ioniche che a
loro volta richiedono un'elevata presenza di canali ionici.

Questo è stato osservato in cellule nervose e muscolari (non è stato
descritto in nessun altro tipo cellulare).

Il potenziale d'azione consiste in una forte depolarizzazione della
membrana cellulare fino a valori positivi; si ha un'inversione, il polo
positivo e quello negativo si invertono, quello positivo diventa
intracellulare mentre quello negativo diventa extracellulare.

Questo fenomeno si manifesta in maniera diversa in cellule diverse, ed è
stato ben descritto nelle cellule nervose, muscolari striate e
cardiache.

L'unità di misura del fenomeno è il millisecondo. Vi è una rapida
depolarizzazione che porta il potenziale da -60 a +10/20/40 mV seguito
da una rapida ripolarizzazione. Il potenziale di membrana scende a
valori più negativi del riposo per ritornare gradualmente al potenziale
di riposo. Nelle cellule cardiache e muscolari lisce questo fenomeno è
un po' meno rapido.

\textbf{(descrizione grafico Jess)}

Il potenziale d'azione ha bisogno di un innesco che consiste in una
debole depolarizzazione di 15 mV rispetto al potenziale basale.

Deve essere raggiunta una certa soglia di depolarizzazione, dopodichè
parte in maniera esplosiva e con un'elevata velocità la depolarizzazione
fino a portare il potenziale a valori positivi, dopodichè la
depolarizzaizone si arresta e a quel punto il potenziale cellulare
ricomincia a ripolarizzarsi con una discesa che riporta il potenziale
fino al valore basale.

Il potenziale d'azione si verifica con una modalità del tipo ``tutto o
nulla'' e dato un certo tipo cellulare si manifesta sempre nello stesso
modo, indipendentemente dallo stimolo iniziale. La depolarizzazione che
porta a superare il livello soglia può dipendere da un impulso più o
meno forte, ma ciò non varia l'intensità del potenziale d'azione
indotto.

La membrana cellulare dopo l'attuazione di un potenziale d'azione entra
in uno stato refrattario; un secondo stimolo, immediatamente successivo
ad un potenziale d'azione, non induce un nuovo potenziale anche se tale
da superare la soglia per provocare un innesco.

È necessario un certo intervallo di tempo prima che la membrana torni
responsiva.

La durata della refrattarietà è legata alla durata del potenziale
d'azione. In una cellula cardiaca si parla di centinaia di millisecondi.

La refrattarietà non scompare improvvisamente ma c'è un periodo di
refrattarietà assoluta in cui nessuno stimolo può indurre un potenziale
di azione, seguita poi da una refrattarietà relativa nella quale la
soglia è più alta del normale.

Per capire il fenomeno sono state studiate le correnti ioniche legate al
potenziale d'azione. Finchè le correnti sono costanti il potenziale è
costante. Se cambia il potenziale significa che stanno cambiando le
correnti.

Il fenomeno è stato descritto la prima volta intorno alla metà del
secolo scorso studiando gli assoni del calamaro. In questo animale si
trovano neuroni piuttosto grandi che hanno assoni altrettanto grandi
facilmente manipolabili e studiabili.

È relativamente facile studiare in vitro, tramite l'utilizzo di
elettrodi, lo stimolo del potenziale d'azione e seguirne la variazione.
Con il metodo del voltage clamp si possono studiare le correnti che
attraversano la membrana.

Studiando questo fenomeno gli sperimentatori notarono una corrente di
ingresso che si sviluppa rapidamente, raggiunge un massimo, dopodichè
tende a scemare.

Successivamente si nota una corrente nulla finchè il fenomeno prosegue
con una corrente in uscita (variazione delle correnti).

Poichè la corrente si modifica continuamente anche il potenziale si
modifica. Infine si raggiuge una corrente finale in uscita che si
manterrà tale finchè verrà mantenuto lo stimolo di depolarizzazione.

Da questo insieme di dati gli studiosi hanno dedotto che nel potenziale
d'azione si verifica un iniziale corrente al sodio in ingresso che poi
svanisce e contemporaneamente si genera una corrente al potassio in
uscita (la prima depolarizza, la seconda ripolarizza).

Dopo un certo tempo, dopo che la membrana si è depolarizzata, le
correnti ioniche saranno pari a zero perchè ormai il potenziale
d'equilibrio sarà stato raggiunto e dunque gli ioni non si muoveranno
più.

Nelle cellule nervose e muscolari sono presenti numerosi canali del
sodio voltaggio dipendenti che si aprono e chiudono rispondendo a una
depolarizzazione della membrana.

Depolarizzando la membrana si ha un aumento progressivo della corrente
del sodio in ingresso, mentre a potenziali più elevati la corrente del
sodio in ingresso tende a ridursi. Intorno a +60mV la corrente cessa
perché il potenziale di membrana vale come il potenziale di equilibrio
del sodio.

A questo punto la forza di trazione sullo ione è zero. Proseguendo nella
depolarizzazione si arriverebbe ad un punto in cui la corrente si
invertirebbe, tuttavia queste non sono condizioni fisiologiche

I canali del sodio lavorano a intervalli negativi nella realtà,
generando corrente e la prima fase del potenziale di membrana. Non si
raggiunge neanche il potenziale di equilibrio del sodio.

Anche i canali potassio sono voltaggio dipendenti e si aprono quando la
membrana si depolarizza. I canali del potassio, a differenza di quelli
del sodio, non si inattivano. La corrente fisiologica di un singolo
canale subisce un aumento della corrente e poi torna a zero.

I canali al sodio sono canali inattivanti; rimangono aperti per un tempo
brevissimo e poi cadono in uno stato inattivo, mentre quelli del
potassio non si inattivano (si chiudono quando la membrana si
ripolarizza).

I canali del potassio non si inattivano sotto lo stimolo, ma rimangono
sempre aperti finchè la membana è depolarizzata e si richiudono quando
la membrana si ripolarizza.

Il potenziale di equilibrio del potassio è più basso del potenziale di
riposo. Si avrà dunque una corrente in uscita sempre più forte man mano
che avviene la depolarizzazione.

Il fenomeno della depolarizzaizone si innesca a causa del raggiungimento
di un \emph{valore soglia}. Inoltre, quando si aprono i canali del sodio
si induce un effetto depolarizzante (questo fenomeno si autoalimenta).
Con la depolarizzazione iniziale cominciano ad aprirsi i canali del
sodio, che subito dopo si disattivano (danno solo una spinta
depolarizzante per un msec o meno).

Più canali del sodio si apriranno più la spinta depolarizzante sarà
forte.

Sotto il livello soglia il fenomeno non parte in maniera così esplosiva.
Se il fenomeno iniziale è così intenso da reclutare una buona quantità
di canali al sodio contemporaneamente la spinta allora sarà tale da
creare una depolarizzazione massiccia dovuta al reclutamento a cascata
di canali. Questo fenomeno proseguirà finchè tutti i canali sodio non
saranno depolarizzati e la spinta scemerà. Man mano che il sodio dà
luogo all'esplosiva depolarizzazione si aprono i canali del potassio,
che una volta tutti aperti creeranno una forte corrente del potassio.

Nel frattempo la corrente del sodio si esaurisce e si mantiene una forte
corrente al potassio in uscita che porta a ripolarizzazione.

Il potenziale di membrana tende ad andare verso il potenziale di
equilibrio del potassio e per questo scende anche a livelli più bassi di
quelli basali iniziali (la membrana è ancora permeabile al potassio). In
seguito la membrana perde la permeabilità al potassio (i canali del
potassio sono voltaggio dipendenti, perciò si richiudono) permettendo al
potenziale di membrana di tornare ai suoi livelli basali.

La refrattarietà si spiega con il fatto che abbiamo i canali del sodio
inattivati, ma i canali del potassio sono ancora parzialmente aperti. La
membrana dunque non reagisce alla depolarizzazione perché la stessa è
ostacolata dai canali al potassio ancora parzialmente aperti.

È la pompa sodio-potassio a ripristinare continuamente le concentrazioni
di questi ioni.

\subsection{La propagazione del potenziale
d'azione}\label{la-propagazione-del-potenziale-dazione}

Il potenziale d'azione non è un evento che si manifestra solo in un
punto della membrana ed è in grado di effettuare un'attività ciclica, ma
si propaga lungo la stessa.

In una membrana, che possiamo considerare come una superficie piatta, il
potenziale d'azione si propaga in maniera circolare allargandosi su di
essa.

Come mai? Il potenziale d'azione provoca una depolarizzazione. La
depolarizzazione rende l'interno della cellula meno negativo e l'esterno
un po' meno positivo; quando si arriva a potenziale 0 non c'è
polarizzazione elettrica, mentre quando si supera il valore dello 0
l'esterno della cellula diventa negativo e l'interno diventa positivo.

Questo evento si manifesta in un certo punto, in un piccolo tratto della
membrana cellulare, ma nelle zone adiacenti tutto è a riposo.

Quindi si avrà una zona della membrana dove si è verificato il
potenziale di azione in cui è stata invertita la polarità e zone della
membrana adiacenti a questo punto con polarità opposte (sullo stesso
lato della membrana). Si ha un'inversione di potenziale.

Due regioni con diverso potenziale elettrico generano un campo elettrico
che agisce sulle cariche elettriche. In questa situazione gli ioni in
soluzione generano correnti elettriche. Ci sono correnti elettriche che
si muovono da regioni esterne alla zone del potenziale d'azione e questo
avviene lungo tutto il bordo del potenziale di azione. Queste correnti
(correnti elettrotoniche) provocano una depolarizzazione delle regioni
adiacenti portando cariche positive all'esterno e negative all'interno.
In questo modo viene depolarizzata la membrana (stessa cosa che succede
aprendo i canali del sodio).

La depolarizzazione è indotta da correnti elettrotoniche che viaggiano
parallelamente alla membrana. La depolarizzazione è blanda ma
sufficiente per innescare un potenziale di azione in zone adiacenti al
primo punto in cui si è verificato il potenziale di azione.

\section{I neuroni}\label{i-neuroni}

La maggior parte dei neuroni contiene 3 principali componenti: un corpo
cellulare e due tipi di \emph{processi neuronali} che partono dal corpo
cellulare, i dendriti e un assone.

Il \textbf{corpo cellulare}, o \emph{soma}, contiene il nucleo e la
maggior parte degli organuli intracellulari.

I \textbf{dendriti} si diramano dal corpo cellulare ricevendo afferenze
da altri neuroni a livello di giunzioni specializzate chiamate
\textbf{sinapsi}. I neuroni sono provvisti anche di un'altra diramazione
che parte dal corpo cellulare, l'\textbf{assone} o \emph{fibra nervosa}.

A differenza del dendrite, la cui funzione è quella di ricevere
informazioni da altri neuroni, il compito dell'assone è quello di
\emph{inviare} informazioni. Generalmente un neurone possiede \emph{un
solo} assone, ma gli assoni possono diramarsi e inviare segnali a più
cellule. Le diramazioni di un assone vengono definite
\textbf{collaterali}.

L'assone serve per la trasmissione di informazioni, che si propagano per
lunghe distanze sotto forma di segnali elettrici definiti
\textbf{potenziali di azione}, rapide e ampie modificazioni del
potenziale di membrana durante le quali l'interno della cellula diviene
positivo rispetto all'esterno.

L'inizio e la fine di un assone sono strutture specializzate del neurone
dette rispettivamente monticolo assonico e terminale assonico.

Il \textbf{monticolo assonico}, che è il sito in cui l'assone si diparte
dal corpo cellulare, è specializzato, nella maggior parte dei neuroni,
nella \emph{genesi} dei potenziali d'azione. Una volta scatenati, i
potenziali d'azione sono trasportati verso il terminare assonico. Il
\textbf{terminale assonico} (o \emph{bottone sinaptico}) è specializzato
nel rilascio del neurotrasmettitore all'arrivo del potenziale d'azione.

Il rilascio del neurotrasmettitore trasmette un segnale ad una
\emph{cellula postsinaptica}, in particolare a un dendrite o al corpo
cellulare di un altro neurone o alle cellule di un organo effettore. Il
neurone il cui temrinale assonico rilascia il neurotrasmettitore è detto
\emph{cellula presinaptica}.

La zona dell'assone presenta molti canali del sodio e dunque presenta
l'eccitabilità giusta per sviluppare i potenziali d'azione.

Il potenziale d'azione si propaga e va sempre nella stessa direzione,
senza tornare mai indietro. Per spiegare coma mai avviene questo si deve
considerare il fenomeno della refrattarietà della membrana.

È la densità dei canali dle sodio a permettere la propagazione
delladepolarizzazione per mezzo delle correnti elettrotoniche.

In un neurone il potenziale di azione avanza e non torna indietro perché
a monte trova la membrana nello stato refrattario, e quando la membrana
esce dallo stato refrattario ormai si trova troppo lontana dalle
correnti elettrotoniche.

Dalla parte del corpo basale il potenziale di azione non si propaga
perché non càè una densità di canali sufficiente per la sua diffusione.

La propagazione del potenziale d'azione ha a che fare con il
funzionamento dei neuroni perchè porta il segnale da una cellula ad
un'altra. La trasmissione del potenziale ha a che fare con la
stimolazione dei movimenti, con la percezione sensoriale, ecc\ldots{}

Tutte queste cose hanno bisogno di un certo tempo di realizzazione, ma
alcune cellule si sono adattate per realizzare i processi con velocità
maggiore per ridurre i tempi di reazione.

Nel sistema nervoso esistono anche cellule diverse dai neuroni, come le
\textbf{cellule gliali}. Queste cellule non servono direttamente alla
trasmissione dei segnali, ma forniscono integrità strutturale al sistema
nervoso, permettendo ai neuroni di svolgere le loro funzioni.

Esistono 5 tipi di cellule gliali:

\begin{itemize}
\itemsep1pt\parskip0pt\parsep0pt
\item
  gli \textbf{astrociti};
\item
  le \textbf{cellule ependimali};
\item
  la \textbf{microglia};
\item
  gli \textbf{oligodendrociti};
\item
  le \textbf{cellule di Schwann}.
\end{itemize}

Tra queste cellule gliali sono quelle di Schwann sono localizzate nel
SNP, mentre le altre si trovano nel SNC.

Le cellule di Schwann hanno un comportamento peculiare: formano un
avvolgimento attorno agli assoni tramite la loro membrana cellulare,
andando a formare una guaina detta \textbf{guaina mielinica}, in modo da
isolarli e permette il passaggio dei potenziali d'azione in modo
efficace e rapido.

Per questo si parla di fibre mielinizzate.

La mielina è formata da strati concentrici di membrane cellulari di
oligodendrociti o cellule di Schwann.

Gli \emph{oligodendrociti} formano la guaina mielinica attorno agli
assoni nel SNC; un oligodendrocita invia proiezioni che forniscono
segmenti di mielina a molti assoni.

Le \emph{cellule di Schwann}, invece, formano la guaina mielinica
attorno agli assoni nel SNP; una cellula di Schwann fornisce mielina a
un singolo assone.

Poichè il doppio strato lipidico della membrana cellulare ha bassa
permeabilità agli ioni, i molti strati di membrana che costituiscono il
rivestimento di mielina di fatto riducono il passaggio di ioni
attraverso la membrana cellulare.

Il doppio strato lipidico funziona come un forte isolante elettrico, e
grazie a questo il potenziale di azione viaggia più velocemente lungo
l'assone.

Tuttavia, esistono delle interruzioni della guaina mielinica, chiamate
\textbf{nodi di Ranvier}, in cui la membrana dell'assone contiene canali
voltaggio-dipendenti per il sodio e il potassio che funzionano nella
trasmissione dei potenziali d'azione permettendo i movimenti ionici
attraverso la membrana.

Il potenziale di azione si sviluppa da un nodo di Ranvier al nodo di
Ranvier successivo, per questo si parla di \emph{conduzione saltatoria}.
Questo aumenta la velocità di trasmissione.

Siccome non tutte le fibre sono mielinizzate, ci sono fibre che
conducino i potenziali di azione in maniera più lenta e altre che lo
conducono in maniera più veloce.

Le fibre più veloci sono i motoneuroni che comandano la muscolatura
scheletrica e che sono piuttosto grandi; conducono lo stimolo ad una
velocità di circa 60-80 m/s. Anche la sensibilità cutanea è molto veloce
(30-60 m/s) fino ad arrivare alle fibre nocicettive, cioè le fibre che
permettono la sensibilità dolorifica e che sono amieliniche; Le fibre
nocicettive hanno un diametro molto piccolo e hanno velocità di
conduzione di circa un centinaio di volte inferiore rispetto a quella
dei motoneuroni.

\textbf{Lezione 20151105}

L'attività dei neuroni in termini di potenziale d'azione consiste nel
produrre potenziale di azione (?). Questi si organizzano in raffiche e
l'attività di un neurone è modulata in base alla loro frequenza. Il
segnale che viaggia lungo l'assone è modulato in frequenza in termini di
potenziale d'azione. In questo modo influenza l'attività di altre
cellule.

Il trasferimento di segnale da una cellula all'altra avviene grazie ad
un dispositivo chiamato sinapsi.

\section{Le sinapsi}\label{le-sinapsi}

Una sinapsi è una giunzione tra due elementi cellulari che consentono il
passaggio di messaggi sotto forma di segnali elettrici.

Le sinapsi in \emph{senso stretto} sono quelle \emph{interneuroniche},
che connettono cioè coppie di neuroni e si stabiliscono di norma tra le
terminazioni di una fibra nervosa ed il soma o i dendriti di un neurone.
Esistono poi sinapsi in cui uno dei due elementi cellulari che si
connettono non è di natura nervosa: in questo caso si parla di
\emph{giunzioni}. Le \emph{giunzioni citoneurali} mettono in
comunicazione le cellule recettrici non nervose di un organo di senso
con le terminazioni di una fibra nervosa afferente sensitiva, mentre le
\emph{giunzioni neuromuscolari} connettono le terminazioni di una fibra
nervosa efferente motoria con una cellula o una fibra muscolare.

Una sinapsi rappresenta sempre un punto di discontinuità strutturale di
una via di comunicazione intercellulare: le membrane dei due elementi
che prendono contatto sinaptico, per quanto vicine tra loro, restano
infatti sempre distinde e separate da uno spazio, la \emph{fessura
sinaptica}. Si parla di cellula presinaptica e di cellula postsinaptica.
La cellula presinaptica interagisce con il terminale assonico, mentre la
cellula postsinaptica interagisce con i dendriti.

Così come il potenziale di azione viaggia sempre in una direzione anche
l'attività della sinapsi è unidirezionale.

Questa situazione rimane fissa nel tempo fino a che, eventualmente, la
sinapsi cessa la sua funzione. Le sinapsi hanno dunque una direzionalità
morfologica e funzionale costante.

Sulla base del meccanismo con cui avviene la trasmissione dei segnali,
si distinguono \emph{sinapsi elettriche} e \emph{sinapsi chimiche}.

Le \textbf{sinapsi elettriche} le troviamo soprattutto negli
invertebrati (meno frequentemente nei vertebrati). In queste sinapsi, la
regione di contatto intercellulare si caratterizza perchè le membrane
pre- e postsinaptiche sono estremamente ravvicinate e assumono la
morfologia tipica delle \emph{giunzioni comunicanti (gap junction)} che
le uniscono. Queste giunzioni mettono in contatto il citoplasma delle
cellule pre- e postsinaptiche con un poro piuttosto grande e poco
selettivo.

Quando arriva un potenaiele d'azione nell'assone della cellula
presinaptica, il potenziale arriva fino in fondo all'assone e induce una
forte depolarizzazione nel punto della sinapsi determinando forti
correnti elettroniche attraverso le giunzioni Gap.

Si induce un potenziale d'azione nella cellula postsinaptica. Vi è una
continuità elettrica tra la cellula pre- e postsinaptica.

Le sinapsi elettriche presentano un'elevata velocità di trasmissione.
Hanno un consumo energetico basso perchè utilizzano correnti passive che
passano per un mexzzo di conduzione già pronto e non sono affaticabili
in quanto la sinapsi può condurre il passaggio di potenziale d'azione in
maniera indefinita nel tempo.

Nelle \textbf{sinapsi chimiche} utilizzano un meccanismo diverso perchè
le due cellule (pre- e postsinaptiche) non sono in contatto fisico tra
loro (le loro membrane sono molto ravvicinate ma non si toccano).

Nelle sinapsi chimiche, un \emph{messaggio elettrico} viene convertito
in un \emph{messaggio chimico} atto a ``scavalcare'' la fessura
sinaptica, per poi essere nuovamente \emph{riconvertito in un messaggio
elettrico}.

In queste sinapsi, la trasmissione richiede la liberazione da parte
dell'elemento presinaptico, in risposta alla depolarizzazione di
quest'ultimo, di un \textbf{neurotrasmettitore}, un composto chimico
capace di attivare l'elemento postsinaptico legandosi a recettori
specifici presenti sulla sua membrana.

Perchè possa avvenire lo stimolo sinaptico la cellula deve intraprendere
un'azione metabolica che coinvolge vie e reazioni e fosforilazioni e
consumi di energia metabolici. La sinapsi chimica richiede dunque un
maggior consumo energetico e presenta affaticabilità (le risorse delle
sinapsi possono esaurirsi nel tempo se gli stimoli mantengono frequenza
elevata per lungo tempo).

Le risorse della sinapsi possono esaurirsi nel corso del tempo se gli
stimoli mantengono una frequenza elevata per lungo tempo (ad esempio può
esaurirsi l'ATP necessario per il meccanismo sinaptico).

La funzionalità di queste sinapsi può essere modulata.

Le sinapsi chimiche possono essere eccitatorie o inibitorie della
cellula postsinaptica.

La cellula presinaptica presenta un rigonfiamento contenente delle
vescicole sinaptiche contenenti il neurotrasmettitore.

Il potenziale di azione si ferma prima del terminale assonico ma ha
effetto sulla sua parte terminale determinando il rilascio delle
vescicole sinaptiche nel ridotto spazio della fessura sinaptica.

Sulla cellula postsinatica si trovano dei recettori per i
neurotrasmettitori che determinano la trasduzione del segnale dalla
cellula pre- alla postsinaptica.

Si parla di fenomeni di traffico vescicolare.

Una delle caretteristiche delle cellule eucariotiche è quella di creare
delle vescicole e indirizzarle in punti specifici della cellula o della
membrana cellulare, per poi generare fusione delle vescicole con la
membrana.

Nei neuroni questi eventi avvengono nella regione sinaptica e qui
determinano lo scarico delle vescicole sinaptiche all'esterno.

Nella fusione delle vescicole è coinvolto il calcio. Più in generale,
ogni volta che abbiamo una secrezione ghiandolare da scarico di
vescicole all'esterno, è sempre coinvolto il calcio.

Nelle sinapsi abbiamo segnali del calcio ogni qual volta abbiamo lo
scarico delle vescicole all'esterno.

L'arrivo del potenziale d'azione e del segnale del calcio sono
estremamente interconnessi. Il potenziale d'azione non arriva all'apice
dell'assone (nella zone della sinapsi) perchè in questa zona non sono
presenti i canali del sodio voltaggio-dipendenti che permettono la
formazione del potenziale d'azione, il quale si ferma qualche decina di
micron prima.

L'ultimo lampo del potenziale di azione induce correnti elettrotoniche
che invadono la terminazione sinaptica.

Le correnti eletroniche depolarizzano la porzione presinaptica in
funzione del potenziale d'azione.

Qui vi sono canali voltaggio dipendenti del calcio che si aprono
permettendo un ingresso del calcio che mette in moto un meccanismo di
traffico vescicolare che fa muovere le vescicole verso la superficie,
cioè verso la plasmamembrana e le fa fondere con essa.

In questo modo ciò che era contenuto nella vescicola diffonderà nella
fessura sinaptica.

Il problema consiste nel far fondere le membrane. Queste normalemente
non si fondono e se lo fanno il processo è molto lento.

\subsection{Il meccanismo di rilascio
vescicolare}\label{il-meccanismo-di-rilascio-vescicolare}

Nell'elemente presinaptico, le vescicole sinaptiche contenenti il
neurotrasmettitore non sono distribuite omogeneamente, ma si raccolgono
in prossimità della densità presinaptica. Dal punto di vista funzionale,
si distinguono due gruppi di vescicole sinaptiche:

\begin{enumerate}
\def\labelenumi{\arabic{enumi}.}
\itemsep1pt\parskip0pt\parsep0pt
\item
  le vescicole di un primo gruppo, detto \textbf{pool di rilascio}, si
  trovano immediatamente a ridosso della membrana presinaptica, in
  corrispondenza delle zone attive, dove vengono predisposte
  all'apertura verso lo spazio sinaptico ed al rilascio del
  neurotrasmettitore in esse contenuto;
\item
  le vescicole del secondo gruppo, detto \textbf{pool di riserva}, si
  trovano a maggiore distanza dalla membrana presinaptica e
  progressivamente meno addensato. Esse sono vincolate al citoscheletro,
  e non sono suscettibili di immediato rilascio al sopraggiungere della
  depolarizzazione presinaptica, ma possono essere svincolate dal
  citoscheletro e indirizzate verso le zone attive per rimpiazzare le
  vescicole del \emph{pool di rilascio} man mano che queste vengono
  consumate.
\end{enumerate}

Il processo di apertura delle vescicole sinaptiche si svolge secondo il
paradigma generale dell'\emph{esocitosi vescicolare}.

Ciascuna tappa del ciclo vescicolare dipende dall'intervento di
specifiche molecole proteiche presenti nell'elemento presinaptico.

Il legame delle vescicole del pool di riserva al citoscheletro è mediato
dalle \textbf{sinapsine}, proteine estrinseche associate al versante
citoplasmatico della membrana vescicolare. Le sinapsine sono in grado di
legare le molecole di actina del citoscheletro. L'interazione delle
sinapsine con i filamenti di actina è modulabile in modo
Ca\(^2\)\(^+\)-dipendente.

L'interazione delle sinapsine con i filamenti di actina è modulabile in
modo Ca\(^2\)\(^+\)-calmodulina-dipendente di tipo II (CaMK-II), una
volta attivata dal complesso \textbf{Ca\(^2\)\(^+\)-CaM}, è in grado di
fosforilare le sinapsine, il che ne riduce l'affinità per l'actina,
promuovendo quindi il distacco delle vescicole.

Questo meccanismo di liberazione delle vescicole del \emph{pool} di
riserva, che da questo punto in poi potranno essere smistate al
\emph{pool} di rilascio, è tanto più attivo quanto più intensa è
l'attività sinaptica.

Il distacco dal citoscheletro di una vescicola del \emph{pool} di
riserva è seguito dalla sua mobilizzazione e dal suo indirizzamento
(``\emph{sorting}'') verso una zona attiva.

Questo processo richiede l'intervento di proteine estrinseche della
membrana vescicolare dette \textbf{Rab3}. Queste sono proteine
monomeriche leganti il GTP e sono dotate di attività GTPasica. Nella
forma legata al GTP Rab3 ``contrassegna'' le vescicole che devono essere
trasportate verso le zone attive.

La successiva tappa del ciclo vescicolare è quella detta di ``attracco''
o \emph{docking}, nella quale la vescicola viene vincolata alla zona
attiva.

In questa fase si stabiliscono interzioni fra proteine vescicolari e
proteine della membrana presinaptica. Fra le proteine implicate nel
\emph{docking} menzioniamo la \textbf{sinaptogamina}, proteina integrale
della membrana vescicolare, e due proteine della membrana presinaptica
delle zone attive, cioè la \textbf{neurezina I} e \textbf{SNAP-25}. La
\emph{sinaptogamina} lega anche un complesso proteico presinaptico.

Al termine del processo di \emph{docking} la vescicola non è ancora
disponibile a essere rilasciata esocitoticamente.

Perchè la vescicola possa essere rilasciata esocitoticamente deve
seguire un ulteriore processo, detto di \emph{priming}, che rende la
vescicola \emph{competente} alla fusione con la membrana presinaptica in
risposta alla depolarizzazione presinaptica e all'aumento della
concentrazione citosolica del Ca\(^2\)\(^+\) nel versante presinaptico.

Il \emph{priming} vescicolare richiede l'interazione fra specifiche
proteine vescicolari e della membrana presinaptica, dette
complessivamente \textbf{proteine SNARE}.

Tali proteine si suddividono in proteine SNARE della membrana
vescicolare, o \textbf{v-SNARE} (presenti sulla vescicola), e proteine
della membrana ``bersaglio'' (\emph{target}: la membrana presinaptica a
livello delle zone attive), o \textbf{t-SNARE} (è la molecola bersaglio
della porzione postsinaptica).

La v-SNARE più importante nel processo di \emph{priming} è la
\textbf{sinaptobrevina} (o \textbf{VAMP}), una proteina integrale di
membrana vescicolare dotata di un solo segmento transmembranario e un
lungo dominio citoplasmatico N-terminale.

Nel corso del \emph{priming}, si instaura una stretta interazione fra le
proteine SNARE. Il procedere di tale interazione sviluppa una potente
forza traente che, al termine del processo, porta la membrana
vescicolare a contatto con la membrana presinaptica. La forza traente
così sviluppata è sufficientemente intensa da scoprire parzialmente il
``\emph{core}'' lipidico di ciascuna membrana, che può così stabilire
un'interazione idrofobica con quello dell'altra membrana.

Le vescicole vengono rifornite al terminale sinaptico dal corpo
cellulare tramite l'assone che funge da convogliatore di vescicole
grazie al citoscheletro.

\subsection{Meccanismi postsinaptici}\label{meccanismi-postsinaptici}

Le molecole del neurotrasmettitore, liberate dall'apertura delle
vescicole sinaptiche, diffondono nello spazio sinaptico e si legano a
specifici recettori chimici presenti nella membrana postsinaptica.

Il ruolo dei recettori postsinaptici va ben oltre quello di semplici
``spie'' dell'avvenuta liberazione del neurotrasmettitore, perchè da
essi dipendono sia il \emph{segno} (eccitamento o inibizione) che
l'intensità della risposta postsinaptica.

Esistono 2 grandi classi di recettori postsinaptici, che si distinguono
per la struttura delle molecole proteiche che li costituiscono e per il
loro modo di operare: i \emph{recettori ad azione diretta} o
\emph{ionotrocipi} (recettori che costituiscono essi stessi un canale
ionico) ed i \emph{recettori ad azione indiretta} o \emph{metabotropici}
(recettori accoppiati a proteine G trimeriche che vanno a stimolare
l'apertura di canali ionici).

\subsubsection{Recettori ionotropici}\label{recettori-ionotropici}

I \emph{recettori ionotropici} (o \emph{recettori-canale}) sono molecole
proteiche che comprendono una porzione recettoriale rivolta verso los
pazio sinaptico ed una porzione strutturata in canale ionico che
attraversa tutto lo spessore della membrana. La proteina è costituita da
più subunità che, essendo disposte in cerchio attorno a un asse
centrale, vengono a delimitare un condotto, costituendo un canale ionico
chemio-dipendente.

In assenza del neurotrasmettitore, il canale è generalmente nello stato
chiuso ed impervio agli ioni.

Quando invece le molecole del neurotrasmettitore si legano ai propri
siti di riconoscimento del dominio recettoriale, il canale passa nello
stato aperto e gli ioni permeanti possono fluirvi secondo il proprio
gradiente elettrochimico.

Ne viene generata una corrente che, a seconda della natura del
recettore-canale e della sua selettività ionica, potrà essere entrante,
quindi depolarizzante, oppure uscente, quindi iperpolarizzante: nel
primo caso la corrente indurrà una depolarizzazione della membrana
postsinaptica della \textbf{potenziale postsinaptico eccitatorio
(EPSP)}, nel secondo caso un'iperpolarizzazione detta \textbf{potenziale
postsinaptico inibitorio (IPSP)}.

I recettori ionotropici agiscono molto rapidamente, questo giustifica il
termine \emph{trasmissione sinaptica rapida}.

\subsubsection{Recettori metabotropici}\label{recettori-metabotropici}

I recettori metabotropici sono molecole proteiche costituite da una
singola catena polipeptidica comprendente 7 \emph{segmenti idrofobici
transmembranari}. Hanno anch'essi uno o più domini recettoriali esposti
allo spazio sinaptico e predisposti al legame col neurotrasmettitore,
tuttavia non formano un canale transmembranario.

La proteina recettoriale sporge verso il versante intracellulare della
membrana con un dominio effettore predisposto a legarsi con una proteina
G trimerica. Quando questa viene attivata dall'interazione con il
recettore, a sua volta attivato dal legame con il suo ligando, può
innescare eventi di vario tipo: può interagire direttamente sul piano
della membrana con un canale ionico, modificandone lo stato di pervietà
oppure può associarsi a un \emph{enzima allosterico di membrana} (es.
adenilaco ciclasi o fosfolipasi C), attivandolo. Ne deriva, in questo
secondo caso, la sintesi di uno o più \emph{secondi messaggeri}, che a
loro volta possono legarsi direttamente a un canale ionico di membrana
oppure attivare a valle altre proteine implicate nella trasduzione del
segnale.

In ultima analisi viene indotta l'apertura o la chiusura di canali
ionici della membrana postsinapticae, a seconda della selettività ionica
di questi ultimi, viene generata una corrente netta di membrana che si
traduce in un EPSP o in un IPSP.

È evidente che i potenziali postsinaptici insorgono con una latenza
tanto maggiore quanto più estesa è la catena di eventi che porta alla
modificazione regolatoria dei canali di membrana.

I potenziali sinaptici generati con questo tipo di meccanismo sono
spesso caratterizzati da un'insorgenza lenta e anche da un lento
decadimento, che li rende assai più duraturi dei potenziali sinaptici
dovuti all'attivazione di recettori ionotropici.

Questo tipo di trasmissione sinaptica viene chiamata \emph{trasmissione
sinaptica lenta}.

Molti neurotrasmettitori ``classici'' (acetilcolina, GABA, glutammato,
serotonina) dispongono di recettori sia ionotropici che metabotropici,
non di rado espressi sulle stesse sinapsi.

Per l'acetilcolina il recettore ionotropico, noto come \emph{recettore
nicotinico}, è un canale ionico di tipo promiscuo che permette il
passaggio di ioni positivi mono- e bivalenti. Permette il passaggio di
una corrente del sodio a causa dell'ingente abbondanza dello ione. Il
recettore metabotropico dell'acetilcolina, invece, è chiamato
\emph{recettore muscarinico} ed è legato a una proteina G che va a
stimolare le correnti ioniche tramite l'utilizzo di secondi messaggeri
come l'AMP-ciclico.

I nomi di questi recettori derivano dal fatto che l'agonista fisiologico
è l'acetilcolina, ma quelli farmacologici sono la \emph{nicotina} e la
\emph{muscarina} (metabolita presente nel fungo \emph{Amanita
muscaria}).

Sia i recettori ionotropici che metabotropici pososno far insorgere
nella cellula postsinaptica dei segnali elettrici per mantenere e
trasmettere l'eccitazione da una cellula all'altra.

Il primo evento elettrico che si produce nella cellula post-sinaptica è
un potenziale post-sinaptico che origina una variazione del potenziale
di membrana (questo non è un potenziale d'azione).

Come già detto i segnali postsinaptici possono essere di due tipi: EPSP
(eccitatorio) e IPSP (inibitorio).

Nelle \emph{sinapsi eccitatorie (EPSP)} il neurotrasmettitore induce
correnti al sodio nella cellula postsinapticha che ne depolarizzano.

Nel punto della sinapsi, sul dendrite della cellula postsinaptica,
abbiamo una depolarizzazione più debole di quella del potenziale
d'azione, ma abbastanza forte da essere sentita dalla cellula e da
creare correti elettrotoniche che viaggiano in parte verso la fine del
dendrite e in parte verso il corpo cellulare tendendo ad invaderlo.

Ogni neurone riceve qualche migliaio di sinapsi. Una sola sinapsi non è
consistente ma tutte insieme, ciascuna eccitatoria e dunque inducente
correnti elettroniche, creano un flusso di correnti elettroniche che
creano una depolarizzazione significativa nel corpo cellulare.

Una zona dove sono molto concentrati i canali per il sodio è il
\emph{cono di emergenza dell'assone}, ovvero la base dell'assone.

I canali sentono le correti elettrotoniche prodotte dalle sinapsi
eccitatorie e se queste correnti superano la soglia di innesco del
potenziale d'azione, questo parte e si propaga lungo l'assone dove si
troverà un'altra sinapsi.

Una singola sinapsi è poca cosa rispetto alle dimensioni della cellula e
genera correnti piccole, mentre più sinapsi insieme possono
\emph{collaborare per sommazione} che può essere \emph{spaziale} o
\emph{temporale}.

Nella \textbf{sommazione spaziale} si ha la concomitanza in vari punti
di stimoli eccitatori nel complesso dendritico che raggiungono così la
soglia di depolarizzazione sommando, mentre nella \textbf{sommazione
temporale} si ha la sommazione di contatti postsinaptici conseguenti nel
tempo che porta a depolarizzazione della membrana. Prima che arrivi il
secondo potenziale la membrana non è ancora tornata in condizioni
standard e quindi con il secondo potenziale si depolarizza ancora di
più.

Le \emph{sinapsi inibitorie (IPSP)} inducono potenziali postsinaptici
inibitori che creano iperpolarizzazione, mediata da canali potassio e
sodio, della membrana.

Il potenziale viene bloccato a potenziale di riposo della membrana, e in
termini di sommazione di potenziali inibitori e eccitatori l'effetto sul
cono di emergenza dell'assone è neutralizzato dal potenziale inibitorio.

La sinapsi eccitatoria è neutralizzata da quella inibitoria.

La scarica di un neurone dipenderà dalla somma di sinapsi eccitatorie ed
inibitorie.

\textbf{L'uscita assonica è l'integrale delle sinapsi dei dendriti (?)}

\subsubsection{Interruzione della trasmissione
sinaptica}\label{interruzione-della-trasmissione-sinaptica}

I canali possono essere desensibilizzati tramite una modificazione
conformazionale. Il neurotrasmetitore può essere presente in quantità
non sufficienti o eccessive.

Inoltre la trasmissione sinaptica può essere interrotta tramite
riassorbimento del neurotrasmettitore o incorporazione per endocitosi o
degradazione enzimatica. L'acetilcolina ad esempio, poù essere degradata
e dunque inattivata dall'\emph{acetilcolinaesterasi}.

\subsubsection{La plasticità
sinaptica}\label{la-plasticituxe0-sinaptica}

È stato scoperto che l'attività di stimolo esercitata su una sinapsi ne
può condizionare l'attività stessa. Questo fenomeno è definito
\textbf{``plasticità sinaptica''}.

Una sinapsi non lavora sempre nello stesso modo, ma il suo funzionamento
è condizionato dalla quantità degli stimoli che riceve.

Più la sinapsi lavora e più diventa idonea al suo lavoro. Si parla si
\emph{``potenziamento sinaptico''}.

Esistono sono vari tipi di potenziamento tra cui:

\begin{enumerate}
\def\labelenumi{\arabic{enumi}.}
\itemsep1pt\parskip0pt\parsep0pt
\item
  la \textbf{facilitazione}, quando una sinapsi è stimolata intensamente
  per tempi brevi;
\item
  il \textbf{potenziamento post-tetanico} (PTP, durata 1-2 min);
\item
  il \textbf{potenziamento a breve termine}, (STP, stimolazioni
  intermedie, durata di decine di min - un'ora);
\item
  il \textbf{potenziamento a lungo termine} (LTP, ore o giori a
  stimolazioni molto forti, 100 Hz, per pochi secondi).
\end{enumerate}

Questi fenomeni sono stati riscontrati analizzando piccoli gruppi di
sinapsi.

\textbf{Nel complesso del sistema nervoso e di regioni specifiche del
cervello, questi fenomeni possono osservare attività macroscopiche a
livello del soggetto (?)}

In cosa consiste il potenziamento? In un miglior funzionamento delle
sinapsi, che più facilmente indurranno una scarica nella cellula
postsinaptica.

La LTP può durare per tempi molto lunghi.

Si è notato che in certe regioni cerebrali, come ad esempio l'ippocampo,
questi fenomeni sono più manifesti. L'ippocampo può essere visto come un
ricciolo marginale della calotta neuronale costituita dagli emisferi
cerebrali, ossia la corteccia.

In questa zona sono stati visti in maniera vistosa fenomeni di LTP.
L'ippocampo è la regione coinvolta nei processi di memoria; danni in
questa regione provocano disturbi seri della memoria.

I recettori coinvolti sono recettori del glutammato. Ne esistono tipi
\emph{NDMA} e \emph{non NMDA} che mediano ingressi di sodio e calcio.

Quando il glutammato agisce i recettori sono bloccati dal Mg.

Una depolarizzazione della membrana, che viene sentita dal canale (pur
non essendo voltaggio-dipendente) provoca il rilascio del Mg dal
\emph{canale NDMA} e a questo punto c'è entrata di sodio e calcio.

I canali \emph{non NMDA} reagiscono direttamente al glutammato creando
una corrente del sodio che depolarizza la membrana. Di questo ne risente
il recettore NMDA che toglie il magnesio e lascia entrare sodio e
calcio. Questo induce il segnale del calcio che attiva delle chinasi
calcio dipendenti.

A questo punto si ha produzione di ossido nitrico tramite la ossido
nitrico sintasi. L'ossido nitrico va ad agire sulla cellula presinaptica
potenziando la sinapsi e dunque il rilascio del neurotrasmettitore.

Questo è il possibile meccanismo cellulare di insorgenza di LTP.

La cellula postsinaptica può influenzare l'attività della cellula
presinaptica.

Allargando la visuale vediamo che si formano plasticità sinaptiche che
generano circuiti preferenziali potenziati. Si pensa che la memoria sia
questa cosa.

\textbf{Lezione 20151109}

\section{Il sistema nervoso}\label{il-sistema-nervoso}

Il sistema nervoso è suddiviso anatomicamente in centrale e periferico.

Il \textbf{sistema nervoso centrale (SNC)} è costituito da encefalo e
midollo spinale, mentre il \textbf{sistema nervoso periferico (SNP)} è
costituito da gangli e nervi.

Nel sistema nervoso esistono due tipi fondamentali di cellule: i
\emph{neuroni} e cellule non neuronali o \emph{cellule gliali}.

Le cellule neuronali sono cellule eccitabili capaci di produrre
potenziali d'azione in risposta ad stimoli elettrici, mentre le cellule
gliali non sono eccitabili.

Si è stimato che il sistema nervoso centrale contiene circa cento
miliardi (10\(^1\)\(^1\)) di neuroni collegati fra loro da centomila
miliardi (10\(^1\)\(^4\)) di sinapsi presenti nell'encefalo e nel
midollo spinale. Mediamente ogni cellula neurale stabilisce almeno 103
sinapsi.

Il SNC è formato per il 75-90\% da cellule gliali (\emph{neuroglia}), e
solo per circa il 10\% da neuroni.

Le cellule gliali si suddividon in: cellule di Schawann,
oligodendrociti, microglia, cellule ependimali e astrociti.

Come già detto i neuroni sono intercollegati tra loro, e ogni 4-5
passaggi il neurone di partenza viene ricontattato da quello finale.
Questo in accordo con il concetto di loop già visto.

Nel SNC lo schema del loop non indica la presenza di una ``geometria'',
ma dice che esistono dei contatti di controllo. Nel sistema nervoso
centrale lo schema funzionale è rappresentato in maniera topografica. È
sempre una topografia a loop.

Le cellule gliali sono cellule che stabiliscono migliaia di contatti con
le cellule neurali. Soprattutto le cellule di Shwann e gli astrociti
(questi rivestono anche i capillari).

A livello cefalico abbiamo una grandissima protezione offerta dal
cranio, ma anche da una serie di membrane connettivali che attutiscono
notevolmente i traumi indotti da urti improvvisi dell'encefalo contro il
cranio stesso.

Questi strati di tessuto vengono chiamati \textbf{meningi} e includono:

\begin{itemize}
\itemsep1pt\parskip0pt\parsep0pt
\item
  la \textbf{dura madre}, è lo strato più esterno formato da uno strato
  spesso e molto denso;
\item
  l'\textbf{aracnoide}, una membrana intermedia a forma di rete;
\item
  la \textbf{pia madre}, è lo strato più interno formato da una membrana
  sottile che riveste tutti i meandri del sistema nervoso centrale.
\end{itemize}

Mentre di norma non vi è alcuno spazio tra la dura madre e l'aracnoide,
tra l'aracnoide e la pia madre vi è uno spazio, lo \textbf{spazio
subaracnoideo}, pieno di \emph{liquido cerebrospinale}.

Il \textbf{liquido cerebrospinale (LCS)} è un liquido limpido che bagna
il SNC; esso ha una composizione simile, ma non identica, al plasma. Il
liquido cerebrospinale non soltanto circonda il sistema nervoso
centrale, ma si insinua anche all'interno di esso, circondando i neuroni
e le cellule gliali e riempiendo alcune cavità presenti all'interno
dell'encefalo e del midollo spinale.

L'encefalo contiene 4 di tali cavità, chiamate \textbf{ventricoli}, che
comunicano tra loro. I due ventricoli laterali a forma di C sono
connessi al terzo ventricolo, mediale, dal \emph{forame
interventricolare}. L'acquedotto cerebrale, chiamato \textbf{acquedotto
di Silvio}, connette il III ventricolo al quarto, che è la continuazione
del \emph{canale centrale}, una lunga e sottile cavità cilindrica che
percorre per tutta la sua lunghezza il midollo spinale.

Il rivestimento interno dei ventricoli e del canale centrale è composto
da cellule gliali, chiamate \textbf{cellule ependimali}, che
costituiscono un tipo particolare di cellule epiteliali.

Il rivestimento dei ventricoli è vascolarizzato e forma un tessuto
chiamato \textbf{plesso coroideo}, che consta di pia madre, capillari e
cellule ependimali.

Il liquido cerebrospinale viene prodotto dal plesso coroideo.

Questi producono un liquido che riversano nel ventricolo alimentando il
liquido cerebrospinale che, in altri punti chiamati villi, viene
assorbito dal sangue fornendo un ricambio continuo.

Appena prodotto il liquido cerebrospinale (LCS) attraversa il sistema
ventricolare ed entra nello spazio subaracnoideo. Il LCS, a livello
subaracnoideo, viene in parte riassorbito nel sangue venoso attraverso
speciali strutture, i \emph{villi aracnoidei}, localizzati alla sommità
dell'encefalo.

Il ricambio completo del LCS avvine a un ritmo di circa tre volte al
giorno. Il LCS occupa un volume di circa 120-150 ml.

Poichè il SNC è completamente circondato dal LCS, galleggia in esso e,
pertanto, il LCS agisce come una struttura ammortizzante che previene le
collisioni del tessuto nervoso con la scatola cranica. Inoltre,
contribuisce a mantenere stabile la composizione ionica all'esterno
delle cellule, e fornisce i nutrienti alle cellule gliali e ai neuroni e
allontana da tali cellule i prodotti di rifiuto.

La composizione del LCS è la stessa del plasma per quanto riguarda il
tenore ionico; la concentrazione del glucosio invece è più bassa, così
come la concentrazione proteica.

Il sistema vascolare e quello nervoso sono collegati da un meccanismo a
loop, anche se la struttura del sistema vascolare non sembra rifletterne
la funzionalità.

Nel sistema nervoso vi è una ricchissima vascolarizzazione (riceve circa
il 15\% del sangue), consuma il 20\% dell'ossigeno e il 50\% del
glucosio consumato dal corpo. A differenza di muscoli e fegato,
l'encefalo non accumula glucosio e non utilizza lipidi come fonte
energetica, perciò il suo metabolismo dipende in toto dall'ossigeno e
dal glucosio che arrivano momento per momento dal sangue.

Questo è il motivo per cui le ischemie cerebrali creano facilmente danni
irreversibili.

Il rapporto tra il sangue e il tessuto del SNC è uno dei più complessi a
causa della barriera ematoencefalica.

\subsection{La barriera
ematoencefalica}\label{la-barriera-ematoencefalica}

Come negli altri tessuti, gli scambi di ossigeno, glucosio ed altre
sostanze tra il sangue e le cellule del SNC si realizzano attraverso le
paretu dei \emph{capillari}. Le pareti dei capillari sono costituite da
un singolos trato sottile di \emph{cellule endoteliali}, cosa che
permette gli scambi gassosi per \emph{diffusione}.

Nella maggior parte dei tessuri, piccole molecole come gas, ioni
inorganici, monosaccaridi ed amminoacidi attraversano liberamente la
parete capillare. Mentre le molecole idrofobiche diffondono attraverso
interruzioni relativamente grandi (\emph{pori}) che si trovano tra le
cellule endoteliali.

Nel SNC il movimento delle moleocle idrofiliche attraverso le pareti dei
capillari é limitato dalla \textbf{barriera ematoencefalica}.
L'esistenza di questa barriera è dovuta alla presenza di giunzioni
strette tra le cellule endoteliali dei cpaillari, che non permettono la
formazione di pori capillari e quinidi limitano la diffusione di
molecole idrofiliche tra le cellule.

Gli astrociti stimolano le cellule endoteliali a sviluppare e mantenere
le giunzioni strette.

Questa barriera protegge il SNC da sostanze tossiche che possono essere
presenti nel sangue, perchè limita il movimento delle molecole
attraverso l'endotelio capillare. Gas e altre molecole idrofobiche
attraversano abbastanza faiclemnte le membrane cellulari, in quanto sono
in grado di attraversare il doppio strato fosfolipidico della membrana
per diffusione semplice. Le sostanze idrofiliche invece, come ioni,
zuccheri ed amminoacidi, non possono attrracersare le mebrane cellulari
mediante diffusione semplice e pertanto devono fare affidamento sul
trasporto mediato per poter attraversare le pareti capillari del SNC.

Poichè il trasporto mediato utilizza proteine di trasporto che sono
specifiche solo per certe molecole, la barriera ematoencefalica è
selettivamente permeabile, permettendo solo ad alcuni composti di
attraversarla.

Molecole come glucosio e amminoacidi possono penetrrare la barriera
ematoencegalica facilmente. Il glucosio è trasportato attrverso la
barriera ematoencefalica da proteine trasportatrici o ``carrier''
GLUT-1.

La barriera ematoencefalica costituisce un problema per le terapie
medicinali, perchè i farmaci devono obbligatoriamente superarla e dunque
essere o lipofili o sfruttare dei trasportatori già presenti.

Il trasporto di sostanze dal sangue al tessuto nervoso è tenuto dunque
sotto un controllo molto stretto.

\subsection{Sostanza grigia e sostanza
bianca}\label{sostanza-grigia-e-sostanza-bianca}

Il SNC ha una disposizione dei neuroni molto ordinata. I corpi
cellulari, i dendriti e i terminali assonici formano agglomerati
(cluster) che appaiono grigi, mentre gli assoni si raggruppano a formare
strutture che appaiono bianche. Si parla, perciò, di \textbf{materia
grigia} e \textbf{materia bianca}.

La sostanza grigia costituisce circa il 40\% del SNC ed è qui che si
realizzano la trasmissione e l'integrazione neuronale. L'altro 60\% dei
SNC è formato da sostanza bianca, costituita per lo più da assoni
mielici, ai quali deve il proprio colore.

La mielina presenta un elevato contenuto lipidico ed è per questo che
appare bianca. Gli assoni mielinici sono specializzati nella
trasmissione rapida delle informazioni.

Guardando la superficie esterna dell'encefalo, è visibile soltanto la
sostanza grigia, perchè la maggior parte della struttura a forma di
globo (chiamata \emph{cervello}) che costituisce la massa dell'encefalo
è interamente coperta da uno strato sottile di materia grigia, chiamato
\textbf{corteccia cerebrale}.

La sostanza bianca è localizzata al di sotto di questo strato;
incastonate sotto la sostanza bianca vi sono piccole aree di sostanza
grugua note come nuclei.

Nel midollo spinale la disposizione è diversa: la sostanza bianca è
posizionata all'esterno, mentre la sostanza grigia si trova all'interno.

Nella sostanza bianca del SNC, gli assoni (anche noti come \textbf{fibre
nervose}) sono organittazi in fasci o tratti che collegano una regione
di sostanza grigia con un'altra. Differenti fasci nervosi sono
classificati in funzione delle regioni che collegano:

\begin{itemize}
\itemsep1pt\parskip0pt\parsep0pt
\item
  i \textbf{fasci di proiezione} connettono la corteccia cerebrale con
  aree encefaliche dislocate a livelli inferiori o con il midollo
  spinale;
\item
  le \textbf{fibre di associazione} connettono un'area della corteccia
  cerebrale con un'altra sullo stesso lato del cervello;
\item
  le \textbf{fibre commissurali} collegano l'area corticale di un
  emisfero con la corrispondente area corticale dell'altro emisfero.
\end{itemize}

La maggioranza delle fibre commissurali è localizzata in una banda di
tessuto chiamata \textbf{corpo calloso}, che collega tra loro i due
\textbf{emisferi cerebrali}. L'encefalo infatti è diviso in due da una
profonda separazione, le due porzioni ``separate'' sono gli emisferi
destro e sinistro.

Alla base dell'encefalo si può notare una protuberanza di tessuto: il
midollo spinale.

\subsection{Il midollo spinale**}\label{il-midollo-spinale}

Il midollo spinale si presenta ocme una struttura di tessuto nervoso di
forma cilindrica che origina dalla porzione terminale del bulbo ed è
circondata dalla \textbf{colonna vertebrale}. Il midollo spinale di un
soggetto adulto è lungo circa \emph{44 cm} e ha un diametro che varia
tra 1 e 1,4 cm.

\subsubsection{I nervi spinali}\label{i-nervi-spinali}

Dal midollo spinale si dipartono, ad intervalli regolari, \emph{31 paia}
di \textbf{nervi spinali}.

I nervi rappresentano il corrispondente delle vie neurali al di fuori
del SNC, quindi nel SNP (sono sempre fasci di fibre). Questi hanno un
rivestimento connettivale che rende il nervo un filamento tangibile. Nei
nervi si trova un fascio di assoni derivante da neuroni di tipo diverso.

Ciascun paio di nervi fuoriesce dalla colonna vertebrale ed è definito
in base alla regione di midollo spinale dal quale origina.

Vi sono:

\begin{itemize}
\itemsep1pt\parskip0pt\parsep0pt
\item
  8 paia di \textbf{nervi cervicali} (C1-C8), che emergono dalle
  vertebre della regione del collo;
\item
  12 paia di \textbf{nervi toracici} (T1-T12), che emergono dalla
  regione toracica;
\item
  5 paia di \textbf{nervi lombari} (L1-L5), che emergono dalla regione
  lombare;
\item
  5 paia di \textbf{nervi sacrali} (S1-S5), che emergono dal coccige;
\item
  un singolo \textbf{nervo coccigeo} (C\(_o\)), che emerge dalla punta
  del coccige.
\end{itemize}

Il midollo spinale si estende solo per i 2/3 della lunghezza della
colonna vertebrale. L'ultimo terzo della colonna non contiene midollo
spinale, ma solo nervi che emergono da essa. Questa regione è conosciuta
come \emph{cauda equina}.

Le numerose fibre nervose che compongono un singolo nervo spinale
viaggiano verso regioni adiacenti del corpo. Pertanto, è possibile
disegnare sulla superficie corporea differenti regioni, chiamate
\emph{dermatomeri}, ciascuna delle quali è innervata da un particolare
nervo spinale.

La faccia non ha una rappresentazione dermatomerica, perchè è innervata
dai \emph{nervi cranici}, che emergono dal cranio.

Le meningi (aracnoide, dura e pia madre), insieme al liquido
cerebrospinale, sono sempre presenti anche nel midollo spinale.

\subsubsection{La sostanza grigia e bianca nel midollo
spinale}\label{la-sostanza-grigia-e-bianca-nel-midollo-spinale}

La sostanza grigia è localizzata in un'area interna a forma di farfalla,
mentre la sostanza bianca è localizzata attorno alla grigia.

La sostanza grigia contiene interneuroni, corpi cellulari e dendriti di
\textbf{neuroni efferenti} (neuroni che viaggiano nei nervi spinali
diretti verso gli organi effettori), e i terminali degli \textbf{assoni
afferenti} (i neuroni afferenti viaggiano nei nervi spinali verso il
midollo spinale, partendo dai recettori sensoriali localizzati alla
periferia del corpo).

La sostanza grigia del midollo spinale è organizzata in modo tale che
differenti tipi di neuroni sono localizzati in differenti regioni.

La sostanza grigia comprende un \textbf{corno dorsale} ed un
\textbf{corno ventrale} in ogni lato.

Il \emph{corno dorsale} comprende la metà dorsale (posteriore) della
sostanza grigia di ogni lato; il \emph{corno ventrale} (anteriore)
comprende quella ventrale.

Le \emph{fibre afferenti} originano dalla periferia come recettori
sensoriali e terminano nel corno dorsale, dove formano sinapsi con
interneuroni o direttamente con neuroni efferenti.

I nervi spinali sono 2 per ogni livello vertebrale ed emergono dal
midollo con due radici (una dorsale e una ventrale) che confluiscono poi
in un unico nervo.

La radice dorsale contiene fibre afferenti, assoni sensoriali il cui
stimolo viaggia verso il SNC. Nella radice ventrale invece, si trovano
fibre motorie il cui potenziale di azione viaggia verso la periferia dal
SNC.

I corpi cellulari dei neuroni afferenti non sono localizzati nel midollo
spinale, ma all'esterno, dove sono raggruppati nei \textbf{gangli delle
radici dorsali} (il termine \emph{ganglio} definisce un gruppo di
neuroni i cui corpi cellulari sono localizzati all'esterno del SNC).

I corpi cellulari dei \emph{neuroni efferenti}, invece, sono localizzati
dentro il midollo spinale. I neuroni efferenti originano nel corno
ventrale e si dirigono verso la periferia, dove formano sinapsi con le
fibre muscolari scheletriche.

Molte delle fibre ascendenti e discendenti si incrociano, ovvero partono
da sinistra e arrivano a destra. Fibre afferenti portano lo stimolo
verso il midollo spinale per poi salire lungo tratti ascendenti
incrociando e continuando a salire. Il tratto discendente proviene dalla
corteccia e va al midollo spinale incrociando e uscendo poi
ventralmente.

\subsubsection{L'encefalo}\label{lencefalo}

L'encefalo consta di 3 parti principali:

\begin{itemize}
\itemsep1pt\parskip0pt\parsep0pt
\item
  il \emph{prosencefalo}, formato da \emph{telencefalo} e
  \emph{diencefalo};
\item
  il \emph{mesencefalo} o \emph{cervelletto};
\item
  il \emph{tronco cefalico}.
\end{itemize}

Il \textbf{prosencefalo}, la parte più ampia e rostrale, è diviso nei
due \emph{emisferi} di destra e sinistra; esso consta del telencefalo
(la corteccia esterna, 80\% del cervello), che è la parte più anteriore,
e del diencefalo.

La corteccia è suddivisa in due zone emisferiche che rivestono la
sostanza bianca dei nuclei della base. La parte superiore è chiamata
``tetto'' e quella inferiore ``base''.

I nuclei della base stanno in basso. Nel diencefalo abbiamo talamo ed
ipotalamo in posizione ventrale.

Centralmente al talamo abbiamo le cavità dei ventricoli cerebrali.
Posteriormente c'è il tronco dell'encefalo che si suddivide in
mesencefalo, ponte e midollo allungato. Al di sopra del ponte abbiamo il
cervelletto. Dall'encefalo emergono i nervi cranici (sono 12) che
innervano gli organi di senso cefalici e varie regioni cefaliche anche
del collo.

(immagine 9.11c p 229)

Il \textbf{cervelletto} è una struttura bilaterale e simmetrica con una
corteccia situata all'esterno e nuclei situati in profondità. Si trova
inferiormente al prosencefalo e dorsalmente al tronco encefalico.

Il cervelletto svolge funzioni fondamentali nel controllo dell'attività
motoria e nel mantenimento dell'equilibrio.

Il \textbf{tronco encefalico} rappresenta la porzione più caudale
dell'encefalo; esso connette il prosencegalo ed il cervelletto con il
midollo spinale.

Il tronco encefalico consta di 3 regioni principali:

\begin{enumerate}
\def\labelenumi{\arabic{enumi}.}
\itemsep1pt\parskip0pt\parsep0pt
\item
  il \textbf{mesencefalo}, la porzione più rostrale che collega il
  tronco encefalico al prosencefalo;
\item
  il \textbf{ponte}, la porzione mediana che si connette al cervelletto;
\item
  il \textbf{midollo allungato} o \textbf{bulbo}, la porzione più
  caudale che si connette al midollo spinale.
\end{enumerate}

All'interno del tronco encefalico originano 10 dei 12 \emph{nervi
cranici}.

All'interno del tronco encefalico vi è anche la \textbf{formazione
reticolare}, una diffusa rete neuronale che svolge un ruolo importante
nel controllo dei cicli sonno-veglia, dell'eccitazione corticale e dello
stato di coscienza. In aggiunta, interviene nella regolazione di molte
funzioni involontarie controllate dal sistema nervoso autonomo, quali la
funzione caridiovascolare e la digestione.

Il \textbf{diencefalo}, che si trova localizzato al di sotto del
cervello, comprende due strutture mediane: il \emph{talamo} e
l'\emph{ipotalamo}.

Il \textbf{talamo} (dorsale) è un aggregato di nuclei sottocorticali
localizzato nel diencefalo. Tutte le informazioni sensoriali seguono un
percorso che include un passaggio attraverso il talamo e quindi verso la
corteccia. La maggior parte dei segnali sensoriali è filtrata e
modificata nel talamo prima di essere trasmessa alla corteccia. In
questo modo, il talamo sembra essere importante nel dirigere
l'attenzione.

Il talamo svolge anche un ruolo nel controllo dei movimenti.

L'\textbf{ipotalamo} (ventrale, inferiormente al talamo) rappresenta il
principale centro di collegamento tra il sistema endocrino e quello
nervoso. In risposta a segnali nervosi o ormonali, l'ipotalamo rilascia
ormoni che regolano il rilascio di altri ormoni dall'adenoipofisi
(ipofisi anteriore). Controlla anche il rilascio di ormoni dall'ipofisi
posteriore, come l'ormone antidiuretico che regola il volume e
l'osmolarità del plasma.

L'ipotalamo influenza anche molti comportamenti; in esso sono presenti
centri nervosi che controllano la sazietà e la fame, e il centro della
sete. Inoltre, essendo parte del sistema limbico, esso influenza le
emozioni ed i comportamenti che da esse dipendono.

La \textbf{corteccia cerebrale} rappresenta la porzione più esterna del
cervello; essa consta di uno strato sottile ed altamente convoluto di
sostanza grigia. Le circonvoluzioni originano da \textbf{solchi}
(invaginazioni) e \textbf{giri} (creste) che permettono all'ampio volume
di sostanza grigia di essere contenuto all'interno della scatola
cranica.

La corteccia cerebrale ha uno spessore che può variare, in relazione
alla localizzazione, da 1,5 a 4 mm. Sebbene la corteccia sia sottile, è
formata da sei strati funzionalmente distinti il cui spessore (e a volte
anche la presenza) dipende dalla localizzazione corticale. I diversi
strati sono composti da cellule differenti; ad esempio nel quarto strato
troviamo cellule stellate che eleaborano le percezioni sensoriali,
mentre nel quinto strato troviamo cellule piramidali che si occupano del
movimento volontario.

La corteccia cerebrale svolge le funzioni cerebrali più elevate ed
evolute, in quanto ci permette di avere percezioni relative all'ambiente
che ci circonda, formulare pensieri, ricordare eventi passati e, infine,
rappresenta l'area da cui partono tutti i comandi per l'esecuzione dei
movimenti.

Ciascun emisfero è diviso in 4 regioni dette \emph{lobi}.

Il \textbf{lobo frontale} rappresenta la parte anteriore del cervello,
si occupa dei movimento volontari e delle attività mentali (cioè il
pensiero). Posteriormente ad esso si trova il \textbf{lobo parietale},
che si occupa delle funzioni sensoriali. Questi due lobi sono separati
dal \emph{solco centrale}, che percorre ciascun emisfero del cervello.

Localizzato posteriormente e inferiormente al lobo parietale vi è il
\textbf{lobo occipitale}, che si occupa dell'elaborazione delle
sensazioni visive. Il \textbf{lobo temporale} è localizzato
inferiormente ai lobi frontale e parietale del cervello; esso è separato
dal lobo frontale da un profondo solco, il \emph{solco laterale} o
\emph{scissura di Silvio}, e si occupa delle percezioni sensoriali
uditive e olfattive.

Molte aree della corteccia cerebrale sono organizzate
\emph{topograficamente} in base alla loro funzione, ossia una proiezione
corporea e dei vari organi sensoriali.

Gli esempi più chiari di tale organizzazione topografico-funzionale sono
rappresentati dalla \emph{corteccia motoria} primaria nel lobo frontale
e dalla \emph{corteccia somatosensoriale} primaria nel lobo parietale.
Le mappe dell'\emph{organizzazione somatotopica} di queste due aree
corticali, in cui parti del corpo vicine sono rappresentate sulla
superficie corticale in regioni vicine, sono definite \textbf{omuncolo
motorio} e \textbf{omuncolo sensoriale}.

Il \textbf{telencefalo} è molto sviluppato soprattutto nei mammiferi.

Subito dopo per importanza troviamo il talamo e l'ipotalamo, che
costituiscono il diencefalo. Il talamo riceve buona parte degli stimoli
sensoriali delle vie afferenti. L'ipotalamo invece ha un ruolo di
stimolazione dell'attività endocrina e ad esso è collegata una ghiandola
chiamata \textbf{ipofisi} (inferiore) e una ghiandola chiamata
\textbf{epifisi} (superiormente).

Il \textbf{sistema limbico} è formato da un insieme di regioni
corticali, nuclei sottocorticali e tratti del prosencefalo strettamente
associati fra loro, coinvolti nelle emozioni, nella memoria e nella
motivazione.

Il sistema limbico è coinvolto in funzioni che regolano le pulsioni
comportamentali di base. Esso include l'\emph{amigdala} (coinvolta nelle
funzioni relative all'aggressività e alla paura) e l'\emph{ippocampo}
(coinvolto nell'apprendimento e nella memoria)

\subsection{Il controllo dei movimenti
volontari}\label{il-controllo-dei-movimenti-volontari}

Una buona parte dell'attività cerebrale riguarda i movimenti volontari.
Gli stimoli che riguardano i movimenti volontari partono tutti dalla
corteccia cerebrale. Nella corteccia avvengono le attività coscienti del
soggetto e quindi anche i movimenti volontari.

Nell'organismo ci sono tutta una serie di motilità che non sono sotto il
controllo del soggeto (es. motilità del cuore e apparato digerente), ma
ce n'è anche una volontaria (quella degli arti, della muscolatura
scheletrica, ecc).

Il controllo dei movimenti parte dalle aree del pensiero; dalle aree
associative (quelle piu' anteriori) partono gli stimoli che vanno alla
\emph{corteccia pre-motoria} che coordina i movimenti e poi passano alla
\emph{corteccia motoria} che manda lo stimolo per l'esecuzione dei
movimenti; questo viene indirizzato ad un preciso muscolo con una
determinata intensità.

L'esecuzione dei comandi motori è esegita mediante invio di segnali
discendneti ai muscoli che devono contrarsi. L'esecuzione del comando
richiede l'attivazione di neuroni efferenti che innervano i muscoli
scheletrici. Questi neuroni efferenti si trovano nel corno ventrale del
midollo spinale e sono chiamati \textbf{motoneuroni inferiori}.

L'attività dei motoneuroni è influenzata da segnali discendenti
provenienti dall'encefalo. Le due vie discendenti importanti nel
controllo dei movimenti volontari sono: i \emph{tratti piramidali}
(cellule dello strato 5) ed i \emph{tratti extrapiramidali}.

I \textbf{tratti piramidali} rappresentano vie dirette dalla corteccia
motoria primaria al midollo spinale. Gli assoni dei neuroni che danno
origine a questi tratti terminano nel \emph{corno ventrale} del midollo
spinale e sono chiamati \textbf{motoneuroni superiori}. Alcuni di questi
motoneuroni stabiliscono sinapsi dirette con i motoneuroni, altre
formano sinapsi con interneuroni.

La maggior parte dei neuroni piramidali incrocia nel SNC a livello del
bulbo.

I tratti piramidali sono coinvolti nel controllo dei movimenti fini e
precisi delle estremità distali degli arti, specialmente avambracci,
mani e dita.

I \textbf{tratti extrapiramidali} includono tutte le vie motorie ald i
fuori del sistema piramidale.

Queste vie formano connessioni indirette tra l'encefalo e il midollo
spinale; ciò sta a significare che i neuroni dei tratti extrapiramidali
non formano sinapsi dirette con i motoneuroni.

In linea generale, il sistema extrapiramidale è coinvolto nel controllo
di molti gruppi muscolari implicati nel mantenimento della postura e
dell'equilibrio, mentre il sistema piramidale è maggiormente coinvolto
nel controllo di piccoli gruppi di muscoli che si contraggono
indipendentemente.

Anche qui abbiamo il concetto di loop; ad esempio una volta che si è
deciso di iniziare a camminare, dopo un po' di tempo che lo si fa, il
processo diventa incosciente e si può camminare senza pensarci. Fino a
quando si realizza che la camminata deve cessare.

A livello periferico il controllo dei movimento gioca molto sugli archi
riflessi.

\textbf{(controllare da qui)}

Un neurone sensoriale propiocettivo (che sta dentro un muscolo) entra
nel midollo dal corno dorsale dove c'è un ganglio che raggruppa i corpi
cellulari. Entrando può allo stesso livello formare sinapsi con un
motoneurone che riceve stimoli anche dalle vie piramidali e va ad
innervare il muscolo stesso. La motilità muscolare è stata controllata
dalla stessa sensibilità sensoriale propiocettiva del muscolo.

Il tronco dell'encefalo comprende mesencefalo ponte e bulbo e la parte
più posteriore è spesso detta midollo allungato. Qui abbiamo una rete
neuronale che non è né un fascio di fibre né un nucleo ben definito.

Esistono poi dei nuclei ben definiti (come quello rosso). Il cervelletto
si sviluppa posteriormente ed interviene nel controllo della motilità. I
nuclei della base intervengono nel controllo del movimento con dei
feedback inibitori alla corteccia.

Il cervelletto contiene l'80\% dei neuroni del corpo. Ha varie
connessioni con la formazione reticolare del tronco dell'encefalo con i
nuclei vestibolari e con il nucleo rubro.

Al cervelletto arrivano delle fibre chiamate \emph{fibre muscoidi} che
sono collegate a vie che arrivano dalla corteccia cerebrale. Il
cervelletto a sua volta invia stimoli alla corteccia motoria attraverso
il talamo, attraverso il nucleo rosso.

Nella sostanza grigia si trova uno strato molecolare, uno delle cellule
del Purkinje ecc\ldots{} Ci sono fibre ascendenti che arrivano allo
strato molecolare facendo sinapsi con le cellule del Purkinje.
Funzionano direttamente le fibre muscose. Gli stimoli in uscita dal
cervelletto sono prodotti dalle cellule del Purkinje (gli stimoli
vengono modulati da queste).

Gli stimoli in entrata sono ricevuti da cellule eccitatorie mentre
quelli in uscita da cellule che ricevono stimoli inibitori. Le cellule
del Purkinje sono inibite dal GABA, che fa sinapsi inibitorie. La
corteccia inibisce l'attività nei nuclei cerebellari e così modula
l'attività motoria cerebrale.

\textbf{(a qui)}

\subsection{Funzoni integrate del SNC}\label{funzoni-integrate-del-snc}

L'encefalo è in grado di ``mettere insieme'' vari stimoli.

Il massimo grado di integrazione si pensa sia a livello dei lobi
frontali dove risiede l'attività del pensiero.

Se sono in grado di rendermi conto di quando tocco un oggetto è grazie
alle attività cerebrali. Se ci fossero dei danni che portano dai
recettori della mano fino alla corteccia non ce ne renderemmo conto.

Percepisco a livello dei centri associativi un evento come una
sensazione tattile. Poi posso pensare di toccare quell'oggetto senza
toccarlo per ricostruire una sensazione senza che questa avvenga
realmente. Questa è un'attività pura di pensiero.

L'attività cerebrale che ``svolgiamo'' può essere percepita sia come
azione sensoriale, che come attività cerebrale. L'attività mentale è
infatti un'attività cerebrale percepita come tale.

Tutto ciò di cui siamo coscienti è attività cerebrale, ma alcune sono
percepite come qualcosa di interagente fisicamente con il corpo e altre
invece che non percepite fisicamente corrispondono all'attività di
pensiero. Il pensiero è la forma più completa di attività associativa.

Esistono attività associative anche legate ad azioni, come il linguaggio
(molto sviluppato nell'essere umano).

Ci sono due aree strettamente connesse al linguaggio: l'\emph{area di
Wernicke}, coinvolta nella comprensione del linguaggio, e l\emph{area di
Broca}, coinvolta nella capacità di parlare e scrivere. Queste due aree
sono collegate da fasci di fibre di interconnessione.

Un'altra attività del cervello è il sonno. Il sonno serve per il
consolidamento dei processi di apprendimento e di memoria, e per il
rafforzamento delle difese immunitarie.

I neuroni hanno attività elettrica molto intensa captabile sulla
superficie del capo. Nel sonno l'attività elettrica cambia, assumendo la
forma di onde ampie e lente.

Esistono die tipi di sonno: il \emph{sonno a onde lente}, caratterizzato
da onde a bassa frequenza, e il \emph{sonno REM}, caratterizzato da onde
ad alta frequenza e da episodi di rapidi movimenti oculari.

Durante il sonno REM possono aumentare la respirazione e la frequenza
cardiaca.

Il ciclo sonno veglia è dovuto al \textbf{sistema reticolare attivamente
ascendente (SRAA)}. Questa regione è critica nel mantenere lo stato di
veglia. Afferenze da quest'area si proiettano alla corteccia, attraverso
stazioni sinaptiche nel talamo, nell'ipotalamo e nel tronco encefalico,
mantendola in uno stato vi veglia che la rende più recettiva ai segnali
in arrivo.

Neurotrasmettitori associati al SRAA sono l'acetilcolina, la
noradrenalina, e la dopamina.

L'adenosina invece porta sonno ad onde lente indotto dal prosencefalo.
La caffeina blocca il rilascio di adenosina. Il ponte rilascia
acetilcolina che induce la fase REM.

Quando si è svegli e vigili, l'elettroencefalogramma (EEG) mostra un
tracciato di onde ad alta frequenza e bassa ampiezza, note come
\emph{onde beta}, che rifletto i segnali elettrici generati da un gran
numero di neuroni in tempi differenti.

Quando si è svegli l'EEG mostra uno schema con onde a frequenza più
bassa e ampiezza maggiore, note come \emph{onde alfa}. In confronto alle
onde beta, le onde alfa rifletto un maggior grado di sincronizzazione
dell'attività elettrica dei neuroni; cioè, i segnali elettrici sono
generati in grandi gruppi di neuroni più o meno nello stesso istante.

Il sonno a onde lente avviene in 4 fasi successive, distinguibili dai
cambiamenti nell'EEG e dalla \emph{soglia del risveglio} (cioè
l'intensità dello stimolo richiesto per svelgiare una persona).

La prima fase ha la soglia del risveglio più bassa, mentre la quarta ha
la soglia più alta. Inoltre, le 4 fasi mostrano un graduale incremento
del grado di sincronizzazione dell'EEG, essendo le onde della quarta
fase le più alte in ampiezza e le più basse in frequenza, indice di
elevata sincronizzazione.

Durante il sonno REM, l'EEG mostra un andamento caratterizzato da onde
ad alta frequenza e bassa ampiezza somiglianti a quelle presenti durante
la veglia nello stato di allerta. La soglia del risveglio è più alta
durante il sonno REM che in qualsiasi altro momento, ma una persona
tende molto più facilemnte a svegliarsi spontaneamente durante questa
fase.

Durante la notte, una persona attraversa le varie fasi del sonno in
maniera ordinata e prevedibile. Quando ci si addormenta, si passa dalla
veglia alla prima fase del sonno a onde lente; da qui si passa con
ordine attraverso le fasi 2, 3 e 4. Circa un'ora dopo l'addormentamento
si ripercorrono le fasi in ordine inverso e quindi si entra nella prima
fase REM. Questi cicli si ripetono alcune volte ad intervalli di circa
90 minuti ma, nel corso della notte, si è portati a passare sempre meno
tempo nella fasi di sonno profondo a onde lente e più tempo nel sonno
REM.

Tra le funzoni intgrate del SNC vi sono anche le emozioni. L'amigdala
sembra giocare un ruolo importante nella paura e nell'ansia;
l'ipotalamo, invece, è associato con i sentimenti di rabbia e
aggressività

Entrambe sono le risposte comportativamente immediate nella vita di un
individuo. Queste sensazioni sono regolate dalla corteccia e danno delle
risposte vegetative motorie e ormonali. Legati all'emozione ci sono i
centri del piacere, che è un sito molto particolare in cui interviene in
sistema dopaminergico.

Altra funzioni integrate dal SNC sono l'apprendimento e la memoria.
Mentre l'apprendimento consiste nell'acquisizione di nuove informazioni
e esperienze, la memoria rappresenta il consolidamento di tali
informaizoni, esperienze o pensieri.

L'apprendimento può essere \emph{associativo} o \emph{non associativo}.

L'apprendimento \emph{associativo} è un tipo di apprendimento che
richiede la capacità di collegare due o più stimoli.

L'apprendimento \emph{non associativo} invece, si realizza in risposta a
stimoli ripetuti ed include i processi di abitudine e sensibilizzazione.
L'abitudine rappresenta una sorta di decrescimento della risposta a
stimoli ripetuti. Al contrario, la sensibilizzazione rappresenta un
incremento della risposta a stimoli ripetuti.

Esistono poi 2 tipi di memoria: la \emph{memoria procedurale} e quella
\emph{dichiarativa}.

La memoria procedurale, o \emph{memoria implicita}, è la memoria delle
capacità motorie e dei comportamenti appresi. Questo tipo di memoria
coinvolge diverse aree encefaliche, inclusi il cervelletto, i nuclei
della base e il ponte.

La memoria dichiarativa, o \emph{memoria esplicita}, rappresenta una
forma di memoria delle esperienze apprese, come fatti, eventi ed altre
cose che possono essere affermate verbalmente. Questo tipo di memoria
coinvolge l'ippocampo.

La memoria si realizza a due livelli: \emph{memoria a breve termine} e
\emph{a lungo termine}.

Nel modello corrente su come l'informazione viene acquisita ed
immagazzinata all;interno della memoria, le informazioni in arrivo prima
entrano nel SNC e poi vengono conservate sotto forma di \textbf{memoria
e breve termine} ( o \emph{memoria di lavoro}), un immagazzinamento
temporaneo di un concetto per pochi secondi o poche ore. Lo spazio per
la memoria a breve termine è limitato e le informaizoni immagazzinate in
questo modo vengono perse se non vengono ulteriormente
\emph{consolidate} sotto forma di \textbf{memoria a lungo termine}, che
può durare per anni o per l'intera vita.

I meccanismi di consolidamento non sono ben conosciuti, ma certamente la
ripetizione aiuta.

La memoria è un processo complesso che coinvole parecchie, se non tutte
le aree encefaliche. Il lobo frontale ha un ruolo cruciale nella memoria
a breve termine, il lobo temporale, incluso l'ippocampo, è necessario
per quella a lungo termine.

\end{document}
