\documentclass[]{article}
\usepackage{lmodern}
\usepackage{amssymb,amsmath}
\usepackage{ifxetex,ifluatex}
\usepackage{fixltx2e} % provides \textsubscript
\ifnum 0\ifxetex 1\fi\ifluatex 1\fi=0 % if pdftex
  \usepackage[T1]{fontenc}
  \usepackage[utf8]{inputenc}
\else % if luatex or xelatex
  \ifxetex
    \usepackage{mathspec}
    \usepackage{xltxtra,xunicode}
  \else
    \usepackage{fontspec}
  \fi
  \defaultfontfeatures{Mapping=tex-text,Scale=MatchLowercase}
  \newcommand{\euro}{€}
\fi
% use upquote if available, for straight quotes in verbatim environments
\IfFileExists{upquote.sty}{\usepackage{upquote}}{}
% use microtype if available
\IfFileExists{microtype.sty}{%
\usepackage{microtype}
\UseMicrotypeSet[protrusion]{basicmath} % disable protrusion for tt fonts
}{}
\ifxetex
  \usepackage[setpagesize=false, % page size defined by xetex
              unicode=false, % unicode breaks when used with xetex
              xetex]{hyperref}
\else
  \usepackage[unicode=true]{hyperref}
\fi
\hypersetup{breaklinks=true,
            bookmarks=true,
            pdfauthor={},
            pdftitle={},
            colorlinks=true,
            citecolor=blue,
            urlcolor=blue,
            linkcolor=magenta,
            pdfborder={0 0 0}}
\urlstyle{same}  % don't use monospace font for urls
\setlength{\parindent}{0pt}
\setlength{\parskip}{6pt plus 2pt minus 1pt}
\setlength{\emergencystretch}{3em}  % prevent overfull lines
\setcounter{secnumdepth}{0}

\date{}

\begin{document}

\section{Il sistema nervoso
efferente}\label{il-sistema-nervoso-efferente}

Fanno parte del \emph{sistema nervoso efferente} quelle vie che portano
gli stimoli nervosi dal SNC ad altri sistemi. Queste vie si dividono tra
sistema \textbf{autonomo}, che conduce gli stimoli involontari, e
\textbf{somatico}, che conduce stimoli volontari (ovvero sotto controllo
del soggetto).

\subsection{Il sistema nervoso
automono}\label{il-sistema-nervoso-automono}

Il sistema nervoso autonomo innerva la maggior parte degli organi e dei
tessuti effettori, incluso il muscolo cardiaco, le cellule muscolari
lisce dei vasi ematici e vari organi viscerali, le ghiandole e il
tessuto adiposo.

Il sistema nervoso ``autonomo'' è definito in tal modo in quanto la sua
attività si svolge in modo inconscio.

Il sistema nervoso autonomo è suddiviso in due sottosistemi, detti
\textbf{sistema simpatico} e \textbf{parasimpatico}, Entrambe le
divisioni del sistema nervoso autonomo innervano la maggior parte degli
organi, un'organizzazione chiamata \emph{duplice innervazione}. Questi
due sistemi spesso innervano gli stessi organi, ma con effetti
antagonisti.

Il sistema nervoso simpatico esce a livello spinale, mentre il
parasimpatico esce dal midollo allungato e dalle ultime vertebre sotto
forma di nervi (per lo più misti).

Tra gli effetti di questi due sistemi vi è ad esempio l'inibizione del
sistema cardovascolare da parte del parasimpatico, mentre il simpatico
lo stimola.

Il sistema nervoso autonomo consta di vie efferenti formate da due
neuroni organizzati in serie tra il SNC e gli organi effettori. I
neuroni comunicano tra loro mediante sinapsi localizzate in strutture
periferiche chiamate \textbf{gangli del sistema nervoso autonomo}. I
neuroni che collegano il SNC ai gangli sono definiti \textbf{neuroni
pregangliari}; quelli che collegano i gangli agli organi effettori sono
detti \textbf{neuroni postgangliari}. All'interno di ciascun ganglio vi
sono i terminali assonici dei neuroni pregangliari e i corpi cellulari
ed i dendriti dei neuroni postgangliari.

Nella porzione tra neurone pre e post gangliare trovo una sinapsi. Il
messaggio viaggia sempre dal SNC all'organo effettore. Gli stimoli sono
potenziali d'azione che viaggiano sotto forma di scariche di potenziale
d'azione. Nei gangli ci sono neuroni intragangliari che possono modulare
le scariche uscendo con frequenza diversa per attività di neuroni
intermedi dentro il ganglio.

\subsubsection{Anatomia del sistema nervoso
simpatico}\label{anatomia-del-sistema-nervoso-simpatico}

Poichè i neuroni pregangliari nel sistema nervoso simpatico emergono
dalle porzioni toraciche e lombari del midollo spinale, il sistema
nervoso simpatico è noto come sistema nervsono autonomo
\emph{toracolombare}.

I neuroni pregangliari originano in una regione di sostanza grigia del
midollo spinale chiamata \textbf{corno laterale}. I neuroni pregangliari
e postgangliari simpatici sono tra loro connessi in 3 modi:

\begin{enumerate}
\def\labelenumi{\arabic{enumi}.}
\itemsep1pt\parskip0pt\parsep0pt
\item
  Nella più comune delle situazioni, i neuroni pregangliari hanno brevi
  assoni che originano nel corno laterale del midollo spinale ed escono
  da questo attraverso la radice ventrale.
\end{enumerate}

Immediatamente dopo che le radici ventrali e dorsali formano il nervo
spinale, l'assone del neurone pregangliare lascia il nervo spinale
attraverso una diramazione chiamata \emph{ramo bianco}, che penetra in
uno o più gangli simpatici localizzati appena fuori dal midollo spinale.
Qui, il neurone pregangliare forma sinapsi con molti neuroni
postgangliari, che hanno lunghi assoni che raggiungono gli organi
effettori.

La maggior parte degli assoni postgangliari ritorna al nervo spinale
attraverso una diramazione chiamata \emph{ramo grigio} e raggiunge
l'organo effettore attraverso un nervo spinale.

I vari gangli simpatici sono tra loro collegati in modo tale da formare
una struttura che decorre parallelamente alla colonna vertebrale, ai due
lati di essa, chiamata \textbf{catena simpatica};

\begin{enumerate}
\def\labelenumi{\arabic{enumi}.}
\setcounter{enumi}{1}
\itemsep1pt\parskip0pt\parsep0pt
\item
  Una seconda possibile organizzazione delle fibre simpatiche si
  verifica quando un gruppo di lunghi neuroni pregangliari innerva
  direttamente tessuti endocrini, come la \emph{midollare della
  ghiandola del surrene}, invece di formare sinapsi con neuroni
  postgangliari.
\end{enumerate}

Ciascuna delle ghiandole surrenali, localizzata nel cuscinetto adiposo
sul ppolo superiore di ciascun rene, è suddivisa in uno strato esterno
corticale ed uno interno midollare.

Quest'ultimo consta di neuroni simpatici postgangliari modificati,
chiamati \textbf{cellule cromaffini}, che si differenziano in cellule
endocrine invece che in neuroni.

In seguito alla stimolazione del sistema nervoso simpatico, la midollare
del surrene rilascia in circolo catecolamine (80\% adrenalina, 20\%
noradrenalina, e tracce trascurabili di dopamina).

La midollare del surrene rilascia i proprio prodotti direttamente nel
sangue, per cui questi prodotti funzionano come ormoni.

\begin{enumerate}
\def\labelenumi{\arabic{enumi}.}
\setcounter{enumi}{2}
\itemsep1pt\parskip0pt\parsep0pt
\item
  Una terza disposizione anatomica delle fibre simpatiche comprende
  neuroni pregangliari che formano sinapsi con neuroni postgangliari in
  strutture chiamate \textbf{gangli collaterali}, situati tra il SNC e
  gli organi effettori.
\end{enumerate}

In questo caso, un neurone pregangliare fuoriesce dal midollo spinale
attraverso la radice ventrale ed entra nella catena simpatica attraverso
un ramo bianco. L'assone del neurone pregangliare attraversa questo
ganglio senza formare sinapsi e raggiunge un ganglio collaterale tramite
un nervo simpatico.

All'interno del ganglio collaterale, il neurone pregangliare forma
sinapsi con molti neuroni postgangliari, che terminano su numerosi
organi bersaglio.

Poichè questi gangli non sono tra loro connessi, come quelli della
catena simpatica, essi permettono al sistema nervoso simpatico di
raggiungere organi bersaglio ben definiti e quindi di esercitare effetti
meno diffusi.

(immagine 11.3 p 306)

\subsubsection{Anatomia del sistema nervoso
parasimpatico}\label{anatomia-del-sistema-nervoso-parasimpatico}

I neuroni pregangliari del sistema nervoso parasimpatico originano nel
tronco encefalico o nel midollo spinale sacrale.

In genere, i neuroni pregangliari parasimpatici sono relativamente
lunghi e terminano in gangli localizzati vicino agli organi effettori. A
livello gangliare, essi formano sinapsi con corti neuroni postgangliari
diretti agli organi effettori.

Nella porzione craniale del sistema parasimpatico, gli assoni dei nervi
pregangliari originano da corpi cellulari localizzati nel tronco
encefalico, che inviano i propri assoni nei nervi cranici.

Nel sistema parasimpatico la cellula pregangliare ha un assone più lungo
che raggiunge una cellula gangliare vicina all'organo effettore, mentre
la postgangliare è più corta. Esempi sono alcuni nervi cranici.

\subsection{I neurotrasmettitori del sistema nervoso
autonomo}\label{i-neurotrasmettitori-del-sistema-nervoso-autonomo}

I due neurotrasmettitori del sistema nervoso periferico sono
l'\emph{acetilcolina} e la \emph{noradrenalina}.

I neuroni che rilasciano acetilcolina sono noti come
\textbf{colinergici}.

L'\emph{acetilcolina} è rilasciata da: + neuroni pregangliari simpatici
e parasimpatici; + neuroni postgangliari parasimpatici; + neuroni
pregangliari simpatici che innervano le cellule cromaffini della
midollare del surrene. In questo caso l'acetilcolina agisce sulle
cellule endocrine della midollare stimolando il rilascio di adrenalina.

La \emph{noradrenalina} è il neurotrasmettitore usato da quasi tutti i
neuroni simpatici postgangliari, che pertanto vengono detti
\textbf{adrenergici}.

L'acetilcolina e la noradrenalina possono entrambe legarsi a differenti
classi e sottoclassi di recettori colinergici e adrenergici.

\subsubsection{I recettori colinergici}\label{i-recettori-colinergici}

Questi recettori si dividono in due categorie: quelli \emph{nicotinici}
e quelli \emph{muscarinici}.

Queste due classi si distinguono in base a studi farmacologici
effettuati utilizzando due agonisti (sostanze chimiche che si legano ad
un recettore e producono lo stesso effetto biologico) dell'acetilcolina:
la \emph{nicotina}, un alcaloide che si trova abbondantemente nella
pianta del tabacco, e la \emph{muscarina}, una sostanza fungina.

I recettori colinergici nicotinici sono localizzati sui corpi cellulari
e sui dendriti dei neuroni postgangliari simpatici e parasimpatici,
sulle cellule cromaffini della midollare del surrene e sulle cellule
muscolari scheletriche.

I recettori muscarinici invece, sono presenti sugli organi effettori del
sistema nervoso parasimpatico, come il cuore, le cellule muscolari lisce
del tratto digerente, ecc\ldots{}

Tutte le sottoclassi di recettori colinergici nicotinici sono associate
a canali permeabili al sodio e al potassio. Quando l'acetilcolina si
lega a tali recettori, questi canali cationici si aprono, permettendo al
sodio di diffondere all'interno della cellula e al potassio di
diffondere all'esterno.

Poichè il Na è lontano dal suo potenziale di equilibrio, il suo flusso
verso l'interno eccederà quello verso l'esterno del K, per cui la
cellula si depolarizza.

Pertanto, i recettori colinergici nicotinici sono associati con la
depolarizzazione, o \emph{eccitazione}, della membrana della cellula
postsinaptica.

Al contrario, tutte le classi di recettori muscarinici sono accoppiate a
proteine G e a secondi messaggeri. Le risposte attivate dal legame
dell'acetilcolina possono essere \emph{sia inibitorie che eccitatorie},
in relazione alle cellule bersaglio e alla natura della via di
trasduzione del segnale coinvolta.

\subsubsection{I recettori adrenergici}\label{i-recettori-adrenergici}

Vi sono due classi principali di recettori adrenergici localizzati in
organi effettori del sistema nervoso simpatico: i \textbf{recettori
alfa} e i \textbf{recettori beta}. Ciascuna di queste classi è
ulteriormente divisa in sottoclassi: \(\alpha\)\(_1\), \(\alpha\)\(_2\),
\(\beta\)\(_1\), \(\beta\)\(_2\) e \(\beta\)\(_3\).

I recettori adrenergici sono accoppiati a proteine G che attivano o
inibiscono sitemi di secondi messaggeri.

Il legame della noradrenalina o dell'adrenalina con un recettore
\textbf{\(\alpha\)\(_1\)} attiva una proteina G che, a sua volta, attiva
l'enzima fosfolipasi C.

Il legame con i recettori \textbf{\(\alpha\)\(_2\)} attiva una proteina
G inibitoria che riduce l'attività dell'adenilato ciclasi, riducendo
così la sintesi di AMP ciclico.

Il legame con i recettori \textbf{\(\beta\)} invece, attiva una proteina
G stimolatrice che incrementa l'attività dell'adenilato ciclasi,
aumentando la sintesi dell'AMP ciclico.

I recettori \(\alpha\) hanno una maggiore affinità per la noradrenalina
rispetto all'adrenalina e sono generalmente eccitatori.

I recettori \(\beta\)\(_1\) e \(\beta\)\(_3\) hanno un'affinità molto
simile per la noradrenalina e l'adrenalina e producono in genere effetti
eccitatori.

I recettori \(\beta\)\(_2\) hanno una maggiore affinità per l'adrenalina
rispetto alla noradrenalina, e in genere producono rispote inibitorie.

Questi recettori sono il bersaglio di vari agenti terapeutici; ad
esempio, per l'ipertensione vengono utilizzati i beta-bloccanti,
l'efedrina invece è utilizzata del trattamento dell'asma in quanto
agonista dei recettori \(\beta\)\(_2\), mentre l'atropina agisce sui
recettori polinergici muscarinici.

\subsection{Le giunzioni
neuroeffettrici}\label{le-giunzioni-neuroeffettrici}

La sinapsi tra un neurone efferente ed il suo organo bersaglio
(effettore) è definita \textbf{giunzione neuroeffettrice}.

Le sinapsi tra i neuroni autonomi postgangliari e i loro organi
effettori differiscono dalle classiche sinapsi tra due neuroni, in
quanto i neuroni postgangliari non inviano i loro terminali assonici su
un numero ben definito di cellule; i neurotrasmettitori infatti, vengono
rilasciati da numerosi rigonfiamenti localizzati ad intervalli quasi
costanti lungo gli assoni, noti come \textbf{varicosità}.

All'interno di queste varicosità, i neurotrasmettitori sono sintetizzati
ed immagazzinati in vescicole.

Le membrane degli assoni contengono i classici canali
voltaggio-dipendenti per il Na e il K che permettono la propagazione dei
potenziali d'azione.

In aggiunta, la membrana, nella regione di ciascuna varicosità, contiene
canali voltaggio-dipendenti per il calcio che si aprono all'arrivo del
potenziale d'azione.

L'arrivo di un potenziale d'azione nella varicosità apre i canali
voltaggio-dipendenti per il calcio che, entrando nel citoplasma, stimola
il rilascio del neurotrasmettitore mediante esocitosi.

Un assone postgangliare ha molte varicosità, per cui un potenziale
d'azione propagato lungo l'assone determina il rilascio del
neurotrasmettitore da tutti i rigonfiamenti.

Poichè la distanza tra le varicosità e l'organo effettore è maggiore
rispetto alla classica fessura sinaptica, il neurotrasmettitore
rilasciato diffonde in un'ampia area dell'organo effettore e si lega ai
recettori posti sulla membrana plasmatica delle cellule di tutto
l'organo bersaglio.

Gli effetti del neurotrasmettitore terminano quando, come nelle sinapsi
neurone-neurone, esso diffonde lontano dai recettori oppure viene
ricaptato o degradato ad opera di enzimi, come per esempio
l'\textbf{acetilcolinesterasi}, localizzata sia sulla membrana del
neurone postgangliare sia sulla membrana delle cellule degli organi
effettori colinergici.

In seguito alla degradazione ad opera dell'acetilcolinesterasi
dell'acetilcolina in acetato e colina, quest'ultima è ricaptata
attivamente all'interno delle varicosità postgangliari ed utilizzata per
sintetizzare altro neurotrasmettitore.

La \textbf{monoamminossidasi} è un altro enzima degradativo che degrada
adrenalina e noradrenalina.

\subsection{Il sistema nervoso
somatico}\label{il-sistema-nervoso-somatico}

A differenza del sistema nervoso autonomo, che controlla le funzooni di
molti organi effetori, il sistema nervoso somatico controlla un solo
tipo di organo effettore, il muscolo scheletrico.

Il sistemanervoso somatico ha un solo tipo di neuroni efferenti, i
motoneuroni, cioè i neuroni che innervano il muscolo scheletrico.

Nel sistema nervoso somatico, un singolo motoneurone collega il sistema
nervoso centrale a una fibra muscolare scheletrica.

I motoneuroni originano nel corno ventrale del midollo spinale e
ricevono segnali da molteplici afferenze.

Un singolo motoneurone innerva molte cellule muscolari (definite
\textbf{fibre muscolari}), ma ciascuna fibra è innervata da un singolo
motoneurone. L'insieme costituito da un motoneurone e dalle cellule
muscolari da esso innervate forma un'\textbf{unità motoria}. Quando un
motoneurone è attivato, stimola la contrazione di tutte le fubre
muscolari presenti nella sua unità.

\subsubsection{La giunzione
neuromuscolare}\label{la-giunzione-neuromuscolare}

Ciascuna diramazione dell'assone di un motoneurone forma sinapsi con una
fibra muscolare scheletrica a livello di una singola regione altamente
specializzata della membrana della cellula muscolare, formando una
\textbf{giunzione muscolare}. I terminali dell'assone del motoneurone,
chiamati \textbf{bottoni sinaptici} o \textbf{bottoni terminali},
immagazzinano e rilasciano acetilcolina, che è l'unico
neurotrasmettitore del sistema nervoso somatico.

Dal lato opposto del bottone sinaptico, sulla membrana della fibra
muscolare, vi è una regione specializzata, la \textbf{placca motrice},
che presenta molte invaginazioni contenenti un elevato numero di
recettori per l'acetilcolina.

Questi recettori rappresentano una varietà dei recettori colinergici
nicotinici.

L'acetilcolina, che è presente tra le invaginazioni della placca
motrice, determina la fine del segnale eccitatoio ed il rilassamento
della fibra muscolare.

Il recettore nicotinico è un canale ionico formato da 5 subunità, di cui
2 alfa che legano l'acetilcolina. Dopo aver legato l'acetilcolina, il
canale si apre permettendo il trasporto di ioni Na\(^+\). L'ingresso di
ioni sodio nella cellula muscolare produce una depolarizzazione che
prende il nome di \textbf{potenziale di placca}. Questo potenziale è
molto forte (circa 70 mV, è molto più forte di quello postsinaptico,
circa 10-15 mV) e da solo è in grado di depolarizzare la membrana
muscolare fino al valore soglia necessario per indurre il potenziale
d'azione.

Il potenziale di placca si propaga sulla fibra della matrice muscolare.

Negli spazi intersinaptici si accumula acetilcolina e qui vengono anche
legati i recettori colinergici nicotinici.

La fine del segnale lo si ha inibendo l'acetilcolinesterasi. Questo
enzima degrada l'acetilcolina e non permette alla cellula di assumere
acetilcolina.

Su queste sinapsi agiscono veleni che possono provocare rilassamento
muscolare e paralisi. I meccanismi di azione di questi veleni sono
diversi:

\begin{itemize}
\itemsep1pt\parskip0pt\parsep0pt
\item
  la \emph{tossina botulinica} viene rilasciata dal batterio Clostridio,
  ed inibisce il rilascio di acetilcolina mandando in paralisi il
  muscolo;
\item
  l'\emph{alfa-bungarotossina} (veleno del cobra, molto potente) agisce
  sul recettore nicotinico bloccando l'apertura dei canali sodio e
  impedendo in questo modo la contrazione della muscolatura;
\item
  il \emph{curaro} (estratto da alcune piante) compete con
  l'acetilcolina (antagonista dell'acetilcolina) e inibisce il
  meccanismo di attivazione del recettore;
\item
  l'\emph{atrossina} (veleno della vedova nera) aumenta il rilasico di
  acetilcolina mandando la muscolatura in contrazione tetanica;
\item
  gli inibitori dell'acetilcolinaesterasi (\emph{pesticidi} e \emph{gas
  nervino}) causano un'inibizione permanente della contrazione muscolare
  dovuta a depolarizzazione permanente e, di conseguenza, paralisi;
\item
  la \emph{succinilcolina} (analogo dell'acetilcolina) è meno sensibile
  all'azione dell'acetilcolinesterasi e aumnta l'attività provocando lo
  stesso effeto degli inibitori dell'acetilcolinesterasi e quindi causa
  paralisi da depolarizzazione.
\end{itemize}

\end{document}
