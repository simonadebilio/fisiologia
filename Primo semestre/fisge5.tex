\documentclass[]{article}
\usepackage{lmodern}
\usepackage{amssymb,amsmath}
\usepackage{ifxetex,ifluatex}
\usepackage{fixltx2e} % provides \textsubscript
\ifnum 0\ifxetex 1\fi\ifluatex 1\fi=0 % if pdftex
  \usepackage[T1]{fontenc}
  \usepackage[utf8]{inputenc}
\else % if luatex or xelatex
  \ifxetex
    \usepackage{mathspec}
    \usepackage{xltxtra,xunicode}
  \else
    \usepackage{fontspec}
  \fi
  \defaultfontfeatures{Mapping=tex-text,Scale=MatchLowercase}
  \newcommand{\euro}{€}
\fi
% use upquote if available, for straight quotes in verbatim environments
\IfFileExists{upquote.sty}{\usepackage{upquote}}{}
% use microtype if available
\IfFileExists{microtype.sty}{%
\usepackage{microtype}
\UseMicrotypeSet[protrusion]{basicmath} % disable protrusion for tt fonts
}{}
\ifxetex
  \usepackage[setpagesize=false, % page size defined by xetex
              unicode=false, % unicode breaks when used with xetex
              xetex]{hyperref}
\else
  \usepackage[unicode=true]{hyperref}
\fi
\hypersetup{breaklinks=true,
            bookmarks=true,
            pdfauthor={},
            pdftitle={},
            colorlinks=true,
            citecolor=blue,
            urlcolor=blue,
            linkcolor=magenta,
            pdfborder={0 0 0}}
\urlstyle{same}  % don't use monospace font for urls
\setlength{\parindent}{0pt}
\setlength{\parskip}{6pt plus 2pt minus 1pt}
\setlength{\emergencystretch}{3em}  % prevent overfull lines
\setcounter{secnumdepth}{0}

\date{}

\begin{document}

\section{Il sistema nervoso
efferente}\label{il-sistema-nervoso-efferente}

Fanno parte del \emph{sistema nervoso efferente} quelle vie che portano
gli stimoli nervosi dal SNC ad altri sistemi. Queste vie si dividono tra
sistema \textbf{autonomo}, che conduce gli stimoli involontari, e
\textbf{somatico}, che conduce stimoli volontari (ovvero sotto controllo
del soggetto).

\subsection{Il sistema nervoso
automono}\label{il-sistema-nervoso-automono}

Il sistema nervoso autonomo innerva la maggior parte degli organi e dei
tessuti effettori, incluso il muscolo cardiaco, le cellule muscolari
lisce dei vasi ematici e vari organi viscerali, le ghiandole e il
tessuto adiposo.

Il sistema nervoso ``autonomo'' è definito in tal modo in quanto la sua
attività si svolge in modo inconscio.

Il sistema nervoso autonomo è suddiviso in due sottosistemi, detti
\textbf{sistema simpatico} e \textbf{parasimpatico}, Entrambe le
divisioni del sistema nervoso autonomo innervano la maggior parte degli
organi, un'organizzazione chiamata \emph{duplice innervazione}. Questi
due sistemi spesso innervano gli stessi organi, ma con effetti
antagonisti.

Il sistema nervoso simpatico esce a livello spinale, mentre il
parasimpatico esce dal midollo allungato e dalle ultime vertebre sotto
forma di nervi (per lo più misti).

Tra gli effetti di questi due sistemi vi è ad esempio l'inibizione del
sistema cardovascolare da parte del parasimpatico, mentre il simpatico
lo stimola.

Il sistema nervoso autonomo consta di vie efferenti formate da due
neuroni organizzati in serie tra il SNC e gli organi effettori. I
neuroni comunicano tra loro mediante sinapsi localizzate in strutture
periferiche chiamate \textbf{gangli del sistema nervoso autonomo}. I
neuroni che collegano il SNC ai gangli sono definiti \textbf{neuroni
pregangliari}; quelli che collegano i gangli agli organi effettori sono
detti \textbf{neuroni postgangliari}. All'interno di ciascun ganglio vi
sono i terminali assonici dei neuroni pregangliari e i corpi cellulari
ed i dendriti dei neuroni postgangliari.

Nella porzione tra neurone pre e post gangliare trovo una sinapsi. Il
messaggio viaggia sempre dal SNC all'organo effettore. Gli stimoli sono
potenziali d'azione che viaggiano sotto forma di scariche di potenziale
d'azione. Nei gangli ci sono neuroni intragangliari che possono modulare
le scariche uscendo con frequenza diversa per attività di neuroni
intermedi dentro il ganglio.

\subsection{Anatomia del sistema nervoso
simpatico}\label{anatomia-del-sistema-nervoso-simpatico}

Poichè i neuroni pregangliari nel sistema nervoso simpatico emergono
dalle porzioni toraciche e lombari del midollo spinale, il sistema
nervoso simpatico è noto come sistema nervsono autonomo
\emph{toracolombare}.

I neuroni pregangliari originano in una regione di sostanza grigia del
midollo spinale chiamata \textbf{corno laterale}. I neuroni pregangliari
e postgangliari simpatici sono tra loro connessi in 3 modi:

\begin{enumerate}
\def\labelenumi{\arabic{enumi}.}
\itemsep1pt\parskip0pt\parsep0pt
\item
  Nella più comune delle situazioni, i neuroni pregangliari hanno brevi
  assoni che originano nel corno laterale del midollo spinale ed escono
  da questo attraverso la radice ventrale.
\end{enumerate}

Immediatamente dopo che le radici ventrali e dorsali formano il nervo
spinale, l'assone del neurone pregangliare lascia il nervo spinale
attraverso una diramazione chiamata \emph{ramo bianco}, che penetra in
uno o più gangli simpatici localizzati appena fuori dal midollo spinale.
Qui, il neurone pregangliare forma sinapsi con molti neuroni
postgangliari, che hanno lunghi assoni che raggiungono gli organi
effettori.

La maggior parte degli assoni postgangliari ritorna al nervo spinale
attraverso una diramazione chiamata \emph{ramo grigio} e raggiunge
l'organo effettore attraverso un nervo spinale.

I vari gangli simpatici sono tra loro collegati in modo tale da formare
una struttura che decorre parallelamente alla colonna vertebrale, ai due
lati di essa, chiamata \textbf{catena simpatica};

\begin{enumerate}
\def\labelenumi{\arabic{enumi}.}
\setcounter{enumi}{1}
\itemsep1pt\parskip0pt\parsep0pt
\item
  Una seconda possibile organizzazione delle fibre simpatiche si
  verifica quando un gruppo di lunghi neuroni pregangliari innerva
  direttamente tessuti endocrini, come la \emph{midollare della
  ghiandola del surrene}, invece di formare sinapsi con neuroni
  postgangliari.
\end{enumerate}

Ciascuna delle ghiandole surrenali, localizzata nel cuscinetto adiposo
sul ppolo superiore di ciascun rene, è suddivisa in uno strato esterno
corticale ed uno interno midollare.

Quest'ultimo consta di neuroni simpatici postgangliari modificati,
chiamati \textbf{cellule cromaffini}, che si differenziano in cellule
endocrine invece che in neuroni.

In seguito alla stimolazione del sistema nervoso simpatico, la midollare
del surrene rilascia in circolo catecolamine (80\% adrenalina, 20\%
noradrenalina, e tracce trascurabili di dopamina).

La midollare del surrene rilascia i proprio prodotti direttamente nel
sangue, per cui questi prodotti funzionano come ormoni.

\begin{enumerate}
\def\labelenumi{\arabic{enumi}.}
\setcounter{enumi}{2}
\itemsep1pt\parskip0pt\parsep0pt
\item
  Una terza disposizione anatomica delle fibre simpatiche comprende
  neuroni pregangliari che formano sinapsi con neuroni postgangliari in
  strutture chiamate \textbf{gangli collaterali}, situati tra il SNC e
  gli organi effettori.
\end{enumerate}

In questo caso, un neurone pregangliare fuoriesce dal midollo spinale
attraverso la radice ventrale ed entra nella catena simpatica attraverso
un ramo bianco. L'assone del neurone pregangliare attraversa questo
ganglio senza formare sinapsi e raggiunge un ganglio collaterale tramite
un nervo simpatico.

All'interno del ganglio collaterale, il neurone pregangliare forma
sinapsi con molti neuroni postgangliari, che terminano su numerosi
organi bersaglio.

Poichè questi gangli non sono tra loro connessi, come quelli della
catena simpatica, essi permettono al sistema nervoso simpatico di
raggiungere organi bersaglio ben definiti e quindi di esercitare effetti
meno diffusi.

(immagine 11.3 p 306)

\subsection{Anatomia del sistema nervoso
parasimpatico}\label{anatomia-del-sistema-nervoso-parasimpatico}

I neuroni pregangliari del sistema nervoso parasimpatico originano nel
tronco encefalico o nel midollo spinale sacrale.

In genere, i neuroni pregangliari parasimpatici sono relativamente
lunghi e terminano in gangli localizzati vicino agli organi effettori. A
livello gangliare, essi formano sinapsi con corti neuroni postgangliari
diretti agli organi effettori.

Nella porzione craniale del sistema parasimpatico, gli assoni dei nervi
pregangliari originano da corpi cellulari localizzati nel tronco
encefalico, che inviano i propri assoni nei nervi cranici.

Nel sistema parasimpatico la cellula pregangliare ha un assone più lungo
che raggiunge una cellula gangliare vicina all'organo effettore, mentre
la postgangliare è più corta. Esempi sono alcuni nervi cranici.

\subsection{I neurotrasmettitori del sistema nervoso
autonomo}\label{i-neurotrasmettitori-del-sistema-nervoso-autonomo}

I due neurotrasmettitori del sistema nervoso periferico sono
l'\emph{acetilcolina} e la \emph{noradrenalina}.

I neuroni che rilasciano acetilcolina sono noti come
\textbf{colinergici}.

L'\emph{acetilcolina} è rilasciata da: + neuroni pregangliari simpatici
e parasimpatici; + neuroni postgangliari parasimpatici; + neuroni
pregangliari simpatici che innervano le cellule cromaffini della
midollare del surrene. In questo caso l'acetilcolina agisce sulle
cellule endocrine della midollare stimolando il rilascio di adrenalina.

La \emph{noradrenalina} è il neurotrasmettitore usato da quasi tutti i
neuroni simpatici postgangliari, che pertanto vengono detti
\textbf{adrenergici}.

L'acetilcolina e la noradrenalina possono entrambe legarsi a differenti
classi e sottoclassi di recettori colinergici e adrenergici.

\subsubsection{I recettori colinergici}\label{i-recettori-colinergici}

Questi recettori si dividono in due categorie: quelli \emph{nicotinici}
e quelli \emph{muscarinici}.

Queste due classi si distinguono in base a studi farmacologici
effettuati utilizzando due agonisti (sostanze chimiche che si legano ad
un recettore e producono lo stesso effetto biologico) dell'acetilcolina:
la \emph{nicotina}, un alcaloide che si trova abbondantemente nella
pianta del tabacco, e la \emph{muscarina}, una sostanza fungina.

I recettori colinergici nicotinici sono localizzati sui corpi cellulari
e sui dendriti dei neuroni postgangliari simpatici e parasimpatici,
sulle cellule cromaffini della midollare del surrene e sulle cellule
muscolari scheletriche.

I recettori muscarinici invece, sono presenti sugli organi effettori del
sistema nervoso parasimpatico, come il cuore, le cellule muscolari lisce
del tratto digerente, ecc\ldots{}

Tutte le sottoclassi di recettori colinergici nicotinici sono associate
a canali permeabili al sodio e al potassio. Quando l'acetilcolina si
lega a tali recettori, questi canali cationici si aprono, permettendo al
sodio di diffondere all'interno della cellula e al potassio di
diffondere all'esterno.

Poichè il Na è lontano dal suo potenziale di equilibrio, il suo flusso
verso l'interno eccederà quello verso l'esterno del K, per cui la
cellula si depolarizza.

Pertanto, i recettori colinergici nicotinici sono associati con la
depolarizzazione, o \emph{eccitazione}, della membrana della cellula
postsinaptica.

Al contrario, tutte le classi di recettori muscarinici sono accoppiate a
proteine G e a secondi messaggeri. Le risposte attivate dal legame
dell'acetilcolina possono essere \emph{sia inibitorie che eccitatorie},
in relazione alle cellule bersaglio e alla natura della via di
trasduzione del segnale coinvolta.

\subsubsection{I recettori adrenergici}\label{i-recettori-adrenergici}

Vi sono due classi principali di recettori adrenergici localizzati in
organi effettori del sistema nervoso simpatico: i \textbf{recettori
alfa} e i \textbf{recettori beta}. Ciascuna di queste classi è
ulteriormente divisa in sottoclassi: \(\alpha\)\(_1\), \(\alpha\)\(_2\),
\(\beta\)\(_1\), \(\beta\)\(_2\) e \(\beta\)\(_3\).

I recettori adrenergici sono accoppiati a proteine G che attivano o
inibiscono sitemi di secondi messaggeri.

Il legame della noradrenalina o dell'adrenalina con un recettore
\textbf{\(\alpha\)\(_1\)} attiva una proteina G che, a sua volta, attiva
l'enzima fosfolipasi C.

Il legame con i recettori \textbf{\(\alpha\)\(_2\)} attiva una proteina
G inibitoria che riduce l'attività dell'adenilato ciclasi, riducendo
così la sintesi di AMP ciclico.

Il legame con i recettori \textbf{\(\beta\)} invece, attiva una proteina
G stimolatrice che incrementa l'attività dell'adenilato ciclasi,
aumentando la sintesi dell'AMP ciclico.

I recettori \(\alpha\) hanno una maggiore affinità per la noradrenalina
rispetto all'adrenalina e sono generalmente eccitatori.

I recettori \(\beta\)\(_1\) e \(\beta\)\(_3\) hanno un'affinità molto
simile per la noradrenalina e l'adrenalina e producono in genere effetti
eccitatori.

I recettori \(\beta\)\(_2\) hanno una maggiore affinità per l'adrenalina
rispetto alla noradrenalina, e in genere producono rispote inibitorie.

Questi recettori sono il bersaglio di vari agenti terapeutici; ad
esempio, per l'ipertensione vengono utilizzati i beta-bloccanti,
l'efedrina invece è utilizzata del trattamento dell'asma in quanto
agonista dei recettori \(\beta\)\(_2\), mentre l'atropina agisce sui
recettori polinergici muscarinici.

\subsection{Le giunzioni
neuroeffettrici}\label{le-giunzioni-neuroeffettrici}

La sinapsi tra un neurone efferente ed il suo organo bersaglio
(effettore) è definita \textbf{giunzione neuroeffettrice}.

Le sinapsi tra i neuroni autonomi postgangliari e i loro organi
effettori differiscono dalle classiche sinapsi tra due neuroni, in
quanto i neuroni postgangliari non inviano i loro terminali assonici su
un numero ben definito di cellule; i neurotrasmettitori infatti, vengono
rilasciati da numerosi rigonfiamenti localizzati ad intervalli quasi
costanti lungo gli assoni, noti come \textbf{varicosità}.

All'interno di queste varicosità, i neurotrasmettitori sono sintetizzati
ed immagazzinati in vescicole.

Le membrane degli assoni contengono i classici canali
voltaggio-dipendenti per il Na e il K che permettono la propagazione dei
potenziali d'azione.

In aggiunta, la membrana, nella regione di ciascuna varicosità, contiene
canali voltaggio-dipendenti per il calcio che si aprono all'arrivo del
potenziale d'azione.

L'arrivo di un potenziale d'azione nella varicosità apre i canali
voltaggio-dipendenti per il calcio che, entrando nel citoplasma, stimola
il rilascio del neurotrasmettitore mediante esocitosi.

Un assone postgangliare ha molte varicosità, per cui un potenziale
d'azione propagato lungo l'assone determina il rilascio del
neurotrasmettitore da tutti i rigonfiamenti.

Poichè la distanza tra le varicosità e l'organo effettore è maggiore
rispetto alla classica fessura sinaptica, il neurotrasmettitore
rilasciato diffonde in un'ampia area dell'organo effettore e si lega ai
recettori posti sulla membrana plasmatica delle cellule di tutto
l'organo bersaglio.

Gli effetti del neurotrasmettitore terminano quando, come nelle sinapsi
neurone-neurone, esso diffonde lontano dai recettori oppure viene
ricaptato o degradato ad opera di enzimi, come per esempio
l'\textbf{acetilcolinesterasi}, localizzata sia sulla membrana del
neurone postgangliare sia sulla membrana delle cellule degli organi
effettori colinergici.

In seguito alla degradazione ad opera dell'acetilcolinesterasi
dell'acetilcolina in acetato e colina, quest'ultima è ricaptata
attivamente all'interno delle varicosità postgangliari ed utilizzata per
sintetizzare altro neurotrasmettitore.

La \textbf{monoamminossidasi} è un altro enzima degradativo che degrada
adrenalina e noradrenalina.

\subsection{Il sistema nervoso
somatico}\label{il-sistema-nervoso-somatico}

A differenza del sistema nervoso autonomo, che controlla le funzooni di
molti organi effetori, il sistema nervoso somatico controlla un solo
tipo di organo effettore, il muscolo scheletrico.

Il sistemanervoso somatico ha un solo tipo di neuroni efferenti, i
motoneuroni, cioè i neuroni che innervano il muscolo scheletrico.

Nel sistema nervoso somatico, un singolo motoneurone collega il sistema
nervoso centrale a una fibra muscolare scheletrica.

I motoneuroni originano nel corno ventrale del midollo spinale e
ricevono segnali da molteplici afferenze.

Un singolo motoneurone innerva molte cellule muscolari (definite
\textbf{fibre muscolari}), ma ciascuna fibra è innervata da un singolo
motoneurone. L'insieme costituito da un motoneurone e dalle cellule
muscolari da esso innervate forma un'\textbf{unità motoria}. Quando un
motoneurone è attivato, stimola la contrazione di tutte le fubre
muscolari presenti nella sua unità.

\subsubsection{La giunzione
neuromuscolare}\label{la-giunzione-neuromuscolare}

Ciascuna diramazione dell'assone di un motoneurone forma sinapsi con una
fibra muscolare scheletrica a livello di una singola regione altamente
specializzata della membrana della cellula muscolare, formando una
\textbf{giunzione muscolare}. I terminali dell'assone del motoneurone,
chiamati \textbf{bottoni sinaptici} o \textbf{bottoni terminali},
immagazzinano e rilasciano acetilcolina, che è l'unico
neurotrasmettitore del sistema nervoso somatico.

Dal lato opposto del bottone sinaptico, sulla membrana della fibra
muscolare, vi è una regione specializzata, la \textbf{placca motrice},
che presenta molte invaginazioni contenenti un elevato numero di
recettori per l'acetilcolina.

Questi recettori rappresentano una varietà dei recettori colinergici
nicotinici.

L'acetilcolina, che è presente tra le invaginazioni della placca
motrice, determina la fine del segnale eccitatoio ed il rilassamento
della fibra muscolare.

Il recettore nicotinico è un canale ionico formato da 5 subunità, di cui
2 alfa che legano l'acetilcolina. Dopo aver legato l'acetilcolina, il
canale si apre permettendo il trasporto di ioni Na\(^+\). L'ingresso di
ioni sodio nella cellula muscolare produce una depolarizzazione che
prende il nome di \textbf{potenziale di placca}. Questo potenziale è
molto forte (circa 70 mV, è molto più forte di quello postsinaptico,
circa 10-15 mV) e da solo è in grado di depolarizzare la membrana
muscolare fino al valore soglia necessario per indurre il potenziale
d'azione.

Il potenziale di placca si propaga sulla fibra della matrice muscolare.

Negli spazi intersinaptici si accumula acetilcolina e qui vengono anche
legati i recettori colinergici nicotinici.

La fine del segnale lo si ha inibendo l'acetilcolinesterasi. Questo
enzima degrada l'acetilcolina e non permette alla cellula di assumere
acetilcolina.

Su queste sinapsi agiscono veleni che possono provocare rilassamento
muscolare e paralisi. I meccanismi di azione di questi veleni sono
diversi:

\begin{itemize}
\itemsep1pt\parskip0pt\parsep0pt
\item
  la \emph{tossina botulinica} viene rilasciata dal batterio Clostridio,
  ed inibisce il rilascio di acetilcolina mandando in paralisi il
  muscolo;
\item
  l'\emph{alfa-bungarotossina} (veleno del cobra, molto potente) agisce
  sul recettore nicotinico bloccando l'apertura dei canali sodio e
  impedendo in questo modo la contrazione della muscolatura;
\item
  il \emph{curaro} (estratto da alcune piante) compete con
  l'acetilcolina (antagonista dell'acetilcolina) e inibisce il
  meccanismo di attivazione del recettore;
\item
  l'\emph{atrossina} (veleno della vedova nera) aumenta il rilasico di
  acetilcolina mandando la muscolatura in contrazione tetanica;
\item
  gli inibitori dell'acetilcolinaesterasi (\emph{pesticidi} e \emph{gas
  nervino}) causano un'inibizione permanente della contrazione muscolare
  dovuta a depolarizzazione permanente e, di conseguenza, paralisi;
\item
  la \emph{succinilcolina} (analogo dell'acetilcolina) è meno sensibile
  all'azione dell'acetilcolinesterasi e aumnta l'attività provocando lo
  stesso effeto degli inibitori dell'acetilcolinesterasi e quindi causa
  paralisi da depolarizzazione.
\end{itemize}

\textbf{lezione 20151116}

\section{Il sistema sensoriale}\label{il-sistema-sensoriale}

La porzione afferente del sitema nervoso periferico trasmette
informazioni dalla periferia al SNC. Le informazioni sono raccolte da
\emph{recettori sensoriali} che rispondono a stimoli specifici. mentre
alcuni di questi recettori vengono eccitati da stimoli provenienti
dall'ambiente esterno, altri, definiti \emph{recettori viscerali},
ricevono stimoli provenienti dall'interno dell'organismo.

Esempi di queste recettori viscerali includono: \emph{chemocettori}
delle pareti dei vasi ematici, che monitorano i livelli di O\(_2\),
CO\(_2\) e H\(^+\); i \emph{barocettori} che rilevano la pressione
ematica; i \emph{meccanocettori} del tratto intestinale.

Sebbene l'encefalo riceva informazioni da questi recettori e le
utilizzi, noi non siamo coscienti di questi stimoli.

Parte della percezione si basa su esperienze precedenti, dunque
individui diversi possono percepire gli stessi stimoli differentemente.

Tutti i recettori inviano stimoli al sistema nervoso centrale ma non
tutti raggiungono la corteccia cerebrale. Il soggetto è conscio solo
degli stimoli che raggiungono la corteccia cerebrale. Altri stimoli
rimangono a lui ignoti dal punto di vista dell'autoconsapevolezza anche
se il SNC non rimane indifferente e reagisce comunque a questi stimoli
tramiti gli efferenti anche se non ce ne si rende conto.

Il potenziale d'azione parte dalla terminazione nervosa e va verso il
SNC. Insorge per diretta stimolazione della terminazione o sotto
stimolazione della cellula recettoriale.

I recettori sensoriali sono strutture neuronali specializzate in grado
di percepire specifiche forme di energia derivanti sia dall'ambiente
esterno che dall'interno dell'organismo.

La forma di energia di uno stimolo è definita \textbf{modalità} (es.
onde luminose, pressione, temperatura\ldots{}). La \textbf{legge delle
energie nervose specifiche} stabilisce che un determinato recettore
sensoriale è specifico per una particolare modalità sensoriale. La
\textbf{modalità} alla quale risponde un recettore è definita
\textbf{stimolo adeguato}. Modalità differenti dagli stimoli adeguati
possono attivare i recettori soltanto se i livelli di energia di
stimolazione sono molto elevati.

La funzione dei recettori sensoriali è la \textbf{trasduzione}, cioè la
conversione di una forma di energia in un'altra. Nella \emph{trasduzione
sensoriale}, i recettori convertono la forma di energia di uno stimolo
sensoriale in modificazioni del potenziale di membrana definite
\textbf{potenziale di recettore} o \textbf{potenziali generatori}. I
potenziali di recettore hanno le stesse caratteristiche dei potenziali
postsinaptici, in quanto sono dei potenziali graduati generati
dall'apertura o dalla chiusura di canali ionici. Maggiore è l'intensità
dello stimolo, maggiore sarà la variazione del potenziale di membrana.

I recettori sensoriali sono di 2 tipi differenti:

\begin{enumerate}
\def\labelenumi{\arabic{enumi}.}
\itemsep1pt\parskip0pt\parsep0pt
\item
  il recettore sensoriale può essere una struttura specializzata
  presente all'estremità periferica di un neurone afferente. Se il
  recettore si depolarizza fino al valore soglia, si avrà un potenziale
  d'azione che verrà propagato dal neurone afferente fino al SNC,
  trasmettendo informazioni riguardanti lo stimolo;
\item
  in altri casi, il recettore sensoriale è costituito da una cellula che
  comunica attraverso una sinapsi chimica con un neurone afferente ad
  essa associato.
\end{enumerate}

Alcuni recettori continuano a rispondere ad uno stimolo per tutta la
durata di applicazione dello stimolo stesso. Tuttavia, molti recettori
\emph{si adattano} allo stimolo, in quanto la loro risposta diminuisce
nel tempo. L'\textbf{adattamento recettoriale} rappresenta il decremento
nel tempo dell'ampiezza del potenziale di recettore in presenza di uno
stimolo costante.

I \textbf{recettori a lento adattamento}, o \textbf{recettori tonici},
hanno bassi livelli di adattamento e pertanto possono dare informazioni
relative all'intensità di uno stimolo prolungato.

I \textbf{recettori a rapido adattamento}, o \textbf{recettori fasici},
si adattano rapidamente e funzionano in maniera ottimale quando devono
rilevare modificazioni dell'intensità dello stimolo. I recettori fasici
rispondono all'inizio dello stimolo, per poi adattarsi a esso. Alcuni di
questi recettori mostrano una seconda e minore risposta al termine dello
stimolo, definita \emph{``risposta off''}.

Le vie nervose specifiche che trasmettono informazioni pertinenti ad una
particolare modalità sono definite \textbf{linee marcate} e ciascuna
modalità sensoriale segue la sua linea marcata. L'attivazione di una via
specifica determina la percezione della modalità associata,
indipendentemente dal reale stimolo che attiva la via.

Un'\textbf{unità sensoriale} comprende un singolo neurone afferente e
tutti i recettori ad esso associati. Tutti i recettori associati con un
determinato neurone afferente sono dello stesso tipo e l'attivazione di
ognuno di essi può generare potenziali d'azione nel neurone afferente.

L'area nella quale uno stimolo adeguato può produrre una risposta
(eccitatoria o inibitoria) nel neurone afferente è definita
\textbf{campo recettivo} di quel neurone; esso corrisponde alla regione
che contiene recettori per quel neurone afferente.

Il neurone afferente che trasmette l'informazione dalla periferia al SNC
è definito \textbf{neurone di primo ordine}. Un singolo neurone di primo
ordine può comunicare con molti interneuroni, causando, nell'ambito del
SNC, una divergenza del segnale. In aggiunta, interneuroni possono
ricevere impulsi convergenti da molti neuroni di primo ordine. Alcuni di
questi interneuroni trasmettono informazioni al talamo, che rappresenta
la regione principale di collegamento per le informazioni sensoriali;
questi interneuroni sono un esempio di \textbf{neuroni di secondo
ordine}.

Nel talamo, questi neuroni di seconod ordine formano sinapsi con
\textbf{neuroni di terzo ordine}, che trasmettono le informazioni alla
corteccia cerebrale, dove si realizza la percezione della sensazione.

\textbf{(immagine 10.6 pag258)}

Questa organizzazione geometrica del sistema nervoso è ideale per creare
delle mappature.

L'\emph{intensità} dello stimolo è codificata dalla \emph{frequenza dei
potenziali d'azione} (\textbf{codice di frequenza}) e dal \emph{numero
dei recettori attivati} (\textbf{codice di popolazione}).

Nel \emph{codice di frequenza}, uno stimolo più intenso dà origine a un
potenziale di recettore più ampio. Se un potenziale graduato (in questo
caso il potenziale del recettore) supera il valore soglia,
depolarizzazioni maggiori possono superare il periodo refrattario
relativo di un potenziale d'azione, generando, di conseguenza, un
secondo potenziale d'azione più rapidamente di quanto non farebbe una
depolarizzazione più debole. Perciò uno stimolo più intenso produce un
aumento della frequenza di scarica dei potenziale d'azione.

Nel \emph{codice di popolazione}, uno stimolo più intenso attiva o
recluta un maggior numero di recettori; questi recettori possono essere
associati con un singolo neurone afferente, oppure lo stimolo può
reclutare recettori associati a differenti neuroni afferenti. In
entrambi i casi, al SNC viene trasmessa una maggior frequenza di
potenziali d'azione in risposta allo stimolo, indicando che lo stimolo è
più forte.

\textbf{(immagine 10.8 p259)}

La precisione con la quale è percepita la localizzazione di uno stimolo
è definita \textbf{acuità} (= capacità di distinguere due stimoli molto
vicini come diversi).

Questa proprietà è ben rappresentata nella sensibilità somatica ma anche
nella retina. La superficie del corpo e della retina sono quelle più
coinvolte nei processi di mappatura topografice spaciale dello stimolo.

Nelle sensazioni associate ai recettori della cute, l'acuità dipende
dalle dimensioni e dal numero dei campi recettivi, dal sovrapporsi degli
stessi e dal fenomeno dell'inibizione laterale. Se è attivato uno
specifico neurone afferente, lo stimolo deve essere localizzato
nell'ambito del campo recettivo di quel neurone. Tuttavia, le dimensioni
dei campi recettivi variano notevolmente nell'organismo.

La localizzazione dello stimolo è migliore nelle regioni innervate da
neuroni con campi recettivi piccoli. La localizzazione è migliorata
dalla sovrapposizione dei campi recettivi di diversi neuroni afferenti.
Questa sovrapposizione migliora la localizzazione tramite due
meccanismi:

\begin{enumerate}
\def\labelenumi{\arabic{enumi}.}
\itemsep1pt\parskip0pt\parsep0pt
\item
  l'attivazione di entrambi i neuroni afferenti da parte di qualsiasi
  stimolo che cada nella regione di sovrapposizione;
\item
  l'inibizione laterale.
\end{enumerate}

Nell'\textbf{inibizione laterale}, uno stimolo che eccita fortemente i
recettori in un'area cutanea inibisce l'attività nelle vie afferenti dei
recettori limitrofi.

La massima acuità è presente nelle labbra o nelle dita delle mani ed è
minima nella schiena.

L'inibizione laterale incrementa l'acuità, in quanto migliora il
contrasto dei segnali nel sistema nervoso. Essa permette la trasmissione
di segnali intensi in alcuni neuroni, sopprimendo la trasmissione di
segnali più deboli provenienti dai neuroni limitrofi.

Ciascun neurone stabilisce sinapsi con un singolo neurone di secondo
ordine. I neuroni afferenti presentano collaterali che comunicano con
interneuroni inibitori del SNC. In questo caso, le collaterali
provenienti dal neurone afferente attivano interneuroni inibitori che
riducono la comunicazione tra neuroni afferenti con campi recettivi
limitrofi e neuroni di secondo ordine.

L'inibizione laterale incrementa l'acuità, in quanto migliora il
contrasto dei segnali nel sistema nervoso. Essa permette la trasmissione
di segnali intensi in alcuni neuroni, sopprimendo la trasmissione di
segnali più deboli provenienti dai neuroni limitrofi.

Il risultato netto sarà una maggiore differenza nella frequenza dei
potenziali d'azione tra i neuroni di secondo ordine rispetto ai neuroni
afferenti ma, ciò che è più importante, la frequenza dei potenziali
d'azione del neuroni che riceve lo stimolo sarà \emph{molto più elevata}
rispetto a quella dei neuroni limitrofi. L'aumentato contrasto
risultante tra segnali neuronali più e meno importanti permetterà una
migliore localizzazine dello stimolo, aumentando l'acuità tattile.

Una misura dell'acuità tattile è data dalla \textbf{discriminazione di
due punti}, vale a dire la capacità di una persona di percepire due
stimoli pressori applicati sulla cute in due punti separati
spazialmente. Al di sotto della \emph{soglia di discriminazione di due
punti} i due stimoli non vengono più percepiti come separati. La
possibilità di discriminare due punti di stimolazione si verifica
soltanto se i due stimoli pressori sono applicati ai campi recettivi di
due diversi neuroni afferenti. Pertanto, più piccoli sono i campi
recettivi, maggiore sarà la capacità di discriminare due punti distinti
e maggiore sarà l'acuità tattile.

In altri sistemi, come quelli olfattivi e uditivi, la localizzaizone
dello stimolo è basata sull'arrivo di stimoli alle due narici o alle due
orecchie con tempi lievementi differenti. Il cervello usa le differenze
temporali nell'arrivo del potenziale d'azione a livello della corteccia
olfattiva o uditiva per determinare la provenienza dello stimolo. Il
sistema nervoso centrale riesce a discriminare quale gruppo di neuroni
si è attivato prima.

\subsection{Recettori
somatosensoriali}\label{recettori-somatosensoriali}

Il sistema somatosensoriale risponde ad una varietà di stimoli
provenienri da molte aree del corpo e, pertanto, utilizza molti tipi di
recettori.

Le sensazioni somestesiche relative a stimoli associati con la
superficie del corpo sono dovute alla presenza di:

\begin{itemize}
\itemsep1pt\parskip0pt\parsep0pt
\item
  \textbf{meccanocettori}, in grado di rilevare stimoli pressori o
  vibrazioni;
\item
  \textbf{termocettori}, in grado di rilevare variazioni di temperatura;
\item
  \textbf{nocicettori}, in grado di rilevare stimoli dannosi per i
  tessuti.
\end{itemize}

\subsubsection{I meccanocettori cutanei}\label{i-meccanocettori-cutanei}

Nella cute c'è un complesso di meccanocettori che vengono attivati da
stimoli pressori. Si parla di sensibilità tattile dovuta a
meccanocettori.

Questi recettori possiedono \emph{fibre A-beta mielinizzate}, le più
grosse del sistema sensoriale.

Le cellule recettoriali sono tutte coincidenti con le terminazioni
neuronali. Esse sono associate a conformazioni caratteristiche della
cute e quindi è stata data tutta una serie di nomi diversi a queste
terminazioni. Ad esempio i \emph{corpuscoli del pacini} sono
terminazioni nervose ricoperte da strati concentrici di connettivo.
Queste lamelle connettivali inducono bella terminazione nervosa
l'apertura dei canali del Na\(^+\) stimolati da azione meccanica.
L'apertura di questi canali induce un potenziale recettoriale, che a sua
volta induce un potenziale d'azione voltaggio-dipendente a livello del
primo nodo di Ranvier (si tratta di fibre mielinizzate).

Questi recettori si trovano nella porzione profonda della cute, con
campi recettoriali ampi (se situati in superficie i campi recettoriali
sono piccoli). Sono recettori fasici a rapido adattamento.

\subsubsection{I termocettori}\label{i-termocettori}

I termocettori reagiscono a stimoli di tipo termico.

Questi recettori sono caratterizzati da \emph{fibre A-delta
mielinizzate} (più piccole delle A-beta) e da \emph{fibre non
mielinizzate di tipo C}.

Ci sono due tipi di termorecettori: quelli per il caldo e quelli per il
freddo.

I \textbf{recettori per il caldo} sono costituiti da terminazioni
nervose libere che rispondono a temperature tra i 30°C e i 45°C; la
frequenza di scarica dei potenziali d'azione aumenta all'aumentare della
temperatura fino a 45°C.

I \textbf{recettori per il freddo} rispondono a variazioni di
temperatura comprese tra i 35°C e i 20°C. La frequenza dei potenziali
dàazione incrementaal diminuire della temperatura ed è massima per
temperature di circa 25°C. I recettori per il freddo rispondono anche a
temperature superiori a 45°C, uno stimolo \emph{``dolorosamente
caldo''}, con la frequenza dei potenziali d'azione che aumenta
all'aumentare della temperatura.

I termorecettori sono terminazioni nervose libere dotate di canali
ionici sensibili alla temperatura.

Sono recettori fasici ed esistono unità sensoriali per il caldo e il
freddo disposte in regioni (miroscopiche) diverse.

\subsubsection{I nocicettori}\label{i-nocicettori}

I nocicettori sono recettori sensoriali responsabili della trasduzione
di \emph{stimoli nocivi} percepiti dal cervello come dolore.

I nocicettori sono costituiti da terminazioni nervose libere che
rispondono a stimoli che inducono danno tissutale (o potenzialmente
dannosi).

Vi sono 3 tipi di nocicettori:

\begin{enumerate}
\def\labelenumi{\arabic{enumi}.}
\itemsep1pt\parskip0pt\parsep0pt
\item
  \textbf{nocicettori meccanici}, che rispondono a stimoli meccanici
  intensi;
\item
  \textbf{nocicettori termici}, che rispondono a temperature elevate
  (\textgreater{} 44°C);
\item
  \textbf{nocicettori polimodali} che rispondono a molti stimoli,
  compresi quelli meccanici e termici, e a sostanze rilasciate dai
  tessuti danneggiati (istamina, prostaglandina, serotonina e
  bradichina).
\end{enumerate}

Questi recettori presentano \emph{fibre A-delta} e \emph{C}, sono
terminazioni nervose libere.

I recettori del dolore sono caratteristici perché hanno \emph{recettori}
\textbf{TRPV1} (un recettore \emph{vanilloide}) che rispondono a calore
eccessivo o sostanze chimiche come la capsaicina (molecola tipica del
peperoncino che abbassa la soglia termica di attivazione del recettore,
per questo provoca una sensazione di calore/bruciore).

Il mentolo invece fa la stessa cosa ma su recettori che rispondono al
freddo. I recettori sono anche nella bocca e nelle prime vie aeree e per
questo le sensazioni si registrano nel cavo orale.

\subsubsection{Corteccia
somatosensoriale}\label{corteccia-somatosensoriale}

La percezione delle sensazioni somatiche provenienti da tutte le parti
del corpo inizia nella corteccia somatosensoriale primaria. Questa è
organizzata in maniera topografica, nel senso che le informazioni
provenienti da aree del corpo vicine sono generalmente proiettate verso
aree corticali contigue.

La corteccia cerebrale ha un'organizzazione colonnare. Essa è sudduvisa
in strati e si individuano delle colonne poiché regioni di superficie
della corteccia corrispondono a regioni di superficie del corpo.

Per ogni area topografica arrivano stimoli diversi, e si hanno colonne
diverse per le varie attività sensoriali.

Vi sono due vie principali che trasmettono informaizoni somatosensoriali
dai recettori periferici al SNC: la \emph{via delle colonne
dorsali-lemnisco mediale} ed il \emph{tratto spinotalamico}.

Queste due vie di trasmissione inviano differenti tipi di informazioni
sensoriali al talamo ed alla corteccia somatosensoriale.

In entrambi i casi, queste due vie afferenti, penetrate nel midollo
spinale, incrociano prima di raggiungere il talamo.

La \textbf{via delle colonne dorsali-melnisco mediale} trasmette
infromazioni da meccanocettori e propiocettori al talamo.

In questa via, i neuroni di primo ordine originano nella periferia e
penetrano nelle corna dorsali del midollo spinale. Mentre collaterali
dell'assone possono terminare nel midollo spinale e stabilire sinapsi
con interneuroni coinvolti nei riflessi spinali, la diramazione
principale dell'assone ascende ipsilateralmente (dallo stesso lato dello
stimolo) nel midollo spinale verso il tronco encefalico, percorrendo le
\textbf{colonne dorsali}. Queste rappresentano dei tratti di sostanza
bianca che passano medialmente e dorsalmente alle corna dorsali del
midollo spinale. I neuroni di primo ordine terminano nei \textbf{nuclei
delle colonne dorsali}, che sono localizzati nel bulbo, dove formano
sinapsi con neuroni di secondo ordine.

Gli assoni dei neuroni di secondo ordine incrociano e passano nel lato
opposto del bulbo, formando un tratto chiamato \emph{lemnisco-mediale},
e quindi ascendendo verso il talamo.

Nel talamo i neuroni di secondo ordine stabiliscono sinapsi con neuroni
di terzo ordine che ritrasmettono le informazioni dal talamo alla
corteccia somatosensoriale.

Lo stimolo arriva nell'emisfero cerebrale dal lato opposto del corpo
rispetto al lato da cui è partito.

Il percorso che compie è:

midollo spinale \(\rightarrow\) midollo allungato \(\rightarrow\) talamo
\(\rightarrow\) corteccia.

Il \textbf{tratto spinotalamico} trasmette informazioni provenienti dai
termocettori e dai nocicettori al talamo.

Il tratto spinotalamico incrocia già nel midollo spinale, prima di
raggiungere il tronco encefalico.

in questa via, i neuroni di primo ordine che originano a livello
periferico a partire da termocettori o nocicettori penetrano nel corno
dorsale del midollo spinale. A tale livello, i neuroni di primo ordine
possono dirigersi verso l'alto o il basso per una brave distanza lungo
il \emph{tratto di Lissauer}, ma alla fine formano sinapsi con neuroni
di secondo ordine presenti nel corno dorsale.

Sono proprio questi neuroni di secondo ordine che, attraversando il
midollo spinale controlateralmente, ascendono nel quadrante
anterolaterale del midollo spinale verso il tronco encefalico e
terminano nel talamo.

Una volta nel talamo formano sinapsi con neuroni di terzo ordine che
terminano nella corteccia somatosensoriale.

I due stimoli (dorale e spinotalamico) arrivano nella stessa porzione
della corteccia ma in colonne differenti pecorrendo diverse vie.

\subsubsection{La percezine del dolore}\label{la-percezine-del-dolore}

Esistono due tipi di dolore, quello rapido e quello lento, che sono
trasmessi da differenti classi di neuroni afferenti.

Il \textbf{dolore rapido} è percepito come una netta sensazione di
puntura facilmente localizzabile; questo tipo di dolore è trasmesso da
\emph{fibre A\(\delta\)}, mieliniche e di piccolo diametro.

Il \textbf{dolore lento} è percepito in modo poco localizzato, dando
origine ad una sensazione che insorge lentamente; esso è trasmesso da
\emph{fibre C}, amieliniche e di piccolo diametro.

Entrambi questi tipi di fibre formano sinapsi con neuroni di secondo
ordine nel cordo dorsale del midollo spinale. La trasmissione sinaptica
a tale livello è resa possibile da differenti neurotrasmettitori. Tale
sostanza, rilasciata dai neuroni afferenti primari, si lega a recettori
sui neuroni di secondo ordine. Questi ultimi ascendono verso il talamo,
mediante il tratto spinotalamico.

La percezione del dolore non è limitata alla superficie corporea, esiste
anche un \emph{dolore viscerale}.

L'attivazione dei recettori viscerali dà origine ad un dolore chiamato
\textbf{dolore riferito} (poichè è stato riferito alla superficie
corporea).

Il dolore riferito è dovuto al fatto che i neuroni di secondo ordine che
ricevono impulsi da afferenze viscerali ricevono anche afferenze
somatiche.

La sensibilità nocicettiva ha un pesante influsso su tutte le attività
cerebrali e si ripercuote sui nuclei profondi, sull'amigdala e
sull'ipotalamo portando reazioni emotive ed endocrine rispettivamente.

I segnali riguardanti informazioni sensoriali possono essere modulati
durante la loro trasmissione lungo le vie sensoriali, attraverso
facilitazione o attenuazione di segnali che possono portare a
cambiamenti nella percezione finale dell'informazione. I segnali
sensoriali possono essere modulati in qualsiasi punto della via in cui
ci sia una sinapsi.

Secondo la \textbf{teoria del controllo a cancello} la percezione del
dolore può essere inibita a livello spinale attraverso afferenze
somatiche non dolorifiche. Questa teoria postula l'esistenza di
un'inibizione sinaptica ad opera di interneuroni spinali sui neuroni di
secondo ordine che trasportano le informazioni dolorifiche.

È come se ci fossero dei cancelli che tengono chiuse le vie del dolore e
che necessitano di un'apertura affinché lo stimolo attraversi il
cancello e percorra la via. I cancelli consistono di interneuroni
inibitori che inibiscono il neurone di secondo ordine impedendo la sua
stimolazione. L'inibizione deve essere vinta affinché il neurone di
secondo ordine venga stimolato.

I cancelli possono essere anche operati da vie che provengono dal
sistema nervoso centrale. Le endorfine, ad esempio, agiscono in questo
modo.

Durante lo stress fisico sono attive vie discendenti che raggiungono il
midollo spinale provenendo dalla corteccia cerebrale e attraversando la
zona periacquiduttale (attorno all'acquedotto di silvio) e portano
stimoli che raggiungono le giunzioni sinaptiche tra i neuroni di primo e
secondo ordine e attivano neuroni inibitori che rilasciano encefalina
(un oppioide endogeno) e diminuiscono le sensazioni dolorifiche.

Il neurotrasmettitore inibisce la sinapsi inibendo la sensazione di
dolore o riducendo la scarica della sinapsi stessa. Questo succede
quando il corpo è sotto stress. È un meccanismo sfruttato
farmacologicamente. La morfina del papavero da oppio fa lo stesso lavoro
ed è dunque usato come potente antidolorifico.

\section{La vista}\label{la-vista}

La vista è dovuta alla percezione di stimoli luminosi, e consiste nella
ricostruzione di tutti i punti luminosi che provengono dal campo visivo
con la stessa disposizione con cui raggiungono il sistema recettoriale.

L'organo in grado di ricostruire l'immagine è l'occhio.

\subsection{Anatomia dell'occhio}\label{anatomia-dellocchio}

L'occhio può essere diviso in 3 strati concentrici:

\begin{itemize}
\itemsep1pt\parskip0pt\parsep0pt
\item
  quello più esterno, formato da \textbf{sclera} e \textbf{cornea}. La
  \emph{sclera} è un tessuto connettivo consistente (forma la parte
  bianca dell'occhio). Nella parte anteriore la sclera dà origine alla
  \emph{cornea}, una struttura trasparente che consente alla lune di
  penetrare nell'occhio;
\item
  lo strato medio è costituito dalla \textbf{coroide}, dal \textbf{corpo
  ciliare} e dall'\textbf{iride}.

  \begin{itemize}
  \itemsep1pt\parskip0pt\parsep0pt
  \item
    La \emph{coroide} è uno strato di tessuto altamente pigmentato posto
    al di sotto della sclera, include i fotorecettori e vasi ematici che
    nutrono lo strato profondo dell'occhio;
  \item
    Il \emph{corpo ciliare} contiene \textbf{muscoli ciliari}, che sono
    attaccati ad una lente, il \emph{cristallino}, attraverso dei
    sottili tendini di tessuto connettivo chiamati \textbf{fibre
    zonulari}. Il \textbf{cristallino} focalizza la luce sulla
    \textbf{retina}, dove l'informazione visiva viene trasdotta. I
    \emph{muscoli ciliari} cambiano la forma del cristallino,
    permettendo la focalizzazione dei raggi luminosi.
  \item
    L'\emph{iride}, che è formata da due strati di cellule muscolari
    pigmentate, è localizzata davanti al cristallino e determina il
    colore degli occhi. La \textbf{pupilla} è un foro, posizionati
    alcentro dell'iride, che permette alla luce di penetrare nella parte
    posteriore dell'occhio (non è una struttura). L'iride regola il
    diametro della pupilla, variando in tal modo la quantità di luce che
    raggiunge la parte posteriore dell'occhio.
  \end{itemize}
\item
  lo strato più interno è rappresentato dalla \textbf{retina}, che è
  formata da tessuto nervoso contenente i fotorecettori (cellule
  sensibili ale onde luminose). I \emph{fotorecettori} sono di due tipi,
  i \textbf{coni} e i \textbf{bastoncelli}, che percepiscono
  rispettivamente la luce intensa e quella soffusa. La retina funzona
  come un \emph{fototrasduttore}, trasformando l'energia luminosa in
  energia elettrica. Nella parte esterna della retina e attaccato alla
  coroide si trova l'\textbf{epitelio pigmentato della retina}. Questa
  struttura ha un'alta concentrazione del pigmento \emph{nero melanina},
  che assorbe la luce che arriva alla parte posteriore dell'occhio,
  impedendo così la riflessione attraverso la retina e la distorsione
  dell'immagine. Due aree della retina sono molto importanti:

  \begin{itemize}
  \itemsep1pt\parskip0pt\parsep0pt
  \item
    la \textbf{fovea}, che rappresenta il punto centrale della retina,
    dove si dirige la luce proveniente dal centro del campo visivo. È
    l'area della retina con la maggiore acuità visiva;
  \item
    il \textbf{disco ottivo}, cioè la porzione della retina attraversata
    dal nervo ottivo e dai vasi ematici che irrorano l'occhio. Poichè
    questa regione è sprovvista di fotorecettori, essa costituisce un
    \textbf{punto cieco} della retina, dove la luce non può generare
    impulsi elettrici e quindi essere percepita.
  \end{itemize}
\end{itemize}

(immagine 10.19 p271)

Il cristallino e il corpo ciliare suddividono l'occhio in due camere
piene di liquido:

\begin{itemize}
\itemsep1pt\parskip0pt\parsep0pt
\item
  davanti al cristallino e al corpo ciliare si trova il \textbf{segmento
  anteriore}. Questo contiene un liquido limpido e acquoso, definito
  \textbf{umor acqueo}, che fornisce nutrienti alla cornea e al
  cristallino. Poichè la cornea e il cristallino sono strutture
  trasparenti che devono essere facilemnte attraversate dalla luce, se
  dipendessero dall'apporto ematico di nutrienti, la presenza dei vasi
  ostruirebbe parzialmente il passaggio della luce. Il segmento
  anteriore è a sua volta diviso in:

  \begin{itemize}
  \itemsep1pt\parskip0pt\parsep0pt
  \item
    \textbf{camera anteriore} (tra cornea e iride);
  \item
    \textbf{camera posteriore} (tra l'iride e il cristallino).
  \end{itemize}
\item
  posterioremente al cristallino e al corpo ciliare vi è una camera
  trasparente (\textbf{camera vitrea} o \textbf{segmento posteriore})
  contenente una sotanza gelatinosa, definita \textbf{umor vitreo}, che
  contribuisce a mantenere la struttura sferica dell'occhio.
\end{itemize}

L'occhi è un sistema sensibile a fotoni compresi tra i 350 e 750 nm
perché a queste lunghezze le particelle si comportano come onde.

La luce possiede tutte le caratteristiche delle onde e può quandi essere
riflessa e rifratta. La \textbf{riflessione} è un fenomeno per il quale
le onde luminose urtano e rimbalzano su una superficie. La
\textbf{rifrazione} rappresenta il fenomeno per il quale le onde
luminose cambiano direzione nel passare attraverso materiali trasparenti
di densità differenti

Quando i raggi luminosi sono perpendicolari alla superficie da
attraversare, non modificano la propria direzione; tuttavia, s ei raggi
attraversano le superfici con angolazioni diverse da quella
perpendicolare, le lenti concave li fanno divergere, mentre quelle
convesse li fanno convergere verso un punto definito \emph{punto
focale}. La distanza tra l'asse maggiore della lente convessa ed il
punto focale è definita \emph{distanza focale}.

Sia la cornea che il cristallino hanno superfici convesse, che
funzionano facendo convergere le onde luminose che penetrano nell'occhio
a livello retinico; in tal modo, l'immagine che si forma nella retina è
a fuoco.

Per vedere l'immagine a fuoco, la luce proveniente da un determinato
punto del \emph{campo visivo} deve convergere in un singolo punto della
retina. Sebbena la cornea abbia un potere di rifrazione maggiore del
cristallino, a causa del maggior raggio di curvatura, il potere di
rifrazione del cristallino può essere variato per permettere la
focalizzazione della luce sulla retina. La capacità del cristallino di
modificare il suo potere di rifrazione nella visione da vicino e da
lontano è definita \textbf{accomodazione}.

La rifrazione dei raggi luminosi quando passano attraverso la cornea e
il cristallino fa sì che l'immagine venga proiettata sulla retina
invertita e capovolta.

Il punto in cui si focalizza l'oggetto dipende anche dalla distanza
dell'oggetto dalla lente. L'occhio non riesce a mettere a fuoco
qualsiasi distanza, ma solo gli oggetti che si trovano entro una certa
distanza; a lunghezze minori si ha un'immagine sfuocata.

L'occhio umano mette a fuoco solo un piano del campo visivo, però può
variare la distanza della messa a fuoco e mettere a fuoco tutti i piani
del campo visivo che sta osservando ma non contemporaneamente. Questa
funzione è chiamata accomodazione.

La forma del cristallino è controllata dal muscolo ciliare, che presenta
fibre disposte concentricamente, mediante la tensione che esos esercita
sulle fibre zonulari che collegano il muscolo ciliare al cristallino.

Maggiore è la concentrazione di un muscolo circolare e minore sarà il
diametro interno del cercio (muscolo ciliare), cui corrisponderà una
minore tensione delle fibre zonulari e una maggiore curvatura del
cristallino.

Per la visione di oggetti distanti il muscolo ciliare è rilasciato, il
che aumenta il diametro del muscolo stesso, tende le fibre zonulari e
riduce la curvatura del cristallino, in modo tale che questo assuma una
forma più schiacciata (minore convessità).

L'accomodazione è sotto il controllo del sistema nervoso parasimpatico,
che attiva la contrazione del muscolo ciliare per la visione da vicino.
In assenza di attività parasimpatica, il muscolo ciliare si rilascia.

Se le onde non sono adeguatamente focalizzate sulla retina, la visione è
distorta.

I difetti visivi più comuni sono:

\begin{itemize}
\itemsep1pt\parskip0pt\parsep0pt
\item
  \textbf{miopia}. La persona può vedere chiaramente gli oggetti vicini,
  ma non quelli distanti, in quanto il cristallino o la cornea
  rifrangono in maniera eccessiva i raggi luminosi; per tale motivo, gli
  oggetti vicini all'occhio possono essere messi a fuoco senza
  accomodazione, ma quelli posti a distanza vengono focalizzati davanti
  alla retina, con conseguente distorsione dell'immagine;
\item
  \textbf{ipermetropia}. Il cristallino o la cornea sono inadeguati in
  relazione alla lunghezza del bulbo oculare; pertanto, gli oggetti a
  distanza possono essere focalizzati sulla retina solo mediante
  accomodazione, il che significa che il cristallino non riesce a
  ottimizzare l'accomodazione in maniera sufficiente nella visione da
  vicino. La luce porveniente da un oggetto vicino all'occhio viene così
  messa a fuoco oltre la retina, generando una distorsione
  dell'immagine;
\item
  \textbf{astigmatismo}. Le irregolarità della superficie della cornea o
  del cristallino alterano la direzione delle onde luminose;
\item
  \textbf{presiopia}. È un indurimento del cristallino che si verifica
  con il passare degli anni. Questo causa una perdita di elasticità del
  cristallino, che riduce la sua capacità di diventare sferico e rende
  difficile l'accomodazione per la visione da vicino;
\item
  \textbf{cataratta}. È un'altra alterazione clinica correlata all'età,
  che provoca un'opacizzazione del cristallino e ne riduce la
  trasparenza;
\item
  \textbf{glaucoma}. È un aumento del volume dell'umor acqueo che
  determina un incremento della pressione nella cavità anteriore del
  bulbo oculare, alterando la forma della cornea e modificando la
  posizione del cristallino. Il cambiamento di posizione del cristallino
  può trasmettere un'aumentata pressione al corpo vitreo, comprimendo i
  vasi ematici che irrorano la retina e generando cecità permanente.
\end{itemize}

Nella \textbf{emmetriopia} o visione normale, invece, una persona vede
bene sia oggetti lontani che vicini.

Gli occhi sono capaci di regolare il quantitativo di luce che penetra in
essi variando il diametro delle pupille.

Nella luce intensa, le pupille sono di diametro ridotto, o
\emph{costrette}, in modo tale che i fotorecettori non vengano
``accecati'' dalla luce troppo intensa. In presenza di luce fioca, al
contrario, le pupille sono \emph{dilatate}, in modo tale da permettere
un maggior passaggio di luce; la dimensione della pupilla è controllata
dall'iride.

L'iride è formata da due trati di cellule muscolari lisce che circondano
la pupilla:

\begin{itemize}
\itemsep1pt\parskip0pt\parsep0pt
\item
  uno strato interno di \textbf{muscolatura circolare}, detto
  \emph{muscolo costrittore}. I muscoli circolari formano cerchi
  concentrici attorno alla pupilla e, quando si contraggono, il diametro
  della pupilla diminuisce;
\item
  uno strato esterno di \textbf{muscolatura radiale}, detto
  \emph{muscolo dilatatore}. Questi muscoli sono organizzati a raggio e,
  quando si contraggono, il diametro della pupilla aumenta.
\end{itemize}

L'iride è sotto controllo del sistema nervoso autonomo. I neuroni
parasimpatici innervano lo strato di cellule muscolari circolari, mentre
i neuroni simpatici innervano le cellule muscolari radiali.

\subsection{La retina}\label{la-retina}

Nella retina, che è costituita da tessuto nervoso, sono localizzati i
fotorecettori: coni e bastoncelli.

Sulla retina i fotorecettori sono condensati principalmente nella fovea.

I \textbf{bastoncelli} permettono la visione in bianco e nero in
condizioni di luce poco intensa o crepuscolare. I \textbf{coni}
forniscono la visione a colori, ma sono attivi soltanto quando la luce è
intensa (visione diurna).

La retina consta di 3 strati distinti:

\begin{enumerate}
\def\labelenumi{\arabic{enumi}.}
\itemsep1pt\parskip0pt\parsep0pt
\item
  uno strato interno formato da \textbf{cellule gangliari}, che inviano
  il loro assone nel nervo ottico;
\item
  uno strato intermedio formato da \textbf{cellule bipolari}, che unisce
  i recettori e le cellule gangliari;
\item
  uno strato esterno contenente i \textbf{fotoreccetori}, coni e
  bastoncelli.
\end{enumerate}

(immagine 10.29 p277)

Nella retina sono presenti anche altre cellule:

\begin{itemize}
\itemsep1pt\parskip0pt\parsep0pt
\item
  le \textbf{cellule orizzontali}, che presentano protuberanze e sono
  disposte orizzontalmente rispetto alle altre, unendo più fotorecettori
  e cellule bipolari tra loro;
\item
  le \textbf{cellule amacrine}, disposte tra le varie cellule gangliari.
\end{itemize}

Tutto quanto è circondato da uno strato fortemente pigmentato che
impedisce l'entrata di luce lateralmente.

I coni e i bastoncelli, essendo posizionati nello strato esterno della
retina, vengono eccitati dalla luce dopo che questa ha attraversato gli
strati retinici interno e medio. Inoltre, i vasi ematici che perfondono
la retina si trovano lungo il percorso dei raggi luminosi, per cui, al
fine di migliorare la trasmissione della luce alla fovea, le cellule
bipolari e quelle gangliari sono disposte lateralmente al centro della
retina. Si crea così una depressione al centro della retina, definita
\textbf{macula lutea}, che circonda la fovea ed è gialla per la presenza
di carotenoidi.

La fovea contiene solo coni; il rapporto tra bastoncelli e coni aumenta
all'aumentare della distanza dalla fovea, fino alla parte periferica
della retina, dove sono presenti solo bastoncelli.

La fovea è la regione centrale di massima acuità visiva. La direzione
del campo visivo corrisponde alla localizzazione della fovea sulla
retina. Il punto in cui si vede meglio del campo visivo è quello verso
cui viene focalizzato lo sguardo.

\subsection{Fototrasduzione}\label{fototrasduzione}

La fototrasduzione rappresenta il fenomeno mediante il quale l'energia
luminosa viene convertita in segnali elettrici. Questo fenomeno si
realizza nei coni e nei bastoncelli.

L'aspetto morfologico di questi due fotorecettori è simile, in quanto
ciascuno è formato da due parti rilevanti, definite \emph{segmento
interno} e \emph{segmento esterno}.

Il segmento esterno contiene invaginazioni della membrana che formano
strati simili a dischi membranosi, contenenti molecole che, assorbendo
l'energia luminosa, permettono ai fotorecettori di eccitarsi. Il
segmento interno contiene il nucleo cellulare e vari organuli; esso
termina con un bottone sinaptico dove sono presenti le vescicole
contenenti il neurotrasmettitore.

L'assorbimento della luce da parte di molecole chiamate
\textbf{fotopigmenti}, contenute nel segmento esterno, rappresenta il
primo evento della fototrasduzione. Nei recettori sono presenti 4 tipi
differenti di fotopigmenti. Ciascuna molecola di fotopigmento contiene
un componente chiamato \textbf{retinale} ed una proteina chiamata
\textbf{opsina}.

Il retinale è comune a tutti i fotopigmenti, mentre il tipo di opsina
presente determina quali lunghezze d'onda sono assorbite da un
determinato pigmento.

L'attivazione di tutte queste varie opsine permette all'occhio la
visione dei colori.

La \textbf{rodopsina} è una proteina di membrana con 7 domini
transmembrana a \(\alpha\)-elica che si trova principalmente nelle
cellule a bastoncello della retina umana che permettono la vista in
bianco e nero.

Queste cellule hanno una forma allungata e nella loro parte apicale
hanno numerosi dischi di membrana con molte rodopsine, costituite da un
pigmento, l'\emph{11-cis-retinale}, sensibile alla luce, legato
all'\emph{opsina}, una proteina della retina.

Il retinale è legato ad un glutammato dell'opsina forma una base di
Schiff, cioè il pigmento rodopsina.

Quando la rodopsina viene a contatto con un fotone di luce, subisce una
fotodecomposizione, o imbianchimento, che porta alla dissociazione della
molecola con formazione di \textbf{retinale tutto-trans}.

L'opsina varia la sua conformazione e diventa \textbf{metarodopsina II},
mentre il retinale tutto trans si stacca finendo nel citosl e uscendo
dalla cellula dove viene captato dall'epitelio pigmentato.

La metarodopsina II non è più sensibile alla luce e viene chiamata
\textbf{opsina scolorita}. Il fotorecettore è meno sensibile alla luce.
Può essere ricostituita la sensibilità rifornendo 11 cis-retinale per
ricostruire il complesso fotosensibile 11-cis-opsina.

\begin{center}\rule{0.5\linewidth}{\linethickness}\end{center}

\textbf{Registrazione}

Fototrasduzione ..

Al buio non arriva luce e \ldots{}

Al buio il forecettore diventa sinapticamente attivo mentre alla luce è
inattivo.

Al buio stimola la cellula bipolare alla luce smette di farlo.
\_\_\_\_\_\_\_\_\_\_\_\_\_\_\_\_\_\_\_\_\_\_\_\_\_\_\_\_\_\_\_\_\_\_\_\_\_\_\_\_\_\_\_\_\_\_\_\_\_\_\_\_\_\_\_\_\_\_\_\_\_\_\_\_\_\_\_\_\_\_\_\_\_\_\_\_\_\_\_\_\_\_\_\_\_\_\_\_\_\_\_\_\_\_\_\_\_\_\_\_\_\_\_\_\_\_\_\_\_\_\_\_\_\_\_\_\_\_\_\_\_\_\_\_\_\_\_\_\_\_\_\_\_\_\_\_\_\_\_\_\_\_\_\_\_\_

Bastoncelli e coni si comportano in maniera diversa. I bastoncelli sono
specializzati per luce debole e sono molto numerosi nella retina. Questi
non sono in grado di percepire il colore perché assorbono solo una
lunghezza d'onda.

Sono in grado di rispondere anche ad un singolo fotone e se sottoposti a
luce intensa sono completamente scoloriti.

In un ambiente poco illuminato i bastoncelli funzionano e i coni no.
Passando ad un ambiente illuminato i bastoncelli sono tutti in funzione
e vengono stimolati intensamente dalla grande quantità di luce che
arriva e il nervo ottico molto stimolato dà la sensazione di abbaglio ma
gradualmente i bastoncelli si scoloriscono e si inizia a vedere
attraverso i coni che meno sensibili alla luce e danno meno l'effetto di
abbagliamento.

La retina è un sistema nervoso con un'organizzazione relativamente
semplice da descrivere ma abbastanza complessa a livello di interazioni
tra neuroni.

Si è capito come si ricostruisce l'immagine del campo visivo e la
determinazione del contrasto.

La retina compie una prima rielaborazione delle informazioni provenienti
dal campo visivo in quanto è un sistema nervoso. Migliaia di batoncelli
convergono su una singola cellula bipolare, mentre di coni ve ne sono
meno.

Anche qui dei abbiamo campi recettoriali che corrispondono a zone della
retina. L'acuità visiva dipende dal fatto di distinguere come separati
due diversi stimoli che agiscono contemporaneamente sulla retina, quindi
percepire due sorgenti luminose come distinte.

Nel caso dei bastoncelli siccome si hanno molte cellule recettoriali su
una cellula nervosa singola si avrà minore acuità, mentre i coni
consentono di vedere con un acuità visiva maggiore.

\end{document}
