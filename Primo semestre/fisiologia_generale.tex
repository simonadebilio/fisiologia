\documentclass[]{article}
\usepackage{lmodern}
\usepackage{amssymb,amsmath}
\usepackage{ifxetex,ifluatex}
\usepackage{fixltx2e} % provides \textsubscript
\ifnum 0\ifxetex 1\fi\ifluatex 1\fi=0 % if pdftex
  \usepackage[T1]{fontenc}
  \usepackage[utf8]{inputenc}
\else % if luatex or xelatex
  \ifxetex
    \usepackage{mathspec}
    \usepackage{xltxtra,xunicode}
  \else
    \usepackage{fontspec}
  \fi
  \defaultfontfeatures{Mapping=tex-text,Scale=MatchLowercase}
  \newcommand{\euro}{€}
\fi
% use upquote if available, for straight quotes in verbatim environments
\IfFileExists{upquote.sty}{\usepackage{upquote}}{}
% use microtype if available
\IfFileExists{microtype.sty}{%
\usepackage{microtype}
\UseMicrotypeSet[protrusion]{basicmath} % disable protrusion for tt fonts
}{}
\ifxetex
  \usepackage[setpagesize=false, % page size defined by xetex
              unicode=false, % unicode breaks when used with xetex
              xetex]{hyperref}
\else
  \usepackage[unicode=true]{hyperref}
\fi
\hypersetup{breaklinks=true,
            bookmarks=true,
            pdfauthor={},
            pdftitle={},
            colorlinks=true,
            citecolor=blue,
            urlcolor=blue,
            linkcolor=magenta,
            pdfborder={0 0 0}}
\urlstyle{same}  % don't use monospace font for urls
\setlength{\parindent}{0pt}
\setlength{\parskip}{6pt plus 2pt minus 1pt}
\setlength{\emergencystretch}{3em}  % prevent overfull lines
\setcounter{secnumdepth}{0}

\date{}

\begin{document}

\section{Fisiologia generale}\label{fisiologia-generale}

Che cos'è la fisiologia? È una disciplina che studia il funzionamento
degli esseri viventi e consiste nella definizione meccanicistica di ciò
che accade nell'organismo di ogni persona. Questa disciplina ti fa
capire che non sei quel che pensi di essere.

La materia vivente è principalmente composta di: H, O, C, N. La
composizione chimica della materia vivente è molto piu' simile a quella
dell'universo e delle stelle, piuttosto che a quella della terra e della
crosta terrestre.

Tra la fine del '700 e l'inizio del '800, dopo la diffusione del
microscopio, ci si rese conto che gli esseri viventi sono organizzati in
cellule. La materia vivente ha una sua unitarietà molto solida, ovvero
tutte le cellule contengono all'incirca gli stessi organelli e nelle
stesse quantità. L'acqua è sempre la sostanza più abbondante all'interno
della cellula e le proteine sono la classe di composti maggiormente
diversificata nell'organismo.

\subsection{L'importanza dell'acqua}\label{limportanza-dellacqua}

L'acqua non è un inerte riempitivo delle strutture organiche, ma le sue
molecole sono molto reattive. Le fondamentali caratteristiche dell'acqua
sono:

\begin{itemize}
\itemsep1pt\parskip0pt\parsep0pt
\item
  possiede una capacità solvente* molto elevata;
\item
  ha un'elevata \emph{capacità termica} ed un \emph{elevato calore
  latente di evaporazione};
\item
  presenta un'\emph{elevata tensione superficiale} che facilita i
  fenomeni di capillarità;
\item
  promuove la formaizone di \emph{legami idrofobici} tra le molecole non
  idrosolubili che vi si trovano immerse.
\end{itemize}

Tutte queste proprietà sono presenti quando l'acqua, alla pressione
atmosferica, si trova allo stato liquido.

Nell'H\(_2\)O, un atomo di O e due atomi di H sono legati da legami
covalenti polari, ovvero: la densità della carica elettrica nell'intorno
dei tre nuclei atomici non è uniforme poichè l'atomo di O contiene un
numero maggiore di protoni di quello dei due H, per cui esso attrae gli
elettroni di legame più fortemente. Questo fa sì che la molecola di
H\(_2\)O, sebbene sia neutra, presenti una distribuzione asimmetrica
delle cariche comportandosi come un \emph{dipolo} che tende ad
orientarsi quando si trova in un campo elettrico.

Questa proprietà conferisce all'acqua la capacità di funzionare da
\emph{solvente} soprattutto per quei composti che nell'acqua si
dissociano in ioni (\emph{elettroliti}).

Cosa succede quando poniamo un elettrolita in acqua? Quando un sale come
NaCl viene posto in acqua, i dipoli idrici subiscono una forte
attrazione elettrostatica da parte degli ioni si positivi che negativi e
vengono spinti a penetrare nella struttura cristallina del sale; essi si
disporranno attorno a ciascuno ione in modo da formare un involucro
detto \textbf{alone di solvatazione}, in cui i dipoli sono orientati con
polarità opposta a della quello ione.

L'acqua è un ottimo solvengte non solo per gli elettroliti, ma anche per
un'ampia gamma di sostanze organiche \emph{polari}. Queste molecole sono
solubili in acqua e perciò dette \textbf{idrofiliche}.

L'acqua è invece un pessimo solvente per quei composti organici (es.
oli) le cui molecole sono \emph{non polari} e che perciò non attraggono
in alcun modo i dipoli idrici. Queste molecole sono insolubili in acqua
e perciò dette \textbf{idrofobiche}.

È poi possibile trovare composti organici le cui molecole possiedono,
disposti in regioni diverse, sia gruppi chimici dissociabili o polari
che gruppi apolari. Questi gruppi vengono detti \textbf{anfipatici}. Un
esempio di composto anfipatico sono i \emph{saponi}: se posti in acqua
costituiscono delle particelle submicroscopiche sferiche dette
\emph{micelle}. Nelle micelle le molecole anfipatiche sono disposte
ordinatamente in modo radiale con le code idrofobiche dirette verso il
centro della micelle e le teste idrofiliche verso l'esterno. Un
conmportamento analogo a quello dei saponi è presentato dai fosfolipidi.

\subsection{L'omeostasi}\label{lomeostasi}

Cosa significa omeostasi? Significa \emph{tendenza a mantenere costante
il sistema}. L'omeostasi è un insieme di processi che mantengono
costanti le condizioni corporee (es: costanza nella composizione chimica
del plasma), ed è garantita da meccanismi autoregolatori. Queste
condizioni devono mantenersi anche al variare delle condizioni esterne.

Si def9inisce \emph{omeostato} un insieme di strutture che permettono di
far sì che il comportamento di un sistema segua un andamento
prestabilito, determinato dalla grandezza del segnale applicato al suo
ingresso. I sistemi di controllo possono essere di due tipi:

\begin{enumerate}
\def\labelenumi{\arabic{enumi}.}
\itemsep1pt\parskip0pt\parsep0pt
\item
  ad \emph{anello aperto};
\item
  ad \emph{anello chiuso} o a \emph{retroazione}.
\end{enumerate}

Il \textbf{controllo ad anello aperto (feedforward)} è più semplice,
perchè la grandezza del sognale in ingresso è indipendente da quella in
uscita. Per applicare questo tipo di controllo occorre avere una buona
conoscenza della dinamica del sistema da controllare: quanto più è
esatta la rappresentazione interna del sistema, tanto più questo tipo di
controllo sarà affidabile. Il segnale in ingresso non entra direttamente
nel sistema da controllare ma nel sistema di controllo e solo
successivamente nel sistema da controllare.

(immagine p14A)

Il \textbf{controllo ad anello chiuso}, invece, funziona diversamente:
immaginiamo che un generico segnale, agendo sull'ingresso \textbf{(I)}
di un sistema, causi un effetto alla sua uscita \textbf{(O)}. Il
\textbf{feedback} consiste nel ``trasferimento all'indietro'' (su I) di
un segnale di retroazione \textbf{(Sr)} ricavato da O, in modo che esso
agisca sulla causa stessa che lo produce.

(immagine p14B)

A seconda che l'anello di retroazione sia ``invertente'' o ``non
invertente'' il segnale in uscita, esistono due tipi di retroazione:

\begin{enumerate}
\def\labelenumi{\arabic{enumi}.}
\itemsep1pt\parskip0pt\parsep0pt
\item
  \emph{negative feedback} quando Sr ha un segno opposto a quello di O;
\item
  \emph{positive feedback}, quando Sr ed O hanno lo stesso segno.
\end{enumerate}

Il \textbf{feedback positivo} è un meccanismo che in natura si presenta
solo in particolari occasioni poichè è \emph{destabilizzante} (allontana
qualunque sistema da un regime stazionario).

Nel \textbf{feedback negativo} o \textbf{controreazione} invece,
possiamo immaginare che il segnale in uscita dal sistema (O) rappresenti
il parametro che deve essere stabilizzato, e che un fattore perturbatore
(I) tenda ad innalzarlo dal suo valore normale. Aumenterà
conseguentemente il segnale di retroazione negativa il quale,
opponendosi ad I, tenderà a riportare O al suo valore normale. In questo
modo il sistema tende a \emph{mantenere stabile} la sua uscita.

Da un punto di vista grafico, supponendo che ci siano due fattori che
agiscono a vicenda con un \emph{feedback negativo} dove al crescere di
un fattore l'altro diminuisce, un fattore agirà come \textbf{y=sen(t)} e
l'altro \textbf{x=cos(t)}, dove \emph{t} è il tempo. Se posti in grafico
questi valori oscillano in maniera sfalsata. Se con gli stessi fattori
costruisco una funzione \textbf{y=f(x)} otterrò un cerchio che ci indica
che, postandoci nel tempo, vengono ripercorsi gli stessi punti. La
distanza di questa funzione dall'origine degli assi è sempre 1.

\subsubsection{Il principio fisiologico
fondamentale}\label{il-principio-fisiologico-fondamentale}

Cos'è un \textbf{agente biologico funzionale}? Un qualsiasi \emph{agente
biologico ereditario} capace di operare una \emph{trasformazione fisica}
(es. enzima). Per ogni agente biologico funzionale esiste almeno un
a.b.f. che lo controlla a monte ed esiste almeno un agente biologico
funzionale che viene controllato da questo a valle. Il numero degli
agenti biologici funzionali presenti in una cellula è finito. Questo
significa che per ogni a.b.f esiste almeno un loop dove l'ultimo a.b.f.
sarà in grado di agire sul primo a.b.f andando a chiudere il ciclo.

L'importanza dell'omeostasi, ovvero la capacità degli esseri viventi di
mantenere costanti i parametri del loro ambiente corporeo, non deve far
pensare che l'ampio spettro delle funzioni vitali non sia in grado di
subire delle piccole variazioni: l'omeostasi varia dunque sul lungo
termine (bisogna tenere conto anche dell'invecchiamento dell'organismo).
Gli organismi viventi si modificano in funzione delle informazioni che
ricevono dall'ambiente in cui vivono e queste modificazioni possono
persistere nel tempo. Questa proprietà prende il nome di
\emph{plasticità} e consiste nella capacità di \textbf{adattamento}
degli organismi: l'organismo può variare il proprio stato funzionale
passando da un iniziale stato di omeostasi ad un nuovo punto di
equilibrio. Un esempio è la preparazione atletica: l'organismo che viene
sottoposto ad esercizio fisico subisce uno stress e reagisce allo stress
con un rinforzo che può progredire mediante adattamento progressivo.
Durante il processo di adattamento avvengono fisicamente delle
trasformazioni nell'organismo che rappresentano il raggiungimento
progressivo di nuovi punti di omeostasi.

Lo schema degli agenti biologici funzionali è dunque plastico: non perde
mai le proprie caratteristiche di base, ma può essere modulato in
risposta a sollecitazioni. Questi cicli possono avere dei punti di
rottura, se eccessivamente sollecitati, che possono essere più o meno
mortali.

\subsection{La membrana cellulare}\label{la-membrana-cellulare}

Nella materia vivente ci sono due ambienti chimici completamenti diversi
e agli antipodi:

\begin{itemize}
\itemsep1pt\parskip0pt\parsep0pt
\item
  l'ambiente \textbf{idrofilo}, la sostanza fondamentale è l'acqua e
  contiene composti idrosolubili;
\item
  l'ambiente \textbf{idrofobo} o \emph{lipofilo}, la sostanza
  fondamentale è costituita da lipidi o grassi.
\end{itemize}

Nell'organismo esistono sostanze tensioattive che formano le membrane
cellulari. A questo gruppo appartengono i \textbf{fosfolipidi}: molecole
che presentano due catene di \emph{acidi grassi} (coda idrofoba)
esterificate su una molecola di \emph{glicerolo} al quale è
ulteriormente esterificato un \emph{gruppo fosfato} (testa idrofila).

Come si dispongono i fosfolipidi in acqua? Normalmente se poniamo dei
lipidi in soluzione acquosa questi formano delle micelle. I fosfolipidi
fanno al stessa cosa, ma possono anche organizzarsi in un \emph{doppio
strato} che risulta stabile in acqua andando a formare una struttura
chiamata \emph{membrana}. Queste membrane presentano due porzioni
idrofile esterne (una a contatto con il citosol e l'altra a contatto con
l'ambiene acquoso esterno alla cellula) e una porzione lipofila interna.
Le membrane fosfolipidiche tenderanno a chiudersi andando a formare
delle vescicole poiché altrimenti, se le membrane fossero piatte come
dei fogli, la porzione interna dei bordi rimarrebbe a contatto con
l'ambiente acquoso circostante.

Le membrane cellulari formano dunque un piccolo ambiente lipidico
all'interno del quale troviamo anche altre molecole: una delle più
importanti è il \emph{colesterolo}, un composto steroideo fortemente
apolare. A cosa serve la presenza del colesterolo? Per mantenere fluida
la membrana.

Cos'è che permette ad un composto grasso d'essere liquido o solido? La
presenza di legami doppi (grassi insaturi, liquidi) o singoli (grassi
saturi, solidi). Il doppio legame introduce un angolo di piegatura nella
catena idrocarburica diverso dagli altri facendo sì che l'attrazione
dovuta alle forze di Van der Waals tra le molecole sia ridotta rispetto
alla forza presente tra molecole lineari (solo legami singoli). Una
forte attrazione data dalle forze di VdW, dovuta alla prevalenze di
legami singoli, darà origine a sostanze solide (come il burro).

Il colesterolo, un grasso saturo, è necessario per regolare la fluidità
della membrana. Se la T è intorno ai 37° il colesterolo impedisce la
mobilità dei fosfolipidi, diminuendo la fluidità della membrana. A T più
basse invece, il colesterolo impedisce l'impaccamento dei fosfolipidi
che diminuirebbe eccessivamente la fluidità della membrana.

I fosfolipidi sono mobili nella membrana, possono ruotare lungo i loro
assi e possono scorrere lateralmente.

Per provare la fluidità delle membrane venne fatto un esperimento:
vennero fatte fondere cellule umane e cellule di topo. Dopo la fusione
vennero marcate e si vide che si erano distribuite lungo tutta la nuova
membrana cellulare, ovvero le membrane non erano rimaste separate ma si
erano fuse e dunque i materiali delle due membrane si erano distribuiti
lungo tutta la nuova membrana dimostrando la loro fluidità.

Nelle membrane cellulari possiamo trovare anche gli
\textbf{sfingolipidi}. I fosfolipidi sono i più abbondanti, ma non sono
gli unici. I primi contengono glicerolo, mentre gli sfingolipidi
contengono un \emph{amminoalcol} (sfingosina con testa polare e coda
apolare) al posto del glicerolo e un acido grasso collegato con la parte
amminica. A volte alla sfingolisina può essere lengato un ulteriore
gruppo -R; se non è presente si parla di \emph{ceramide}.

Gli sfingolipidi sono maggiormente presenti in determinate zone della
membrana dette zattere lipidiche (zone più spesse perché le catene degli
sfingolipidi sono un po' più lunghe di quelle dei fosfolipidi) e
contengono proteine che non sono presenti nelle altre zone.

La membrana esterna forma una barriera tra l'ambiente intra- ed extra-
cellulare. Le membrane permettono il trasporto selettivo di nutrienti,
prodotti di rifiuto e metaboliti tra i vari compartimenti cellulari. Le
membrane servono a formare vescicole per catturare e secernere
macromolecole e altre particelle.

Nelle membrane sono presenti molte proteine che possono essere:

\begin{itemize}
\itemsep1pt\parskip0pt\parsep0pt
\item
  proteine integrali di membrana, se sono incorporate nella membrana
  mediante domini idrofobici;
\item
  proteine di superficie, se sono associate o agganciate alla membrana
  ma non hanno parti idrofobiche immerse nello strato fosfolipidico.
\end{itemize}

Le proteine di membrana possono svolgere diversi ruoli:

\begin{itemize}
\itemsep1pt\parskip0pt\parsep0pt
\item
  possono avere un ruolo strutturale (citoscheletrico);
\item
  possono servire per attaccare la cellula ad un substrato (strutture di
  adesione cellulare);
\item
  possono far attaccare le cellule l'una all'altra;
\item
  possono essere dei trasportatori (trasporto di membrana);
\item
  possono essere dei recettori (capaci di legare molecole segnale che
  inducono processi interni alla cellula)\ldots{}
\end{itemize}

Le proteine di membrana, se la attraversano, possiedono una parte
extracellulare che molto spesso porta legate delle molecole zuccherine
lineari o ramificate, ed in questo caso sono dette \emph{proteine
glicosilate}. Questo fa si che la superficie cellulare sia ricoperta di
zuccheri, molecole molto idrofiliche che rendono lo strato acquoso
subito presente esternamente alla cellula molto denso formando una
struttura chiamata \textbf{glicocalice}. Il glicocalice si oppone
ulteriormente ad un contatto troppo diretto della cellula con il mondo
esterno svolgendo dunque una funzione protettiva, e rappresenta inoltre
un fattore di riconoscimento per le cellule.

\section{Diffusione e osmosi}\label{diffusione-e-osmosi}

Una parte della fisiologia si occupa di capire se attraverso le membrane
biologiche possono passare delle sostanze e, se passano, come passano.
Se non passasse nulla e la cellula fosse uno spazio totalmente isolato
dal resto dell'universo, dovrebbe usare il materiale che contiene
all'interno per compiere i processi che la tengono in vita e riciclare
tutti gli scarti dei suoi processi per ricreare i cicli. Le cellule
nella realtà scambiano continuamente diversi tipi di molecole con
l'esterno. Ma la cellula è anche separata dall'esterno, e questo ci
permette di capire che le membrane possono essere penetrate e
oltrepassate da diversi tipi molecolari.

La fisiologia applica a questi fenomeni le leggi della fisica.

Alcuni di questi fenomeni avvengono spontaneamente, senza l'intervento
del metabolismo, e sono i fenomeni di diffusione e osmosi. I doppi
strati lipidici (bilayer) sono impermeabili a molte molecole e ioni
essenziali. Poiché la membrana cellulare ha una natura lipidica tutte le
molecole apolari possono attraversarla tramite diffusione semplice,
mentre le sostanze polari e gli ioni non riescono a oltrepassarla.

Sostanze inorganiche apolari di enorme importanza per la cellula e
capaci di attraversare la membrana sono invece i gas atmosferici (O2 e
N2).

Cos'è la diffusione? La materia, così come la si conosce è costituita da
cellule che hanno un loro dinamismo (non sono statiche). Tanto più le
cose sono dinamiche tanto più emergono fenomeni come quelli di
diffusione: le molecole più mobili sono i gas, poi vi sono i liquidi ed
infine i solidi.

Questro dinamismo fa sì che nelle sostanze in cui le molecole hanno la
possibilità di muoversi si crei il fenomeno della \emph{diffusione}: se
in una certa regione c'è abbondanza di una certa sostanza o gas, questa
tenderà a spostarsi diffondendo. Quando c'è una diffusione non omogenea
si parla di \emph{gradiente}.

Se le molecole non sono distribuite uniformemente sui due lati di una
membrana le loro diverse concentrazioni formano un gradiente, ovvero una
forma di energia potenziale. L'energia libera della soluzione viene
definita con: \(\Delta\)G= RT ln{[}S{]}

Dove:

\begin{itemize}
\itemsep1pt\parskip0pt\parsep0pt
\item
  R è la costante dei gas;
\item
  T è la temperatura in Kelvin;
\item
  S è la concentrazione ed è espressa in moli.
\end{itemize}

L'energia del gradiente, invece, viene definita con:

\(\Delta\)G = energia esterna -- energia interna

ovvero

\(\Delta\)G= RT ln{[}S{]}o -- RT ln{[}S{]}i = RT ln{[}S{]}o/RT
ln{[}S{]}i

Questa viene detta legge di Van't Hoff.

In una soluzione acquosa si possono avere 3 modalità di flusso che si
differenziano per la natura della driving force che determina il moto:

\begin{itemize}
\itemsep1pt\parskip0pt\parsep0pt
\item
  flusso di massa in cui la driving force è generata da una differenza
  di pressione idraulica;
\item
  la \textbf{diffusione}, generata da una differenza di concentrazione
  delle particelle;
\item
  la migrazione in campo elettrico di particelle elettricamente cariche
  (ioni), in cui la driving force è generata da una differenza di
  potenziale elettrico.
\end{itemize}

La \emph{diffusione} si verifica quando tra due regioni di una soluzione
esiste una differenza di concentraizone; si ha allora un flusso
particellare dalla regione a concentrazione magiore verso quelle a
concentrazione minore, finchè le particelle non sono distribuite in modo
omogeneo in tutto lo spazio disponibile.

La membrana cellulare permette alla cellula di avere una concentrazione
di una molecola X al suo interno, diversa dalla concentrazione della
stessa molecole X all'esterno della cellula. La presenza di due
soluzioni omogenee, ma con una diversa concentrazione di soluti, dalle
due parti della membrana crea un potenziale che premerà sulla membrana
stessa.

La cellula crea queste situazioni proprio per il fatto che esiste una
membrana che isola l'esterno dall'interno. All'interno e all'esterno
della membrana si possono avere concentrazioni diverse di un dato
elemento. Quello che succede dentro la membrana può essere trascurabile
rispetto al fenomeno complessivo che questa differenza di concentrazione
determina nella cellula.

Il processo osmotico è spontaneo, non richiede energia aggiunta. Questo
è impossibile per il \emph{primo principio della termodinamica}
(l'energia di un sistema termodinamico isolato non si crea né si
distrugge, ma si trasforma, passando da una forma a un'altra), abbiamo
un apparente paradosso.

Se consideriamo due ambienti 1 e 2 separati da una sezione ideale S, ci
attendiamo che il numero di molecole di soluto che nell'unità di tempo
escono da 1 (e entrano in 2) sia \emph{proporzionale} al numero di
molecole che sono presenti per unità di volume, cioè alla
\emph{concentrazione C1} (e viceversa).

Questo può essere espresso tramite l'\textbf{equazione di Fick} o
\emph{legge di diffusione}:

\textbf{Fd=D\(\Delta\) (C1-C2)}

(Fd sta per flusso diffusionale netto)

Dove \textbf{D} è il \emph{coefficiente di diffusione} (dipende dalla
natura dei partecipanti al processo e dalla temperatura) e \textbf{A} è
l'\emph{area della sezione interessata al processo diffusivo}.

Se consideriamo due compartimenti 1 e 2 separai da una parete con pori
di lunghezza \emph{delta} per una superficie totale \emph{F} e
differenza di concentrazione del soluto \(\delta\)C= C\(_1\)-C\(_2\), il
soluto diffonde da 1 verso 2.

Il tasso di diffusione è dunque \emph{direttamente proporzionale} alla
differenza di concentrazione, alla temperatura e alla superficie di
scambio, mentre è \emph{inversamente proporzionale} alla distanza, alle
dimensioni delle molecole e alla viscosità del solvente.

Questa equazione è valida nel caso di una membrana ideale. Supponiamo
ora che i due compartimenti siano separati da una barriera reale,
costituita da una membrana (M) di spessore deltax e composizion propria,
e che le particelle siano spinte solo da un gradiente di concentrazione.
In queste condizioni, il passaggio di una generica sestanza (i) può
essere descritto dalla semplice equazione di Fick purchè il coefficiente
di diffusione venga sostituito dal coefficiente di permeabilità P.

\textbf{F\(_i\)=P\(_i\) (C1-C2)}

Il coefficiente di permeabilità di una data sostanza in una membrana
omogenea comprende tutti i fattori che ne condizionano il passaggio:

\textbf{P\(_i\) = \(\beta\)\(_i\) (D\(_m\) - \(\Delta\)x)}

dove \textbf{D\(_m\)} è il \emph{coefficiente di diffusione della
sostanza} nel materiale costitutivo della membrana e
\textbf{\(\Delta\)x} lo \emph{spessore della membrana}.
\textbf{\(\beta\)\(_i\)} invece, esprime \emph{la capacità della
sostanza di passare da uno spazio acquoso ad uno lipidico}, ovvero di
potersi disciogliere nel materiale che costituisce la membrana.

Il coefficiente di permeabilità \textbf{P\(_i\)} esprime la
\emph{velocità con cui una sostanza può attraversare la membrana}.

Se la permeabilità è 0, il flusso attraverso la membrana sarà 0 (nessun
flusso) indipendentemente dal gradiente.

Se il coefficiente di permeabilità invece è 1, la velocità di diffusione
dipenderà solo dal gradiente.

\subsection{Diffusione facilitata}\label{diffusione-facilitata}

Nella membrana cellulare ci sono delle proteine che creano dei pori che
permettono il passaggio di sostanze che altrimenti non passerebbero.

La diffusione facilitata è sempre un processo spontaneo che non richiede
l'aggiunta di energia, ma che richiede la presenza di elementi
aggiuntivi nella membrana (proteine). La cinetica di trasporto è diversa
rispetto a quella della diffusione semplice: nella \emph{diff. semplice}
la velocità di diffusione è in relazione lineare con il gradiente di
concentrazione della molecola che diffonde, mentre nella \emph{diff.
facilitata} la velocità di diffusione dipende dalla disponibilità delle
molecole che aiutano la diffusione. Quando queste molecole sono sature
l'aumento della concentrazione della molecola che diffonde non
incrementa ulteriormente il tasso di diffusione. Questo significa che
avrò per un po' una crescita lineare e poi arriverò ad un plateau oltre
il quale l'aumento del soluto che diffonde non servirà più per aumentare
la velocità di diffusione.

\subsection{L'osmosi}\label{losmosi}

Quando due soluzioni diversamente concentrate vengono in contatto, non
si osserva solo la diffusione delle molecole del soluto dalla soluzione
più concentrata verso quella meno concentrata, ma anche la diffusione
delle molecole d'acqua nella \emph{direzione opposta}.

Questo flusso diffusionale dell'acqua prende il nome di \textbf{osmosi},
e la forza che determina il flusso diffusionale dell'acqua riferita
all'unità di superficie prende il nome di \textbf{pressione osmotica}.

Affinchè avvenga il fenomeno osmotico è necessario che le due soluzioni
a concentrazione diversa siano separate da una \emph{membrana permeabile
al solvente ma non al soluto}.

La forza che spinge le molecole idriche a passare attraverso la membrana
semipermeabile si traduce in un aumento della pressione idrostatica
all'interno del recipiente con più soluto al suo interno, pressione che
aumenta fino ad equilibrare esattamente, a livello di membrana, la forza
che spinge le molecole idriche ad attraversarla (\emph{pressione
osmotica}).

L'acqua tenderà a passare dalla zona meno concentrata a quella più
concentrata.

la relazione quantitativa tra la differenza di pressione osmotica (delta
pigreco) è data dalla \textbf{legge di van't Hoff}:

\textbf{\(\Delta\) \(\pi\) = RT (C\(_1\)-C\(_2\))}

Una delle conseguenze della legge di van't Hoff della pressione osmotica
è che \textbf{NON} dipende dalla natura delle particelle di soluto, ma
solo dal loro numero per unità di volume.

Le cellule risentono della pressione osmotica poiché presentano una
membrana plasmatica che lascia sempre passare l'acqua (questo passaggio
può essere controllato).

Il gradiente osmotico tra l'ambiente esterno ed interno della cellula è
creato da tutti quei soluti, detti \textbf{osmoliti}, che non possono
assolutamente attraversare la membrana.

Se le concentrazioni degli osmoliti dentro e fuori sono uguali
ovviamente non si ha l'effetto osmotico. Normalmente le cellule animali
sono in condizioni di equilibrio osmotico, mentre se non c'è equilibrio
possono rigonfiarsi o avvizzire.

Nel plasma e nei globuli rossi abbiamo una concentrazione di 300
mOsmoles/L (ambiente \emph{isotonico}). Se i globuli rossi vengono posti
in una \emph{soluzione ipotonica} (es 150 mOsmoles/L), la cellula
rigonfia fino a scoppiare, mentre se i globuli rossi vengono posti in
una \emph{soluzione ipertonica} (es 500 mOsmoles/L) la cellula
raggrinzisce.

\section{Il trasporto di membrana}\label{il-trasporto-di-membrana}

Le cellule non rappresentano un sistema chiuso, ma scambiano
costantemente sostanze con l'esterno. Questo scambio viene fatto in
diversi modi che complessivamente vengono definiti come ``trasporto di
membrana''. Poiché la membrana ricopre interamente la cellula e non
presenta ``buchi'', deve esistere un modo per le molecole di
attraversarla. I fisiologi catalogano il trasporto di membrana creando
in primo luogo due grandi divisioni:

\begin{enumerate}
\def\labelenumi{\arabic{enumi}.}
\itemsep1pt\parskip0pt\parsep0pt
\item
  \textbf{trasporto attivo}, le particelle possono essere trasportate
  anche \emph{contro gradiente} e l'energia necessaria al trasporto
  proviene dal metabolismo cellulare;
\item
  \textbf{trasporto passivo}, le particelle attraversano la membrana
  cellulare solo \emph{in favore di un gradiente} e l'energia necessaria
  al trasporto proviene dalla dissipazione del gradiente che muove le
  particelle.
\end{enumerate}

Possiamo avere trasporto attivo o passivo sia in ingresso che in uscita
dalla cellula.

Il \emph{trasporto passivo} può essere:

\begin{enumerate}
\def\labelenumi{\arabic{enumi}.}
\itemsep1pt\parskip0pt\parsep0pt
\item
  \textbf{semplice}, se non servono intermediari affinchè la sostanza
  diffonda attraverso la membrana;
\item
  \textbf{facilitato}, se la sostanza non può attraversare
  spontaneamente la membrana ma utilizza delle proteine trasportatrici.
\end{enumerate}

Per il \emph{trasporto semplice} serve una sostanza che possa diffondere
attraverso la membrana ed un gradiente osmotico tra l'interno e
l'esterno della cellula.

Un esempio di diffusione semplice è il passaggio dell'acqua (a volte può
sfruttare anche il trasporto facilitato grazie alle \emph{acquaporine}),
dei gas atmosferici o le sostanze lipidiche. Per diffusione facilitata
inoltre, passano sostanze come i principali zuccheri del metabolismo
(glucosio e fruttosio) e i principali ioni minerali (Na, K, Cl, Ca)
necessari per la vita della cellula.

Il \emph{trasporto attivo} invece viene suddiviso in:

\begin{itemize}
\itemsep1pt\parskip0pt\parsep0pt
\item
  \emph{primario}, crea i gradienti di concentrazione e viene generato
  da speciali trasportatori detti \textbf{pompe} che consumano energia
  sotto forma di ATP (dette anche pompe ATPasiche o ATP dipendenti);
\item
  \emph{secondario}, non viene speso direttamente ATP ma viene sfruttata
  la differenza di potenziale elettrochimico creata dai trasportatori
  primari che pompano ioni al di fuori della cellula, il trasportatore
  secondario sfrutta l'energia di questo gradiente per trasportare
  un'altra sostanza contro gradiente.
\end{itemize}

Il trasporto attivo secondario viene poi diviso in:

\begin{itemize}
\itemsep1pt\parskip0pt\parsep0pt
\item
  \textbf{uniporto}, è il trasporto secondario di una sola sostanza che
  si muove sfruttando la differenza di potenziale elettrochimico creato
  da trasportatori primari;
\item
  \textbf{costrasporto}, è il trasporto contemporaneo di due specie
  ioniche o di altri soluti.
\end{itemize}

Il cotrasporto avviene grazie alla presenza di proteine di membrana
dette \textbf{cotrasportatori}: queste proteine possono trasportare 2 o
3 molecole diverse di cui una contro gradiente e l'altra secondo
gradiente. Come già detto non c'è consumo diretto di ATP ma c'è
dissipazione di un gradiente che sarà stato precedentemente creato da un
trasportatore attivo primario. Nel cotrasporto si ha la diffusione di
una sostanza lungo il suo gradiente ed è proprio questa diffusione a
creare i presupposti affinché lo stesso trasportatore possa far
attraversare la membrana ad un'altra sostanza contro gradiente (le
sostanze trasportate possono essere anche tre).

Il cotrasporto si divide poi in:

\begin{itemize}
\itemsep1pt\parskip0pt\parsep0pt
\item
  \textbf{antiporto}, è il trasporto contemporaneo di due specie ioniche
  o di altri soluti che si muovono in \emph{direzioni diverse}
  attraverso la membrana. Una delle due sostanze viene lasciata fluire
  secondo gradiente, da un compartimento ad alta concentrazione ad uno a
  bassa concentrazione. Questo genera l'energia entropica necessaria per
  guidare il trasporto dell'altro soluto contro gradiente, da bassa ad
  alta concentrazione;
\item
  \textbf{simporto}, usa il flusso di un soluto secondo gradiente per
  muovere un'altra molecola contro gradiente ma il movimento avviene in
  questo caso attraversando la membrana nella \emph{stessa direzione}.
\end{itemize}

Nei tipi di trasporto possiamo poi includere fenomeni di
\textbf{esocitosi} e \textbf{endocitosi}. In questo caso le sostanze non
vengono trasportate molecola per molecola ma come massa di materia
grazie a movimenti della membrana plasmatica che ingloba una certa
quantità di materiale. L'endocitosi e l'esocitosi implicano sempre la
formazione di vescicole.

\section{Trasporto attivo primario: le
pompe}\label{trasporto-attivo-primario-le-pompe}

Le pompe sono trasportatori di membrana, ovvero proteine di membrana
formate da 3 porzioni: una parte esterna e glicosilata, una parte
citosolica (molto più grande di quella esterna) ed una parte immersa
nella membrana.

Queste proteine sono capaci di accoppiare il trasporto contro gradiente
di diversi substrati con la defosforilazione di una molecola di ATP.
L'idrolisi dell'ATP libera energia che il trasportatore può utilizzare
per far avvenire il passaggio di una sostanza da un lato all'altro della
membrana.

I trasportatori di membrana sono presenti anche all'interno della
cellula sulle membrana di alcuni organuli cellulari (es: lisosomi,
mitocondri\ldots{}). Trasportano numerosi tipi di composti (noi vedremo
soprattutto gli ioni minerali)

Le proteine responsabili dei trasporti attivi primari si possono
suddividere in 3 gruppi:

\begin{itemize}
\itemsep1pt\parskip0pt\parsep0pt
\item
  le ABC ATPasi;
\item
  le H-ATPasi (comprende le F,V e A ATP-asi);
\item
  le P-ATPasi.
\end{itemize}

\textbf{(IMMAGINE p87)}

Le \textbf{F-ATPasi} (o ATP sintasi) si trovano nei mitocondri e sono
coinvolte nella sintetizzazione dell'ATP; queste ATPasi funzionano in
modo opposto rispetto alle altre, ovvero sinsetizzano ATP sfruttando
l'energia di un trasporto ionico secondo gradiente invece di trasportare
contro gradiente ioni sfruttando l'energia prodotta dall'idrolisi
dell'ATP.

Le \textbf{V-ATPasi} acidificano il lume interno dei vacuoli e di altri
organuli endocellulari portando al loro interno ioni H+.

Le F e le V-ATPasi sono costituite da due gruppi di subunità
(\emph{complessi}) denominati F1 e F0 o V1 e V0. I gruppi F1 e V1 sono
citoplasmatici e presentano i siti catalitici per l'ATP, mentre le
porzioni F0 e V0 sono transmembrana e sono la sede del trasporto degli
idrogenioni.

\textbf{(IMMAGINE P89)}

Le \textbf{A-ATPasi} si trovano esclusivamente negli Archea (funzione
delle F-ATPasi e struttura simile alle V-ATPasi).

Le \textbf{P-ATPasi} pompano ioni Na\(^+\), K\(^+\),
Ca\(^2\)\(^+\)\ldots{} hanno un intermedio fosforilato. La molecola del
trasportatore viene fosforilata e trasportata dal compartimento
extracellulare a quello intracellulare.

In una normale condizione fisiologica esistono diversi gradienti di
membrana (indispensabili per la vita delle cellule). La fisiologia di
una cellula ruota attorno ad alcuni ioni minerali che sono quelli più
abbondanti nei liquidi biologici.

I più abbondanti in assoluto sono il Na e il K, seguiti da Cl e Ca.
Questi ioni hanno sempre concentrazioni diverse dentro e fuori la
cellula:

\begin{itemize}
\itemsep1pt\parskip0pt\parsep0pt
\item
  \textbf{Na\(^+\)} \textgreater{} nell'ambiente esterno (140 mmol
  nell'ambiente esterno, 5 mmol nel citosol);
\item
  \textbf{K\(^+\)} \textgreater{} nel citosol (5-15 mmol nell'ambiente
  esterno, 145 mmol nel citosol);
\item
  \textbf{Cl\(^-\)} \textgreater{} nell'ambiente esterno (110 mmol
  nell'ambiente esterno, 4 mmol nel citosol);
\item
  \textbf{Ca\(^2\)\(^+\)} \textgreater{} nell'ambiente esterno (2,5-5
  mmol nell'ambiente esterno, 0,0001 mmol nel citosol).
\end{itemize}

Nel momento in cui uno di questi gradienti cambia la cellula muore,
dunque la cellula tende a mantenere questi gradienti sempre costanti.

Questi gradienti sono mantenuti dal trasporto attivo primario che li
crea ed in parte da quello secondario che sfruttando questi gradienti
fondamentali ne tiene stabili altri.

\subsection{Le P-ATPasi}\label{le-p-atpasi}

Queste proteine presentano un meccanismo di trasporto comune a tutte,
anche se è stato individuato con precisione solo nelle pompe SERCA e
nella Na\(^+\)/K\(^+\)-ATPasi.

Il loro modello di funzionamento è chiamato ``Post-Albers'' e assume che
la proteina trasportatrice possa assumere \emph{due conformazioni}: +
nella conformaizone \textbf{E1} i siti di legame sono esposti dal lato
citoplasmatico ed hanno alta affinità per i substrati che devono essere
trasportati dall'altro lato della membrana e bassa affinità per i
substrati che sono trasportati in senso opposto; + nella conformazione
\textbf{E2} gli stessi siti sono esposti al lato extracellulare della
membrana ed hanno bassa affinità per i substrati che sono stati legati
in precedenza al lato intracellulare ed alta affinità per i substrati
che saranno importati nel citoplasma.

Come avviene il cambiamento conformazionale?

Il ciclo parte con la proteina nella conformazione E1 in cui presenta
sia un sito di legame ad alta affinità per l'ATP (che vi si lega) che i
siti di legame per i substrati ad alta affinità.

\begin{enumerate}
\def\labelenumi{\arabic{enumi}.}
\itemsep1pt\parskip0pt\parsep0pt
\item
  Si forma il legame tra i substrati presenti nel citosol e i siti di
  legame ad alta affinità. L'energia liberata da tale legame provoca un
  primo riarrangiamento della proteina;
\item
  il cambiamento conformazionale provoca la chiusura delle vie di
  accesso ai siti di legame verso il citoplasma intrappolando il
  substrato che si era legato. Questo viene detto stato di
  fosforilazione E1-P;
\item
  si ha un cambiamento nella proteina che riduce l'affinità dei siti di
  legame per i substrati che aveva legato precedentemente, mentre
  aumenta l'affinità per i substrati da legare nell'ambiente
  extracellulare. Contemporaneamente si apre un varco di uscita al lato
  extracellulare della membrana dove, per la ridotta affinità dei siti
  di legame, vengono rilasciati i substrati precedentemente legati. A
  questi siti di legame vengono poi legati i nuovi substrati. Durante
  questo passaggio la proteina passa dalla conformazione E1-P a quella
  E2-P;
\item
  il legame dei nuovi substrati ai siti di legame causa la chiusura di
  questo varco ed il conseguente intrappolamento dei substrati.
\end{enumerate}

Questo processo avviene in maniera ciclica permettendo il continuo
legame e trasporto di molecole da una parte all'altra della cellula.

\subsubsection{\texorpdfstring{Pompa
Na\(^+\)/K\(^+\)}{Pompa Na\^{}+/K\^{}+}}\label{pompa-nak}

Il Na\(^+\) è lo ione più importante extracellularmente, mentre il
K\(^+\) è il più importante intracellularmente. Esiste una
Na\(^+\)/K\(^+\) ATP, o ATPasi sodio-potassio dipendente o pompa
sodio-potassio. È una pompa di tipo P, localizzata nella membrana
plasmatica, che sfrutta un intermedio fosforilato.

Questo trasportatore crea il gradiente del Na\(^+\) e del K\(^+\), che
sono i più importanti dal punto di vista quantitativo. Movimenta la
maggior quantità di materiale attraverso la membrana.

Trasporta contemporaneamente ioni Na\(^+\) fuori dalla cellula e ioni
K\(^+\) all'interno, entrambi contro il loro gradiente elettrochimico.
Questa pompa è essenziale per l'eccitabilità dei neuroni e del muscolo
poichè è essenziale per l'equilibrio elettrico e osmotico della cellula.

\textbf{(riguardare ELETTROMAGNETISMO su appunti fisica)}

Questa proteina è formata da due subunità:

\begin{itemize}
\itemsep1pt\parskip0pt\parsep0pt
\item
  una \textbf{subunità alfa} formata da 10 domini transmembrna che
  sporge in prevalenza nel citosol. Questa subunità porta i siti di
  legame per l'ATP e per i cationi;
\item
  una \textbf{subunità beta} più piccola, formata da un solo dominio
  transmembrana, che sporge verso l'esterno con una porzione
  glicosilata.
\end{itemize}

La proteina viene fosforilata, tramite l'idrolisi di una molecola di
ATP, mentre si trova in uno stato in cui la proteina presenta elevata
affinità per il legame di Na\(^+\) sul lato citosolico (dove sodio ce
n'è ``poco''). Questa fosforilazione produce una variazione
conformazionale della proteina. A questo punto la zona che lega il
Na\(^+\) (può ospitare fino a 3 ioni alla volta) si apre all'interno
della cellula dove il Na+ è molto concentrato.

Questa variazione conformazionale rende i siti di legame per il Na+
molto meno affini allo ione permettendogli di venire rilasciato
all'interno della cellula. L'affinità per il ligando si presenta con un
valore simile a quello della concentrazione dello ione. Quando la
proteina varia la sua conformazione rivolgendo i siti di legame verso
l'esterno il coefficiente di dissociazione deve salire a 100/150 mmol.

La fosforilazione della proteina non dura molto a lungo (parliamo di
millisecondi) e quando viene defosforilata l'affinità per il Na\(^+\)
diminuisce ma aumenta quella per il K\(^+\).

Ogni \emph{3 ioni Na\(^+\)} che escono ne entrano \emph{2 di K\(^+\)}.
Questo trasporto viene definito \textbf{elettrogenico} poiché crea una
differenza di potenziale elettrico tra l'esterno e l'interno della
cellula a causa del differente numero di cariche positive trasportate.

\subsubsection{\texorpdfstring{Pompa protonica o H\(^+\)/K\(^+\)
ATPasi}{Pompa protonica o H\^{}+/K\^{}+ ATPasi}}\label{pompa-protonica-o-hk-atpasi}

Questo trasportatore opera un \emph{antiporto} molto simile alla
Na\(^+\)/K\(^+\)-ATPasi (è un'ATPasi dei tipo P) ma, a differenza di
quest'ultima è \emph{elettroneutra}. Questo trasportatore estrude 2 ioni
H\(^+\) ed intrude 2 ioni K\(^+\). È tipico delle cellule ossintiche
delle ghiandole gastriche della mucosa dello stomaco per
l'acidificazione del chimo.

\subsubsection{\texorpdfstring{Pompa
Ca\(^2\)\(^+\)-ATPasi}{Pompa Ca\^{}2\^{}+-ATPasi}}\label{pompa-ca2-atpasi}

Questi trasportatori attivi primari dovrebbero essere più propriamente
denominati Ca\(^2\)\(^+\)/H\(^+\)-ATPasi e sono anch'essi molto
espressi.

Ne possiamo riconoscere almeno due tipi distinti:

\begin{enumerate}
\def\labelenumi{\arabic{enumi}.}
\itemsep1pt\parskip0pt\parsep0pt
\item
  Le \textbf{pompe PMCA} o Ca\(^2\)\(^+\) ATPasi della plasma membrana,
  trasportano \emph{uno ione Ca\(^2\)\(^+\)} all'esterno della cellula e
  \emph{uno ione H\(^+\)} all'interno della cellula per ogni molecola di
  ATP idrolizzata. Ne esistono 4 isoforme.
\item
  le \textbf{SERCA} o Ca\(^2\)\(^+\) ATPasi del reticolo
  sarco-endoplasmatico, trasportano verso il lume del reticolo
  endoplasmatico \emph{due ioni Ca\(^2\)\(^+\)} ed un certo numero di
  ioni H\(^+\) (per ogni coppia di Ca\(^2\)\(^+\) vengono traslocati un
  po' meno di 4 ioni H\(^+\)). È il principale sistema di omeostasi del
  muscolo scheletrico e cardiaco.
\end{enumerate}

Questa proteina movimenta gli ioni sulla membrana del reticolo
endoplasmatico, mentre le altre due viste movimentano gli ioni sulla
membrana plasmatica.

\subsubsection{\texorpdfstring{Cotrasporto
3Na\(^+\)/2Ca\(^2\)\(^+\)}{Cotrasporto 3Na\^{}+/2Ca\^{}2\^{}+}}\label{cotrasporto-3na2ca2}

Questo è un antiporto.

Questo trasportatore sfrutta il gradiente del Na\(^+\) facendolo entrare
nella cellula e contemporaneamente porta Ca\(^2\)\(^+\) fuori dalla
cellula.

È un processo che tende ad innalzare il potenziale eletttrico
transmembranario a causa del diverso numero di cariche spostate.

\subsubsection{\texorpdfstring{Controtrasporto
Na\(^+\)/H\(^+\)}{Controtrasporto Na\^{}+/H\^{}+}}\label{controtrasporto-nah}

Questi trasportatori provvedono all'espulsione nello spazio
extracellulare degli ioni H\(^+\) liberati nel citosol dalla
deidrogenazione dei substrati. L'ingresso di ioni Na\(^+\) comporta
un'equivalente uscita di H\(^+\) in direzione opposta.

Questi trasportatori sono molto espressi nel rene dove rappresentano il
principale meccanismo di acidificazione dell'urina.

Nel rene avviene la filtrazione del sangue (liquido di partenza =
plasma), il liquido finale è l'urina (molto diversa dal plasma). Tutte
le differenze tra il plasma e l'urina sono dovute all'attività di
trasportatori che agiscono nei tubicini dove passa il plasma.

\subsection{Trasportatori mediati da
vescicole}\label{trasportatori-mediati-da-vescicole}

Molecole ed aggregati sovramolecolari di varie dimensioni possono
entrare o uscire dalle cellule con un meccanismo completamente diverso
da quelli considerati finora: superando la emmbrana plasmatica
\emph{racchiusi in vescicole}.

\textbf{Endocitosi.} Quando il carico delle vescicole è una goccia
fluida, la vescicola è di minor diametro e si parla di
\textbf{pinocitosi}. Solitamente la pinocitosi è \emph{mediata da
recettori}, essendo innescata dal contatto delle molecole che devono
essere introdotte nella cellula con specifici recettori della membrana
plasmatica.

Quando invece viene unglobato un corpuscolo solido, la vescicola che lo
include ha diametro maggiore e il processo è detto \textbf{fagocitosi}.
La fagocitosi richiede l'\emph{intervento del citoscheletro} ed in
particolare, una volta che il materiale da fagocitare ha attivato
specifici recettori di membrana, mette in moto i microfilamenti di
actina ad esso sottostanti, assumendo il chiaro carattere di azione
motoria.

\textbf{Esocitosi.} Qui la sequenza degli eventi è inversa a quella
dell'endocitosi: essa inizia con l'adesione alla superficie interna
della membrana plasmatica di una vescicola in arrivo dal citoplasma,
carica del materiale che dev'essere espulso dalla cellula. Seguono la
fusione e l'apertura della vescicola, che rende libero il suo contenuto
nell'ambiente.

Le vescicole destinate a fondersi con la membrana plasmatica provengono
dal complesso di Golgi.

\subsection{Il trasporto
transepiteliale}\label{il-trasporto-transepiteliale}

Negli organismi superiori, ogni scambio di materia con l'ambiente
avviene attraverso particolari tessuti denominati \emph{epiteli}.

Mentre gli epiteli pluristratificati delimitano e proteggono
l'organismo, quelli monostratificati assolvono al compito di regolare
gli scambi materiali tra il suo \emph{ambiente interno} e l'ambiente in
cui esso vive, svolgendo una funzione analoga alla membrana plasmatica
della singola cellula.

L'alterazione del trasporto transepiteliale causa malattie come la
fibrosi cistica, malattia dovuta ad un trasporto anomalo del cloro.

\section{La comunicazione cellulare}\label{la-comunicazione-cellulare}

Le cellule possono comunicare tra loro sia per contatto che a distanza
grazie all'utilizzo di molecole che funzionano come messaggeri chimici.

Per la comunicazione cellulare serve:

\begin{itemize}
\itemsep1pt\parskip0pt\parsep0pt
\item
  un messaggero chimico;
\item
  un recettore, ovvero una molecola in grado di recepire la molecola
  segnale;
\item
  un marchingegno cellulare che converta il segnale ricevuto dal
  recettore in un'attività della cellula.
\end{itemize}

Quando un recettore cellulare lega una molecola segnale ad esso
indirizzata modifica lo stato funzionale della una cellula (= la cellula
stava facendo qualcosa e comincia a fare qualcos'altro). Ci sono
sostanze che hanno una capacità enorme di modificare lo stato funzionale
di una certa cellula, dette ``ormoni'' (ne basta una concentrazione
molto bassa perché si modifichi lo stato funzionale della cellula
interessata).

\subsection{I recettori cellulari}\label{i-recettori-cellulari}

I recettori chimici sono i sensori dei messaggeri extracellulari. Essi
sono sempre \textbf{molecole proteiche} foggiate in modo da legare con
\emph{altissima affinità} quelle dei messaggeri (chiamati genericamente
\emph{ligandi}).

Si tratta di legami relativamente \emph{labili} (ponti idrogeno, forze
di van der Waals\ldots{}) e quindi facilmente reversibili.

La molecola segnale solitamente arriva da un'altra cellula, entra in
contatto con la cellula che possiede il recettore e si lega al recettore
attivandolo.

Il recettore, dopo essere entrato in contatto con la molecola segnale,
modifica la sua conformazione legando altre proteine della cellula, che
a loro volta legano altre proteine, fino ad attivare degli enzimi che
cambieranno lo stato funzionale della cellula. Questo processo viene
chiamato \textbf{``trasduzione del segnale''}.

Quando avvengono le comunicazioni cellulari? Sempre. Non esistono
cellule vive che non siano coinvolte in fenomeni di comunicazione
cellulare.

Nelle colture cellulari in vitro non tutti i tipi di cellule possono
essere coltivate, ma per quelle che lo sono è sempre essenziale che ci
siano dei fattori di crescita nel terreno, altrimenti la cellula andrà
in apoptosi. Solitamente viene utilizzato il siero fetale bovino poichè
contiene molti fattori di crescita.

I recettori possono essere:

\begin{itemize}
\itemsep1pt\parskip0pt\parsep0pt
\item
  \textbf{intracellulari} (o endocellulari), ubicati nel citoplasma o
  nel nucleo, accessibili soltanto ai messaggeri extracellulari
  \textbf{liposolubili} che possono superare la membrana della cellula
  bersaglio. Generalmente esplicano al loro azione regolando
  direttamente svariati processi di trascrizione genica che presiedono
  alla sintesi di nuove proteine;
\item
  \textbf{di superficie} (o membranali), ubicati sulla membrana
  plasmatica, riservati ai ligandi \textbf{idrosolubili} i quali, per le
  dimensioni ed altre caratteristiche delle loro molecole, non possono
  superare la membrana della cellula bersaglio.
\end{itemize}

Gli \emph{endocrinologi} studiano da dove arriva a dove va la molecola
segnale. Si distingue un segnale \textbf{endocrino} quando le due
cellule sono situate a distanza, ed il messaggio deve viaggiare nel
torrente sanguigno.

Un altro segnale è quello \textbf{paracrino}, qui non serve il passaggio
della molecola segnale nel circolo sanguigno poichè il punto di origine
e di arrivo della molecola sono molto vicini (cellule vicine), dunque la
moleocla devve attraversare solo lo spazio intracellulare.

Possiamo distinguere anche un segnale \textbf{autocrino} se la molecola
segnale rilasciata da una cellula trova recettori sulla propria membrana
(es. fattori di crescita).

Il \textbf{ligando} consiste nella la molecola segnale (ormone,
feromone, ione, neurotrasmettitore, farmaco, etc\ldots{}) che si lega in
maniera specifica a un sito sulla molecola del recettore (situato sulla
superficie o all'intenro della cellula bersaglio).

Esistono metodi raffinati (es. HPLC cromatogragfia liquida ad alta
prestazione) con i quali si è possibile individuare queste molecole
quando presenti nel sangue; tramite questi metodi è stato osservato che
la maggior parte di queste molecole, quando stanno agendo, sono a
concentrazioni di 10\^{}-9/-10 molare.

Per definire l'affinità del recettore per una molecola possiamo fare un
grafico dove l'\textbf{asse y} indica la \textbf{percentuale di molecole
recettore legate} e l'\textbf{asse x} indica la \textbf{concentrazione
del ligando nel sangue}. È possibile stimolare la cellula bersaglio con
una concentrazione crescente di molecole segnale e valutare quante
molecole vengono legate dai recettori. Il grafico mostrerà una crescita
inizialmente esponenziale e poi evidenzierà un valore di plateau che
indica che tutti i recettori sono stati legati.

Le cellule spesso hanno una frazione di \emph{recettori ``di riserva''},
così che la concentrazione del ligando che fornisce la massima risposta
è minore della concentrazione del ligando che saturerebbe i recettori.

L'intensità di una risposta biologica a un ligando è generalmente
proporzionale al numero di recettori occupati:

{[}RL{]} = {[}R{]}{[}L{]}/k

Per una data concentrazione di ligando, cellule con più recettori
avranno più recettori occupati. L'interazione tra la molecola segnale e
il recettore è sempre basata su legami deboli, perciò il tempo che il
ligando rimane attaccato al recettore è molto breve.

Più il ligando è affine al recettore, più le cellule saranno sensibili
al ligando.

\subsection{Molecole segnale di tipo
gassoso}\label{molecole-segnale-di-tipo-gassoso}

\subsubsection{L'ossido nitrico (NO)}\label{lossido-nitrico-no}

Il monossido di azoto (molecola inquinante dal punto di vista
ambientale) viene prodotta dalle cellule ed è dunque presente
all'interno dei tessuti. L'ossido nitrico è prodotto nelle cellule dagli
enzimi \textbf{NO sintasi (NOS)} a partire da \emph{arginina} che viene
convertita in \emph{citrullina} liberando monossido di azoto.

Il primo effetto di questo enzima che fu scoperto riguardava la sua
azione come \emph{vasodilatatore}: il monossido di azoto agisce sulla
\textbf{guanosima monofosfato ciclico (cGMP)} che inibisce la
contrazione della muscolatura liscia causando il rilassamento dei vasi e
il loro ``ingrandimento''.

Il monossido di azoto viene prodotto dall'endotelio dei vasi ed essendo
un gas può diffondere bene nelle cellule vicine.

Un'altro gas è il \textbf{monossido ci carbonio (CO)}, prodotto
dall'organismo tramite l'attività della \textbf{eme-ossigenasi} e che
può agire come \emph{molecola segnale delle macchinasi}.

L'attuale teoria sulla quale si basa tutta la biologia è la
\textbf{``Teoria del codice genetico''}: la materia vivente viene
prodotta secondo più processi partendo dalla decodificazione delle
informazione contenute in certe molecole (DNA). Attraverso questo
processo l'informazione può essere trasmessa alla progenie. Questa
teoria spiega tutti i fenomeni che vengono osservati nella materia
vivente.

Solo piccole parti di DNA vengono trascritte in RNA, e solo piccole
porzioni di RNA vengono tradotte in proteine. Le proteine sono poi
responsabili di quasi tutto quello che avviene negli esseri viventi.

Questa teoria è focalizzata a spiegare come nasce la materia vivente ma
pochissimo su come questa possa rimanere in vita.

Le basi di questo sistema sono state individuate nel 1870 circa, e
all'incirca negli anni '50 è stato individuato un modo per poter
manipolare il DNA così da poter mutare le proteine che desideriamo.

\subsection{Molecole segnale di tipo
organico}\label{molecole-segnale-di-tipo-organico}

Molte molecole segnale sono state chiamate ``ormoni''.

Le molecole segnale vengono suddivise in lipofile e idrofile ed
intervengono in maniera diversa sulla cellula bersaglio (quella che
contiene i recettori).

\subsubsection{I recettori
intracellulari}\label{i-recettori-intracellulari}

I ligandi liposolubili sono veicolati fino alle cellule bersaglio dai
liquidi circolanti (sangue e plasma), legati a speciali \emph{proteine
vettrici} per formare complessi idrofili. Per raggiungere i recettori
intracellulari, le molecole liposolubili vengono separate dalla proteina
vettrice ed attraversano la membrana della cellula bersaglio in forma
idrofobica.

I recettori intracellulari, una volta attivati, accedono al DNA
nucleare, sul quale operano come \emph{fattori di trascrizione}. Essi
infatti facilitano o inibiscono la trascrizione di particolari geni
bersaglio, quindi la sintesi di nuove proteine (sia strutturali che
enzimatiche).

Tra le molecole lipofile troviamo:

\begin{itemize}
\itemsep1pt\parskip0pt\parsep0pt
\item
  gli \textbf{ormoni steroidei} e gli \textbf{steroidi}. Questi
  viaggiano nel sangue combinati con \emph{proteine carrier} e non in
  maniera libera (non sono solubili nel siero), diffondono passivamente
  attraverso la plasmamembrana e legano recettori intracellulari. Hanno
  una \emph{struttura comune tetraciclica} che è propria anche del
  colesterolo e di un idrocarburo chiamato sterano. L'estradiolo ha un
  anello aromatico che gli altri ormoni steroidei non hanno. Es.
  steroidi sessuali, corticosteroidi\ldots{};
\item
  gli \textbf{ormoni tiroidei}, la famiglia delle molecole della
  \textbf{vitamina D}, l'\textbf{acido retinoico}\ldots{}
\end{itemize}

Un'altra grande categoria di molecole segnale sono \textbf{molecole di
natura peptidica (polimeri di amminoacidi)}. Di solito non hanno grandi
dimensioni ma in alcusi casi, come l'insulina, possono essere grandi
come una proteina. L'insulina è formata da due catene di amminoacidi
unite da vari ponti disolfuro, viaggia nel sangue e può influenzare
l'attività in vari tessuti.

Anche gli ormoni luteizzante e follicolo stimolante sono proteine.

Vi sono poi i \textbf{neurotrasmettitori} che agiscono sul sistema
nervoso o neuromuscolare. Questa è una categoria molto eterogenea di
composti dove tutti condividono una caratteristica: \emph{possiedono
sempre un atomo di azoto (composti organo-azotati)}. Queste molecole
hanno la capacità di condizionare l'attività dei neuroni e delle cellule
muscolari.

Anche la caffeina e la cocaina hanno lo stesso effetto. La cocaina è una
molecola organica contenente azoto pericolosa per come agisce sul
sistema nervoso. Gli alcaloidi invece, sono molecole contenenti azoto
che vengono prodotte dalle piante e che mimano l'azione delle molecole
fisiologiche.

Vi sono poi gli \textbf{eicosanoidi}, ovvero molecole lipidiche capaci
di agire su recettori di superficie come le prostaglandine e le
prostacicline. Queste molecole agiscono come \emph{segnali paracrini} o
\emph{autocrini} stimolando una varietà di risposte fra cui
l'aggregazione delle piastrine, la risposta infiammatoria e la
contrazione della muscolatura liscia. Hanno una notevole importanza
nelle situazioni al confine tra fisiologia e patologia.

La molecola capostipite di tutte queste sostanze è l'\textbf{acido
arachidonico} (un acido grasso tetrainsaturo presente nelle membrane
cellulari); questa molecola si può facilmente ripiegare su se stessa
(grazie ai doppi legami situati al centro della molecola) e offrire
l'azione a degli enzimi, in particolare le \textbf{cicloossigenasi
(COX)}.

Le COX producono una molecola ciclia a partire dall'acido arachidonico:
le \textbf{prostaglandine}. Queste sono alla base di tutti gli
eicosanoidi, come per esempio i \emph{tromboxani} e le
\emph{prostacicline}. La COX è l'enzima chiave per la produzione di
tutte queste molecole e siccome alcune sono mediatori dell'infiammazione
(scatenano l'infiammazione), la ciclo ossigenasi è un bersaglio anche
dei farmaci antinfiammatori. Un farmaco antinfiammatorio di largo uso è
l'aspirina (il cui principio attivo è l'acido salicilico o
acetil-salicilico per far durare di più l'effetto della molecola) che è
un inibitore della cicloossigenasi.

Le prostaglandine scatenano l'infiammazione con annesso dolore perché
vengono attivate anche le terminazioni nervose del dolore, si formano il
rossore e il gonfiore a causa della vasocostrizione. L'aspirina può
lenire questi sintomi.

L'aspirina inibisce anche l'aggregazione delle piastrine e la
coagulazione del sangue (inibizione dei tromboxani).

Come mai queste molecole agiscono solo su alcuni tipi cellulari e non su
tutti? Perchè hanno bisogno di recettori che vengono espressi solo da
alcuni tipi cellulari.

\textbf{NOTA BENE:} il termine recettore è utilizzato sia per indicare
le molecole che interagiscono con molecole segnale, sia per indicare
degli organi o delle cellule presenti negli organi sensoriali (es. coni
e bastoncelli sono recettori visivi).

\subsubsection{I recettori di membrana}\label{i-recettori-di-membrana}

Molti ligandi esplicano la loro azione legandosi a recettori chimici che
sono \emph{glicoproteine intrinseche} della membrana plasmatica delle
cellule bersaglio.

Per poter funzionare un recettore membranale deve comprendere almeno due
domini:

\begin{itemize}
\itemsep1pt\parskip0pt\parsep0pt
\item
  un \textbf{dominio recettoriale} che presenta uno o più siti di legame
  per la molecola segnale;
\item
  un \textbf{dominio effettore} che, dopo essere stato attivato dalla
  formazione del complesso ligando-recettore, innesca la risposta
  cellulare.
\end{itemize}

Esistono 3 classi di recettori membranali:

\begin{itemize}
\itemsep1pt\parskip0pt\parsep0pt
\item
  recettori \textbf{ionotropici} o recettori legati a \textbf{canali
  ionici};
\item
  recettori \textbf{metabotropici}. Questi si dividono in due gruppi:
\end{itemize}

\begin{enumerate}
\def\labelenumi{\arabic{enumi}.}
\itemsep1pt\parskip0pt\parsep0pt
\item
  recettori legati alle \textbf{proteina G} (o a 7 domini
  transmembranari);
\item
  recettori operanti per \textbf{via enzimatica};
\end{enumerate}

\begin{itemize}
\itemsep1pt\parskip0pt\parsep0pt
\item
  la classe delle \textbf{proteine adesive}.
\end{itemize}

Per ognuna di queste categorie abbiamo moltissimi recettori diversi.

\paragraph{Recettori-canale}\label{recettori-canale}

Questi recettori, quando sono attivati dal ligando extracellulare,
aprono nella loro molecola un condotto transmembranario che consente il
transito di ioni, determinando una pronta \textbf{variazione del
potenziale di membrana}. La molecola di questi recettori presenta una
\textbf{porzione recettrice} esposta al lato extracellulare della
membrana e dotata di uno o più \emph{siti di legame} per la molecola del
ligando, ed una \textbf{porzione effettrice}, costituita da un
\emph{canale ionico} che attraversa tutto lo spessore della membrana.

Il canale possiede almeno un \emph{gate} che ne controlla lo stato di
apertura/chiusura provvisto di un \emph{filtro di selettività} che lo
rende permeabile solo a determinate specie ioniche.

Questi canali agisocno facendo passare una corrente ionica poichè,
quando attivati, permettono il passaggio di ioni attraverso la membrana
(diffusione facilitata).

I tre tipi di recettori-canale meglio conosciuti sono:

\begin{itemize}
\itemsep1pt\parskip0pt\parsep0pt
\item
  i recettori per l'\textbf{acetilcolina}, alla cui attivazione è
  affidata la trasmissione sinaptica. Il canale ionico che essi
  costituiscono è relativamente poco selettivo, perchè percorribile
  dalla maggior parte dei cationi (Na+, K+, Ca2+, Mg2+) presenti nei
  liquidi fisiologici;
\item
  i recettori per l'\textbf{acido glutammico};
\item
  i recettori per l'\textbf{acido gamma-amminobutirrico}.
\end{itemize}

\paragraph{Recettori accoppiati a proteine
G}\label{recettori-accoppiati-a-proteine-g}

Queste sono \emph{proteine intrinseche di membrana}. La loro larga
diffusione nella membrana di tutte le cellule è in accordo con la
capacità che essi hanno di attivare/inibire una grande varietà di
processi intracellulari.

La sequenza di eventi intracellulari che fa seguito all'attivazione di
questi recettori si svolge in questo modo:

\begin{enumerate}
\def\labelenumi{\arabic{enumi}.}
\itemsep1pt\parskip0pt\parsep0pt
\item
  il recettore, attivato dal ligando extracellulare, comunica il segnale
  ad una proteina G trimerica che trasferisce l'attivazione all'enzima
  produttore del secondo messaggero. Nel più comune dei casi l'enzima su
  cui agisce la proteina G trimerica è l'\emph{adenilato ciclasi} ed il
  secondo messaggero che viene prodotto è l'\emph{adenosin-monofosfato
  ciclico};
\item
  in alcuni casi i secondi messaggeri sono capaci di attivare
  direttamente canali ionici o di aumentare la concentrazione
  intracellulare degli ioni Ca\(^2\)\(^+\), ma più frequentemente
  attivano una \textbf{protein-chinasi}, che diviene capace di rendere
  operative (per \emph{via fosforilativa}) le proteine-bersaglio.
\end{enumerate}

\textbf{IMMAGINE p137 (5-6)}

Le \textbf{protein-chinasi} costituiscono un'ampia famiglia di
\emph{proteine fosforilanti}, destinate primariamente a regolare
l'attività delle proteine-bersaglio attuatrici delle varie risposte
cellulari.

Tra i ligandi che attivano i recettori accoppiati a proteine G
trimeriche troviamo ormoni, neuritrasmettitori\ldots{}

Le proteine G sono formate da 7 segmenti transmembranari dei recettori
``serpentini'' sono connessi a 6 anse: 3 extracellulari e 3
intracellulari. La terza ansa intracellulare contiene il \emph{dominio
di interazione con la proteina G trimerica}.

L'estremità N-terminale della catena polipeptidic si sviluppa
nell'ambiente extracellulare e contiene, oltre a diversi siti di
possibile glicosilazione, il \emph{dominio recettoriale} contenente i
\emph{siti di legame} per il ligando. L'estremità C-terminale si estende
nel citosol e contiene diversi \emph{siti regolatori}.

Recettori di questo tipo sono ad esempio i recettori \(\beta\)
dell'adrenalina.

\textbf{IMMAGINE p137 (5-7)}

\paragraph{Recettori operanti per via
enzimatica}\label{recettori-operanti-per-via-enzimatica}

Questi recettori metabotropici sono proteine membranali che presentano
tutte un \emph{dominio recettoriale} che sporge al lato
\emph{extracellulare} come un'antenna, ed un \emph{dominio effettore}
che sporge al lato \emph{intracellulare}; i due domini sono connessi da
un \emph{singolo segmento transmembranario}.

Il dominio effettore ha già in sé la funzione di \emph{enzima} e può
attivare direttamente una proteina-bersaglio senza l'intermediazione di
proteine G trimeriche.

Nella maggior parte dei casi l'attività enzimatica del dominio effettore
è di tipo protein-chinasico, capace di fosforilare le proteine-bersaglio
su residui di \emph{tirosina}. I recettori che presentano queste
caratteristiche sono chiamati \textbf{recettori tirosin-chinasici} e
sono i più numerosi.

Recettore ad attività tirosinachinasica sono in grado di fosforilare la
tirosina presente sulla proteina (gli altri siti di fosforilazione delle
proteine sono la \emph{serina} e la \emph{treonina}); i recettori si
possono autofosforilare (fosforilano la tirosina dello stesso
recettore). Un altro recettore fosforilato è il guanilatociclasico
(recettore che produce GMP, guanosinmonofosfato ciclico).

Possiamo avere anche delle moleocle che inibiscono i recettori.

Una molecola che si lega ad un recettore e \emph{attiva il sistema di
trasduzione del segnale} viene detta \textbf{agonista}. Una molecola che
si lega ad un recettore ma \emph{non lo attiva}, cioè non genera un
segnale nella cellula è definita \textbf{antagonista}.

Questo è un fatto molto importante in farmacologia, perché può esserci
interesse nell'attivare o inattivare un recettore nelle terapie.

Ad esempio nel morbo di Parkinson (malattia neurale che ha a che fare
con i movimenti), la cura più utilizzata consiste nell'uso di agonisti
dei recettori della dopamina (recettori presenti nei neuroni che legano
la dopamina, un neurotrasmettitore tra i più abbondanti nel sistema
nervoso). La malattia è caratterizzato da una deficienza nei sistemi
dopaminergici. Gli agonisti sopperiscono alla carenza di dopamina e
mantengono in vita le funzioni neuronali che nel morbo tendono a
spegnersi.

Sempre sui recettori della dopamina c'è un esempio in cui si interviene
tramite antagonismo: la schizofrenia - sindrome bipolare (malattia
nervosa che influenza il comportamento del soggetto). In questo caso il
problema nasce da un'iperattività dei recettori dopaminergici che fa
insorgere i problemi legati a questa sindrome ed è utile usare
antagonisti della dopamina. Naturalmente possono esserci effetti
collaterale (si interferisce con i meccanismi di controllo motorio).

\subsubsection{I recettori
endocellulari}\label{i-recettori-endocellulari}

I recettori endocellulari sono ubicati nel citoplasma o nel nucleo,
accessibili soltanto ai messaggeri extracellulari \emph{liposolubili}
che possono superare la membrana della cellula bersaglio.

I ligandi liposolubili sono veicolati fino alle cellule bersaglio dai
liquidi circolanti (es. sangue), legati a speciali \emph{proteine
vettrici} per formare complessi idrofili. Per raggiungere i recettori
intracellulari, le molecole liposolubili vengono separate dalla proteina
vettrice ed attraversano la membrana della cellula bersaglio in forma
idrofobica.

I recettori intracellulari, una volta che siano stati attivati dallo
specifico ligando, \emph{accedono al DNA nucleare}, sul quale
\emph{operano come fattori di trascrizione}. Essi infatti facilitano o
inibiscono la trascrizione di particolari geni bersaglio, quindi la
\emph{sintesi di nuove proteine}, sia strutturali che enzimatiche.

La molecola dei recettori intracellulari comprende 3 domini
fondamentali:

\begin{enumerate}
\def\labelenumi{\arabic{enumi}.}
\itemsep1pt\parskip0pt\parsep0pt
\item
  un \emph{dominio recettoriale} (di legame al ligando) lipofilo,
  situato al lato C-terminale;
\item
  un breve \emph{dominio effettore} (di legame al DNA), che occupa la
  parte centrale della catena polipeptidica;
\item
  un \emph{dominio regolatore} (di regolazione, idrofilo) tramite il
  quale può essere modificata l'affinità del recettore per il ligando.
\end{enumerate}

Le molecole dei recettori intracellulari, in assenza del legame con
l'ormone, restano inattive perchè vincolare ad una \emph{proteina
inibitrice}. Quando la molecola ormonale si combina col dominio
recettoriale, il complesso inibitore si distacca e ciò attiva il
recettore, rendendo disponibili nel dominio effettore i siti di legame
per il DNA o permetendo la traslocazione del recettore dal citoplasma al
nucleo.

Il complesso recettore-ormone si lega allora, all'interno del nucleo, a
specifiche sequenze di DNA denominate \emph{elementi di risposta
all'ormone} (\textbf{HRE}, Hormone Responsive Elements). Gli HRE
risiedono sempre nella \emph{regione promoter}, quindi \emph{a monte}
del gene che presiede alla sintesi della proteina richiesta dal
messaggio ormonale.

I recettori intracellulari si distinguono in: 1. \textbf{recettori
citosolici} specifici per gli ormoni della corteccia surrenale; 2.
\textbf{recettori nucleari} che comprendono i recettori: a. per gli
\emph{ormoni sessuali} (androgeni, estrogeni e progesterone); b. per gli
\emph{ormoni tiroidei}; c. per le vitamine D e A; d. per l'\emph{acido
retinoico}

\section{La trasduzione del segnale}\label{la-trasduzione-del-segnale}

Un tipico meccanismo della trasduzione del segnale è l'attivazione di
una \emph{``cascata di fosforilazione''}. In queste cascate si attivano
in serie delle protein-chinasi che fosforilano altre proteine una dopo
l'altra.

\subsection{La via dell'adenosin-monofosfato ciclico
(cAMP)}\label{la-via-delladenosin-monofosfato-ciclico-camp}

Il cAMP agisce da secondo messaggero. Si forma
dall'\textbf{adenosin-trifosfato} (ATP) per eliminazione di
\emph{pirofosfato}. Questa reazione richiede, in presenza di ioni Mg2+,
l'intervento di un enzima membranale: l'\textbf{adenilato ciclasi}.

Dal luogo di formazione il cAMP raggiunge per diffusione le strutture
intracellulari sulle quali deve esplicare la sua azione. Questa è
generalmente \emph{fasica} (cioè limitata nel tempo): dopo l'arrivo del
messaggero extracellulare che ne ha evocato la formazione, il livello
citosolico di cAMP \emph{aumenta di 5-10 volte in 5-10 secondi} per la
rapida sintesi di nuove molecole, controbilanciata in breve tempo da una
degradazione altrettanto rapida. La demolizione avviene ad opera della
\textbf{cAMP fosfodiesterasi}, enzima che idrolizza il cAMP ad AMP.

\subsubsection{L'adenilato ciclasi (AC)}\label{ladenilato-ciclasi-ac}

L'adenilato ciclasi è costituita da una molecola di notevoli dimensioni.
La sua lunga catena polipeptidica è inserita nella membrana plasmatica
con \emph{due subunità} di 6 segmenti idrofobici; questi danno origine
ad un \emph{unico dominio catalitico} che pesca nel citosol, dove si
estendono anche le due estremità dell'intera molecola.

Il dominio catalitico dell'AC è una struttura anulare allungata,
contenente al suo interno il \emph{sito ATPasico}, mentre ai due poli
esterni dell'anello si trovano i \emph{siti di legame} per le subunità
alfa delle proteine G di controllo.

\subsubsection{Proteine G trimeriche ed attivazione
dell'AC}\label{proteine-g-trimeriche-ed-attivazione-dellac}

Nello svolgere la sua azione di accoppiamento tra un recettore a 7
segmenti transmembranari e l'adenilato ciclasi, la molecola di una
proteina G \emph{trimerica} mantiene un costante rapporto con il
foglietto interno della membrana cellulare grazie alla presenza di acidi
grassi a lunga catena legati alle subunità alfa e gamma.

La proteina G ha una struttura trimerica, cioè con \emph{3 subunità
(alfa, beta, gamma)}, debolmente vincolate allo strato fosfolipidico
membranale da un \emph{gruppo prenilico} appartenente alla subunità
gamma.

La subunità alfa dispone di \emph{3 siti di legame}:

\begin{itemize}
\itemsep1pt\parskip0pt\parsep0pt
\item
  uno per il recettore a 7 segmenti transmembranari;
\item
  uno per il nucleotide guanilico (GTP o GDP);
\item
  uno per la molecola dell'adenilato ciclasi.
\end{itemize}

Inoltre ha la facoltà di staccarsi dal complesso beta/gamma e di
migrare, nella forma attivata, fino a raggiungere la molecola dell'AC ed
agire su di essa.

L'intero processo avviene in forma ciclica:

\begin{enumerate}
\def\labelenumi{\arabic{enumi}.}
\itemsep1pt\parskip0pt\parsep0pt
\item
  \textbf{condizione di riposo:} il ligando non ha ancora raggiunto ed
  attivato il recettore a 7 domini transmembranari, la subunità alfa è
  ancora legata al complesso beta/gamma ed è inattivata (si trova nella
  forma G\_alfa-GDP);
\item
  \textbf{innesco dell'attivazione:} il ligando raggiunge il recettore
  causando la sostituzione GDP -\textgreater{} GTP. Il terzo gruppo
  fosfato del GTP provoca una variazione conformazionale della subunità
  alfa che coinvolge due amminoacidi ``switch'': una glicina e una
  treonina. Questo causa il distacco della subunità alfa-GTP dal
  complesso beta/gamma e rende la subunità alfa libera di migrare fino a
  raggiungere l'AC; questa, attivata, trasforma l'ATP in cAMP;
\item
  \textbf{ritorno alla condizione di riposo:} il legame alfa-GTP con
  l'AC stimola l'azione GTPasica della subunità alfa, che fa passare
  alfa-GTP nella forma inattiva alfa-GDP (con liberazione di Pi). A
  questo punto alfa-GDP si distacca dall'AC, si lega nuovamente al
  complesso beta/gamma e ripristina la condizione di quiescenza.
\end{enumerate}

\textbf{IMMAGINE p147}

Dopo che la subunità alfa ha attivato la proteina bersaglio si
auto-disattiva idrolizzando il GTP. La subunità alfa si dissocia dalla
proteina bersaglio e si ricombina col complesso beta-gamma,
ricostituendo la proteina G inattiva.

\textbf{NOTA BENE}: in questo ciclo non esiste un vero punto di partenza
e di arrivo. La molecola segnale è l'innesco ma in realtà, considerando
l'organismo intero, ci rendiamo conto che anche la molecola segnale è
inserita in un complesso di eventi causa effetto che ritorna sempre al
punto di partenza.

Il recettore attivato viene disattivato tramite:

\begin{itemize}
\itemsep1pt\parskip0pt\parsep0pt
\item
  fosforilazione da parte della chinasi del recettore legato a proteina
  G (GRK);
\item
  legame di una molecola arrestina;
\item
  endocitosi del recettore.
\end{itemize}

Il cAMP ha un'influenza notevole sull'attivazione delle cellule: induce
le cellule a fare ``cose in più'' oltre al metabolismo basale.

La \textbf{fosfodiesterasi} è l'enzima capace di degradare il cAMP così
che non raggiunga concentrazioni troppo elevate che porterebbero la
cellula fuori dall'omeostasi.

Viene detto \emph{``ciclico''} poiché l'unico fosfato presente nel AMP
forma un ponte con un idrossile del ribosio formando un anello dovuto al
fosfato che ciclizza con l'idrossile.

La fosfodiesterasi rompe questo anello. Una forma equivalente si trova
in presenza di guanina invece che adenina (cGMP).

In una cellula nervosa, in presenza di serotonina (il SN presenta
recettori a proteina G per la serotonina) si nota la produzione di cAMP.

\subsubsection{Com'è che il cAMP attiva le
cellule?}\label{comuxe8-che-il-camp-attiva-le-cellule}

Il cAMP è in grado di ``mettere in moto'' il metabolismo cellulare.

Questo perchè il cAMP attiva la \textbf{protein chinasi A (PKA)}. La
chinasi forma un compleso con un'altra proteina, l'AMP ciclico si
attacca a quest'altra proteina staccandosi dalla chinasi che una volta
libera può attivarsi.

La PKA può attivare una cascata di fosforilazione.

\textbf{Es:} adrenalina. Si attacca al recettore beta-adrenergico che
lega una proteina G la quale attiva l'adenilato ciclasi che forma cAMP
la quale attiva la fosforilasi chinasi che a sua volta attiva la
fosforilasi (passa dalla conformazione b alla a) (?) Questi processi
devono essere veloci.

Una molecola, rispetto ad una cellula, è molto piccola, quindi i
recettori che reagiscono con i ligandi sono anch'essi molto piccoli.
Ogni cellula possiede molte molecole di un certo recettore localizzate
sulla superficie. Questi eventi avvengono tutti sulla superficie. Per
far sì che una cellula si metta in movimento occorre influenzare il
volume della cellula.

La cascata di fosforilazione può essere interpretato come
un'amplificazione del segnale.

Funzioni della PKA: attivazione di fattori di trascrizione. La PKA
attiva anche il CREB (un fattore di trascrizione) che si lega al suo
elemento regolatore-dipendente che è un promotore di un gene.

Mediante fosforilazione a cascata si ha una velocizzazione del processo.
Il meccanismo a proteina G si è evoluto sia nell'attivazione che
nell'inibizione. Le proteine G non agiscono solo tramite AMP ciclico ma
anche tramite altri fattori come GMP ciclico, calcio, ecc\ldots{} Ci
sono malattie legate a disfunzioni della proteina G, come colera e
pertosse.

\subsubsection{I recettori legati ad attività
enzimatica}\label{i-recettori-legati-ad-attivituxe0-enzimatica}

I recettori legati ad attività enzimatica hanno un aspetto comune
composto da 3 porzioni: + un'ampia \textbf{porzione extracellulare}
(lega la molecola segnale); + un \textbf{dominio transmembrana}; + una
\textbf{porzione intracellulare} con attività tirosin chinasica.

Questi recettori legano fattori di crescita (è un po' diverso il
recettore per l'insulina).

I \textbf{recettori tirosin chinasici} legano ormoni peptidici
idrosolubili. Il ligando stimola l'attività enzimatica del recettore che
a sua volta induce una cascata di fosforilazione. Questi recettori sono
implicati nella proliferazione e nel differenziamento cellulare,
promuovo la sopravvivenza delle cellule.

Dopo aver legato il ligando, il recettore si dimerizza e si fosforilano
a vicenda; a questi dimeri si aggregano altre proteine il cui nucleo di
partenza è formato dall'insieme di \textbf{ras}, \textbf{grb2} e
\textbf{sos} e produce la cascata di fosforilazione. Il punto finale
della cascata è l'attivazione di geni per un fattore di trascrizione.

\subsubsection{Il metabolismo del cGMP}\label{il-metabolismo-del-cgmp}

\textbf{cGMP} = guanosin monofosfato ciclico

Questa molecola deriva dal GTP ad opera di due enzimi:

\begin{itemize}
\itemsep1pt\parskip0pt\parsep0pt
\item
  \textbf{guanilato ciclasi solubile (sGC)};
\item
  \textbf{guanilato ciclasi di membrana (pGC)}.
\end{itemize}

Il GTP viene degradato a GMP dalla \textbf{fosfodiesterasi} (è una
famiglia di enzimi che degradata abbastanza indifferentemente cAMP e
cGMP). Il cGMP attiva il metabolismo, ne troviamo forme sia di membrana
che solubili. Il cGMP agisce da secondo messaggero.

\subsection{Via di segnale dipendente da
NO/cGMP}\label{via-di-segnale-dipendente-da-nocgmp}

Il \textbf{GMP ciclico} (Guanosin-monofosfato ciclico o cGMP) è un
secondo messaggero generato dalla ciclizzazione di una molecola di
\emph{guanosin-trifosfato (GTP)} ad opera due diversi enzimi:

\begin{itemize}
\itemsep1pt\parskip0pt\parsep0pt
\item
  il primo è la \emph{guanilato ciclasi di membrana} intrinseca ad un
  recettore esterno per il peptide natriuretico atriale, che attiva il
  dominio interno che poi produce GMP ciclico;
\item
  il secondo è la \textbf{guanilato ciclasi citosolica (GC)}. Questa è
  una proteina formata da 2 subunità e contenente un gruppo eme (con
  ferro Fe\(^2\)\(^+\)) collegato ad un amminoacido istidina della
  subunità beta e a 4 atomi di azoto.
\end{itemize}

Questo enzima agisce dopo l'attivazione da parte di monossido d'azoto
(NO) che entra nella cellula per diffusione. Il NO può anche essere
prodotto dall'enzima \textbf{NO sintasi (NOS)}. Il NO lega un atomo di
Fe del gruppo eme inserito tra le subunità alfa e beta.

Lo ione ferroso dell'eme è essenziale per l'attivazione dell'enzima.

L'ossigeno del NO attira il Fe\(^2\)\(^+\) del gruppo eme carico
positivamente apportando una modifica della struttura dell'eme che
induce una modifica sulla proteina tramite l'istidina.

La modifica conformazionale dell'enzima induce attivazione del sito
attivo dell'enzima e quindi produzione di cGMP da guanosin trifosfato.

Uno degli effetti di questo meccanismo è la vasodilatazione. Il NO è
infatti il più potente vasodilatatore conosciuto (combatte
l'ipertensione arteriosa).

Il meccanismo di vasodilatazione:

\begin{itemize}
\itemsep1pt\parskip0pt\parsep0pt
\item
  Il NO viene prodotto da cellule endoteliali in risposta ad
  acetilcolina, GABA, adrenalina; I vasi sono formati da cellule
  endoteliali che li circondano e da fibre muscolari che ne causano la
  dilatazione e la contrazione;
\item
  diffonde dalle cellule endoteliali al muscolo liscio;
\item
  nelle cellule muscolari attiva la guanilato ciclasi;
\item
  la GC produce cGMP che attiva una pompa SERCA;
\item
  si abbassa la concentrazione di Ca\(^2\)\(^+\) nella cellula;
\item
  le fibre del muscolo liscio si rilassano e quindi si ha
  vasodilatazione.
\end{itemize}

Tra i farmaci storici utilizzati come vasodilatatori il NO è un
componente della \emph{nitroglicerina} che veniva usata contro angina e
infarto (la niutroglicerina venendo metabolizzata accentua la produzione
di monossido di azoto permettendo un miglioramento della pressione
sanguigna coronarica).

L'infarto del miocardio è dovuto a insufficiente dilatazione coronarica.

La nitroglicerana è trinitrossidata.

\section{Il potenziale di membrana}\label{il-potenziale-di-membrana}

Negli esseri viventi esistono dei fenomeni elettrici.
L'elettrocardiogramma è una registrazione di un'attività elettrica;
rende evidente l'attività elettrica complessiva del cuore che genera un
campo elettrico così forte che può essere registrato con degli elettrodi
appoggiati sulla superficie del corpo. Tutte le cellule presentano
attività elettrica. Questa è legata alla sopravvivenza stessa delle
cellule.

Le cellule sono generatori di elettricità. Il campo elettrico generato
da un potenziale elettrico mette in moto tutti i fenomeni elettrici
della cellula. Il potenziale elettrico si stabilisce a cavallo della
membrana cellulare che risulta polarizzata. La cellula presenta un polo
negativo ed uno positivo e può indurre correnti elettriche e dunque
agire da generatore elettrico.

Le proprietà elettriche degli esseri viventi sono rese possibili dal
fatto che gli esseri viventi vivono in soluzioni acquose. Qui sono
presenti ioni minerali (e organici, anche se non creano fenomeni
elettrici) in concentrazioni tali da generare fenomeni elettrici. Gli
gli ioni sodio, cloro e potassio creano il potenziale elettrico e sono
gli elementi basilari dell'attività elettrica delle cellule. In realtà
ci sarebbero anche gli ioni calcio in quanto anch'essi producono
fenomeni elettrici ma solo in conseguenza al potenziale elettrico.

Poiché gli ioni sono carichi elettricamente se un gradiente ionico si
stabilisce a cavallo della membrana si può generare un gradiente
elettrochimico.

Lo ione potassio è più concentrato all'esterno che all'interno della
cellula, così come il sodio. Il cloro invece è più concentrato
all'interno che all'esterno.

Le membrane cellulari presentano un \emph{potenziale di equilibrio} ed
un \emph{potenziale di diffusione}.

Perchè si formi un \textbf{potenziale di equilibrio} devono essere
presenti le seguenti condizioni:

\begin{itemize}
\itemsep1pt\parskip0pt\parsep0pt
\item
  deve essere presente una membrana che divide i due comparti;
\item
  nei comparti deve esistere almeno uno ione permeabile alla membrana ed
  uno impermeabile;
\item
  i due ioni devono presentare concentrazioni diverse ai due lati della
  membrana.
\end{itemize}

Perchè si formi un \textbf{potenziale di diffusione} devono essere
presenti le seguenti condizioni:

\begin{itemize}
\itemsep1pt\parskip0pt\parsep0pt
\item
  gli ioni devono essere presenti in concentrazioni diverse ai due lati
  della membrana;
\item
  la membrana deve presentare una permeabilità diversa al loro
  passaggio.
\end{itemize}

Il potenziale di equilibrio è statico mentre quello di diffusione è
dinamico.

Nel potenziale di equilibrio le correnti elettriche valgono zero.

Nel potenziale di diffusione sono presenti delle correnti elettriche ma
se il potenziale di diffusione è costante significa che le correnti
elettriche sono in pareggio tra le correnti in ingresso e quelle in
uscita (la somma delle correnti in ingresso è uguale a quella delle
correnti in uscita). Per questo nel sistema la corrente è nulla e il
potenziale di diffusione è costante.

Il potenziale di equilibrio è invece costante quando ci sono correnti
nulle.

Nel caso in cui una membrana sia permeabile a due ioni nello stesso
modo, anche a seguito di un momentaneo divario si otterrà una situazione
di equilibrio e non si formerà nessun potenziale elettrico. Se invece la
membrana è permeabile ad uno ione e non all'altro, lo ione permeabile si
muoverà dal lato in cui è più concentrato a quello in cui è meno
concentrato. Lo ione non permeabile invece non potrà diffondere e ogni
volta che uno ione diffonderà e attraversà la membrana creerà uno
squilibrio di cariche su entrambi i lati della membrana. Gli ioni
positivi, man mano che passano, si trovano a dover affrontare un
gradiente elettrico che li respinge dal lato da cui sono venuti. Lo
stesso ione è soggetto a due forze contrastanti che andranno ad
equivalersi fino a raggiungere uno stato di equilibrio. In questo modo
si ottiene un potenziale di membrana di equilibrio. La membrana è il
punto in cui il potenziale si viene a creare.

Un potenziale elettrico è una situazione che genera un campo elettrico.

I fenomeni fisici sono modellizzati tramite la matematica.

La cosa che caratterizza il potenziale di equilibrio è la concentrazione
dello ione permeabile allo stato di equilibrio. All'equilibrio tutti gli
ioni permeabili hanno un potenziale di equilibrio uguale al potenziale
di membrana.

E\(_i\) = V\(_m\)

Dove E\(_i\) indica il potenziale di equilibrio dello ione permeabile e
V\(_m\) il potenziale di membrana.

\subsection{L'equazione di Nerst}\label{lequazione-di-nerst}

\textbf{V = (RT/z\(_i\)F) ln {[}C\(_i\){]}\(_1\)/{[}C\(_i\){]}\(_2\)}

Dove:

\begin{itemize}
\itemsep1pt\parskip0pt\parsep0pt
\item
  \emph{z\(_i\)} è la carica dello ione presa con il suo segno;
\item
  \emph{F} è la costante di Faraday;
\item
  C\(_i\) è la concentrazione dalle 2 parti della membrana degli ioni
  permeabili.
\end{itemize}

\textbf{E\(_i\) = V\(_m\) = V = (RT/z\(_i\)F) ln
{[}C\(_i\){]}\(_1\)/{[}C\(_i\){]}\(_2\)}

Ad una temperatura di 18-20°C:

E\(_i\) = V\(_m\) = V = (0,058/z\(_i\)) ln
{[}C\(_i\){]}\(_1\)/{[}C\(_i\){]}\(_2\)

Per uno ione monovalente con carica 1, il potenziale varia di 0.0085
volt tutte le volte che il rapporto tra le concentrazioni varia di un
fattore 10.

Nella creazione del potenziale di equilibrio gli ioni permeabili sono
soggetti alla forza generata dal gradiente chimico e a quella generata
dal gradiente elettrico.

Ci saranno dunque due flussi diversi dovuti alle due forze. Il
contributo al flusso ionico del campo elettrico è dato dalla
\emph{equazione di Planck}, mentre il contributo al flusso ionico del
gradiente di concentrazione è dato dalla \emph{legge di Fick}.

(J\(_i\))\(_d\) = - D\(\alpha\){[}C\(_i\){]}

Il segno meno è una convenzione.

Dove D è il coefficiente di diffusione.

(J\(_i\))\(_e\) = -\(\mu\)\(_i\) z\(_i\) C\(\Delta\)V

Dove:

\begin{itemize}
\itemsep1pt\parskip0pt\parsep0pt
\item
  \(\mu\)\(_i\) è la motilità ionica;
\item
  \(\Delta\) è la forza che li fa spostare.
\end{itemize}

J\(_i\) = -D\(_i\)(A{[}C\(_i\){]}+ z\(_i\){[}C\$\_i{]}F/RT \(\Delta\)V)

Dove: + J\(_i\) è il flusso totale dovuto al gradiente elettrochimico e
chimico dello ione. All'equilibrio le correnti elettriche valgono 0 per
uno ione; + D\(_i\) è la costante di diffusione, sempre diversa da 0.

J\(_i\) diventa 0 quando C\(_i\) diventa 0. Questo succede quando
C\(_i\) è uguale all'opposto del secondo C\(_i\).

Quindi:

\(\delta\){[}C\(_i\){]} / \(\delta\)x = z\(_i\){[}C\(_i\){]}F / RT
\(\delta\) {[}V{]} / \(\delta\)x

Si ha un'equazione differenziale risolvibile per concentrazione o
potenziale.

Se la risolvo per potenziale avrò:

E\(_i\) = V\(_i\)\(_n\) - V\(_o\)\(_u\)\(_t\) = RT/ z\(_i\)F ln
{[}C\(_i\){]}\(_o\)\(_u\)\(_t\) / {[}C\(_i\){]}\(_i\)\(_n\)

E\(_i\) è il potenziale di Nerst.

L'equazione di Nerst è applicabile alla cellula perchè sono stati
calcolati i potenziali ionici interni ed esterni.

Inserendo i valori nella formula si trovano i potenziali di equilibrio
del potassio {[}-87,9 mV{]} e del sodio {[}54,8 mV{]}.

Il flusso dipendente da gradiente elettrochimico è la somma di flusso
imposto dal gradiente elettrico e di quello imposto dal gradiente
chimico.

Nel potenziale di diffusione si ha una situazione simile a quella
precedente ma questa volta si ha che gli ioni permeano ma con
permeabilità diverse. Si crea un divario elettrico tra i due lati perché
da un lato ci sarà un eccesso di cariche positive e dall'altro di
negative. Qui si capisce la natura fisica del potenziale con accumulo di
cariche ai due lati della membrana. C'è un affollamento eccessivo di
cariche positive su di un lato e di cariche negative sull'altro. Ciò
crea il potenziale elettrico.

Il potenzile è di membrana ma siccome tende a decadere grazie alla
diffusione si dice che è un \emph{potenziale di diffusione}. C'è un
momento di potenziale di membrana che però poi andrà a scemare.

\section{Dipendenza del potenziale di membrana dai vari
ioni}\label{dipendenza-del-potenziale-di-membrana-dai-vari-ioni}

Il potenziale di equilibrio del K\(^+\) è più vicino a quello di
membrana rispetto a quello del Na\(^+\) perchè P\(_K\) \textgreater{}
P\(_N\)\(_a\).

Se aumenta P\(_N\)\(_a\) il potenziale di membrana si sposta verso il
valore del potenziale di equilibrio del Na\(^+\).

Il potenziale di equilibrio del Cl\(^-\) è molto simile al potenziale di
membrana, quindi variazioni di P\(_C\)\(_l\) non hanno una grande
influenza sul potenziale di membrana.

Il potenziale di membrana è più vicino al potenziale di equilibrio degli
ioni maggiormente permeabili.

Se varia la permeabilità di uno ione varierà il potenziale di membrana.

Siccome il potenziale di equilibrio dello ione varia a seconda delle sue
concentrazioni, varierà anche il potenziale di membrana (la membrana è
sensibile alle concentrazioni degli ioni).

Le concentrazioni ioniche nei fluidi extracellulari e nel citosol si
mantengono praticamente sempre costanti e non influenzano il potenziale
di membrana.

Questo fattore può essere usato sperimentalmente per studi sull'attività
elettrica tra cellule o nella cellula inducendo depolarizzazione (si può
aumentare la concentrazione extracellulare del potassio).

Con il termine \textbf{depolarizzazione} si indica una \emph{diminuzione
in valore assoluto} del potenziale di membrana (il valore si sposta
verso lo 0 da valori negativi). Il modo più semplice per indurre una
depolarizzazione è aumentare la concentrazione esterna del potassio. Se
si abbassa il potenziale di equilibrio dello ione potassio si abbassa
anche il potenziale di membrana.

Se la concentrazione esterna del potassio aumenta, il potenziale di
membrana aumenta come valore numerico ma si depolarizza andando verso un
potenziale nullo.

Il potenziale di membrana di una cellula è un potenziale di diffusione e
non un potenziale di equilibrio.

Fintanto che il potenziale è stabile le correnti sono globalmente nulle,
ovvero si ha un bilancio in pareggio tra le correnti in ingresso e
quelle in uscita.

Le concentrazioni ioniche nei fluidi extracellulari e nel citosol
cellulare si mantengono praticamente sempre costanti; questo fattore può
essere usato sperimentalmente per studi sull'attività elettrica tra le
cellule o nella cellula, inducendo depolarizzazione.

Si prendono in considerazione correnti del sodio e del potassio, il
cloro non ha affetto perché ha potenziale quasi identico a quello di
membrana.

Le correnti di diffusione del cloro sono pressoché nulle, mentre le
correnti di diffusione del potassio e del sodio non sono nulle perché
hanno potenziale diverso da quello di membrana.

Una corrente negativa indica una corrente in ingresso nella cellula
(Na\(^+\)), mentre una corrente positiva indica una corrente in uscita
(K\(^+\)).

Una cellula deve trovarsi in una situazione in cui la corrente in
ingresso è data quasi totalmente da ioni sodio mentre quella in uscita è
data quasi totalmente da ioni potassio. La corrente del potassio deve
essere uguale all'opposto della corrente del sodio (I\(_K\) =
-I\(_N\)\(_a\)). Si ottiene che il rapporto delle conduttanze vale circa
11. Questo ci fa capire che la permeabilità al potassio è poco più di 10
volte quella al sodio nella membrana. Lo ione sodio è sottoposto ad una
forza maggiore del potassio. Una elevata forza elettromotrice muove gli
ioni sodio.

Gli ioni sodio sono esposti ad un forte campo elettrico, mentre quelli
potassio ad uno debole. La forte forza che agisce sugli ioni sodio va
moltiplicata per una piccola conduttanza (la \textbf{conduttanza} è
l'\emph{espressione quantitativa di un conduttore ad essere percorso da
corrente elettrica}), viceversa quella piccola degli ioni potassio viene
moltiplicata per una elevata conduttanza. Questi due prodotti forniscono
la stessa corrente.

Le correnti sono in pareggio e garantiscono l'elettroneutralità ma gli
ioni si muovono.

Entrano ioni sodio ed escono ioni potassio.

Questa è la condizione cellulare a riposo.

Finchè le correnti sono costanti, uguali e opposte come direzione, si
mantiene il potenziale di membrana.

Le concentrazioni intracellulari di sodio e potassio non rimangono
costanti a causa dei continui scambi.

Si ha dissipamento del potenziale di diffusione perché le concentrazioni
vanno ad avvicinarsi così come il potenziale di equilibrio. La
situazione è mantenuta da meccanismi che riportano gli ioni indietro.

Questo sistema è la \textbf{pompa sodio-potassio} la quale pompa ioni
sodio e potassio alla stessa velocità con cui compiono il passaggio
inverso ed impedisce che vengano dissipati quei differenziali. Questa
pompa mantiene la cellula in riposo con un potenziale di diffusione.
Inizialmente non era ben chiaro come gli ioni attraversassero la
membrana.

L'equazione di Goldman parte dal presupposto che la membrana sia un
conduttore permeato da una corrente ionica in tutta la sua matrice
fisica. In realtà gli ioni attraversano la membrana solo in punti
precisi ossia pori realizzati in particolari proteine chiamate
\textbf{canali ionici}.

I canali ionici sono formati da proteine collocate nella membrana che
sporgono da entrambi i lati. Alcuni canali ionici sono sempre aperti
(es. canali del cloro e del potassio, questi sono gli ioni maggiormente
permeabili), altri sono canali \textbf{``gate''} che possono essere
aperti e chiusi da operatori che possono essere di natura diversa.

Si distinguono tre casi:

\begin{itemize}
\itemsep1pt\parskip0pt\parsep0pt
\item
  \emph{canali operati da ligando} (si comportano come i recettori) in
  cui una molecola segnale interagisce con il canale e lo fa aprire;
\item
  \emph{canali operati da forza meccanica} come stiramenti e estensioni
  applicate alla membrana cellulare;
\item
  \emph{canali operati da voltaggio e da potenziale di membrana}, si
  aprono e si chiudono a seconda del potenziale di membrana in
  particolari momenti in cui questo effettua delle variazioni.
\end{itemize}

I \textbf{canali del sodio voltaggio-dipendenti}. Possono essere pensati
come una sorta di tubo realizzato da tanti passaggi trasmembrana
organizzati a moduli (o domini) costituiti da 6 passaggi transmembrana.

In questi domini ci sono sporgenze extracellulari, motivi peptidici che
sporgono, e che possono occludere in maniera efficace o non il poro
centrale.

Uno dei passaggi transmembrana di ciascun dominio è un sensore del
voltaggio. Questa è una porzione della molecola in grado di effettuare
una modificazione conformazionale se cambia il potenziale di membrana. È
una porzione ricca di cariche che risentono della variaziano del
potenziale di membrana producendo un riorientamento della parte della
molecola che chiude o riapre il poro.

Il canale è chiuso o aperto a seconda del valore del potenziale di
membrana. I canali si aprono quando la cellula si depolarizza passando
da valori negativi a valori che vanno verso lo zero.

I \textbf{canali del potassio} sono un po' più complessi e ne esistono
di vari tipi.

Alcuni canali del potassio sono formati da \emph{tetrameri} ossia canali
costiutiti da 4 subunità che formano un poro centrale. Canali di questo
tipo sono voltaggio o calcio dipendenti.

I \textbf{canali leak} sono canali formati da dimeri e sono tipici della
cellula a riposo.

I \textbf{canali del potassio rettificanti anomali (o inward
rectifiers)} sono canali del potassio a 2 \(/alpha\) -eliche
transmembrana (2 STM) che permangono aperti in condizioni di
iperpolarizzazione della membrana e si chiudono quanto questa si
depolarizza. Sono tipici delle cellule muscolari cardiache, stabilizzano
il potenziale quando la membrana è nello stato di riposo, ma quando uno
stimolo sovrasoglia induce un potenziale d'azione, i canali si chiudono
e permettono allo stimolo una durata maggiore. Fanno parte di questa
famiglia i canali del potassio attivati da proteine G.

In tutte queste situazioni so realizza un poro centrale che può essere
chiuso o aperto.

La cellula riesce ad avere proprietà elettriche mantenendo allo stesso
tempo un equilibrio osmotico. L'acqua può infiltrarsi attraverso lo
strato lipidico ma la permeabilità dell'acqua può essere maggiore se
nella membrana si trovano le aquaporine.

\subsection{Equilibrio di Donnan}\label{equilibrio-di-donnan}

Frederick Donnan ha dimostrato che tra due soluzioni acquose separate da
una membrana che sia \emph{impermeabile ad uno solo dei soluti} si
stabilisce un equilibrio garantito da una \emph{differenza di
potenziale} transmebranaria.

Una conseguenza dell'equilibrio di Donnan è che tra i due compartimenti
si stabilisce una \emph{differenza di pressione osmotica}, maggiore nel
compartimento contenente lo ione non diffusibile.

All'equilibrio la membrana assume potenziale di equilibrio.

Il prodotto degli ioni diffusibili ad un lato della membrana è uguale al
prodotto degli ioni diffusibili dall'altro lato della stessa.

All'equilibrio le soluzioni sono elettroneutre. La somma delle cariche
positive è uguale alla somma delle cariche negative. La somma delle
concentrazioni delle sostanze diffusibili da una parte è maggiore di
quella delle stesse dall'altra. Un settore è quello che sta dentro la
cellula, mentre l'altro è quello che sta fuori.

Se si applica alla cellula il modello precedente, si ottiene un
equilibrio con potenziale di membrana e una concentrazione osmotica
intracellulare maggiore di quella del mezzo esterno. In questo modo la
cellula raggiunge l'equilibrio di Donnan e va in equilibrio
elettrochimico, ma la non in equilibrio osmotico. Di conseguenza, per
raggiungere una situazione di equilibrio osmotico, la cellula tenderebbe
a rigonfiare, perdendo però l'equilibrio di Donnan. Questo modello
porterebbe a lisi cellulare.

Abbiamo dunque una continua oscillazione fra i due equilibri.
Introducendo il sodio (Na) extracellulare la cellula raggiunge
l'equilibrio di Donnan. Così ci sono ioni non diffusibili intra e ioni
non diffusibili extracellulari. Si parla di doppio equilibrio di Donnan.
Ci sono cariche diffusibili e non.

L'aggiunta di Na rappresenta un controbilanciamento degli anioni
intracellulari non diffusibili. Il doppio equilibrio non è ancora la
situazione reale della cellula perché non prende in considerazione il
fatto che si tratta di un \emph{potenziale di diffusione} (visti i
continui spostamenti).

Si parla di modello \textbf{``pump and leak''}. Questo è il modello
fisiologico della cellula vicino a quello di Donnan ed è un potenziale
di diffusione mediato da movimenti ionici attivi dovuti alla pompa Na-K
e mediato da correnti.

Essendo gli ioni diversamente permeabili esiste una separazione di
cariche strettamente prossima alla membrana. L'interno presenta una
maggiore quantità di cariche negative mentre l'esterno presenta una
maggiore quantità di cariche positive.

\subsubsection{Regolazione del volume
cellulare}\label{regolazione-del-volume-cellulare}

Le cellule regolano il volume cellulare grazie a trasportatori di vario
tipo. Se il mezzo è ipotonico aumentano il volume rigonfiando, mentre se
il mezzo è ipertonico raggrinziscono.

Dopo una prima fase di aggiustamento isosmotico tornano al volume
iniziale.

Questo è un fenomeno fondamentale: la cellula risponde alla variazione
osmotica mettendo in atto degli accorgimenti che le permettono di
tornare al volume iniziale.

Questi accorgimenti vengono chiamati \textbf{RVD} (Regulatory Volume
Decrease), è un fenomeno che segue all'esposizione della cellula ad una
soluzione ipotonica, e \textbf{RVI} (Regulatory Volume Increase), è un
fenomeno che segue all'esposizione della cellula ad una soluzione
ipertonica.

Con l'RVD la cellula espelle osmoliti e perde acqua riducendo il proprio
volume (sfrutta la permeabilità ai canali potassio e cloro, gli ioni più
permeabili). L'RVI invece, porta all'acquisizione di osmoliti per
acquisire acqua tramite l'aumento della permeabilità al sodio (facendoli
entrare all'interno della cellula), ma anche altri ioni per mezzo di
cotrasportatori sodio-dipendenti.

Gli ioni entrano e la pompa sodio-potassio porta fuori Na\(^+\) e dentro
K\(^+\) per mantenere l'equilibrio. Nei trasportatori transepiteliali le
cellule lavorano sul lato basolaterale in RVD, sul lato apicale con RVI
sfruttando il movimento di ioni da un lato all'altro dell'epitelio.

Nell'elettrofisiologia si tende a considerare una cellula come un
circuito elettrico. Si considerano potenziale, corrente e conduttanza
insieme alla capacità. La \textbf{capacità} è una caratteristica che
impegna cariche in alternativa alla resistenza, ed è una caratteristica
della membrana.

La membrana è assimilabile ad un \emph{conduttore con una}
\textbf{resistenza} ma anche ad un condensatore con una capacità. Il
condensatore accumula cariche su una superficie, proprio come è in grado
di fare la membrana cellulare. Queste cariche non generano corrente ma
sono ``catturate'' dalla membrana.

La \textbf{resistenza} è la matrice attraverso cui passano le cariche
che generano corrente (sono i canali ionici della cellula).

Le proprietà elettriche delle cellule sono studiate in
elettrofisiologia. L'elettrofisiologia stimola le cellule
elettricamente, effettua delle misurazioni e ne ricava dei dati sulle
proprietà elettriche delle cellule.

Le modalità operative utilizzate sono due:

\begin{itemize}
\itemsep1pt\parskip0pt\parsep0pt
\item
  \textbf{current clamp}, significa \emph{``a corrente bloccata''}. In
  questo caso si lavora con cellule vive (in genere con buone correnti
  come i neuroni e le cellule muscolari) normalmente isolate, mantenute
  in coltura e montate su un vetrino in una camera posta su un
  microscopio collegato a \emph{micromanipolatori}. All'interno delle
  cellule in esame si inseriscono degli elettrodi (tubicini di vetro
  pieni di una soluzione che conduce elettricità), e la cellula viene
  posta in un circuito. L'elettrodo presenta una punta inserita nella
  cellula e una punta inserita nel bagno di perfusione che contiene la
  cellula. Al momento dell'immersione dell'elettrodo nel bagno
  contenente le cellule, vengono inviati dei gradini rettangolari di
  potenziale, di solito di 5 mV di ampiezza e qualche ms di durata. Per
  misurare il potenziale si utilizza un voltmetro che misura potenziali
  dell'ordine dei millivolt.
\end{itemize}

In questa modalità di lavoro si applicano alle cellule correnti di
intensità note che vengono attivati istantaneamente e disattivati
istantaneamente (rettangolari). La durata degli impulsi è dell'ordine
dei millisecondi altrimenti la membrana cellulare si rovinerebbe.
L'applicazione della corrente produce una registrazione di potenziale.
Se cambiano le correnti che passano attraverso la membrana cambia anche
il potenziale. Questo esperimento consente di misurare resistenza e
capacità.

Il potenziale di membrana in ogni istante del processo è uguale al
potenziale \textbf{(???)}

(immagine p191)

V\(_m\) = V\(_0\) + (V\(_f\)-V\(_0\)) e\^{}(t/Rm + m)

(immagine1)

Non si ha una relazione lineare, la curva è la reazione effettiva.

Con correnti crescenti si hanno potenziali più depolarizzati.

La resistenza specifica è molto elevata rispetto a quella dei componenti
elettrici degli elettrodomestici mentre la capacità è molto bassa.

La resistenza della membrana non è costante. Quando gli sperimentatori
hanno cercato di assimilare la membrana cellulare ad un circuito
elettrico speravano fosse comodo applicare le leggi della fisica che
regolano l'andamento dei parametri elettrici in un circuito elettrico.

Interessante è la relazione tra intensità di corrente, potenziale,
resistenza o conduttanza. La legge di Ohm mette in relazione potenziale
e conduttanza.

In maniera grafica è rappresentata da una retta di equazione I = gV e I
= 1/R * V.

Su una stessa cellula è possibile applicare correnti sempre più forti e
registrare i potenziali conseguenti. Se la cellula si comporta come un
conduttore ohmico ci deve essere una relazione lineare tra corrente
applicata e potenziale registrato.

Metto poi in relazione facendo il grafico con i valori di corrente e il
potenziale registrato per ognuno di essi. Verifico se portano ad una
relazione lineare. La linearità ci dice che la conduttanza è costante.

Gli sperimentatori si accorsero che ciò nella realtà non si verifica.
Infatti la conduttanza della membrana, man mano che il potenziale
aumenta e la cellula si depolarizza, aumenta.

La conduttanza aumenta con la depolarizzazione. Vedi schema. Si è capito
il perché studiando l'attività dei canali.

Per fare questo è stato necessario adottare un'altra modalità di lavoro
che è il \textbf{voltage clamp} (letteralmente ``blocco del
voltaggio''). Questa metodica sperimentale è atta a \emph{separare la
componente resistiva e capacitiva} e a misuratre nello specifico solo
quella resistiva al variare del potenziale di membrana.

Per fare questo viene utilizzato un \textbf{amplificatore
differenziale}. Questo è uno strumento che riceve ai suoi ingressi (+ e
-) due segnali di potenziale elettrico, che moltiplica per un fattore
arbitrario; esso restituisce quindi in uscita un segnale di potenziale
che è il risultato della sottrazione (o comunque combinazione lineare)
dei segnali in ingresso. Il potenziale in uscita può poi essere
trasformato in un segnale di corrente tramite un convertitore
voltaggio-corrente.

Un microelettrodo inserito in una cellula ne ``legge'' il potenziale di
membrava \textbf{V\(_m\)} (rispetto a un elettrodo di riferimento posto
nella camera portacampione) e lo invia a uno degli ingressi
dell'amplificatore differenziale. Lo sperimentatore invia quindi
all'altro ingresso un valore di potenziale elettrico
\textbf{V\(_c\)\(_l\)} (\emph{potenziale di clamp o di comando}) che
vuole imporre alla membrana. L'amplificatore opera il confronto tra i
due segnali ed eroga una corrente di segno e ampiezza tali da
minimizzare la differenza tra V\(_m\) e V\(_c\)\(_l\). Il tutto funziona
con la logica di un feedback negativo e la corrente iniettata
dall'amplificatore è pertanto detta \emph{corrente di feedbacK}.

Nel metodo del current clamp si lavora a corrente bloccata, mentre in
quello di voltage clamp si misura la corrente tenendo fermo il
potenziale (l'opposto).

L'attività dei canali risponde alla variazione di potenziale, perciò man
mano che la cellula si depolarizza ci sono canali ionici che si aprono
in risposta alla variazione. L'apertura dei canali porta ad un aumento
della conduttanza perché si aprono nuove vie alla corrente ionica.

La cellula \emph{non} è un conduttore ohmico; nella cellula minime
perturbazioni di corrente portano ad ampie variazioni di resistenza e
conduttanza.

Nel voltage clamp si può creare un tracciato di corrente per vedere sia
la componente capacitiva che quella resistiva. Quando viene applicato il
potenziale la corrente schizza subito ad un livello molto alto per poi
abbassarsi ed equilibrarsi. La prima è la corrente capacitiva. Sono una
quota di cariche che vanno a caricare il condensatore. La funzione
resistiva si sviluppa nel tempo fino ad assumere un valore massimo e
costante. La relazione tra le componenti non è lineare.

Man mano che si accumulavano dati sul ruolo dei vari ioni è diventato
importante studiare l'attività dei singoli canali specifici per
determinati ioni. Una tecnica per selezionare i canali è chiamata
\textbf{patch clamp}. Strappa frammenti di membrana facendoli aderire
alla punta di un elettrodo. Si utilizzano elettrodi con punte più larghe
che succhiano pezzi di membrana strappandola via. L'elettrodo non viene
inserito a perforare la membrana. L'adesione della membrana
all'elettrodo deve essere perfetta per impedire fughe di correnti che
non vengono registrate. La corrente che entra nell'elettrodo deve
passare attraverso la membrana e se l'adesione dell'elettrodo non è
perfetta l'esperimento non è valido. Si parla di \textbf{seal} come
esperimento. Con un buon seal si registra solo che correnti passano
attraverso l'elettrodo considerando solo i canali che si trovano in
prossimità della superficie dell'elettrodo. Si registrano anche correnti
di singolo canale (picoampère).

Le correnti ci danno anche la visualizzazione di come funziona un canale
e l'idea di canale chiuso e aperto. Un canale è un sistema che funziona
come un tutto o un nulla: se aperto passa corrente, se chiuso non passa.
Se il canale è inattivo non passa corrente anche se in realtà abbiamo
brevi momenti di apertura del canale che causa correnti di alcuni
picoampère. Se il canale è attivato da voltaggio prevalgono i momenti di
passaggio di corrente a quelli di assenza di corrente. La corrente è
anche più forte. Per studiare un prendonocanale alcuni fanno esprimere
il gene di quel canale (che è una proteina o più proteine) in un modello
cellulare opportuno tra cellule che esprimono pochi canali come oociti
di rana o rospo (poche correnti basse e dimensioni grandi che si
ottengono facilmente e sono facilmente manipolabili) altre cellule sono
le HEK (cellule di rene di embrione umano che esprimono poche correnti e
sono facilmente transfettabili e quindi è facile fargli esprimere un
gene alieno). La cellula esprime il DNA del gene per il canale e nelle
cellule che esprimono queste canale effettuare misure di patch clamp.
Saranno prevalenti i canali indotti tramite manipolazione genetica e si
può registrare l'attività di questi canali. In questo modo si possono
caratterizzare le attività dei canali ionici. Queste si caratterizzano
realizzando due tipi di dati (dato di corrente e relazione tra corrente
e potenziale). Si usa il grafico I-V oppure le tracce di corrente.

\section{La via del calcio}\label{la-via-del-calcio}

Il Ca di cui parleremo è la presenza dello ione all'interno delle
cellule.

Come tutti gli ioni che troviamo negli esseri viventi ce n'è una quota
dentro le cellule e una fuori. Noi ci occuperemo soprattutto del calcio
intracellulare. Il calcio intracellulare partecipa alle trasduzioni del
segnale. Siamo sempre nell'ambito della comunicazione cellulare.

Nella cellula sono stato descritti molti sistemi capaci di trasportare
lo ione da una parte all'altra, chiamati \emph{sistemi di mobilizzazione
attiva o passiva dello ione calcio}.

Il Ca è presente nella cellula sia come ione libero che come ione legato
ad altre molecole, soprattutto proteine (legano preferenzialmente ioni
bivalenti). Anche gli ioni monovalenti possono legarsi alle proteine ma
lo fanno con una minore affinità. Poi c'è una sistuaizone particolare
che è quella dei pori ionici.

In parte il Ca\(^2\)\(^+\) è uno ione libero nel citosol e in parte è
presente negli organelli cellulari (libero o legato a proteine).

La quantità di Ca\(^2\)\(^+\) presente come ione libero è molto piccola,
ma in realtà nella cellula non è che ci sia poco calcio (questo è quasi
tutto legato a proteine o sequestrato negli organuli, e anche qui è
prevalentemente legato a proteine).

La concentrazione del calcio citosolico si dice che è intorno a 100 nM
ma può essere anche più bassa (è una concentraizone molto bassa =
10\(^-\)\(^7\) M), mentre nel mezzo extracellulare lo ione calcio ha una
concentrazione di qualche mM (intorno a 10\(^-\)\(^3\) M).

la membrana cellulare divide dunque due soluzioni in cui il calcio ha
concentrazioni diverse: all'esterno circa da 2 a 5 10\(^-\)\(^3\) M
mentre nel citosol è attorno a 10\(^-\)\(^7\) M.

La quantità di calcio per volume all'intenro della cellula è quasi
uguale a quella all'esterno, ma è legato a proteine.

Il calcio può presentare degli aumenti transitori, di durata variabile
(da qualche mSec ad alcuni minuti) dove la concetrazione può salire per
poi ridiscendere. Concentraizoni molto elevate (oltre 1000 nM) diventano
tossiche per la cellula fino a mandare la cellula in apoptosi o
addiritura in necrosi.

Dentro la cellula il calcio è mantenuto a livelli molto bassi in
soluzione perchè ci sono dei sistemi tampone, detti \emph{buffer del
calcio}, e perchè questo è quasi tutto legato a proteine.

Il calcio conservato negli organuli ha indotto i fisiologi a parlare di
\emph{riserva intracellulare di calcio} (o store, cioè magazzini del
calcio) anche perchè il Ca\(^2\)\(^+\) può essere rilasicato da queste
riserve.

Le riserve che mobilizzano più facilemtne il calcio sono rapppresentate
dal reticolo endoplasmatico (si possono individuare 3 tipologie
morfologiche di RE = liscio, ruvido e membrana nucleare).

In questi organuli la concentrazione del Ca\(^2\)\(^+\) è intorno al
microM.

Nel RE ci sono varie proteine che legano il Ca\(^2\)\(^+\) e che dunque
favoriscono l'accumulo dello stesso. Proteine di questo tipo sono
presenti anche nel citosol.

Lo ione calcio presenta un forte gradiente (uno dei più forti che si
conoscono negli esseri viventi, ma non il più forte). La membrana
cellulare è pochissimo permeabile al Ca\(^2\)\(^+\) e il potenziale di
equilibrio del calcio è molto lontano da quello di membrana. C'è un
forte gradiente elettrochimico che spinge lo ione ad entrare nella
cellula, ma non può farlo perchè la membrana è impermeabile.
Occasionalmente, tuttavia, la membrana cellulare può diventare
temporaneamente permeabile al calcio grazie all'apertura dei
\emph{canali del calcio}. A questo punto abbiamo degli afflussi di
Ca\(^2\)\(^+\) dall'esterno all'interno della cellula, con aumento della
concentrazione interna dello ione.

Quando si attiva una corrente del sodio o del potassio superiore a
quello basale si ha un flusso e un deflusso di ioni. Queste correnti
però non sono tali da modificare le concentrazioni intra- ed
extracellulari delgi ioni perchè ne movimentano troppi pochi.

Nel caso del calcio le concentrazioni però sono molto più basse e qundi
è sufficiente un ingresso dello ione dovuto all'apertura dei canali per
provocare una modficazione della concentrazione citosolica dello ione
(cosa che non accade per Na\(^+\), K\(^+\) e Cl\(^-\)). Questi aumenti
sono transitori.

Fenomeni di questo genere sono stati chiamati \emph{``segnali del
calcio''}.

Il \emph{potenziale di equilibiro del Ca\(^2\)\(^+\)} è di \textbf{130
mV}.

Lo stesso fenomeno può verificarsi anche dagli organuli al citosol.
Sugli organuli ci sono dei canali che possono aprirsi originando un
deflusso degli ioni dagli stessi al citosol.

I canali del calcio sono \emph{canali ionici}: sono proteine di membrana
che presentano un poro attaverso il quale può essere condotto uno ione
attraverso la membrana. Lo ione passa sempre \emph{o dall'esterno al
citosol o dalle riserve intracellulari al citosol}. Non è mai stato
osservato un trasporto nel senso opposto.

Nella plasmamembrana esistono diversi tipi di canali del calcio che sono
alla base della sensibilità, del movimento, ecc.

Questi canali possono rispondere ad una variazione di voltaggio, oppure
essere regolati da recettori (cioè rispondono ad un ligando, i più
tipici sono glutammato, serotonina e acetilcolina, sono tutti
neurotrasmettitori).

Questi canali li troviamo nel tessuto muscolare e nel tessuto nervoso
(ma non solo).

I \textbf{canali del calcio voltaggio-dipendenti} generalmente non
operano isolatamente, ma lavorano in sinergia con altri tipi di canali
ionici. I canali voltaggio-dipendenti del calcio sono attivato dalla
depolarizzazione, per cui la corrente I\(_C\)\(_a\), evocata da uno
stimolo depolarizzante, assume facilmente carattere autorigenerativo ed
è in grado di generare potenziali d'azione.

Oltre che nella genesi dle potenziale d'azione, i canali
voltaggio-dipendenti del calcio svolgono un ruolo cruciale nel regolare
l'ingresso nel citosol, attraverso la membrana plasmatica, dello ione
Ca\(^2\)\(^+\), un fattore attivo in molti processi essenziali nelle
funzioni delle cellule (es. promotore del rilascio del
neurotrasmettitore alle giunzioni sinaptiche, attivatore dell'apparato
contrattile nelle fibre muscolari, ecc).

Misure elettrofisiologiche delle correnti di calcio in cellule native
hanno rivelato l'esistenza di due gruppi di canali del calcio
voltaggio-dipendenti, sulla base della voltaggio-dipendenza della loro
attivazione:

\begin{itemize}
\itemsep1pt\parskip0pt\parsep0pt
\item
  i \textbf{canali del calcio a bassa soglia} (\emph{Low Voltage
  Activated, LVA}) si attivano in risposta a modeste depolarizzazioni
  della membrana plasmatica e mostrano una inattivazione
  voltaggio-dipendente e rapida;
\item
  i \textbf{canali del calcio ad alta soglia} (\emph{High Voltage
  Activated, HVA}) richiedono depolarizzazioni più ampie per aprirsi e
  si attivano molto più lentamente.
\end{itemize}

I canli voltaggio-dipendenti hanno un funzionamento descritto da 3
stati: chiuso, aperto e inattivo. I canali del calcio si aprono per
depolarizzazione della membrana, ciò significa che se la membrana è al
suo potenziale di riposo il canale è chiuso, mentre se la cellula si
depolarizza il canale si apre.

Una \textbf{depolarizzazione}, in biologia, è la \emph{diminuzione del
valore assoluto del potenziale di membrana di una cellula}. Così, quando
il potenziale di membrana di una cellula si avvicina allo zero, si ha
una depolarizzazione.

Il canale può presentarsi anche in \emph{forma inattiva} e in questo
caso non reagisce alla depolarizzazione (dunque rimane sempre chiuso).

La proteina costitutiva dei canali voltaggio-dipendenti del calcio,
detta subunità \(/alpha\)\(_1\) è attraversata assialmente dal ``poro''
che viene percorso dagli ioni Ca\(^2\)\(^+\) che vi fluiscono quando il
canale si trova nello stato aperto. La subunità \(/alpha\)\(_1\) è
solitamente associata a 4 subunità accessorie di minor dimensione
(\(/alpha\)\(_2\), \(/beta\) e \(/delta\)). La presenza in alcune delle
loro anse citosoliche di siti fosforilabili fa pensare che esse possano
svolgere una funzione regolatrice delle caratteristiche funzionali della
subunità \(/alpha\)\(_1\) cui sono associate.

I vari tipi di canali del calcio sono accomunati da una sensibilità più
o meno spiccata ai cationi bi- e tri- valenti Cd\(^2\)\(^+\),
Ni\(^2\)\(^+\), Co\(^2\)\(^+\) e la La\(^3\)\(^+\), che esercitano
un'azione bloccante.

Tra i \textbf{canali attivati da ligandi} troviamo il \emph{recettore
del glutammato}, l'\emph{N-metil-D-aspartato} (detto anche
\textbf{NMDA}), che ha azione diretta sul recettore causando l'apertura
del poro, mentre il manganese stimola il canale.

Il \emph{glutammato} è un neurotrasmettitore delle cellule nervose.

Nelle sinapsi, quando arriva uno stimolo, si ha l'apertura di più canali
del calcio nella cellula presinaptica. Uno degli effetti del segnale del
calcio è l'esocitosi, che permette il rilascio delle vescicole che
contengono il neurotrasmettitore (nella cellula presinaptica sono canali
voltaggio-dipedenti). Ponendo che le vescicole contengano glutammato,
questo invade la camera sinaptica legandosi ai recettori sull'altra
cellula.

Altri tipi di canali che reagisocno a vari stimoli sono i \textbf{TRP
(Transient Receptor Potential)}. Questi canali costituiscono un'ampia
superfamiglia di canali versatili che sono espressi praticamente in
tutti i tipi cellulari nei vertebrati e negli invertebrati. Questi
canali sono cruciali nell'innesco di meccanismi che permettono
l'ingresso di Ca\(^2\)\(^+\), Mg\(^2\)\(^+\) e tracce di altri ioni
metallici in risposta a stimoli intra- ed extracellulari in un ampio
spettro di processi fisiologici.

Questi sono canali generalmente \emph{non selettivi}.

I canali TRP sono proteine di membrana intrinseche con sei domini
transmembranari e una regione che ne costituisce il poro, permeabile ai
cationi.

Uno dei più studiati è il gruppo dei \textbf{vanilloidi (TRPV)} perchè
coinvolti, ad esempio, nella sensibilità olfattiva e termica (anche le
risposte al peperoncino o al mentolo).

\subsection{\texorpdfstring{Il rilascio intracellulare del
Ca\(^2\)\(^+\)}{Il rilascio intracellulare del Ca\^{}2\^{}+}}\label{il-rilascio-intracellulare-del-ca2}

Esistono diversi meccanismi tramite cui le riserve interne di calcio
possono essere liberate.

I \textbf{canali IP\(_3\)R dipendenti} (recettori dell'\emph{inositolo
trifosfato}, IP3) sono enormi proteine presenti sul reticolo
endoplasmatico formate da \emph{4 subunità} (ognuna da 313 kDa) con un
grosso \emph{dominio citosolico} e un più piccolo \emph{dominio
intrareticolo}. Le 4 subunità formano una struttura circolare con un
poro centrale.

La transizione del canale dalla conformazione chiusa a quella aperta
viene stimolata dall'inositolo trifosfato (innesco) e dal calcio. L'IP3
è un polialcol (i polialcoli sono molecole simili agli zuccheri ma
presentanti tutte funzioni alcoliche, mentre gli zuccheri hanno una
funzione aldeidica e le altre sono funzioni alcoliche).

L'IP3 ha \emph{3 funzioni alcoliche esterificate a 3 gruppi fosfato}.

In particolare la molecola che ci interessa è l'\emph{1,4,5 trifosfato}.

Questa molecola deriva da un \emph{fosfolipide di membrana}, il
\textbf{fosfatidilinositolo}, che può essere fosforilato diventando
\textbf{fosfatidilinositolo} 4,5-bisfosfato. Successivamente si forma
l'\textbf{inositolo 1,4,5-trifosfato}; il terzo gruppo fosfato deriva
dal fostato che teneva la molecola attaccata al fosfolipide, lasciando
così una moleocla di \emph{diacilglierolo} nella membrana. L'inositolo
si stacca dalla membrana per mezzo dell'enzima \textbf{fosfolipasi C},
diffonde nel citosol e si lega al canale.

Il recettore è formato da 4 subunità e può legare fino a 4 molecole di
IP3. Maggiore è la quantità di IP3 rilasciato, maggiore è la quantità di
calcio rilasciato. Tutto questo è messo in moto da una fosfolipasi che
viene attivata da una proteina G, la quale a sua volta viene attivata da
un ormone.

Un altro tipo di recettore è l'\textbf{RYR}. Questo è un complesso
proteico transmembrana del reticolo sarco/endoplasmatico formato da 4
subunità. Questo canale non ha un agonista fisiologico preciso: il
principale è il Ca, quando la sua concentrazione citosolica aumenta il
canale si apre.

Questo tipo di canale risponde ad un moderato aumento del calcio
citosolico (tra 0,1 e 1 \(/mu\)M). Nel muscolo striato tuttavia, le
concentrazioni che lo attivano vanno anche oltre il \(/mu\)M, ma passato
questo valore il Ca diventa inattivante. Per questo si pensa che siano
presenti due siti nel canale: uno a bassa affinità e uno ad alta
affinità che inibiscono o stimolano il canale.

Per i canali RYR sono stati scoperti vari agonisti farmacologici quali
eparina, caffeina e rianodina (quest'ultima è la molecola agonista più
nota). La rianodina a concentrazioni basse stimola l'apertura del canale
mentre a concentrazioni alte ne stimola la chiusura (sempre rispetto
alla concentrazione basale).

\textbf{Lessico:}

\begin{itemize}
\itemsep1pt\parskip0pt\parsep0pt
\item
  si dice \emph{``ingresso''} quando uno ione entra nella cellula
  dall'esterno;
\item
  si dice \emph{``rilascio''} quando uno ione arriva da una riserva
  intracellulare;
\item
  si dice \emph{``espulsione''} o \emph{``estrusione''} quando il calcio
  viene trasferito dal citosol all'esterno;
\item
  si dice \emph{``sequestro''} quando lo ione viene trasferito dal
  citosol agli organuli.
\end{itemize}

Con questi due recettori abbiamo appena visto fenomeni di rilascio del
calcio. Anche questi generano segnali del calcio.

Sono state anche descritte anche mobilitazioni dello ione che \emph{non}
generano segnali del calcio; queste sono dovute a correnti di trasporto
passivo o facilitato. Questo fenomeno viene definito come *``ingresso
capacitativo del Ca\(^2\)\(^+\).

Le attività di sequestro del calcio portano lo ione dentro le riserve
mentre le attività di rilascio lo portano al di fuori di esse.

Se in una cellula prevalgono le attività di rilascio gli ``store''
intracellulari tendono a scaricarsi e siccome il calcio citosolico poi
viene anche espulso, tutta la cellula si scarica.

Quanto più intenso è il segnale del calcio proveniente dalle riserve,
tanto più la cellula si impoverisce. L'ingresso capacitativo del
Ca\(^2\)\(^+\) si oppone a questo fenomeno ricaricando gli store senza
tuttavia indurre un segnale del calcio. Sono stati identificati diversi
meccanismi detti \textbf{SOC} (\emph{Store-Operated Channels}, è un
canale), \textbf{CRAC} (Calcium Release Activated Channels, fenomeno) e
SOCE. \textbf{CONTROLLARE}

È stato poi individuato un fenomeno secondo il quale nel RE sono
presenti delle proteine chiamate STIM che avvertono la diminuzione delle
riserve del calcio all'interno del reticolo. Si è postulato che il
processo di \textbf{CCE} (\emph{Capacitative Calcium Entry}) necessiti
dell'interazione di due proteine, \textbf{Orai} e \textbf{STIM}
(\emph{Stromal Interaction Molecule}); la prima è il canale vero e
proprio localizzato nella membrana plasmatica, l'altra il modulatore di
Orai localizzato nella membrana del RE.

Le proteine STIM si aggregano e si legano a canali ORAI attivandoli. A
questo punto il Ca\(^2\)\(^+\) entra nella cellula e viene sequestrato
nel reticolo ricaricandolo, passando probabilmente dal citosol.

\section{\texorpdfstring{Meccanismi Off del segnale del
Ca\(^2\)\(^+\)}{Meccanismi Off del segnale del Ca\^{}2\^{}+}}\label{meccanismi-off-del-segnale-del-ca2}

Questo ione si trova ad una concentrazione citosolica molto bassa,
mentre all'esterno della cellula si trova in soluzione ad una
concentrazione comparabile a quella del K\(^+\) (ricade tra gli ioni più
abbondanti).

La differente concentrazione del calcio tra l'esterno e l'interno della
cellula permette dei temporanei aumenti della concentrazione dello ione.
Questo fenomeno non avviene per altri ioni che rimangono praticamente
sempre costanti in soluzione nel citosol e all'esterno.

Il Ca\(^2\)\(^+\) è lo ione che presenta le più ampie variazioni a
livello citosolico (altri ioni sono lo zinco, il rame, ma sono fenomeni
molto piu contenuti).

Le variaizoni del calcio hanno un forte impatto sul metabolismo
cellulare.

Com'è possibile che la cellula mantenga la concentrazione di questo ione
nel citosol molto bassa? Questo meccanismo viene detto ``omeostasi del
calcio''.

\subsection{L'omeostasi del calcio}\label{lomeostasi-del-calcio}

Il Ca\(^2\)\(^+\) lega facilmente molte proteine e questo è già un
motivo per cui la sua concentrazione tende a rimanere molto bassa.
Questo ione si lega a delle molecole che a loro volta si legano alla
cellula e viene accumulato internamente alla cellula come riserva.

Un'elevata concentrazione del calcio (alcune centinaia di nM) non viene
tollerata dalla cellula per tempi lunghi, ovvero tempi superiori a
qualche decina di sec.~Superata una concentrazione di 1 \(\mu\)M la
cellula inizia a subire danni. Tuttavia questa è una situazione che
fisiologicamente non si verifica mai.

In seguito al verificarsi di aumenti transitori del calcio, questo viene
rapidamente riportato alle concentrazioni basali, tra 50 e 100 nM (non
c'è un valore fisso), da una serie di sistemi.

I principali sistemi sono due pompe del calcio: \textbf{SERCA}, una
calcio ATPai del reticolo sarco/endoplasmatico e \textbf{PMCA}, una
pompa presente sulla membrana plasmatica.

Questi sono trasportatori attivi primari, ossia creano il gradiente del
calcio. Consumano ATP per trasportare lo ione contro un gradiente molto
accentuato.

Sulla membrana esiste poi un altro trasportatore. Questo è un
\textbf{antiporto sodio-calcio}, ovvero un cotrasportatore che trasporta
sodio all'interno della cellula e calcio all'esterno di essa. In mote
cellule il suo ruolo è fondamentale, tanto che se inibito causa la
perdita dell'omeostasi del calcio nella cellula.

Questo antiporto è estremamente dipendente dalla \emph{pompa
sodio-potassio} la quale mantiene il gradiente del sodio; se questa
viene inibita salta l'omeostasi del calcio perché viene meno la funzione
dell'antiporto.

LA \textbf{PMCA} è una pompa del calcio presente sulla membrana
plasmatica e in grado di trasportare lo ione dal citosol all'esterno (lo
espelle dalla cellula).

Esistono 4 isoforme di questa pompa, di cui due specifiche del tessuto
nervoso. I tessuti con le maggiori dinamiche del calcio sono quello
nervoso e quello muscolare (sono le cellule che presentano le più forti
correnti del calcio e dunque anche i più potenti sistemi di omeostasi).
Le correnti del calcio si studiano molto meglio in queste cellule
piuttosto che in altri tipi cellulari dove è piu difficile metterle in
evidenza.

Altri tessuti presentano altre isoforme.

Questa pompa è presente in tutti i tipi cellulari perchè è il principale
fattore di omeostasi del calcio in quasi tutti i tipi cellulari.

Nel sistema nervoso ne troviamo due isoforme specifiche.

Questa pompa presenta 10 segmenti transmembrana e grosse parti
citosoliche. Può essere fosforilata e legare la calmodulina.

Il suo funzionamente può essere regolato da vari eventi metabolici
cellulari.

Dal punto di vista biochimico-fisico il trasportatore ha un'elevata
affinità e bassa capacità.

Questo vuol dire che la porma riesce a legare lo ione anche a
concentraizoni bassissime, ma ha un tasso di trasporto lento (trasporta
moli al sec). Entrambe queste caratteristiche, ovvero la K\(_m\) e la
V\(_m\)\(_a\)\(_x\), possono modificarsi se la pompa viene modificata o
se lega la calmodulina.

Un sistema analogo è presente sul reticolo endoplasmatico ed è chiamato
\textbf{SERCA}.

Il reticolo endoplasmatico della cellula muscolare striata è chiamato
\emph{sarco-endoplasmatico} perchè ha una conformazione e una
funzionalità particolare.

Questa pompa presenta \emph{3 isoforme}: una del muscolo scheletrico,
una del muscolo cardiaco e una presente negli altri tessuti.

Questa pompa ha un'attività importante nel tessuto muscolare striato
dove è espressa ad altissimi livelli e funziona maggiormente quanto più
il calcio citosolico aumenta.

L'\textbf{antiporto} invece lavora con bassa affinità ed elevata
capacità: ha un'affinità per lo ione \emph{inferiore} a quella della
PMCA, ma ha una capacità (ovvero una velocità) di trasporto maggiore di
quello della PMCA (trasporta circa 2000 ioni al secondo per ogni singolo
antiporto).

La stechiometria in alcuni tessuti è di \emph{3 Na\(^+\) : 1
Ca\(^2\)\(^+\)} (ma non sempre, ad esempio nei bastoncelli della retina
è \emph{4 Na\(^+\) : 1 Ca\(^2\)\(^+\) + 1 K\(^+\)}.

Anche questo tipo di trasportatore espelle lo ione dalla cellula (come
la PMCA) mentre la SERCA sequestra lo ione all'intenro del reticolo
sarcoendoplasmatico.

In ogni caso tutte queste strutture sottraggono ioni Ca\(^2\)\(^+\) al
citosol quando la concentrazione è abbondante.

L'antiporto lavora solo a livelli molto alti di calcio e con resa molto
grande.

A livelli bassi di calcio si lavora molto più lentamente, agisce in
questo modo la PMCA. Agiscono anche quando la concentrazione dello ione
è estremamente bassa. I sistemi a bassa affinità possono lavorare solo
oltre ad una certa concentrazione dello ione poichè al di sotto di essa
non riescono più a legarlo a causa di un'affinità troppa bassa.

I sistemi con un'elevata affinità invece, riescono a legare il calcio
anche quando ha concentrazioni bassissime e lo rilasciano dove le
concentrazioni sono elevate.

Dal punto di vista chimico-fisico l'attività di questi trasportatori è
controintuitiva perchè possono legare lo ione dove la concentrazione è
bassa e liberarlo dove è alta. La cellula adotta vari sistemi con vari
gradi di affinità: quando il Ca\(^2\)\(^+\) è alto intervengono sistemi
con bassa affinità ma elevata capacità perchè il Ca\(^2\)\(^+\) deve
essere portato via in fretta.

Ci sono poi dei sistemi che regolano dinamiche ancora più lente del
calcio. Tra questi troviamo i mitocondri, che possono sequestrare il
calcio dal citosol e accumularlo al loro interno per poi rilasciarlo al
momento opportuno.

I mitocondri dispongono di vari trasportatori del calcio sulla membrana
interna.

La membrana esterna è poco selettiva mentre quella interna è molto
selettiva.

Inoltre la membrana interna del mitocondrio è carica elettricamente e ha
un potenziale di membrana molto forte a causa del trasporto di ioni
idrogeno al suo esterno. Il mitocondrio è un potentissimo trasportatore
di ioni idrogeno.

In condizioni basali il Ca\(^2\)\(^+\) tende ad entrare nel mitocondrio.
Nel mitocondrio la concentrazione del calcio è molto più elevata che nel
citosol, eppure esso è in grado di incorporare passivamente lo ione.

La concentrazione del calcio nel mitocondrio è circa doppia rispetto a
quella del citosol.

La forza elettrochimica trainante che agisce sullo ione è data dalla
differenza tra il potenziale elettrochimico dello ione meno il
potenziale di membrana del mitocondrio. Il gradiente elettrochimico
spinge lo ione all'interno del mitocondrio, ovvero nel compartimento in
cui esso è più concentrato per effetto di un potenziale mitocondriale
elevato.

Tuttavia questo non è l'unico movimento che lo ione può fare attraverso
la membrana mitocondriale.

Non sono stati descritti veri e propri canali del calcio mitocondriali.
I principali sistemi utilizzati dal mitocondrio sono:

\begin{itemize}
\itemsep1pt\parskip0pt\parsep0pt
\item
  un \emph{uniporto}, trasferisce lo ione dal citosol all'interno del
  mitocondrio che è negativo rispetto al citosol per via del forte
  gradiente protonico della membrana interna;
\item
  un \emph{antiporto Na/Ca} che veicola il calcio all'esterno;
\item
  un \emph{antiporto H/Ca} (calcio all'esterno);
\item
  un trasportatore capace di sequestrare il calcio rapidamente quando lo
  ione aumenta a livello citosolico.
\end{itemize}

Fisiologicamente il mitocondrio è impegnato soprattutto in attività di
sequestro del calcio. Questo avviene generalemnte quando lo ione
raggiunge concentrazioni elevate nel citosol o quando le concentrazioni
sono elevate per lunghi periodi.

Si pensa che l'accumulo dello ione nel mitocondrio diventi significativo
quando questo supera la concentrazione citosolica di 1 \(\mu\)M.

A questi livelli di Ca\(^2\)\(^+\) nel citosol il mitocondrio diventa il
principale sistema di abbasamento della concentrazione del
Ca\(^2\)\(^+\) nel citosol, dopodichè intervengono anche gli altri
sistemi visti.

Nei mitocondri si trova anche il \textbf{PTP (Permeability Transition
Pore)}, un sistema legato all'apoptosi e attraverso cui il calcio esce
massicciamente dall'organello.

Esiste dunque un'interazione tra calcio e apoptosi nella quale il
mitocondrio è profondamente coinvolto.

Il gradiente del Ca\(^2\)\(^+\) è un gradiente molto forte che richiede
molta forza per essere creato e mantenuto. Questa forza non è tutta a
carico di un solo sistema ma ce n'è più di uno:

\begin{itemize}
\itemsep1pt\parskip0pt\parsep0pt
\item
  la \textbf{Ca\(^2\)\(^+\) ATPasi}, che mantiene un livello bassissimo
  di calcio nel citosol;
\item
  l'\textbf{antiporto Na\(^+\)/Ca\(^2\)\(^+\)}, che ha una funzione
  essenziale nel picco del calcio;
\item
  il \textbf{mitocondrio}, che ha una funzione molto importante quando
  il calcio tende a stazionare a livelli elevati. Un esempio è quando la
  cellula è stimolata a lungo da un ormone che fa aumentare il
  Ca\(^2\)\(^+\). In questo caso lo ione tenderebbe a salire fino a
  valori non fisiologici e dannosi per la cellula. Quando lo stimolo
  cessa il mitocondrio aiuta la cellula a ristabilire le concentrazioni
  fisiologiche.
\end{itemize}

Il calcio è l'unico ione a dare manifestazioni molto evidenti poichè
praticamente non è presente nel citosol; basta un minimo scompenso
nell'omeostasi perchè la sua concentrazione aumenti, mentre per farla
ritornare ai livelli adeguati (bassi) devono intervenire sistemi con una
potenza notevole rispetto all'omeostasi di tutti gli altri ioni.

Tramite lo studio di cellule in coltura cresciute in una situazione dove
le concentrazioni del Ca\(^2\)\(^+\) sono lontane dall'omeostasi, si è
notato che la cellula perde il citoscheletro diventando sferica,
successivamente inizia a rilasciare delle vescicole (\emph{blebbing}) e
alla fine la membrana degenera (la cellula perde l'impermeabilità e va
in rapida degenerazione).

Questo è un disordine del calcio rapido e brutale che porta la cellula a
morte per disfacimento, una morte non descrivibile con dei processi
biologici. In questo caso si parla di \emph{necrosi}.

Anche un disordine di minore intensità ma prolungato nel tempo (il
calcio aumenta più gradualmente ma perdura nel tempo) può portare la
cellula alla morte tramite un processo chiamato \emph{apoptosi}. In
questo tipo di fenomeno si generano un ingresso e un rilascio di calcio
dal RE.

Quando le cellule vanno in apoptosi presentano delle modificazioni
morfologiche che possono essere descritte perchè ripetibili (si
osservano sempre in corrispondenza di questo fenomeno).

In una cellula che si predispone all'apoptosi abbiamo delle
modificazioni mitocondriali che portano alla formazione di un \emph{poro
di transizione}; questo è un complesso molecolare che si forma sulla
membrana mitocondriale interna e che la rende permeabile a molti soluti
tra cui il calcio. Un altro evento è il \emph{rilascio del citocromo C}
sempre attraverso il poro di transizione. Il citocromo C è abbondante
nella matrice e induce la cascata di apoptosi nel citosol tramite
l'attivazione delle \textbf{capsasi} (proteasi). In questa serie di
eventi si ha ingresso di calcio prima della formazione del poro, mentre
dopo la sua formazione si ha rilascio di calcio dal mitocondrio.

Il forte rilasico di calcio durante questi fenomeni aumenta il fenomeno
stesso.

Perchè il calcio esce dal mitocondrio quando si forma il poro di
transizione? Tendenzialmente il mitocondrio non rilascia Ca\(^2\)\(^+\)
ma lo sequestra.

Anche il mitocondrio presenta una sua omeostasi del calcio, ma con la
formazione del poro si rompe l'equilibrio delle pompe e si ha di
conseguenza una caduta del potenziale di membrana mitocondriale
(\emph{psi}). La caduta del potenziale di membrana è uno dei segnali
principali dell'inizio del processo di apoptosi.

La caduta del potenziale inverte il gradiente elettrochimico del
Ca\(^2\)\(^+\) così che il potenziale di membrana tenda ad andare verso
il potenziale di equilibrio, facendo sì che il Ca\(^2\)\(^+\) esca dal
mitocondrio (ne possiede estremamente di più rispetto al citosol).

Il poro di transizione si origina dall'aggregazione di molte molecole,
alcune della membrana esterna e alcune della membrana interna (si forma
un ponte tra le due membrane). Le moleocle coinvolte sono molte.

Questo fenomeno causa una caduta del potenziale di membrana, una caduta
della concentrazione dell'ATP e un aumento delle ROS.

\subsection{\texorpdfstring{La decodificazione del segnale del
Ca\(^2\)\(^+\)}{La decodificazione del segnale del Ca\^{}2\^{}+}}\label{la-decodificazione-del-segnale-del-ca2}

Il sistema cellulare utilizza lo ione come segnale facendo avvenire
delle variazioni di concentrazione dello ione stesso (per fare questo
deve lavorare a bassi livelli di concentrazione). Lo ione oscilla tra
concentrazioni molto basse (tra 100 nM e 1 \(\mu\)M) che gli permettono
di variare in tempi molto brevi (si parla di millisecondi), perchè
smuovere concentrazioni maggiori richiederebbe tempi troppo lunghi.

Ma cosa avviene quando il calcio aumenta a livello citosolico? Questo
segnale viene decodificato metabolicamente dalla cellula.

Lo stimolo indotto dallo ione può tradursi o nell'entità dell'aumento o
nella frequenza delle oscillazioni. La frequenza delle oscillazioni
riflette l'intensità dello stimolo extracellulare.

Queste sono situazioni che si verificano dopo l'applicazione dell'ormone
vasopressina a concentrazioni crescenti. La potenza dell'azione ormonale
si traduce in una maggior frequenza di oscillazione.

La conversione di queste oscillazioni in variazioni metaboliche è dovuta
a proteine tra cui la \emph{calmodulina} e la \emph{troponina c}.

La clamodulina è una piccola proteina citosolica specializzata nel
legame con il Ca\(^2\)\(^+\). Questa proteina presenta due zone
globulari e una lineare. Nelle due zone globulari sono presenti dei siti
di legame del calcio. Questi siti di legame sono motivi tipici di
proteine che legano il calcio. Nel mondo delle proteine esistono dei
motivi che si ripetono.

Il motivo delle proteine che lega il calcio è chiamato \textbf{``mano
EF''}. Questa regione ha una struttura elica-ansa-elica (ricordano due
dita di una mano come pollice ed indice).

Per ogni sito di legame del calcio la calmodulina ha un sito
\emph{mano-EF} (in totale 4).

La calmodulina ha un'affinità per il calcio tarata appena al di sopra
del livello di base del calcio, quindi una costante di affinità intorno
al 100 nM, o poco sopra. La costante di affinità di una molecola per
un'altra ci dice quale sarà l'intervallo o la zona di livelli di
concentrazione nei quali l'interazione avverrà preferenzialmente.

La calmodulina lega il calcio non appena questo sale sopra il livello
basale. Questa proteina ``sente'' il calcio non appena si verifica un
segnale del calcio, dopodichè lega lo ione. A questo punto la
\textbf{CAM (calcio-calmodulina)} può interagire con altre proteine
(questo è il passaggio che conduce dall'azione dello ione al
metabolismo).

Le altre proteine con cui può interagire la CAM sono prevalentemente
enzimi e lo ione condiziona altri enzimi grazie al suo legame con la
calmodulina.

Lo ione può poi attivare direttamente altre proteine. Uno dei fattori
principali attivati dal calcio è la \textbf{protein-chinasi C} (\emph{c}
sta per calcio).

Ne esistono molte isoforme con azioni differenti. L'attivazione della
protein chinasi C, come la A, ha una ripercussione notevole sul
metabolismo cellulare.

Le chinasi sono enzimi multifunzione perchè attivano varie cose nella
cellula. Il calcio e l'AMP-ciclico sono detti ``secondi messaggeri''
perchè aumentano nella cellula in risposta a vari ormoni.

\section{Il potenziale d'azione}\label{il-potenziale-dazione}

Il potenziale di azione cellulare è un potenziale di diffusione che si
mantiene costante fintanto che le correnti sono costanti (queste sono
costanti finchè la permeabilità della membrana ai relativi ioni è
costante).

Il potenziale d'azione può essere pensato come una ``fuga''
dall'omeostasi con ritorno.

Osserviamo il potenziLe d'azione in cellule ben dotate di canali ionici.
Il potenziale d'azione consiste in una variazione repentina del
potenziale di membrana e richiede dunque forti correnti ioniche che a
loro volta richiedono un'elevata presenza di canali ionici.

Questo è stato osservato in cellule nervose e muscolari (non è stato
descritto in nessun altro tipo cellulare).

Il potenziale d'azione consiste in una forte depolarizzazione della
membrana cellulare fino a valori positivi; si ha un'inversione, il polo
positivo e quello negativo si invertono, quello positivo diventa
intracellulare mentre quello negativo diventa extracellulare.

Questo fenomeno si manifesta in maniera diversa in cellule diverse, ed è
stato ben descritto nelle cellule nervose, muscolari striate e
cardiache.

L'unità di misura del fenomeno è il millisecondo. Vi è una rapida
depolarizzazione che porta il potenziale da -60 a +10/20/40 mV seguito
da una rapida ripolarizzazione. Il potenziale di membrana scende a
valori più negativi del riposo per ritornare gradualmente al potenziale
di riposo. Nelle cellule cardiache e muscolari lisce questo fenomeno è
un po' meno rapido.

\textbf{(descrizione grafico Jess)}

Il potenziale d'azione ha bisogno di un innesco che consiste in una
debole depolarizzazione di 15 mV rispetto al potenziale basale.

Deve essere raggiunta una certa soglia di depolarizzazione, dopodichè
parte in maniera esplosiva e con un'elevata velocità la depolarizzazione
fino a portare il potenziale a valori positivi, dopodichè la
depolarizzaizone si arresta e a quel punto il potenziale cellulare
ricomincia a ripolarizzarsi con una discesa che riporta il potenziale
fino al valore basale.

Il potenziale d'azione si verifica con una modalità del tipo ``tutto o
nulla'' e dato un certo tipo cellulare si manifesta sempre nello stesso
modo, indipendentemente dallo stimolo iniziale. La depolarizzazione che
porta a superare il livello soglia può dipendere da un impulso più o
meno forte, ma ciò non varia l'intensità del potenziale d'azione
indotto.

La membrana cellulare dopo l'attuazione di un potenziale d'azione entra
in uno stato refrattario; un secondo stimolo, immediatamente successivo
ad un potenziale d'azione, non induce un nuovo potenziale anche se tale
da superare la soglia per provocare un innesco.

È necessario un certo intervallo di tempo prima che la membrana torni
responsiva.

La durata della refrattarietà è legata alla durata del potenziale
d'azione. In una cellula cardiaca si parla di centinaia di millisecondi.

La refrattarietà non scompare improvvisamente ma c'è un periodo di
refrattarietà assoluta in cui nessuno stimolo può indurre un potenziale
di azione, seguita poi da una refrattarietà relativa nella quale la
soglia è più alta del normale.

Per capire il fenomeno sono state studiate le correnti ioniche legate al
potenziale d'azione. Finchè le correnti sono costanti il potenziale è
costante. Se cambia il potenziale significa che stanno cambiando le
correnti.

Il fenomeno è stato descritto la prima volta intorno alla metà del
secolo scorso studiando gli assoni del calamaro. In questo animale si
trovano neuroni piuttosto grandi che hanno assoni altrettanto grandi
facilmente manipolabili e studiabili.

È relativamente facile studiare in vitro, tramite l'utilizzo di
elettrodi, lo stimolo del potenziale d'azione e seguirne la variazione.
Con il metodo del voltage clamp si possono studiare le correnti che
attraversano la membrana.

Studiando questo fenomeno gli sperimentatori notarono una corrente di
ingresso che si sviluppa rapidamente, raggiunge un massimo, dopodichè
tende a scemare.

Successivamente si nota una corrente nulla finchè il fenomeno prosegue
con una corrente in uscita (variazione delle correnti).

Poichè la corrente si modifica continuamente anche il potenziale si
modifica. Infine si raggiuge una corrente finale in uscita che si
manterrà tale finchè verrà mantenuto lo stimolo di depolarizzazione.

Da questo insieme di dati gli studiosi hanno dedotto che nel potenziale
d'azione si verifica un iniziale corrente al sodio in ingresso che poi
svanisce e contemporaneamente si genera una corrente al potassio in
uscita (la prima depolarizza, la seconda ripolarizza).

Dopo un certo tempo, dopo che la membrana si è depolarizzata, le
correnti ioniche saranno pari a zero perchè ormai il potenziale
d'equilibrio sarà stato raggiunto e dunque gli ioni non si muoveranno
più.

Nelle cellule nervose e muscolari sono presenti numerosi canali del
sodio voltaggio dipendenti che si aprono e chiudono rispondendo a una
depolarizzazione della membrana.

Depolarizzando la membrana si ha un aumento progressivo della corrente
del sodio in ingresso, mentre a potenziali più elevati la corrente del
sodio in ingresso tende a ridursi. Intorno a +60mV la corrente cessa
perché il potenziale di membrana vale come il potenziale di equilibrio
del sodio.

A questo punto la forza di trazione sullo ione è zero. Proseguendo nella
depolarizzazione si arriverebbe ad un punto in cui la corrente si
invertirebbe, tuttavia queste non sono condizioni fisiologiche

I canali del sodio lavorano a intervalli negativi nella realtà,
generando corrente e la prima fase del potenziale di membrana. Non si
raggiunge neanche il potenziale di equilibrio del sodio.

Anche i canali potassio sono voltaggio dipendenti e si aprono quando la
membrana si depolarizza. I canali del potassio, a differenza di quelli
del sodio, non si inattivano. La corrente fisiologica di un singolo
canale subisce un aumento della corrente e poi torna a zero.

I canali al sodio sono canali inattivanti; rimangono aperti per un tempo
brevissimo e poi cadono in uno stato inattivo, mentre quelli del
potassio non si inattivano (si chiudono quando la membrana si
ripolarizza).

I canali del potassio non si inattivano sotto lo stimolo, ma rimangono
sempre aperti finchè la membana è depolarizzata e si richiudono quando
la membrana si ripolarizza.

Il potenziale di equilibrio del potassio è più basso del potenziale di
riposo. Si avrà dunque una corrente in uscita sempre più forte man mano
che avviene la depolarizzazione.

Il fenomeno della depolarizzaizone si innesca a causa del raggiungimento
di un \emph{valore soglia}. Inoltre, quando si aprono i canali del sodio
si induce un effetto depolarizzante (questo fenomeno si autoalimenta).
Con la depolarizzazione iniziale cominciano ad aprirsi i canali del
sodio, che subito dopo si disattivano (danno solo una spinta
depolarizzante per un msec o meno).

Più canali del sodio si apriranno più la spinta depolarizzante sarà
forte.

Sotto il livello soglia il fenomeno non parte in maniera così esplosiva.
Se il fenomeno iniziale è così intenso da reclutare una buona quantità
di canali al sodio contemporaneamente la spinta allora sarà tale da
creare una depolarizzazione massiccia dovuta al reclutamento a cascata
di canali. Questo fenomeno proseguirà finchè tutti i canali sodio non
saranno depolarizzati e la spinta scemerà. Man mano che il sodio dà
luogo all'esplosiva depolarizzazione si aprono i canali del potassio,
che una volta tutti aperti creeranno una forte corrente del potassio.

Nel frattempo la corrente del sodio si esaurisce e si mantiene una forte
corrente al potassio in uscita che porta a ripolarizzazione.

Il potenziale di membrana tende ad andare verso il potenziale di
equilibrio del potassio e per questo scende anche a livelli più bassi di
quelli basali iniziali (la membrana è ancora permeabile al potassio). In
seguito la membrana perde la permeabilità al potassio (i canali del
potassio sono voltaggio dipendenti, perciò si richiudono) permettendo al
potenziale di membrana di tornare ai suoi livelli basali.

La refrattarietà si spiega con il fatto che abbiamo i canali del sodio
inattivati, ma i canali del potassio sono ancora parzialmente aperti. La
membrana dunque non reagisce alla depolarizzazione perché la stessa è
ostacolata dai canali al potassio ancora parzialmente aperti.

È la pompa sodio-potassio a ripristinare continuamente le concentrazioni
di questi ioni.

\subsection{La propagazione del potenziale
d'azione}\label{la-propagazione-del-potenziale-dazione}

Il potenziale d'azione non è un evento che si manifestra solo in un
punto della membrana ed è in grado di effettuare un'attività ciclica, ma
si propaga lungo la stessa.

In una membrana, che possiamo considerare come una superficie piatta, il
potenziale d'azione si propaga in maniera circolare allargandosi su di
essa.

Come mai? Il potenziale d'azione provoca una depolarizzazione. La
depolarizzazione rende l'interno della cellula meno negativo e l'esterno
un po' meno positivo; quando si arriva a potenziale 0 non c'è
polarizzazione elettrica, mentre quando si supera il valore dello 0
l'esterno della cellula diventa negativo e l'interno diventa positivo.

Questo evento si manifesta in un certo punto, in un piccolo tratto della
membrana cellulare, ma nelle zone adiacenti tutto è a riposo.

Quindi si avrà una zona della membrana dove si è verificato il
potenziale di azione in cui è stata invertita la polarità e zone della
membrana adiacenti a questo punto con polarità opposte (sullo stesso
lato della membrana). Si ha un'inversione di potenziale.

Due regioni con diverso potenziale elettrico generano un campo elettrico
che agisce sulle cariche elettriche. In questa situazione gli ioni in
soluzione generano correnti elettriche. Ci sono correnti elettriche che
si muovono da regioni esterne alla zone del potenziale d'azione e questo
avviene lungo tutto il bordo del potenziale di azione. Queste correnti
(correnti elettrotoniche) provocano una depolarizzazione delle regioni
adiacenti portando cariche positive all'esterno e negative all'interno.
In questo modo viene depolarizzata la membrana (stessa cosa che succede
aprendo i canali del sodio).

La depolarizzazione è indotta da correnti elettrotoniche che viaggiano
parallelamente alla membrana. La depolarizzazione è blanda ma
sufficiente per innescare un potenziale di azione in zone adiacenti al
primo punto in cui si è verificato il potenziale di azione.

\section{I neuroni}\label{i-neuroni}

La maggior parte dei neuroni contiene 3 principali componenti: un corpo
cellulare e due tipi di \emph{processi neuronali} che partono dal corpo
cellulare, i dendriti e un assone.

Il \textbf{corpo cellulare}, o \emph{soma}, contiene il nucleo e la
maggior parte degli organuli intracellulari.

I \textbf{dendriti} si diramano dal corpo cellulare ricevendo afferenze
da altri neuroni a livello di giunzioni specializzate chiamate
\textbf{sinapsi}. I neuroni sono provvisti anche di un'altra diramazione
che parte dal corpo cellulare, l'\textbf{assone} o \emph{fibra nervosa}.

A differenza del dendrite, la cui funzione è quella di ricevere
informazioni da altri neuroni, il compito dell'assone è quello di
\emph{inviare} informazioni. Generalmente un neurone possiede \emph{un
solo} assone, ma gli assoni possono diramarsi e inviare segnali a più
cellule. Le diramazioni di un assone vengono definite
\textbf{collaterali}.

L'assone serve per la trasmissione di informazioni, che si propagano per
lunghe distanze sotto forma di segnali elettrici definiti
\textbf{potenziali di azione}, rapide e ampie modificazioni del
potenziale di membrana durante le quali l'interno della cellula diviene
positivo rispetto all'esterno.

L'inizio e la fine di un assone sono strutture specializzate del neurone
dette rispettivamente monticolo assonico e terminale assonico.

Il \textbf{monticolo assonico}, che è il sito in cui l'assone si diparte
dal corpo cellulare, è specializzato, nella maggior parte dei neuroni,
nella \emph{genesi} dei potenziali d'azione. Una volta scatenati, i
potenziali d'azione sono trasportati verso il terminare assonico. Il
\textbf{terminale assonico} (o \emph{bottone sinaptico}) è specializzato
nel rilascio del neurotrasmettitore all'arrivo del potenziale d'azione.

Il rilascio del neurotrasmettitore trasmette un segnale ad una
\emph{cellula postsinaptica}, in particolare a un dendrite o al corpo
cellulare di un altro neurone o alle cellule di un organo effettore. Il
neurone il cui temrinale assonico rilascia il neurotrasmettitore è detto
\emph{cellula presinaptica}.

La zona dell'assone presenta molti canali del sodio e dunque presenta
l'eccitabilità giusta per sviluppare i potenziali d'azione.

Il potenziale d'azione si propaga e va sempre nella stessa direzione,
senza tornare mai indietro. Per spiegare coma mai avviene questo si deve
considerare il fenomeno della refrattarietà della membrana.

È la densità dei canali dle sodio a permettere la propagazione
delladepolarizzazione per mezzo delle correnti elettrotoniche.

In un neurone il potenziale di azione avanza e non torna indietro perché
a monte trova la membrana nello stato refrattario, e quando la membrana
esce dallo stato refrattario ormai si trova troppo lontana dalle
correnti elettrotoniche.

Dalla parte del corpo basale il potenziale di azione non si propaga
perché non càè una densità di canali sufficiente per la sua diffusione.

La propagazione del potenziale d'azione ha a che fare con il
funzionamento dei neuroni perchè porta il segnale da una cellula ad
un'altra. La trasmissione del potenziale ha a che fare con la
stimolazione dei movimenti, con la percezione sensoriale, ecc\ldots{}

Tutte queste cose hanno bisogno di un certo tempo di realizzazione, ma
alcune cellule si sono adattate per realizzare i processi con velocità
maggiore per ridurre i tempi di reazione.

Nel sistema nervoso esistono anche cellule diverse dai neuroni, come le
\textbf{cellule gliali}. Queste cellule non servono direttamente alla
trasmissione dei segnali, ma forniscono integrità strutturale al sistema
nervoso, permettendo ai neuroni di svolgere le loro funzioni.

Esistono 5 tipi di cellule gliali:

\begin{itemize}
\itemsep1pt\parskip0pt\parsep0pt
\item
  gli \textbf{astrociti};
\item
  le \textbf{cellule ependimali};
\item
  la \textbf{microglia};
\item
  gli \textbf{oligodendrociti};
\item
  le \textbf{cellule di Schwann}.
\end{itemize}

Tra queste cellule gliali sono quelle di Schwann sono localizzate nel
SNP, mentre le altre si trovano nel SNC.

Le cellule di Schwann hanno un comportamento peculiare: formano un
avvolgimento attorno agli assoni tramite la loro membrana cellulare,
andando a formare una guaina detta \textbf{guaina mielinica}, in modo da
isolarli e permette il passaggio dei potenziali d'azione in modo
efficace e rapido.

Per questo si parla di fibre mielinizzate.

La mielina è formata da strati concentrici di membrane cellulari di
oligodendrociti o cellule di Schwann.

Gli \emph{oligodendrociti} formano la guaina mielinica attorno agli
assoni nel SNC; un oligodendrocita invia proiezioni che forniscono
segmenti di mielina a molti assoni.

Le \emph{cellule di Schwann}, invece, formano la guaina mielinica
attorno agli assoni nel SNP; una cellula di Schwann fornisce mielina a
un singolo assone.

Poichè il doppio strato lipidico della membrana cellulare ha bassa
permeabilità agli ioni, i molti strati di membrana che costituiscono il
rivestimento di mielina di fatto riducono il passaggio di ioni
attraverso la membrana cellulare.

Il doppio strato lipidico funziona come un forte isolante elettrico, e
grazie a questo il potenziale di azione viaggia più velocemente lungo
l'assone.

Tuttavia, esistono delle interruzioni della guaina mielinica, chiamate
\textbf{nodi di Ranvier}, in cui la membrana dell'assone contiene canali
voltaggio-dipendenti per il sodio e il potassio che funzionano nella
trasmissione dei potenziali d'azione permettendo i movimenti ionici
attraverso la membrana.

Il potenziale di azione si sviluppa da un nodo di Ranvier al nodo di
Ranvier successivo, per questo si parla di \emph{conduzione saltatoria}.
Questo aumenta la velocità di trasmissione.

Siccome non tutte le fibre sono mielinizzate, ci sono fibre che
conducino i potenziali di azione in maniera più lenta e altre che lo
conducono in maniera più veloce.

Le fibre più veloci sono i motoneuroni che comandano la muscolatura
scheletrica e che sono piuttosto grandi; conducono lo stimolo ad una
velocità di circa 60-80 m/s. Anche la sensibilità cutanea è molto veloce
(30-60 m/s) fino ad arrivare alle fibre nocicettive, cioè le fibre che
permettono la sensibilità dolorifica e che sono amieliniche; Le fibre
nocicettive hanno un diametro molto piccolo e hanno velocità di
conduzione di circa un centinaio di volte inferiore rispetto a quella
dei motoneuroni.

\textbf{Lezione 20151105}

L'attività dei neuroni in termini di potenziale d'azione consiste nel
produrre potenziale di azione (?). Questi si organizzano in raffiche e
l'attività di un neurone è modulata in base alla loro frequenza. Il
segnale che viaggia lungo l'assone è modulato in frequenza in termini di
potenziale d'azione. In questo modo influenza l'attività di altre
cellule.

Il trasferimento di segnale da una cellula all'altra avviene grazie ad
un dispositivo chiamato sinapsi.

\section{Le sinapsi}\label{le-sinapsi}

Una sinapsi è una giunzione tra due elementi cellulari che consentono il
passaggio di messaggi sotto forma di segnali elettrici.

Le sinapsi in \emph{senso stretto} sono quelle \emph{interneuroniche},
che connettono cioè coppie di neuroni e si stabiliscono di norma tra le
terminazioni di una fibra nervosa ed il soma o i dendriti di un neurone.
Esistono poi sinapsi in cui uno dei due elementi cellulari che si
connettono non è di natura nervosa: in questo caso si parla di
\emph{giunzioni}. Le \emph{giunzioni citoneurali} mettono in
comunicazione le cellule recettrici non nervose di un organo di senso
con le terminazioni di una fibra nervosa afferente sensitiva, mentre le
\emph{giunzioni neuromuscolari} connettono le terminazioni di una fibra
nervosa efferente motoria con una cellula o una fibra muscolare.

Una sinapsi rappresenta sempre un punto di discontinuità strutturale di
una via di comunicazione intercellulare: le membrane dei due elementi
che prendono contatto sinaptico, per quanto vicine tra loro, restano
infatti sempre distinde e separate da uno spazio, la \emph{fessura
sinaptica}. Si parla di cellula presinaptica e di cellula postsinaptica.
La cellula presinaptica interagisce con il terminale assonico, mentre la
cellula postsinaptica interagisce con i dendriti.

Così come il potenziale di azione viaggia sempre in una direzione anche
l'attività della sinapsi è unidirezionale.

Questa situazione rimane fissa nel tempo fino a che, eventualmente, la
sinapsi cessa la sua funzione. Le sinapsi hanno dunque una direzionalità
morfologica e funzionale costante.

Sulla base del meccanismo con cui avviene la trasmissione dei segnali,
si distinguono \emph{sinapsi elettriche} e \emph{sinapsi chimiche}.

Le \textbf{sinapsi elettriche} le troviamo soprattutto negli
invertebrati (meno frequentemente nei vertebrati). In queste sinapsi, la
regione di contatto intercellulare si caratterizza perchè le membrane
pre- e postsinaptiche sono estremamente ravvicinate e assumono la
morfologia tipica delle \emph{giunzioni comunicanti (gap junction)} che
le uniscono. Queste giunzioni mettono in contatto il citoplasma delle
cellule pre- e postsinaptiche con un poro piuttosto grande e poco
selettivo.

Quando arriva un potenaiele d'azione nell'assone della cellula
presinaptica, il potenziale arriva fino in fondo all'assone e induce una
forte depolarizzazione nel punto della sinapsi determinando forti
correnti elettroniche attraverso le giunzioni Gap.

Si induce un potenziale d'azione nella cellula postsinaptica. Vi è una
continuità elettrica tra la cellula pre- e postsinaptica.

Le sinapsi elettriche presentano un'elevata velocità di trasmissione.
Hanno un consumo energetico basso perchè utilizzano correnti passive che
passano per un mexzzo di conduzione già pronto e non sono affaticabili
in quanto la sinapsi può condurre il passaggio di potenziale d'azione in
maniera indefinita nel tempo.

Nelle \textbf{sinapsi chimiche} utilizzano un meccanismo diverso perchè
le due cellule (pre- e postsinaptiche) non sono in contatto fisico tra
loro (le loro membrane sono molto ravvicinate ma non si toccano).

Nelle sinapsi chimiche, un \emph{messaggio elettrico} viene convertito
in un \emph{messaggio chimico} atto a ``scavalcare'' la fessura
sinaptica, per poi essere nuovamente \emph{riconvertito in un messaggio
elettrico}.

In queste sinapsi, la trasmissione richiede la liberazione da parte
dell'elemento presinaptico, in risposta alla depolarizzazione di
quest'ultimo, di un \textbf{neurotrasmettitore}, un composto chimico
capace di attivare l'elemento postsinaptico legandosi a recettori
specifici presenti sulla sua membrana.

Perchè possa avvenire lo stimolo sinaptico la cellula deve intraprendere
un'azione metabolica che coinvolge vie e reazioni e fosforilazioni e
consumi di energia metabolici. La sinapsi chimica richiede dunque un
maggior consumo energetico e presenta affaticabilità (le risorse delle
sinapsi possono esaurirsi nel tempo se gli stimoli mantengono frequenza
elevata per lungo tempo).

Le risorse della sinapsi possono esaurirsi nel corso del tempo se gli
stimoli mantengono una frequenza elevata per lungo tempo (ad esempio può
esaurirsi l'ATP necessario per il meccanismo sinaptico).

La funzionalità di queste sinapsi può essere modulata.

Le sinapsi chimiche possono essere eccitatorie o inibitorie della
cellula postsinaptica.

La cellula presinaptica presenta un rigonfiamento contenente delle
vescicole sinaptiche contenenti il neurotrasmettitore.

Il potenziale di azione si ferma prima del terminale assonico ma ha
effetto sulla sua parte terminale determinando il rilascio delle
vescicole sinaptiche nel ridotto spazio della fessura sinaptica.

Sulla cellula postsinatica si trovano dei recettori per i
neurotrasmettitori che determinano la trasduzione del segnale dalla
cellula pre- alla postsinaptica.

Si parla di fenomeni di traffico vescicolare.

Una delle caretteristiche delle cellule eucariotiche è quella di creare
delle vescicole e indirizzarle in punti specifici della cellula o della
membrana cellulare, per poi generare fusione delle vescicole con la
membrana.

Nei neuroni questi eventi avvengono nella regione sinaptica e qui
determinano lo scarico delle vescicole sinaptiche all'esterno.

Nella fusione delle vescicole è coinvolto il calcio. Più in generale,
ogni volta che abbiamo una secrezione ghiandolare da scarico di
vescicole all'esterno, è sempre coinvolto il calcio.

Nelle sinapsi abbiamo segnali del calcio ogni qual volta abbiamo lo
scarico delle vescicole all'esterno.

L'arrivo del potenziale d'azione e del segnale del calcio sono
estremamente interconnessi. Il potenziale d'azione non arriva all'apice
dell'assone (nella zone della sinapsi) perchè in questa zona non sono
presenti i canali del sodio voltaggio-dipendenti che permettono la
formazione del potenziale d'azione, il quale si ferma qualche decina di
micron prima.

L'ultimo lampo del potenziale di azione induce correnti elettrotoniche
che invadono la terminazione sinaptica.

Le correnti eletroniche depolarizzano la porzione presinaptica in
funzione del potenziale d'azione.

Qui vi sono canali voltaggio dipendenti del calcio che si aprono
permettendo un ingresso del calcio che mette in moto un meccanismo di
traffico vescicolare che fa muovere le vescicole verso la superficie,
cioè verso la plasmamembrana e le fa fondere con essa.

In questo modo ciò che era contenuto nella vescicola diffonderà nella
fessura sinaptica.

Il problema consiste nel far fondere le membrane. Queste normalemente
non si fondono e se lo fanno il processo è molto lento.

\subsection{Il meccanismo di rilascio
vescicolare}\label{il-meccanismo-di-rilascio-vescicolare}

Nell'elemente presinaptico, le vescicole sinaptiche contenenti il
neurotrasmettitore non sono distribuite omogeneamente, ma si raccolgono
in prossimità della densità presinaptica. Dal punto di vista funzionale,
si distinguono due gruppi di vescicole sinaptiche:

\begin{enumerate}
\def\labelenumi{\arabic{enumi}.}
\itemsep1pt\parskip0pt\parsep0pt
\item
  le vescicole di un primo gruppo, detto \textbf{pool di rilascio}, si
  trovano immediatamente a ridosso della membrana presinaptica, in
  corrispondenza delle zone attive, dove vengono predisposte
  all'apertura verso lo spazio sinaptico ed al rilascio del
  neurotrasmettitore in esse contenuto;
\item
  le vescicole del secondo gruppo, detto \textbf{pool di riserva}, si
  trovano a maggiore distanza dalla membrana presinaptica e
  progressivamente meno addensato. Esse sono vincolate al citoscheletro,
  e non sono suscettibili di immediato rilascio al sopraggiungere della
  depolarizzazione presinaptica, ma possono essere svincolate dal
  citoscheletro e indirizzate verso le zone attive per rimpiazzare le
  vescicole del \emph{pool di rilascio} man mano che queste vengono
  consumate.
\end{enumerate}

Il processo di apertura delle vescicole sinaptiche si svolge secondo il
paradigma generale dell'\emph{esocitosi vescicolare}.

Ciascuna tappa del ciclo vescicolare dipende dall'intervento di
specifiche molecole proteiche presenti nell'elemento presinaptico.

Il legame delle vescicole del pool di riserva al citoscheletro è mediato
dalle \textbf{sinapsine}, proteine estrinseche associate al versante
citoplasmatico della membrana vescicolare. Le sinapsine sono in grado di
legare le molecole di actina del citoscheletro. L'interazione delle
sinapsine con i filamenti di actina è modulabile in modo
Ca\(^2\)\(^+\)-dipendente.

L'interazione delle sinapsine con i filamenti di actina è modulabile in
modo Ca\(^2\)\(^+\)-calmodulina-dipendente di tipo II (CaMK-II), una
volta attivata dal complesso \textbf{Ca\(^2\)\(^+\)-CaM}, è in grado di
fosforilare le sinapsine, il che ne riduce l'affinità per l'actina,
promuovendo quindi il distacco delle vescicole.

Questo meccanismo di liberazione delle vescicole del \emph{pool} di
riserva, che da questo punto in poi potranno essere smistate al
\emph{pool} di rilascio, è tanto più attivo quanto più intensa è
l'attività sinaptica.

Il distacco dal citoscheletro di una vescicola del \emph{pool} di
riserva è seguito dalla sua mobilizzazione e dal suo indirizzamento
(``\emph{sorting}'') verso una zona attiva.

Questo processo richiede l'intervento di proteine estrinseche della
membrana vescicolare dette \textbf{Rab3}. Queste sono proteine
monomeriche leganti il GTP e sono dotate di attività GTPasica. Nella
forma legata al GTP Rab3 ``contrassegna'' le vescicole che devono essere
trasportate verso le zone attive.

La successiva tappa del ciclo vescicolare è quella detta di ``attracco''
o \emph{docking}, nella quale la vescicola viene vincolata alla zona
attiva.

In questa fase si stabiliscono interzioni fra proteine vescicolari e
proteine della membrana presinaptica. Fra le proteine implicate nel
\emph{docking} menzioniamo la \textbf{sinaptogamina}, proteina integrale
della membrana vescicolare, e due proteine della membrana presinaptica
delle zone attive, cioè la \textbf{neurezina I} e \textbf{SNAP-25}. La
\emph{sinaptogamina} lega anche un complesso proteico presinaptico.

Al termine del processo di \emph{docking} la vescicola non è ancora
disponibile a essere rilasciata esocitoticamente.

Perchè la vescicola possa essere rilasciata esocitoticamente deve
seguire un ulteriore processo, detto di \emph{priming}, che rende la
vescicola \emph{competente} alla fusione con la membrana presinaptica in
risposta alla depolarizzazione presinaptica e all'aumento della
concentrazione citosolica del Ca\(^2\)\(^+\) nel versante presinaptico.

Il \emph{priming} vescicolare richiede l'interazione fra specifiche
proteine vescicolari e della membrana presinaptica, dette
complessivamente \textbf{proteine SNARE}.

Tali proteine si suddividono in proteine SNARE della membrana
vescicolare, o \textbf{v-SNARE} (presenti sulla vescicola), e proteine
della membrana ``bersaglio'' (\emph{target}: la membrana presinaptica a
livello delle zone attive), o \textbf{t-SNARE} (è la molecola bersaglio
della porzione postsinaptica).

La v-SNARE più importante nel processo di \emph{priming} è la
\textbf{sinaptobrevina} (o \textbf{VAMP}), una proteina integrale di
membrana vescicolare dotata di un solo segmento transmembranario e un
lungo dominio citoplasmatico N-terminale.

Nel corso del \emph{priming}, si instaura una stretta interazione fra le
proteine SNARE. Il procedere di tale interazione sviluppa una potente
forza traente che, al termine del processo, porta la membrana
vescicolare a contatto con la membrana presinaptica. La forza traente
così sviluppata è sufficientemente intensa da scoprire parzialmente il
``\emph{core}'' lipidico di ciascuna membrana, che può così stabilire
un'interazione idrofobica con quello dell'altra membrana.

Le vescicole vengono rifornite al terminale sinaptico dal corpo
cellulare tramite l'assone che funge da convogliatore di vescicole
grazie al citoscheletro.

\subsection{Meccanismi postsinaptici}\label{meccanismi-postsinaptici}

Le molecole del neurotrasmettitore, liberate dall'apertura delle
vescicole sinaptiche, diffondono nello spazio sinaptico e si legano a
specifici recettori chimici presenti nella membrana postsinaptica.

Il ruolo dei recettori postsinaptici va ben oltre quello di semplici
``spie'' dell'avvenuta liberazione del neurotrasmettitore, perchè da
essi dipendono sia il \emph{segno} (eccitamento o inibizione) che
l'intensità della risposta postsinaptica.

Esistono 2 grandi classi di recettori postsinaptici, che si distinguono
per la struttura delle molecole proteiche che li costituiscono e per il
loro modo di operare: i \emph{recettori ad azione diretta} o
\emph{ionotrocipi} (recettori che costituiscono essi stessi un canale
ionico) ed i \emph{recettori ad azione indiretta} o \emph{metabotropici}
(recettori accoppiati a proteine G trimeriche che vanno a stimolare
l'apertura di canali ionici).

\subsubsection{Recettori ionotropici}\label{recettori-ionotropici}

I \emph{recettori ionotropici} (o \emph{recettori-canale}) sono molecole
proteiche che comprendono una porzione recettoriale rivolta verso los
pazio sinaptico ed una porzione strutturata in canale ionico che
attraversa tutto lo spessore della membrana. La proteina è costituita da
più subunità che, essendo disposte in cerchio attorno a un asse
centrale, vengono a delimitare un condotto, costituendo un canale ionico
chemio-dipendente.

In assenza del neurotrasmettitore, il canale è generalmente nello stato
chiuso ed impervio agli ioni.

Quando invece le molecole del neurotrasmettitore si legano ai propri
siti di riconoscimento del dominio recettoriale, il canale passa nello
stato aperto e gli ioni permeanti possono fluirvi secondo il proprio
gradiente elettrochimico.

Ne viene generata una corrente che, a seconda della natura del
recettore-canale e della sua selettività ionica, potrà essere entrante,
quindi depolarizzante, oppure uscente, quindi iperpolarizzante: nel
primo caso la corrente indurrà una depolarizzazione della membrana
postsinaptica della \textbf{potenziale postsinaptico eccitatorio
(EPSP)}, nel secondo caso un'iperpolarizzazione detta \textbf{potenziale
postsinaptico inibitorio (IPSP)}.

I recettori ionotropici agiscono molto rapidamente, questo giustifica il
termine \emph{trasmissione sinaptica rapida}.

\subsubsection{Recettori metabotropici}\label{recettori-metabotropici}

I recettori metabotropici sono molecole proteiche costituite da una
singola catena polipeptidica comprendente 7 \emph{segmenti idrofobici
transmembranari}. Hanno anch'essi uno o più domini recettoriali esposti
allo spazio sinaptico e predisposti al legame col neurotrasmettitore,
tuttavia non formano un canale transmembranario.

La proteina recettoriale sporge verso il versante intracellulare della
membrana con un dominio effettore predisposto a legarsi con una proteina
G trimerica. Quando questa viene attivata dall'interazione con il
recettore, a sua volta attivato dal legame con il suo ligando, può
innescare eventi di vario tipo: può interagire direttamente sul piano
della membrana con un canale ionico, modificandone lo stato di pervietà
oppure può associarsi a un \emph{enzima allosterico di membrana} (es.
adenilaco ciclasi o fosfolipasi C), attivandolo. Ne deriva, in questo
secondo caso, la sintesi di uno o più \emph{secondi messaggeri}, che a
loro volta possono legarsi direttamente a un canale ionico di membrana
oppure attivare a valle altre proteine implicate nella trasduzione del
segnale.

In ultima analisi viene indotta l'apertura o la chiusura di canali
ionici della membrana postsinapticae, a seconda della selettività ionica
di questi ultimi, viene generata una corrente netta di membrana che si
traduce in un EPSP o in un IPSP.

È evidente che i potenziali postsinaptici insorgono con una latenza
tanto maggiore quanto più estesa è la catena di eventi che porta alla
modificazione regolatoria dei canali di membrana.

I potenziali sinaptici generati con questo tipo di meccanismo sono
spesso caratterizzati da un'insorgenza lenta e anche da un lento
decadimento, che li rende assai più duraturi dei potenziali sinaptici
dovuti all'attivazione di recettori ionotropici.

Questo tipo di trasmissione sinaptica viene chiamata \emph{trasmissione
sinaptica lenta}.

Molti neurotrasmettitori ``classici'' (acetilcolina, GABA, glutammato,
serotonina) dispongono di recettori sia ionotropici che metabotropici,
non di rado espressi sulle stesse sinapsi.

Per l'acetilcolina il recettore ionotropico, noto come \emph{recettore
nicotinico}, è un canale ionico di tipo promiscuo che permette il
passaggio di ioni positivi mono- e bivalenti. Permette il passaggio di
una corrente del sodio a causa dell'ingente abbondanza dello ione. Il
recettore metabotropico dell'acetilcolina, invece, è chiamato
\emph{recettore muscarinico} ed è legato a una proteina G che va a
stimolare le correnti ioniche tramite l'utilizzo di secondi messaggeri
come l'AMP-ciclico.

I nomi di questi recettori derivano dal fatto che l'agonista fisiologico
è l'acetilcolina, ma quelli farmacologici sono la \emph{nicotina} e la
\emph{muscarina} (metabolita presente nel fungo \emph{Amanita
muscaria}).

Sia i recettori ionotropici che metabotropici pososno far insorgere
nella cellula postsinaptica dei segnali elettrici per mantenere e
trasmettere l'eccitazione da una cellula all'altra.

Il primo evento elettrico che si produce nella cellula post-sinaptica è
un potenziale post-sinaptico che origina una variazione del potenziale
di membrana (questo non è un potenziale d'azione).

Come già detto i segnali postsinaptici possono essere di due tipi: EPSP
(eccitatorio) e IPSP (inibitorio).

Nelle \emph{sinapsi eccitatorie (EPSP)} il neurotrasmettitore induce
correnti al sodio nella cellula postsinapticha che ne depolarizzano.

Nel punto della sinapsi, sul dendrite della cellula postsinaptica,
abbiamo una depolarizzazione più debole di quella del potenziale
d'azione, ma abbastanza forte da essere sentita dalla cellula e da
creare correti elettrotoniche che viaggiano in parte verso la fine del
dendrite e in parte verso il corpo cellulare tendendo ad invaderlo.

Ogni neurone riceve qualche migliaio di sinapsi. Una sola sinapsi non è
consistente ma tutte insieme, ciascuna eccitatoria e dunque inducente
correnti elettroniche, creano un flusso di correnti elettroniche che
creano una depolarizzazione significativa nel corpo cellulare.

Una zona dove sono molto concentrati i canali per il sodio è il
\emph{cono di emergenza dell'assone}, ovvero la base dell'assone.

I canali sentono le correti elettrotoniche prodotte dalle sinapsi
eccitatorie e se queste correnti superano la soglia di innesco del
potenziale d'azione, questo parte e si propaga lungo l'assone dove si
troverà un'altra sinapsi.

Una singola sinapsi è poca cosa rispetto alle dimensioni della cellula e
genera correnti piccole, mentre più sinapsi insieme possono
\emph{collaborare per sommazione} che può essere \emph{spaziale} o
\emph{temporale}.

Nella \textbf{sommazione spaziale} si ha la concomitanza in vari punti
di stimoli eccitatori nel complesso dendritico che raggiungono così la
soglia di depolarizzazione sommando, mentre nella \textbf{sommazione
temporale} si ha la sommazione di contatti postsinaptici conseguenti nel
tempo che porta a depolarizzazione della membrana. Prima che arrivi il
secondo potenziale la membrana non è ancora tornata in condizioni
standard e quindi con il secondo potenziale si depolarizza ancora di
più.

Le \emph{sinapsi inibitorie (IPSP)} inducono potenziali postsinaptici
inibitori che creano iperpolarizzazione, mediata da canali potassio e
sodio, della membrana.

Il potenziale viene bloccato a potenziale di riposo della membrana, e in
termini di sommazione di potenziali inibitori e eccitatori l'effetto sul
cono di emergenza dell'assone è neutralizzato dal potenziale inibitorio.

La sinapsi eccitatoria è neutralizzata da quella inibitoria.

La scarica di un neurone dipenderà dalla somma di sinapsi eccitatorie ed
inibitorie.

\textbf{L'uscita assonica è l'integrale delle sinapsi dei dendriti (?)}

\subsubsection{Interruzione della trasmissione
sinaptica}\label{interruzione-della-trasmissione-sinaptica}

I canali possono essere desensibilizzati tramite una modificazione
conformazionale. Il neurotrasmetitore può essere presente in quantità
non sufficienti o eccessive.

Inoltre la trasmissione sinaptica può essere interrotta tramite
riassorbimento del neurotrasmettitore o incorporazione per endocitosi o
degradazione enzimatica. L'acetilcolina ad esempio, poù essere degradata
e dunque inattivata dall'\emph{acetilcolinaesterasi}.

\subsubsection{La plasticità
sinaptica}\label{la-plasticituxe0-sinaptica}

È stato scoperto che l'attività di stimolo esercitata su una sinapsi ne
può condizionare l'attività stessa. Questo fenomeno è definito
\textbf{``plasticità sinaptica''}.

Una sinapsi non lavora sempre nello stesso modo, ma il suo funzionamento
è condizionato dalla quantità degli stimoli che riceve.

Più la sinapsi lavora e più diventa idonea al suo lavoro. Si parla si
\emph{``potenziamento sinaptico''}.

Esistono sono vari tipi di potenziamento tra cui:

\begin{enumerate}
\def\labelenumi{\arabic{enumi}.}
\itemsep1pt\parskip0pt\parsep0pt
\item
  la \textbf{facilitazione}, quando una sinapsi è stimolata intensamente
  per tempi brevi;
\item
  il \textbf{potenziamento post-tetanico} (PTP, durata 1-2 min);
\item
  il \textbf{potenziamento a breve termine}, (STP, stimolazioni
  intermedie, durata di decine di min - un'ora);
\item
  il \textbf{potenziamento a lungo termine} (LTP, ore o giori a
  stimolazioni molto forti, 100 Hz, per pochi secondi).
\end{enumerate}

Questi fenomeni sono stati riscontrati analizzando piccoli gruppi di
sinapsi.

\textbf{Nel complesso del sistema nervoso e di regioni specifiche del
cervello, questi fenomeni possono osservare attività macroscopiche a
livello del soggetto (?)}

In cosa consiste il potenziamento? In un miglior funzionamento delle
sinapsi, che più facilmente indurranno una scarica nella cellula
postsinaptica.

La LTP può durare per tempi molto lunghi.

Si è notato che in certe regioni cerebrali, come ad esempio l'ippocampo,
questi fenomeni sono più manifesti. L'ippocampo può essere visto come un
ricciolo marginale della calotta neuronale costituita dagli emisferi
cerebrali, ossia la corteccia.

In questa zona sono stati visti in maniera vistosa fenomeni di LTP.
L'ippocampo è la regione coinvolta nei processi di memoria; danni in
questa regione provocano disturbi seri della memoria.

I recettori coinvolti sono recettori del glutammato. Ne esistono tipi
\emph{NDMA} e \emph{non NMDA} che mediano ingressi di sodio e calcio.

Quando il glutammato agisce i recettori sono bloccati dal Mg.

Una depolarizzazione della membrana, che viene sentita dal canale (pur
non essendo voltaggio-dipendente) provoca il rilascio del Mg dal
\emph{canale NDMA} e a questo punto c'è entrata di sodio e calcio.

I canali \emph{non NMDA} reagiscono direttamente al glutammato creando
una corrente del sodio che depolarizza la membrana. Di questo ne risente
il recettore NMDA che toglie il magnesio e lascia entrare sodio e
calcio. Questo induce il segnale del calcio che attiva delle chinasi
calcio dipendenti.

A questo punto si ha produzione di ossido nitrico tramite la ossido
nitrico sintasi. L'ossido nitrico va ad agire sulla cellula presinaptica
potenziando la sinapsi e dunque il rilascio del neurotrasmettitore.

Questo è il possibile meccanismo cellulare di insorgenza di LTP.

La cellula postsinaptica può influenzare l'attività della cellula
presinaptica.

Allargando la visuale vediamo che si formano plasticità sinaptiche che
generano circuiti preferenziali potenziati. Si pensa che la memoria sia
questa cosa.

\textbf{Lezione 20151109}

\section{Il sistema nervoso}\label{il-sistema-nervoso}

Il sistema nervoso è suddiviso anatomicamente in centrale e periferico.

Il \textbf{sistema nervoso centrale (SNC)} è costituito da encefalo e
midollo spinale, mentre il \textbf{sistema nervoso periferico (SNP)} è
costituito da gangli e nervi.

Nel sistema nervoso esistono due tipi fondamentali di cellule: i
\emph{neuroni} e cellule non neuronali o \emph{cellule gliali}.

Le cellule neuronali sono cellule eccitabili capaci di produrre
potenziali d'azione in risposta ad stimoli elettrici, mentre le cellule
gliali non sono eccitabili.

Si è stimato che il sistema nervoso centrale contiene circa cento
miliardi (10\(^1\)\(^1\)) di neuroni collegati fra loro da centomila
miliardi (10\(^1\)\(^4\)) di sinapsi presenti nell'encefalo e nel
midollo spinale. Mediamente ogni cellula neurale stabilisce almeno 103
sinapsi.

Il SNC è formato per il 75-90\% da cellule gliali (\emph{neuroglia}), e
solo per circa il 10\% da neuroni.

Le cellule gliali si suddividon in: cellule di Schawann,
oligodendrociti, microglia, cellule ependimali e astrociti.

Come già detto i neuroni sono intercollegati tra loro, e ogni 4-5
passaggi il neurone di partenza viene ricontattato da quello finale.
Questo in accordo con il concetto di loop già visto.

Nel SNC lo schema del loop non indica la presenza di una ``geometria'',
ma dice che esistono dei contatti di controllo. Nel sistema nervoso
centrale lo schema funzionale è rappresentato in maniera topografica. È
sempre una topografia a loop.

Le cellule gliali sono cellule che stabiliscono migliaia di contatti con
le cellule neurali. Soprattutto le cellule di Shwann e gli astrociti
(questi rivestono anche i capillari).

A livello cefalico abbiamo una grandissima protezione offerta dal
cranio, ma anche da una serie di membrane connettivali che attutiscono
notevolmente i traumi indotti da urti improvvisi dell'encefalo contro il
cranio stesso.

Questi strati di tessuto vengono chiamati \textbf{meningi} e includono:

\begin{itemize}
\itemsep1pt\parskip0pt\parsep0pt
\item
  la \textbf{dura madre}, è lo strato più esterno formato da uno strato
  spesso e molto denso;
\item
  l'\textbf{aracnoide}, una membrana intermedia a forma di rete;
\item
  la \textbf{pia madre}, è lo strato più interno formato da una membrana
  sottile che riveste tutti i meandri del sistema nervoso centrale.
\end{itemize}

Mentre di norma non vi è alcuno spazio tra la dura madre e l'aracnoide,
tra l'aracnoide e la pia madre vi è uno spazio, lo \textbf{spazio
subaracnoideo}, pieno di \emph{liquido cerebrospinale}.

Il \textbf{liquido cerebrospinale (LCS)} è un liquido limpido che bagna
il SNC; esso ha una composizione simile, ma non identica, al plasma. Il
liquido cerebrospinale non soltanto circonda il sistema nervoso
centrale, ma si insinua anche all'interno di esso, circondando i neuroni
e le cellule gliali e riempiendo alcune cavità presenti all'interno
dell'encefalo e del midollo spinale.

L'encefalo contiene 4 di tali cavità, chiamate \textbf{ventricoli}, che
comunicano tra loro. I due ventricoli laterali a forma di C sono
connessi al terzo ventricolo, mediale, dal \emph{forame
interventricolare}. L'acquedotto cerebrale, chiamato \textbf{acquedotto
di Silvio}, connette il III ventricolo al quarto, che è la continuazione
del \emph{canale centrale}, una lunga e sottile cavità cilindrica che
percorre per tutta la sua lunghezza il midollo spinale.

Il rivestimento interno dei ventricoli e del canale centrale è composto
da cellule gliali, chiamate \textbf{cellule ependimali}, che
costituiscono un tipo particolare di cellule epiteliali.

Il rivestimento dei ventricoli è vascolarizzato e forma un tessuto
chiamato \textbf{plesso coroideo}, che consta di pia madre, capillari e
cellule ependimali.

Il liquido cerebrospinale viene prodotto dal plesso coroideo.

Questi producono un liquido che riversano nel ventricolo alimentando il
liquido cerebrospinale che, in altri punti chiamati villi, viene
assorbito dal sangue fornendo un ricambio continuo.

Appena prodotto il liquido cerebrospinale (LCS) attraversa il sistema
ventricolare ed entra nello spazio subaracnoideo. Il LCS, a livello
subaracnoideo, viene in parte riassorbito nel sangue venoso attraverso
speciali strutture, i \emph{villi aracnoidei}, localizzati alla sommità
dell'encefalo.

Il ricambio completo del LCS avvine a un ritmo di circa tre volte al
giorno. Il LCS occupa un volume di circa 120-150 ml.

Poichè il SNC è completamente circondato dal LCS, galleggia in esso e,
pertanto, il LCS agisce come una struttura ammortizzante che previene le
collisioni del tessuto nervoso con la scatola cranica. Inoltre,
contribuisce a mantenere stabile la composizione ionica all'esterno
delle cellule, e fornisce i nutrienti alle cellule gliali e ai neuroni e
allontana da tali cellule i prodotti di rifiuto.

La composizione del LCS è la stessa del plasma per quanto riguarda il
tenore ionico; la concentrazione del glucosio invece è più bassa, così
come la concentrazione proteica.

Il sistema vascolare e quello nervoso sono collegati da un meccanismo a
loop, anche se la struttura del sistema vascolare non sembra rifletterne
la funzionalità.

Nel sistema nervoso vi è una ricchissima vascolarizzazione (riceve circa
il 15\% del sangue), consuma il 20\% dell'ossigeno e il 50\% del
glucosio consumato dal corpo. A differenza di muscoli e fegato,
l'encefalo non accumula glucosio e non utilizza lipidi come fonte
energetica, perciò il suo metabolismo dipende in toto dall'ossigeno e
dal glucosio che arrivano momento per momento dal sangue.

Questo è il motivo per cui le ischemie cerebrali creano facilmente danni
irreversibili.

Il rapporto tra il sangue e il tessuto del SNC è uno dei più complessi a
causa della barriera ematoencefalica.

\subsection{La barriera
ematoencefalica}\label{la-barriera-ematoencefalica}

Come negli altri tessuti, gli scambi di ossigeno, glucosio ed altre
sostanze tra il sangue e le cellule del SNC si realizzano attraverso le
paretu dei \emph{capillari}. Le pareti dei capillari sono costituite da
un singolos trato sottile di \emph{cellule endoteliali}, cosa che
permette gli scambi gassosi per \emph{diffusione}.

Nella maggior parte dei tessuri, piccole molecole come gas, ioni
inorganici, monosaccaridi ed amminoacidi attraversano liberamente la
parete capillare. Mentre le molecole idrofobiche diffondono attraverso
interruzioni relativamente grandi (\emph{pori}) che si trovano tra le
cellule endoteliali.

Nel SNC il movimento delle moleocle idrofiliche attraverso le pareti dei
capillari é limitato dalla \textbf{barriera ematoencefalica}.
L'esistenza di questa barriera è dovuta alla presenza di giunzioni
strette tra le cellule endoteliali dei cpaillari, che non permettono la
formazione di pori capillari e quinidi limitano la diffusione di
molecole idrofiliche tra le cellule.

Gli astrociti stimolano le cellule endoteliali a sviluppare e mantenere
le giunzioni strette.

Questa barriera protegge il SNC da sostanze tossiche che possono essere
presenti nel sangue, perchè limita il movimento delle molecole
attraverso l'endotelio capillare. Gas e altre molecole idrofobiche
attraversano abbastanza faiclemnte le membrane cellulari, in quanto sono
in grado di attraversare il doppio strato fosfolipidico della membrana
per diffusione semplice. Le sostanze idrofiliche invece, come ioni,
zuccheri ed amminoacidi, non possono attrracersare le mebrane cellulari
mediante diffusione semplice e pertanto devono fare affidamento sul
trasporto mediato per poter attraversare le pareti capillari del SNC.

Poichè il trasporto mediato utilizza proteine di trasporto che sono
specifiche solo per certe molecole, la barriera ematoencefalica è
selettivamente permeabile, permettendo solo ad alcuni composti di
attraversarla.

Molecole come glucosio e amminoacidi possono penetrrare la barriera
ematoencegalica facilmente. Il glucosio è trasportato attrverso la
barriera ematoencefalica da proteine trasportatrici o ``carrier''
GLUT-1.

La barriera ematoencefalica costituisce un problema per le terapie
medicinali, perchè i farmaci devono obbligatoriamente superarla e dunque
essere o lipofili o sfruttare dei trasportatori già presenti.

Il trasporto di sostanze dal sangue al tessuto nervoso è tenuto dunque
sotto un controllo molto stretto.

\subsection{Sostanza grigia e sostanza
bianca}\label{sostanza-grigia-e-sostanza-bianca}

Il SNC ha una disposizione dei neuroni molto ordinata. I corpi
cellulari, i dendriti e i terminali assonici formano agglomerati
(cluster) che appaiono grigi, mentre gli assoni si raggruppano a formare
strutture che appaiono bianche. Si parla, perciò, di \textbf{materia
grigia} e \textbf{materia bianca}.

La sostanza grigia costituisce circa il 40\% del SNC ed è qui che si
realizzano la trasmissione e l'integrazione neuronale. L'altro 60\% dei
SNC è formato da sostanza bianca, costituita per lo più da assoni
mielici, ai quali deve il proprio colore.

La mielina presenta un elevato contenuto lipidico ed è per questo che
appare bianca. Gli assoni mielinici sono specializzati nella
trasmissione rapida delle informazioni.

Guardando la superficie esterna dell'encefalo, è visibile soltanto la
sostanza grigia, perchè la maggior parte della struttura a forma di
globo (chiamata \emph{cervello}) che costituisce la massa dell'encefalo
è interamente coperta da uno strato sottile di materia grigia, chiamato
\textbf{corteccia cerebrale}.

La sostanza bianca è localizzata al di sotto di questo strato;
incastonate sotto la sostanza bianca vi sono piccole aree di sostanza
grugua note come nuclei.

Nel midollo spinale la disposizione è diversa: la sostanza bianca è
posizionata all'esterno, mentre la sostanza grigia si trova all'interno.

Nella sostanza bianca del SNC, gli assoni (anche noti come \textbf{fibre
nervose}) sono organittazi in fasci o tratti che collegano una regione
di sostanza grigia con un'altra. Differenti fasci nervosi sono
classificati in funzione delle regioni che collegano:

\begin{itemize}
\itemsep1pt\parskip0pt\parsep0pt
\item
  i \textbf{fasci di proiezione} connettono la corteccia cerebrale con
  aree encefaliche dislocate a livelli inferiori o con il midollo
  spinale;
\item
  le \textbf{fibre di associazione} connettono un'area della corteccia
  cerebrale con un'altra sullo stesso lato del cervello;
\item
  le \textbf{fibre commissurali} collegano l'area corticale di un
  emisfero con la corrispondente area corticale dell'altro emisfero.
\end{itemize}

La maggioranza delle fibre commissurali è localizzata in una banda di
tessuto chiamata \textbf{corpo calloso}, che collega tra loro i due
\textbf{emisferi cerebrali}. L'encefalo infatti è diviso in due da una
profonda separazione, le due porzioni ``separate'' sono gli emisferi
destro e sinistro.

Alla base dell'encefalo si può notare una protuberanza di tessuto: il
midollo spinale.

\subsection{Il midollo spinale**}\label{il-midollo-spinale}

Il midollo spinale si presenta ocme una struttura di tessuto nervoso di
forma cilindrica che origina dalla porzione terminale del bulbo ed è
circondata dalla \textbf{colonna vertebrale}. Il midollo spinale di un
soggetto adulto è lungo circa \emph{44 cm} e ha un diametro che varia
tra 1 e 1,4 cm.

\subsubsection{I nervi spinali}\label{i-nervi-spinali}

Dal midollo spinale si dipartono, ad intervalli regolari, \emph{31 paia}
di \textbf{nervi spinali}.

I nervi rappresentano il corrispondente delle vie neurali al di fuori
del SNC, quindi nel SNP (sono sempre fasci di fibre). Questi hanno un
rivestimento connettivale che rende il nervo un filamento tangibile. Nei
nervi si trova un fascio di assoni derivante da neuroni di tipo diverso.

Ciascun paio di nervi fuoriesce dalla colonna vertebrale ed è definito
in base alla regione di midollo spinale dal quale origina.

Vi sono:

\begin{itemize}
\itemsep1pt\parskip0pt\parsep0pt
\item
  8 paia di \textbf{nervi cervicali} (C1-C8), che emergono dalle
  vertebre della regione del collo;
\item
  12 paia di \textbf{nervi toracici} (T1-T12), che emergono dalla
  regione toracica;
\item
  5 paia di \textbf{nervi lombari} (L1-L5), che emergono dalla regione
  lombare;
\item
  5 paia di \textbf{nervi sacrali} (S1-S5), che emergono dal coccige;
\item
  un singolo \textbf{nervo coccigeo} (C\(_o\)), che emerge dalla punta
  del coccige.
\end{itemize}

Il midollo spinale si estende solo per i 2/3 della lunghezza della
colonna vertebrale. L'ultimo terzo della colonna non contiene midollo
spinale, ma solo nervi che emergono da essa. Questa regione è conosciuta
come \emph{cauda equina}.

Le numerose fibre nervose che compongono un singolo nervo spinale
viaggiano verso regioni adiacenti del corpo. Pertanto, è possibile
disegnare sulla superficie corporea differenti regioni, chiamate
\emph{dermatomeri}, ciascuna delle quali è innervata da un particolare
nervo spinale.

La faccia non ha una rappresentazione dermatomerica, perchè è innervata
dai \emph{nervi cranici}, che emergono dal cranio.

Le meningi (aracnoide, dura e pia madre), insieme al liquido
cerebrospinale, sono sempre presenti anche nel midollo spinale.

\subsubsection{La sostanza grigia e bianca nel midollo
spinale}\label{la-sostanza-grigia-e-bianca-nel-midollo-spinale}

La sostanza grigia è localizzata in un'area interna a forma di farfalla,
mentre la sostanza bianca è localizzata attorno alla grigia.

La sostanza grigia contiene interneuroni, corpi cellulari e dendriti di
\textbf{neuroni efferenti} (neuroni che viaggiano nei nervi spinali
diretti verso gli organi effettori), e i terminali degli \textbf{assoni
afferenti} (i neuroni afferenti viaggiano nei nervi spinali verso il
midollo spinale, partendo dai recettori sensoriali localizzati alla
periferia del corpo).

La sostanza grigia del midollo spinale è organizzata in modo tale che
differenti tipi di neuroni sono localizzati in differenti regioni.

La sostanza grigia comprende un \textbf{corno dorsale} ed un
\textbf{corno ventrale} in ogni lato.

Il \emph{corno dorsale} comprende la metà dorsale (posteriore) della
sostanza grigia di ogni lato; il \emph{corno ventrale} (anteriore)
comprende quella ventrale.

Le \emph{fibre afferenti} originano dalla periferia come recettori
sensoriali e terminano nel corno dorsale, dove formano sinapsi con
interneuroni o direttamente con neuroni efferenti.

I nervi spinali sono 2 per ogni livello vertebrale ed emergono dal
midollo con due radici (una dorsale e una ventrale) che confluiscono poi
in un unico nervo.

La radice dorsale contiene fibre afferenti, assoni sensoriali il cui
stimolo viaggia verso il SNC. Nella radice ventrale invece, si trovano
fibre motorie il cui potenziale di azione viaggia verso la periferia dal
SNC.

I corpi cellulari dei neuroni afferenti non sono localizzati nel midollo
spinale, ma all'esterno, dove sono raggruppati nei \textbf{gangli delle
radici dorsali} (il termine \emph{ganglio} definisce un gruppo di
neuroni i cui corpi cellulari sono localizzati all'esterno del SNC).

I corpi cellulari dei \emph{neuroni efferenti}, invece, sono localizzati
dentro il midollo spinale. I neuroni efferenti originano nel corno
ventrale e si dirigono verso la periferia, dove formano sinapsi con le
fibre muscolari scheletriche.

Molte delle fibre ascendenti e discendenti si incrociano, ovvero partono
da sinistra e arrivano a destra. Fibre afferenti portano lo stimolo
verso il midollo spinale per poi salire lungo tratti ascendenti
incrociando e continuando a salire. Il tratto discendente proviene dalla
corteccia e va al midollo spinale incrociando e uscendo poi
ventralmente.

\subsubsection{L'encefalo}\label{lencefalo}

L'encefalo consta di 3 parti principali:

\begin{itemize}
\itemsep1pt\parskip0pt\parsep0pt
\item
  il \emph{prosencefalo}, formato da \emph{telencefalo} e
  \emph{diencefalo};
\item
  il \emph{mesencefalo} o \emph{cervelletto};
\item
  il \emph{tronco cefalico}.
\end{itemize}

Il \textbf{prosencefalo}, la parte più ampia e rostrale, è diviso nei
due \emph{emisferi} di destra e sinistra; esso consta del telencefalo
(la corteccia esterna, 80\% del cervello), che è la parte più anteriore,
e del diencefalo.

La corteccia è suddivisa in due zone emisferiche che rivestono la
sostanza bianca dei nuclei della base. La parte superiore è chiamata
``tetto'' e quella inferiore ``base''.

I nuclei della base stanno in basso. Nel diencefalo abbiamo talamo ed
ipotalamo in posizione ventrale.

Centralmente al talamo abbiamo le cavità dei ventricoli cerebrali.
Posteriormente c'è il tronco dell'encefalo che si suddivide in
mesencefalo, ponte e midollo allungato. Al di sopra del ponte abbiamo il
cervelletto. Dall'encefalo emergono i nervi cranici (sono 12) che
innervano gli organi di senso cefalici e varie regioni cefaliche anche
del collo.

(immagine 9.11c p 229)

Il \textbf{cervelletto} è una struttura bilaterale e simmetrica con una
corteccia situata all'esterno e nuclei situati in profondità. Si trova
inferiormente al prosencefalo e dorsalmente al tronco encefalico.

Il cervelletto svolge funzioni fondamentali nel controllo dell'attività
motoria e nel mantenimento dell'equilibrio.

Il \textbf{tronco encefalico} rappresenta la porzione più caudale
dell'encefalo; esso connette il prosencegalo ed il cervelletto con il
midollo spinale.

Il tronco encefalico consta di 3 regioni principali:

\begin{enumerate}
\def\labelenumi{\arabic{enumi}.}
\itemsep1pt\parskip0pt\parsep0pt
\item
  il \textbf{mesencefalo}, la porzione più rostrale che collega il
  tronco encefalico al prosencefalo;
\item
  il \textbf{ponte}, la porzione mediana che si connette al cervelletto;
\item
  il \textbf{midollo allungato} o \textbf{bulbo}, la porzione più
  caudale che si connette al midollo spinale.
\end{enumerate}

All'interno del tronco encefalico originano 10 dei 12 \emph{nervi
cranici}.

All'interno del tronco encefalico vi è anche la \textbf{formazione
reticolare}, una diffusa rete neuronale che svolge un ruolo importante
nel controllo dei cicli sonno-veglia, dell'eccitazione corticale e dello
stato di coscienza. In aggiunta, interviene nella regolazione di molte
funzioni involontarie controllate dal sistema nervoso autonomo, quali la
funzione caridiovascolare e la digestione.

Il \textbf{diencefalo}, che si trova localizzato al di sotto del
cervello, comprende due strutture mediane: il \emph{talamo} e
l'\emph{ipotalamo}.

Il \textbf{talamo} (dorsale) è un aggregato di nuclei sottocorticali
localizzato nel diencefalo. Tutte le informazioni sensoriali seguono un
percorso che include un passaggio attraverso il talamo e quindi verso la
corteccia. La maggior parte dei segnali sensoriali è filtrata e
modificata nel talamo prima di essere trasmessa alla corteccia. In
questo modo, il talamo sembra essere importante nel dirigere
l'attenzione.

Il talamo svolge anche un ruolo nel controllo dei movimenti.

L'\textbf{ipotalamo} (ventrale, inferiormente al talamo) rappresenta il
principale centro di collegamento tra il sistema endocrino e quello
nervoso. In risposta a segnali nervosi o ormonali, l'ipotalamo rilascia
ormoni che regolano il rilascio di altri ormoni dall'adenoipofisi
(ipofisi anteriore). Controlla anche il rilascio di ormoni dall'ipofisi
posteriore, come l'ormone antidiuretico che regola il volume e
l'osmolarità del plasma.

L'ipotalamo influenza anche molti comportamenti; in esso sono presenti
centri nervosi che controllano la sazietà e la fame, e il centro della
sete. Inoltre, essendo parte del sistema limbico, esso influenza le
emozioni ed i comportamenti che da esse dipendono.

La \textbf{corteccia cerebrale} rappresenta la porzione più esterna del
cervello; essa consta di uno strato sottile ed altamente convoluto di
sostanza grigia. Le circonvoluzioni originano da \textbf{solchi}
(invaginazioni) e \textbf{giri} (creste) che permettono all'ampio volume
di sostanza grigia di essere contenuto all'interno della scatola
cranica.

La corteccia cerebrale ha uno spessore che può variare, in relazione
alla localizzazione, da 1,5 a 4 mm. Sebbene la corteccia sia sottile, è
formata da sei strati funzionalmente distinti il cui spessore (e a volte
anche la presenza) dipende dalla localizzazione corticale. I diversi
strati sono composti da cellule differenti; ad esempio nel quarto strato
troviamo cellule stellate che eleaborano le percezioni sensoriali,
mentre nel quinto strato troviamo cellule piramidali che si occupano del
movimento volontario.

La corteccia cerebrale svolge le funzioni cerebrali più elevate ed
evolute, in quanto ci permette di avere percezioni relative all'ambiente
che ci circonda, formulare pensieri, ricordare eventi passati e, infine,
rappresenta l'area da cui partono tutti i comandi per l'esecuzione dei
movimenti.

Ciascun emisfero è diviso in 4 regioni dette \emph{lobi}.

Il \textbf{lobo frontale} rappresenta la parte anteriore del cervello,
si occupa dei movimento volontari e delle attività mentali (cioè il
pensiero). Posteriormente ad esso si trova il \textbf{lobo parietale},
che si occupa delle funzioni sensoriali. Questi due lobi sono separati
dal \emph{solco centrale}, che percorre ciascun emisfero del cervello.

Localizzato posteriormente e inferiormente al lobo parietale vi è il
\textbf{lobo occipitale}, che si occupa dell'elaborazione delle
sensazioni visive. Il \textbf{lobo temporale} è localizzato
inferiormente ai lobi frontale e parietale del cervello; esso è separato
dal lobo frontale da un profondo solco, il \emph{solco laterale} o
\emph{scissura di Silvio}, e si occupa delle percezioni sensoriali
uditive e olfattive.

Molte aree della corteccia cerebrale sono organizzate
\emph{topograficamente} in base alla loro funzione, ossia una proiezione
corporea e dei vari organi sensoriali.

Gli esempi più chiari di tale organizzazione topografico-funzionale sono
rappresentati dalla \emph{corteccia motoria} primaria nel lobo frontale
e dalla \emph{corteccia somatosensoriale} primaria nel lobo parietale.
Le mappe dell'\emph{organizzazione somatotopica} di queste due aree
corticali, in cui parti del corpo vicine sono rappresentate sulla
superficie corticale in regioni vicine, sono definite \textbf{omuncolo
motorio} e \textbf{omuncolo sensoriale}.

Il \textbf{telencefalo} è molto sviluppato soprattutto nei mammiferi.

Subito dopo per importanza troviamo il talamo e l'ipotalamo, che
costituiscono il diencefalo. Il talamo riceve buona parte degli stimoli
sensoriali delle vie afferenti. L'ipotalamo invece ha un ruolo di
stimolazione dell'attività endocrina e ad esso è collegata una ghiandola
chiamata \textbf{ipofisi} (inferiore) e una ghiandola chiamata
\textbf{epifisi} (superiormente).

Il \textbf{sistema limbico} è formato da un insieme di regioni
corticali, nuclei sottocorticali e tratti del prosencefalo strettamente
associati fra loro, coinvolti nelle emozioni, nella memoria e nella
motivazione.

Il sistema limbico è coinvolto in funzioni che regolano le pulsioni
comportamentali di base. Esso include l'\emph{amigdala} (coinvolta nelle
funzioni relative all'aggressività e alla paura) e l'\emph{ippocampo}
(coinvolto nell'apprendimento e nella memoria)

\subsection{Il controllo dei movimenti
volontari}\label{il-controllo-dei-movimenti-volontari}

Una buona parte dell'attività cerebrale riguarda i movimenti volontari.
Gli stimoli che riguardano i movimenti volontari partono tutti dalla
corteccia cerebrale. Nella corteccia avvengono le attività coscienti del
soggetto e quindi anche i movimenti volontari.

Nell'organismo ci sono tutta una serie di motilità che non sono sotto il
controllo del soggeto (es. motilità del cuore e apparato digerente), ma
ce n'è anche una volontaria (quella degli arti, della muscolatura
scheletrica, ecc).

Il controllo dei movimenti parte dalle aree del pensiero; dalle aree
associative (quelle piu' anteriori) partono gli stimoli che vanno alla
\emph{corteccia pre-motoria} che coordina i movimenti e poi passano alla
\emph{corteccia motoria} che manda lo stimolo per l'esecuzione dei
movimenti; questo viene indirizzato ad un preciso muscolo con una
determinata intensità.

L'esecuzione dei comandi motori è esegita mediante invio di segnali
discendneti ai muscoli che devono contrarsi. L'esecuzione del comando
richiede l'attivazione di neuroni efferenti che innervano i muscoli
scheletrici. Questi neuroni efferenti si trovano nel corno ventrale del
midollo spinale e sono chiamati \textbf{motoneuroni inferiori}.

L'attività dei motoneuroni è influenzata da segnali discendenti
provenienti dall'encefalo. Le due vie discendenti importanti nel
controllo dei movimenti volontari sono: i \emph{tratti piramidali}
(cellule dello strato 5) ed i \emph{tratti extrapiramidali}.

I \textbf{tratti piramidali} rappresentano vie dirette dalla corteccia
motoria primaria al midollo spinale. Gli assoni dei neuroni che danno
origine a questi tratti terminano nel \emph{corno ventrale} del midollo
spinale e sono chiamati \textbf{motoneuroni superiori}. Alcuni di questi
motoneuroni stabiliscono sinapsi dirette con i motoneuroni, altre
formano sinapsi con interneuroni.

La maggior parte dei neuroni piramidali incrocia nel SNC a livello del
bulbo.

I tratti piramidali sono coinvolti nel controllo dei movimenti fini e
precisi delle estremità distali degli arti, specialmente avambracci,
mani e dita.

I \textbf{tratti extrapiramidali} includono tutte le vie motorie ald i
fuori del sistema piramidale.

Queste vie formano connessioni indirette tra l'encefalo e il midollo
spinale; ciò sta a significare che i neuroni dei tratti extrapiramidali
non formano sinapsi dirette con i motoneuroni.

In linea generale, il sistema extrapiramidale è coinvolto nel controllo
di molti gruppi muscolari implicati nel mantenimento della postura e
dell'equilibrio, mentre il sistema piramidale è maggiormente coinvolto
nel controllo di piccoli gruppi di muscoli che si contraggono
indipendentemente.

Anche qui abbiamo il concetto di loop; ad esempio una volta che si è
deciso di iniziare a camminare, dopo un po' di tempo che lo si fa, il
processo diventa incosciente e si può camminare senza pensarci. Fino a
quando si realizza che la camminata deve cessare.

A livello periferico il controllo dei movimento gioca molto sugli archi
riflessi.

\textbf{(controllare da qui)}

Un neurone sensoriale propiocettivo (che sta dentro un muscolo) entra
nel midollo dal corno dorsale dove c'è un ganglio che raggruppa i corpi
cellulari. Entrando può allo stesso livello formare sinapsi con un
motoneurone che riceve stimoli anche dalle vie piramidali e va ad
innervare il muscolo stesso. La motilità muscolare è stata controllata
dalla stessa sensibilità sensoriale propiocettiva del muscolo.

Il tronco dell'encefalo comprende mesencefalo ponte e bulbo e la parte
più posteriore è spesso detta midollo allungato. Qui abbiamo una rete
neuronale che non è né un fascio di fibre né un nucleo ben definito.

Esistono poi dei nuclei ben definiti (come quello rosso). Il cervelletto
si sviluppa posteriormente ed interviene nel controllo della motilità. I
nuclei della base intervengono nel controllo del movimento con dei
feedback inibitori alla corteccia.

Il cervelletto contiene l'80\% dei neuroni del corpo. Ha varie
connessioni con la formazione reticolare del tronco dell'encefalo con i
nuclei vestibolari e con il nucleo rubro.

Al cervelletto arrivano delle fibre chiamate \emph{fibre muscoidi} che
sono collegate a vie che arrivano dalla corteccia cerebrale. Il
cervelletto a sua volta invia stimoli alla corteccia motoria attraverso
il talamo, attraverso il nucleo rosso.

Nella sostanza grigia si trova uno strato molecolare, uno delle cellule
del Purkinje ecc\ldots{} Ci sono fibre ascendenti che arrivano allo
strato molecolare facendo sinapsi con le cellule del Purkinje.
Funzionano direttamente le fibre muscose. Gli stimoli in uscita dal
cervelletto sono prodotti dalle cellule del Purkinje (gli stimoli
vengono modulati da queste).

Gli stimoli in entrata sono ricevuti da cellule eccitatorie mentre
quelli in uscita da cellule che ricevono stimoli inibitori. Le cellule
del Purkinje sono inibite dal GABA, che fa sinapsi inibitorie. La
corteccia inibisce l'attività nei nuclei cerebellari e così modula
l'attività motoria cerebrale.

\textbf{(a qui)}

\subsection{Funzoni integrate del SNC}\label{funzoni-integrate-del-snc}

L'encefalo è in grado di ``mettere insieme'' vari stimoli.

Il massimo grado di integrazione si pensa sia a livello dei lobi
frontali dove risiede l'attività del pensiero.

Se sono in grado di rendermi conto di quando tocco un oggetto è grazie
alle attività cerebrali. Se ci fossero dei danni che portano dai
recettori della mano fino alla corteccia non ce ne renderemmo conto.

Percepisco a livello dei centri associativi un evento come una
sensazione tattile. Poi posso pensare di toccare quell'oggetto senza
toccarlo per ricostruire una sensazione senza che questa avvenga
realmente. Questa è un'attività pura di pensiero.

L'attività cerebrale che ``svolgiamo'' può essere percepita sia come
azione sensoriale, che come attività cerebrale. L'attività mentale è
infatti un'attività cerebrale percepita come tale.

Tutto ciò di cui siamo coscienti è attività cerebrale, ma alcune sono
percepite come qualcosa di interagente fisicamente con il corpo e altre
invece che non percepite fisicamente corrispondono all'attività di
pensiero. Il pensiero è la forma più completa di attività associativa.

Esistono attività associative anche legate ad azioni, come il linguaggio
(molto sviluppato nell'essere umano).

Ci sono due aree strettamente connesse al linguaggio: l'\emph{area di
Wernicke}, coinvolta nella comprensione del linguaggio, e l\emph{area di
Broca}, coinvolta nella capacità di parlare e scrivere. Queste due aree
sono collegate da fasci di fibre di interconnessione.

Un'altra attività del cervello è il sonno. Il sonno serve per il
consolidamento dei processi di apprendimento e di memoria, e per il
rafforzamento delle difese immunitarie.

I neuroni hanno attività elettrica molto intensa captabile sulla
superficie del capo. Nel sonno l'attività elettrica cambia, assumendo la
forma di onde ampie e lente.

Esistono die tipi di sonno: il \emph{sonno a onde lente}, caratterizzato
da onde a bassa frequenza, e il \emph{sonno REM}, caratterizzato da onde
ad alta frequenza e da episodi di rapidi movimenti oculari.

Durante il sonno REM possono aumentare la respirazione e la frequenza
cardiaca.

Il ciclo sonno veglia è dovuto al \textbf{sistema reticolare attivamente
ascendente (SRAA)}. Questa regione è critica nel mantenere lo stato di
veglia. Afferenze da quest'area si proiettano alla corteccia, attraverso
stazioni sinaptiche nel talamo, nell'ipotalamo e nel tronco encefalico,
mantendola in uno stato vi veglia che la rende più recettiva ai segnali
in arrivo.

Neurotrasmettitori associati al SRAA sono l'acetilcolina, la
noradrenalina, e la dopamina.

L'adenosina invece porta sonno ad onde lente indotto dal prosencefalo.
La caffeina blocca il rilascio di adenosina. Il ponte rilascia
acetilcolina che induce la fase REM.

Quando si è svegli e vigili, l'elettroencefalogramma (EEG) mostra un
tracciato di onde ad alta frequenza e bassa ampiezza, note come
\emph{onde beta}, che rifletto i segnali elettrici generati da un gran
numero di neuroni in tempi differenti.

Quando si è svegli l'EEG mostra uno schema con onde a frequenza più
bassa e ampiezza maggiore, note come \emph{onde alfa}. In confronto alle
onde beta, le onde alfa rifletto un maggior grado di sincronizzazione
dell'attività elettrica dei neuroni; cioè, i segnali elettrici sono
generati in grandi gruppi di neuroni più o meno nello stesso istante.

Il sonno a onde lente avviene in 4 fasi successive, distinguibili dai
cambiamenti nell'EEG e dalla \emph{soglia del risveglio} (cioè
l'intensità dello stimolo richiesto per svelgiare una persona).

La prima fase ha la soglia del risveglio più bassa, mentre la quarta ha
la soglia più alta. Inoltre, le 4 fasi mostrano un graduale incremento
del grado di sincronizzazione dell'EEG, essendo le onde della quarta
fase le più alte in ampiezza e le più basse in frequenza, indice di
elevata sincronizzazione.

Durante il sonno REM, l'EEG mostra un andamento caratterizzato da onde
ad alta frequenza e bassa ampiezza somiglianti a quelle presenti durante
la veglia nello stato di allerta. La soglia del risveglio è più alta
durante il sonno REM che in qualsiasi altro momento, ma una persona
tende molto più facilemnte a svegliarsi spontaneamente durante questa
fase.

Durante la notte, una persona attraversa le varie fasi del sonno in
maniera ordinata e prevedibile. Quando ci si addormenta, si passa dalla
veglia alla prima fase del sonno a onde lente; da qui si passa con
ordine attraverso le fasi 2, 3 e 4. Circa un'ora dopo l'addormentamento
si ripercorrono le fasi in ordine inverso e quindi si entra nella prima
fase REM. Questi cicli si ripetono alcune volte ad intervalli di circa
90 minuti ma, nel corso della notte, si è portati a passare sempre meno
tempo nella fasi di sonno profondo a onde lente e più tempo nel sonno
REM.

Tra le funzoni intgrate del SNC vi sono anche le emozioni. L'amigdala
sembra giocare un ruolo importante nella paura e nell'ansia;
l'ipotalamo, invece, è associato con i sentimenti di rabbia e
aggressività

Entrambe sono le risposte comportativamente immediate nella vita di un
individuo. Queste sensazioni sono regolate dalla corteccia e danno delle
risposte vegetative motorie e ormonali. Legati all'emozione ci sono i
centri del piacere, che è un sito molto particolare in cui interviene in
sistema dopaminergico.

Altra funzioni integrate dal SNC sono l'apprendimento e la memoria.
Mentre l'apprendimento consiste nell'acquisizione di nuove informazioni
e esperienze, la memoria rappresenta il consolidamento di tali
informaizoni, esperienze o pensieri.

L'apprendimento può essere \emph{associativo} o \emph{non associativo}.

L'apprendimento \emph{associativo} è un tipo di apprendimento che
richiede la capacità di collegare due o più stimoli.

L'apprendimento \emph{non associativo} invece, si realizza in risposta a
stimoli ripetuti ed include i processi di abitudine e sensibilizzazione.
L'abitudine rappresenta una sorta di decrescimento della risposta a
stimoli ripetuti. Al contrario, la sensibilizzazione rappresenta un
incremento della risposta a stimoli ripetuti.

Esistono poi 2 tipi di memoria: la \emph{memoria procedurale} e quella
\emph{dichiarativa}.

La memoria procedurale, o \emph{memoria implicita}, è la memoria delle
capacità motorie e dei comportamenti appresi. Questo tipo di memoria
coinvolge diverse aree encefaliche, inclusi il cervelletto, i nuclei
della base e il ponte.

La memoria dichiarativa, o \emph{memoria esplicita}, rappresenta una
forma di memoria delle esperienze apprese, come fatti, eventi ed altre
cose che possono essere affermate verbalmente. Questo tipo di memoria
coinvolge l'ippocampo.

La memoria si realizza a due livelli: \emph{memoria a breve termine} e
\emph{a lungo termine}.

Nel modello corrente su come l'informazione viene acquisita ed
immagazzinata all;interno della memoria, le informazioni in arrivo prima
entrano nel SNC e poi vengono conservate sotto forma di \textbf{memoria
e breve termine} ( o \emph{memoria di lavoro}), un immagazzinamento
temporaneo di un concetto per pochi secondi o poche ore. Lo spazio per
la memoria a breve termine è limitato e le informaizoni immagazzinate in
questo modo vengono perse se non vengono ulteriormente
\emph{consolidate} sotto forma di \textbf{memoria a lungo termine}, che
può durare per anni o per l'intera vita.

I meccanismi di consolidamento non sono ben conosciuti, ma certamente la
ripetizione aiuta.

La memoria è un processo complesso che coinvole parecchie, se non tutte
le aree encefaliche. Il lobo frontale ha un ruolo cruciale nella memoria
a breve termine, il lobo temporale, incluso l'ippocampo, è necessario
per quella a lungo termine.

\section{Il sistema nervoso
efferente}\label{il-sistema-nervoso-efferente}

Fanno parte del \emph{sistema nervoso efferente} quelle vie che portano
gli stimoli nervosi dal SNC ad altri sistemi. Queste vie si dividono tra
sistema \textbf{autonomo}, che conduce gli stimoli involontari, e
\textbf{somatico}, che conduce stimoli volontari (ovvero sotto controllo
del soggetto).

\subsection{Il sistema nervoso
automono}\label{il-sistema-nervoso-automono}

Il sistema nervoso autonomo innerva la maggior parte degli organi e dei
tessuti effettori, incluso il muscolo cardiaco, le cellule muscolari
lisce dei vasi ematici e vari organi viscerali, le ghiandole e il
tessuto adiposo.

Il sistema nervoso ``autonomo'' è definito in tal modo in quanto la sua
attività si svolge in modo inconscio.

Il sistema nervoso autonomo è suddiviso in due sottosistemi, detti
\textbf{sistema simpatico} e \textbf{parasimpatico}, Entrambe le
divisioni del sistema nervoso autonomo innervano la maggior parte degli
organi, un'organizzazione chiamata \emph{duplice innervazione}. Questi
due sistemi spesso innervano gli stessi organi, ma con effetti
antagonisti.

Il sistema nervoso simpatico esce a livello spinale, mentre il
parasimpatico esce dal midollo allungato e dalle ultime vertebre sotto
forma di nervi (per lo più misti).

Tra gli effetti di questi due sistemi vi è ad esempio l'inibizione del
sistema cardovascolare da parte del parasimpatico, mentre il simpatico
lo stimola.

Il sistema nervoso autonomo consta di vie efferenti formate da due
neuroni organizzati in serie tra il SNC e gli organi effettori. I
neuroni comunicano tra loro mediante sinapsi localizzate in strutture
periferiche chiamate \textbf{gangli del sistema nervoso autonomo}. I
neuroni che collegano il SNC ai gangli sono definiti \textbf{neuroni
pregangliari}; quelli che collegano i gangli agli organi effettori sono
detti \textbf{neuroni postgangliari}. All'interno di ciascun ganglio vi
sono i terminali assonici dei neuroni pregangliari e i corpi cellulari
ed i dendriti dei neuroni postgangliari.

Nella porzione tra neurone pre e post gangliare trovo una sinapsi. Il
messaggio viaggia sempre dal SNC all'organo effettore. Gli stimoli sono
potenziali d'azione che viaggiano sotto forma di scariche di potenziale
d'azione. Nei gangli ci sono neuroni intragangliari che possono modulare
le scariche uscendo con frequenza diversa per attività di neuroni
intermedi dentro il ganglio.

\subsection{Anatomia del sistema nervoso
simpatico}\label{anatomia-del-sistema-nervoso-simpatico}

Poichè i neuroni pregangliari nel sistema nervoso simpatico emergono
dalle porzioni toraciche e lombari del midollo spinale, il sistema
nervoso simpatico è noto come sistema nervsono autonomo
\emph{toracolombare}.

I neuroni pregangliari originano in una regione di sostanza grigia del
midollo spinale chiamata \textbf{corno laterale}. I neuroni pregangliari
e postgangliari simpatici sono tra loro connessi in 3 modi:

\begin{enumerate}
\def\labelenumi{\arabic{enumi}.}
\itemsep1pt\parskip0pt\parsep0pt
\item
  Nella più comune delle situazioni, i neuroni pregangliari hanno brevi
  assoni che originano nel corno laterale del midollo spinale ed escono
  da questo attraverso la radice ventrale.
\end{enumerate}

Immediatamente dopo che le radici ventrali e dorsali formano il nervo
spinale, l'assone del neurone pregangliare lascia il nervo spinale
attraverso una diramazione chiamata \emph{ramo bianco}, che penetra in
uno o più gangli simpatici localizzati appena fuori dal midollo spinale.
Qui, il neurone pregangliare forma sinapsi con molti neuroni
postgangliari, che hanno lunghi assoni che raggiungono gli organi
effettori.

La maggior parte degli assoni postgangliari ritorna al nervo spinale
attraverso una diramazione chiamata \emph{ramo grigio} e raggiunge
l'organo effettore attraverso un nervo spinale.

I vari gangli simpatici sono tra loro collegati in modo tale da formare
una struttura che decorre parallelamente alla colonna vertebrale, ai due
lati di essa, chiamata \textbf{catena simpatica};

\begin{enumerate}
\def\labelenumi{\arabic{enumi}.}
\setcounter{enumi}{1}
\itemsep1pt\parskip0pt\parsep0pt
\item
  Una seconda possibile organizzazione delle fibre simpatiche si
  verifica quando un gruppo di lunghi neuroni pregangliari innerva
  direttamente tessuti endocrini, come la \emph{midollare della
  ghiandola del surrene}, invece di formare sinapsi con neuroni
  postgangliari.
\end{enumerate}

Ciascuna delle ghiandole surrenali, localizzata nel cuscinetto adiposo
sul ppolo superiore di ciascun rene, è suddivisa in uno strato esterno
corticale ed uno interno midollare.

Quest'ultimo consta di neuroni simpatici postgangliari modificati,
chiamati \textbf{cellule cromaffini}, che si differenziano in cellule
endocrine invece che in neuroni.

In seguito alla stimolazione del sistema nervoso simpatico, la midollare
del surrene rilascia in circolo catecolamine (80\% adrenalina, 20\%
noradrenalina, e tracce trascurabili di dopamina).

La midollare del surrene rilascia i proprio prodotti direttamente nel
sangue, per cui questi prodotti funzionano come ormoni.

\begin{enumerate}
\def\labelenumi{\arabic{enumi}.}
\setcounter{enumi}{2}
\itemsep1pt\parskip0pt\parsep0pt
\item
  Una terza disposizione anatomica delle fibre simpatiche comprende
  neuroni pregangliari che formano sinapsi con neuroni postgangliari in
  strutture chiamate \textbf{gangli collaterali}, situati tra il SNC e
  gli organi effettori.
\end{enumerate}

In questo caso, un neurone pregangliare fuoriesce dal midollo spinale
attraverso la radice ventrale ed entra nella catena simpatica attraverso
un ramo bianco. L'assone del neurone pregangliare attraversa questo
ganglio senza formare sinapsi e raggiunge un ganglio collaterale tramite
un nervo simpatico.

All'interno del ganglio collaterale, il neurone pregangliare forma
sinapsi con molti neuroni postgangliari, che terminano su numerosi
organi bersaglio.

Poichè questi gangli non sono tra loro connessi, come quelli della
catena simpatica, essi permettono al sistema nervoso simpatico di
raggiungere organi bersaglio ben definiti e quindi di esercitare effetti
meno diffusi.

(immagine 11.3 p 306)

\subsection{Anatomia del sistema nervoso
parasimpatico}\label{anatomia-del-sistema-nervoso-parasimpatico}

I neuroni pregangliari del sistema nervoso parasimpatico originano nel
tronco encefalico o nel midollo spinale sacrale.

In genere, i neuroni pregangliari parasimpatici sono relativamente
lunghi e terminano in gangli localizzati vicino agli organi effettori. A
livello gangliare, essi formano sinapsi con corti neuroni postgangliari
diretti agli organi effettori.

Nella porzione craniale del sistema parasimpatico, gli assoni dei nervi
pregangliari originano da corpi cellulari localizzati nel tronco
encefalico, che inviano i propri assoni nei nervi cranici.

Nel sistema parasimpatico la cellula pregangliare ha un assone più lungo
che raggiunge una cellula gangliare vicina all'organo effettore, mentre
la postgangliare è più corta. Esempi sono alcuni nervi cranici.

\subsection{I neurotrasmettitori del sistema nervoso
autonomo}\label{i-neurotrasmettitori-del-sistema-nervoso-autonomo}

I due neurotrasmettitori del sistema nervoso periferico sono
l'\emph{acetilcolina} e la \emph{noradrenalina}.

I neuroni che rilasciano acetilcolina sono noti come
\textbf{colinergici}.

L'\emph{acetilcolina} è rilasciata da: + neuroni pregangliari simpatici
e parasimpatici; + neuroni postgangliari parasimpatici; + neuroni
pregangliari simpatici che innervano le cellule cromaffini della
midollare del surrene. In questo caso l'acetilcolina agisce sulle
cellule endocrine della midollare stimolando il rilascio di adrenalina.

La \emph{noradrenalina} è il neurotrasmettitore usato da quasi tutti i
neuroni simpatici postgangliari, che pertanto vengono detti
\textbf{adrenergici}.

L'acetilcolina e la noradrenalina possono entrambe legarsi a differenti
classi e sottoclassi di recettori colinergici e adrenergici.

\subsubsection{I recettori colinergici}\label{i-recettori-colinergici}

Questi recettori si dividono in due categorie: quelli \emph{nicotinici}
e quelli \emph{muscarinici}.

Queste due classi si distinguono in base a studi farmacologici
effettuati utilizzando due agonisti (sostanze chimiche che si legano ad
un recettore e producono lo stesso effetto biologico) dell'acetilcolina:
la \emph{nicotina}, un alcaloide che si trova abbondantemente nella
pianta del tabacco, e la \emph{muscarina}, una sostanza fungina.

I recettori colinergici nicotinici sono localizzati sui corpi cellulari
e sui dendriti dei neuroni postgangliari simpatici e parasimpatici,
sulle cellule cromaffini della midollare del surrene e sulle cellule
muscolari scheletriche.

I recettori muscarinici invece, sono presenti sugli organi effettori del
sistema nervoso parasimpatico, come il cuore, le cellule muscolari lisce
del tratto digerente, ecc\ldots{}

Tutte le sottoclassi di recettori colinergici nicotinici sono associate
a canali permeabili al sodio e al potassio. Quando l'acetilcolina si
lega a tali recettori, questi canali cationici si aprono, permettendo al
sodio di diffondere all'interno della cellula e al potassio di
diffondere all'esterno.

Poichè il Na è lontano dal suo potenziale di equilibrio, il suo flusso
verso l'interno eccederà quello verso l'esterno del K, per cui la
cellula si depolarizza.

Pertanto, i recettori colinergici nicotinici sono associati con la
depolarizzazione, o \emph{eccitazione}, della membrana della cellula
postsinaptica.

Al contrario, tutte le classi di recettori muscarinici sono accoppiate a
proteine G e a secondi messaggeri. Le risposte attivate dal legame
dell'acetilcolina possono essere \emph{sia inibitorie che eccitatorie},
in relazione alle cellule bersaglio e alla natura della via di
trasduzione del segnale coinvolta.

\subsubsection{I recettori adrenergici}\label{i-recettori-adrenergici}

Vi sono due classi principali di recettori adrenergici localizzati in
organi effettori del sistema nervoso simpatico: i \textbf{recettori
alfa} e i \textbf{recettori beta}. Ciascuna di queste classi è
ulteriormente divisa in sottoclassi: \(\alpha\)\(_1\), \(\alpha\)\(_2\),
\(\beta\)\(_1\), \(\beta\)\(_2\) e \(\beta\)\(_3\).

I recettori adrenergici sono accoppiati a proteine G che attivano o
inibiscono sitemi di secondi messaggeri.

Il legame della noradrenalina o dell'adrenalina con un recettore
\textbf{\(\alpha\)\(_1\)} attiva una proteina G che, a sua volta, attiva
l'enzima fosfolipasi C.

Il legame con i recettori \textbf{\(\alpha\)\(_2\)} attiva una proteina
G inibitoria che riduce l'attività dell'adenilato ciclasi, riducendo
così la sintesi di AMP ciclico.

Il legame con i recettori \textbf{\(\beta\)} invece, attiva una proteina
G stimolatrice che incrementa l'attività dell'adenilato ciclasi,
aumentando la sintesi dell'AMP ciclico.

I recettori \(\alpha\) hanno una maggiore affinità per la noradrenalina
rispetto all'adrenalina e sono generalmente eccitatori.

I recettori \(\beta\)\(_1\) e \(\beta\)\(_3\) hanno un'affinità molto
simile per la noradrenalina e l'adrenalina e producono in genere effetti
eccitatori.

I recettori \(\beta\)\(_2\) hanno una maggiore affinità per l'adrenalina
rispetto alla noradrenalina, e in genere producono rispote inibitorie.

Questi recettori sono il bersaglio di vari agenti terapeutici; ad
esempio, per l'ipertensione vengono utilizzati i beta-bloccanti,
l'efedrina invece è utilizzata del trattamento dell'asma in quanto
agonista dei recettori \(\beta\)\(_2\), mentre l'atropina agisce sui
recettori polinergici muscarinici.

\subsection{Le giunzioni
neuroeffettrici}\label{le-giunzioni-neuroeffettrici}

La sinapsi tra un neurone efferente ed il suo organo bersaglio
(effettore) è definita \textbf{giunzione neuroeffettrice}.

Le sinapsi tra i neuroni autonomi postgangliari e i loro organi
effettori differiscono dalle classiche sinapsi tra due neuroni, in
quanto i neuroni postgangliari non inviano i loro terminali assonici su
un numero ben definito di cellule; i neurotrasmettitori infatti, vengono
rilasciati da numerosi rigonfiamenti localizzati ad intervalli quasi
costanti lungo gli assoni, noti come \textbf{varicosità}.

All'interno di queste varicosità, i neurotrasmettitori sono sintetizzati
ed immagazzinati in vescicole.

Le membrane degli assoni contengono i classici canali
voltaggio-dipendenti per il Na e il K che permettono la propagazione dei
potenziali d'azione.

In aggiunta, la membrana, nella regione di ciascuna varicosità, contiene
canali voltaggio-dipendenti per il calcio che si aprono all'arrivo del
potenziale d'azione.

L'arrivo di un potenziale d'azione nella varicosità apre i canali
voltaggio-dipendenti per il calcio che, entrando nel citoplasma, stimola
il rilascio del neurotrasmettitore mediante esocitosi.

Un assone postgangliare ha molte varicosità, per cui un potenziale
d'azione propagato lungo l'assone determina il rilascio del
neurotrasmettitore da tutti i rigonfiamenti.

Poichè la distanza tra le varicosità e l'organo effettore è maggiore
rispetto alla classica fessura sinaptica, il neurotrasmettitore
rilasciato diffonde in un'ampia area dell'organo effettore e si lega ai
recettori posti sulla membrana plasmatica delle cellule di tutto
l'organo bersaglio.

Gli effetti del neurotrasmettitore terminano quando, come nelle sinapsi
neurone-neurone, esso diffonde lontano dai recettori oppure viene
ricaptato o degradato ad opera di enzimi, come per esempio
l'\textbf{acetilcolinesterasi}, localizzata sia sulla membrana del
neurone postgangliare sia sulla membrana delle cellule degli organi
effettori colinergici.

In seguito alla degradazione ad opera dell'acetilcolinesterasi
dell'acetilcolina in acetato e colina, quest'ultima è ricaptata
attivamente all'interno delle varicosità postgangliari ed utilizzata per
sintetizzare altro neurotrasmettitore.

La \textbf{monoamminossidasi} è un altro enzima degradativo che degrada
adrenalina e noradrenalina.

\subsection{Il sistema nervoso
somatico}\label{il-sistema-nervoso-somatico}

A differenza del sistema nervoso autonomo, che controlla le funzooni di
molti organi effetori, il sistema nervoso somatico controlla un solo
tipo di organo effettore, il muscolo scheletrico.

Il sistemanervoso somatico ha un solo tipo di neuroni efferenti, i
motoneuroni, cioè i neuroni che innervano il muscolo scheletrico.

Nel sistema nervoso somatico, un singolo motoneurone collega il sistema
nervoso centrale a una fibra muscolare scheletrica.

I motoneuroni originano nel corno ventrale del midollo spinale e
ricevono segnali da molteplici afferenze.

Un singolo motoneurone innerva molte cellule muscolari (definite
\textbf{fibre muscolari}), ma ciascuna fibra è innervata da un singolo
motoneurone. L'insieme costituito da un motoneurone e dalle cellule
muscolari da esso innervate forma un'\textbf{unità motoria}. Quando un
motoneurone è attivato, stimola la contrazione di tutte le fubre
muscolari presenti nella sua unità.

\subsubsection{La giunzione
neuromuscolare}\label{la-giunzione-neuromuscolare}

Ciascuna diramazione dell'assone di un motoneurone forma sinapsi con una
fibra muscolare scheletrica a livello di una singola regione altamente
specializzata della membrana della cellula muscolare, formando una
\textbf{giunzione muscolare}. I terminali dell'assone del motoneurone,
chiamati \textbf{bottoni sinaptici} o \textbf{bottoni terminali},
immagazzinano e rilasciano acetilcolina, che è l'unico
neurotrasmettitore del sistema nervoso somatico.

Dal lato opposto del bottone sinaptico, sulla membrana della fibra
muscolare, vi è una regione specializzata, la \textbf{placca motrice},
che presenta molte invaginazioni contenenti un elevato numero di
recettori per l'acetilcolina.

Questi recettori rappresentano una varietà dei recettori colinergici
nicotinici.

L'acetilcolina, che è presente tra le invaginazioni della placca
motrice, determina la fine del segnale eccitatoio ed il rilassamento
della fibra muscolare.

Il recettore nicotinico è un canale ionico formato da 5 subunità, di cui
2 alfa che legano l'acetilcolina. Dopo aver legato l'acetilcolina, il
canale si apre permettendo il trasporto di ioni Na\(^+\). L'ingresso di
ioni sodio nella cellula muscolare produce una depolarizzazione che
prende il nome di \textbf{potenziale di placca}. Questo potenziale è
molto forte (circa 70 mV, è molto più forte di quello postsinaptico,
circa 10-15 mV) e da solo è in grado di depolarizzare la membrana
muscolare fino al valore soglia necessario per indurre il potenziale
d'azione.

Il potenziale di placca si propaga sulla fibra della matrice muscolare.

Negli spazi intersinaptici si accumula acetilcolina e qui vengono anche
legati i recettori colinergici nicotinici.

La fine del segnale lo si ha inibendo l'acetilcolinesterasi. Questo
enzima degrada l'acetilcolina e non permette alla cellula di assumere
acetilcolina.

Su queste sinapsi agiscono veleni che possono provocare rilassamento
muscolare e paralisi. I meccanismi di azione di questi veleni sono
diversi:

\begin{itemize}
\itemsep1pt\parskip0pt\parsep0pt
\item
  la \emph{tossina botulinica} viene rilasciata dal batterio Clostridio,
  ed inibisce il rilascio di acetilcolina mandando in paralisi il
  muscolo;
\item
  l'\emph{alfa-bungarotossina} (veleno del cobra, molto potente) agisce
  sul recettore nicotinico bloccando l'apertura dei canali sodio e
  impedendo in questo modo la contrazione della muscolatura;
\item
  il \emph{curaro} (estratto da alcune piante) compete con
  l'acetilcolina (antagonista dell'acetilcolina) e inibisce il
  meccanismo di attivazione del recettore;
\item
  l'\emph{atrossina} (veleno della vedova nera) aumenta il rilasico di
  acetilcolina mandando la muscolatura in contrazione tetanica;
\item
  gli inibitori dell'acetilcolinaesterasi (\emph{pesticidi} e \emph{gas
  nervino}) causano un'inibizione permanente della contrazione muscolare
  dovuta a depolarizzazione permanente e, di conseguenza, paralisi;
\item
  la \emph{succinilcolina} (analogo dell'acetilcolina) è meno sensibile
  all'azione dell'acetilcolinesterasi e aumnta l'attività provocando lo
  stesso effeto degli inibitori dell'acetilcolinesterasi e quindi causa
  paralisi da depolarizzazione.
\end{itemize}

\textbf{lezione 20151116}

\section{Il sistema sensoriale}\label{il-sistema-sensoriale}

La porzione afferente del sitema nervoso periferico trasmette
informazioni dalla periferia al SNC. Le informazioni sono raccolte da
\emph{recettori sensoriali} che rispondono a stimoli specifici. mentre
alcuni di questi recettori vengono eccitati da stimoli provenienti
dall'ambiente esterno, altri, definiti \emph{recettori viscerali},
ricevono stimoli provenienti dall'interno dell'organismo.

Esempi di queste recettori viscerali includono: \emph{chemocettori}
delle pareti dei vasi ematici, che monitorano i livelli di O\(_2\),
CO\(_2\) e H\(^+\); i \emph{barocettori} che rilevano la pressione
ematica; i \emph{meccanocettori} del tratto intestinale.

Sebbene l'encefalo riceva informazioni da questi recettori e le
utilizzi, noi non siamo coscienti di questi stimoli.

Parte della percezione si basa su esperienze precedenti, dunque
individui diversi possono percepire gli stessi stimoli differentemente.

Tutti i recettori inviano stimoli al sistema nervoso centrale ma non
tutti raggiungono la corteccia cerebrale. Il soggetto è conscio solo
degli stimoli che raggiungono la corteccia cerebrale. Altri stimoli
rimangono a lui ignoti dal punto di vista dell'autoconsapevolezza anche
se il SNC non rimane indifferente e reagisce comunque a questi stimoli
tramiti gli efferenti anche se non ce ne si rende conto.

Il potenziale d'azione parte dalla terminazione nervosa e va verso il
SNC. Insorge per diretta stimolazione della terminazione o sotto
stimolazione della cellula recettoriale.

I recettori sensoriali sono strutture neuronali specializzate in grado
di percepire specifiche forme di energia derivanti sia dall'ambiente
esterno che dall'interno dell'organismo.

La forma di energia di uno stimolo è definita \textbf{modalità} (es.
onde luminose, pressione, temperatura\ldots{}). La \textbf{legge delle
energie nervose specifiche} stabilisce che un determinato recettore
sensoriale è specifico per una particolare modalità sensoriale. La
\textbf{modalità} alla quale risponde un recettore è definita
\textbf{stimolo adeguato}. Modalità differenti dagli stimoli adeguati
possono attivare i recettori soltanto se i livelli di energia di
stimolazione sono molto elevati.

La funzione dei recettori sensoriali è la \textbf{trasduzione}, cioè la
conversione di una forma di energia in un'altra. Nella \emph{trasduzione
sensoriale}, i recettori convertono la forma di energia di uno stimolo
sensoriale in modificazioni del potenziale di membrana definite
\textbf{potenziale di recettore} o \textbf{potenziali generatori}. I
potenziali di recettore hanno le stesse caratteristiche dei potenziali
postsinaptici, in quanto sono dei potenziali graduati generati
dall'apertura o dalla chiusura di canali ionici. Maggiore è l'intensità
dello stimolo, maggiore sarà la variazione del potenziale di membrana.

I recettori sensoriali sono di 2 tipi differenti:

\begin{enumerate}
\def\labelenumi{\arabic{enumi}.}
\itemsep1pt\parskip0pt\parsep0pt
\item
  il recettore sensoriale può essere una struttura specializzata
  presente all'estremità periferica di un neurone afferente. Se il
  recettore si depolarizza fino al valore soglia, si avrà un potenziale
  d'azione che verrà propagato dal neurone afferente fino al SNC,
  trasmettendo informazioni riguardanti lo stimolo;
\item
  in altri casi, il recettore sensoriale è costituito da una cellula che
  comunica attraverso una sinapsi chimica con un neurone afferente ad
  essa associato.
\end{enumerate}

Alcuni recettori continuano a rispondere ad uno stimolo per tutta la
durata di applicazione dello stimolo stesso. Tuttavia, molti recettori
\emph{si adattano} allo stimolo, in quanto la loro risposta diminuisce
nel tempo. L'\textbf{adattamento recettoriale} rappresenta il decremento
nel tempo dell'ampiezza del potenziale di recettore in presenza di uno
stimolo costante.

I \textbf{recettori a lento adattamento}, o \textbf{recettori tonici},
hanno bassi livelli di adattamento e pertanto possono dare informazioni
relative all'intensità di uno stimolo prolungato.

I \textbf{recettori a rapido adattamento}, o \textbf{recettori fasici},
si adattano rapidamente e funzionano in maniera ottimale quando devono
rilevare modificazioni dell'intensità dello stimolo. I recettori fasici
rispondono all'inizio dello stimolo, per poi adattarsi a esso. Alcuni di
questi recettori mostrano una seconda e minore risposta al termine dello
stimolo, definita \emph{``risposta off''}.

Le vie nervose specifiche che trasmettono informazioni pertinenti ad una
particolare modalità sono definite \textbf{linee marcate} e ciascuna
modalità sensoriale segue la sua linea marcata. L'attivazione di una via
specifica determina la percezione della modalità associata,
indipendentemente dal reale stimolo che attiva la via.

Un'\textbf{unità sensoriale} comprende un singolo neurone afferente e
tutti i recettori ad esso associati. Tutti i recettori associati con un
determinato neurone afferente sono dello stesso tipo e l'attivazione di
ognuno di essi può generare potenziali d'azione nel neurone afferente.

L'area nella quale uno stimolo adeguato può produrre una risposta
(eccitatoria o inibitoria) nel neurone afferente è definita
\textbf{campo recettivo} di quel neurone; esso corrisponde alla regione
che contiene recettori per quel neurone afferente.

Il neurone afferente che trasmette l'informazione dalla periferia al SNC
è definito \textbf{neurone di primo ordine}. Un singolo neurone di primo
ordine può comunicare con molti interneuroni, causando, nell'ambito del
SNC, una divergenza del segnale. In aggiunta, interneuroni possono
ricevere impulsi convergenti da molti neuroni di primo ordine. Alcuni di
questi interneuroni trasmettono informazioni al talamo, che rappresenta
la regione principale di collegamento per le informazioni sensoriali;
questi interneuroni sono un esempio di \textbf{neuroni di secondo
ordine}.

Nel talamo, questi neuroni di seconod ordine formano sinapsi con
\textbf{neuroni di terzo ordine}, che trasmettono le informazioni alla
corteccia cerebrale, dove si realizza la percezione della sensazione.

\textbf{(immagine 10.6 pag258)}

Questa organizzazione geometrica del sistema nervoso è ideale per creare
delle mappature.

L'\emph{intensità} dello stimolo è codificata dalla \emph{frequenza dei
potenziali d'azione} (\textbf{codice di frequenza}) e dal \emph{numero
dei recettori attivati} (\textbf{codice di popolazione}).

Nel \emph{codice di frequenza}, uno stimolo più intenso dà origine a un
potenziale di recettore più ampio. Se un potenziale graduato (in questo
caso il potenziale del recettore) supera il valore soglia,
depolarizzazioni maggiori possono superare il periodo refrattario
relativo di un potenziale d'azione, generando, di conseguenza, un
secondo potenziale d'azione più rapidamente di quanto non farebbe una
depolarizzazione più debole. Perciò uno stimolo più intenso produce un
aumento della frequenza di scarica dei potenziale d'azione.

Nel \emph{codice di popolazione}, uno stimolo più intenso attiva o
recluta un maggior numero di recettori; questi recettori possono essere
associati con un singolo neurone afferente, oppure lo stimolo può
reclutare recettori associati a differenti neuroni afferenti. In
entrambi i casi, al SNC viene trasmessa una maggior frequenza di
potenziali d'azione in risposta allo stimolo, indicando che lo stimolo è
più forte.

\textbf{(immagine 10.8 p259)}

La precisione con la quale è percepita la localizzazione di uno stimolo
è definita \textbf{acuità} (= capacità di distinguere due stimoli molto
vicini come diversi).

Questa proprietà è ben rappresentata nella sensibilità somatica ma anche
nella retina. La superficie del corpo e della retina sono quelle più
coinvolte nei processi di mappatura topografice spaciale dello stimolo.

Nelle sensazioni associate ai recettori della cute, l'acuità dipende
dalle dimensioni e dal numero dei campi recettivi, dal sovrapporsi degli
stessi e dal fenomeno dell'inibizione laterale. Se è attivato uno
specifico neurone afferente, lo stimolo deve essere localizzato
nell'ambito del campo recettivo di quel neurone. Tuttavia, le dimensioni
dei campi recettivi variano notevolmente nell'organismo.

La localizzazione dello stimolo è migliore nelle regioni innervate da
neuroni con campi recettivi piccoli. La localizzazione è migliorata
dalla sovrapposizione dei campi recettivi di diversi neuroni afferenti.
Questa sovrapposizione migliora la localizzazione tramite due
meccanismi:

\begin{enumerate}
\def\labelenumi{\arabic{enumi}.}
\itemsep1pt\parskip0pt\parsep0pt
\item
  l'attivazione di entrambi i neuroni afferenti da parte di qualsiasi
  stimolo che cada nella regione di sovrapposizione;
\item
  l'inibizione laterale.
\end{enumerate}

Nell'\textbf{inibizione laterale}, uno stimolo che eccita fortemente i
recettori in un'area cutanea inibisce l'attività nelle vie afferenti dei
recettori limitrofi.

La massima acuità è presente nelle labbra o nelle dita delle mani ed è
minima nella schiena.

L'inibizione laterale incrementa l'acuità, in quanto migliora il
contrasto dei segnali nel sistema nervoso. Essa permette la trasmissione
di segnali intensi in alcuni neuroni, sopprimendo la trasmissione di
segnali più deboli provenienti dai neuroni limitrofi.

Ciascun neurone stabilisce sinapsi con un singolo neurone di secondo
ordine. I neuroni afferenti presentano collaterali che comunicano con
interneuroni inibitori del SNC. In questo caso, le collaterali
provenienti dal neurone afferente attivano interneuroni inibitori che
riducono la comunicazione tra neuroni afferenti con campi recettivi
limitrofi e neuroni di secondo ordine.

L'inibizione laterale incrementa l'acuità, in quanto migliora il
contrasto dei segnali nel sistema nervoso. Essa permette la trasmissione
di segnali intensi in alcuni neuroni, sopprimendo la trasmissione di
segnali più deboli provenienti dai neuroni limitrofi.

Il risultato netto sarà una maggiore differenza nella frequenza dei
potenziali d'azione tra i neuroni di secondo ordine rispetto ai neuroni
afferenti ma, ciò che è più importante, la frequenza dei potenziali
d'azione del neuroni che riceve lo stimolo sarà \emph{molto più elevata}
rispetto a quella dei neuroni limitrofi. L'aumentato contrasto
risultante tra segnali neuronali più e meno importanti permetterà una
migliore localizzazine dello stimolo, aumentando l'acuità tattile.

Una misura dell'acuità tattile è data dalla \textbf{discriminazione di
due punti}, vale a dire la capacità di una persona di percepire due
stimoli pressori applicati sulla cute in due punti separati
spazialmente. Al di sotto della \emph{soglia di discriminazione di due
punti} i due stimoli non vengono più percepiti come separati. La
possibilità di discriminare due punti di stimolazione si verifica
soltanto se i due stimoli pressori sono applicati ai campi recettivi di
due diversi neuroni afferenti. Pertanto, più piccoli sono i campi
recettivi, maggiore sarà la capacità di discriminare due punti distinti
e maggiore sarà l'acuità tattile.

In altri sistemi, come quelli olfattivi e uditivi, la localizzaizone
dello stimolo è basata sull'arrivo di stimoli alle due narici o alle due
orecchie con tempi lievementi differenti. Il cervello usa le differenze
temporali nell'arrivo del potenziale d'azione a livello della corteccia
olfattiva o uditiva per determinare la provenienza dello stimolo. Il
sistema nervoso centrale riesce a discriminare quale gruppo di neuroni
si è attivato prima.

\subsection{Recettori
somatosensoriali}\label{recettori-somatosensoriali}

Il sistema somatosensoriale risponde ad una varietà di stimoli
provenienri da molte aree del corpo e, pertanto, utilizza molti tipi di
recettori.

Le sensazioni somestesiche relative a stimoli associati con la
superficie del corpo sono dovute alla presenza di:

\begin{itemize}
\itemsep1pt\parskip0pt\parsep0pt
\item
  \textbf{meccanocettori}, in grado di rilevare stimoli pressori o
  vibrazioni;
\item
  \textbf{termocettori}, in grado di rilevare variazioni di temperatura;
\item
  \textbf{nocicettori}, in grado di rilevare stimoli dannosi per i
  tessuti.
\end{itemize}

\subsubsection{I meccanocettori cutanei}\label{i-meccanocettori-cutanei}

Nella cute c'è un complesso di meccanocettori che vengono attivati da
stimoli pressori. Si parla di sensibilità tattile dovuta a
meccanocettori.

Questi recettori possiedono \emph{fibre A-beta mielinizzate}, le più
grosse del sistema sensoriale.

Le cellule recettoriali sono tutte coincidenti con le terminazioni
neuronali. Esse sono associate a conformazioni caratteristiche della
cute e quindi è stata data tutta una serie di nomi diversi a queste
terminazioni. Ad esempio i \emph{corpuscoli del pacini} sono
terminazioni nervose ricoperte da strati concentrici di connettivo.
Queste lamelle connettivali inducono bella terminazione nervosa
l'apertura dei canali del Na\(^+\) stimolati da azione meccanica.
L'apertura di questi canali induce un potenziale recettoriale, che a sua
volta induce un potenziale d'azione voltaggio-dipendente a livello del
primo nodo di Ranvier (si tratta di fibre mielinizzate).

Questi recettori si trovano nella porzione profonda della cute, con
campi recettoriali ampi (se situati in superficie i campi recettoriali
sono piccoli). Sono recettori fasici a rapido adattamento.

\subsubsection{I termocettori}\label{i-termocettori}

I termocettori reagiscono a stimoli di tipo termico.

Questi recettori sono caratterizzati da \emph{fibre A-delta
mielinizzate} (più piccole delle A-beta) e da \emph{fibre non
mielinizzate di tipo C}.

Ci sono due tipi di termorecettori: quelli per il caldo e quelli per il
freddo.

I \textbf{recettori per il caldo} sono costituiti da terminazioni
nervose libere che rispondono a temperature tra i 30°C e i 45°C; la
frequenza di scarica dei potenziali d'azione aumenta all'aumentare della
temperatura fino a 45°C.

I \textbf{recettori per il freddo} rispondono a variazioni di
temperatura comprese tra i 35°C e i 20°C. La frequenza dei potenziali
dàazione incrementaal diminuire della temperatura ed è massima per
temperature di circa 25°C. I recettori per il freddo rispondono anche a
temperature superiori a 45°C, uno stimolo \emph{``dolorosamente
caldo''}, con la frequenza dei potenziali d'azione che aumenta
all'aumentare della temperatura.

I termorecettori sono terminazioni nervose libere dotate di canali
ionici sensibili alla temperatura.

Sono recettori fasici ed esistono unità sensoriali per il caldo e il
freddo disposte in regioni (miroscopiche) diverse.

\subsubsection{I nocicettori}\label{i-nocicettori}

I nocicettori sono recettori sensoriali responsabili della trasduzione
di \emph{stimoli nocivi} percepiti dal cervello come dolore.

I nocicettori sono costituiti da terminazioni nervose libere che
rispondono a stimoli che inducono danno tissutale (o potenzialmente
dannosi).

Vi sono 3 tipi di nocicettori:

\begin{enumerate}
\def\labelenumi{\arabic{enumi}.}
\itemsep1pt\parskip0pt\parsep0pt
\item
  \textbf{nocicettori meccanici}, che rispondono a stimoli meccanici
  intensi;
\item
  \textbf{nocicettori termici}, che rispondono a temperature elevate
  (\textgreater{} 44°C);
\item
  \textbf{nocicettori polimodali} che rispondono a molti stimoli,
  compresi quelli meccanici e termici, e a sostanze rilasciate dai
  tessuti danneggiati (istamina, prostaglandina, serotonina e
  bradichina).
\end{enumerate}

Questi recettori presentano \emph{fibre A-delta} e \emph{C}, sono
terminazioni nervose libere.

I recettori del dolore sono caratteristici perché hanno \emph{recettori}
\textbf{TRPV1} (un recettore \emph{vanilloide}) che rispondono a calore
eccessivo o sostanze chimiche come la capsaicina (molecola tipica del
peperoncino che abbassa la soglia termica di attivazione del recettore,
per questo provoca una sensazione di calore/bruciore).

Il mentolo invece fa la stessa cosa ma su recettori che rispondono al
freddo. I recettori sono anche nella bocca e nelle prime vie aeree e per
questo le sensazioni si registrano nel cavo orale.

\subsubsection{Corteccia
somatosensoriale}\label{corteccia-somatosensoriale}

La percezione delle sensazioni somatiche provenienti da tutte le parti
del corpo inizia nella corteccia somatosensoriale primaria. Questa è
organizzata in maniera topografica, nel senso che le informazioni
provenienti da aree del corpo vicine sono generalmente proiettate verso
aree corticali contigue.

La corteccia cerebrale ha un'organizzazione colonnare. Essa è sudduvisa
in strati e si individuano delle colonne poiché regioni di superficie
della corteccia corrispondono a regioni di superficie del corpo.

Per ogni area topografica arrivano stimoli diversi, e si hanno colonne
diverse per le varie attività sensoriali.

Vi sono due vie principali che trasmettono informaizoni somatosensoriali
dai recettori periferici al SNC: la \emph{via delle colonne
dorsali-lemnisco mediale} ed il \emph{tratto spinotalamico}.

Queste due vie di trasmissione inviano differenti tipi di informazioni
sensoriali al talamo ed alla corteccia somatosensoriale.

In entrambi i casi, queste due vie afferenti, penetrate nel midollo
spinale, incrociano prima di raggiungere il talamo.

La \textbf{via delle colonne dorsali-melnisco mediale} trasmette
infromazioni da meccanocettori e propiocettori al talamo.

In questa via, i neuroni di primo ordine originano nella periferia e
penetrano nelle corna dorsali del midollo spinale. Mentre collaterali
dell'assone possono terminare nel midollo spinale e stabilire sinapsi
con interneuroni coinvolti nei riflessi spinali, la diramazione
principale dell'assone ascende ipsilateralmente (dallo stesso lato dello
stimolo) nel midollo spinale verso il tronco encefalico, percorrendo le
\textbf{colonne dorsali}. Queste rappresentano dei tratti di sostanza
bianca che passano medialmente e dorsalmente alle corna dorsali del
midollo spinale. I neuroni di primo ordine terminano nei \textbf{nuclei
delle colonne dorsali}, che sono localizzati nel bulbo, dove formano
sinapsi con neuroni di secondo ordine.

Gli assoni dei neuroni di secondo ordine incrociano e passano nel lato
opposto del bulbo, formando un tratto chiamato \emph{lemnisco-mediale},
e quindi ascendendo verso il talamo.

Nel talamo i neuroni di secondo ordine stabiliscono sinapsi con neuroni
di terzo ordine che ritrasmettono le informazioni dal talamo alla
corteccia somatosensoriale.

Lo stimolo arriva nell'emisfero cerebrale dal lato opposto del corpo
rispetto al lato da cui è partito.

Il percorso che compie è:

midollo spinale \(\rightarrow\) midollo allungato \(\rightarrow\) talamo
\(\rightarrow\) corteccia.

Il \textbf{tratto spinotalamico} trasmette informazioni provenienti dai
termocettori e dai nocicettori al talamo.

Il tratto spinotalamico incrocia già nel midollo spinale, prima di
raggiungere il tronco encefalico.

in questa via, i neuroni di primo ordine che originano a livello
periferico a partire da termocettori o nocicettori penetrano nel corno
dorsale del midollo spinale. A tale livello, i neuroni di primo ordine
possono dirigersi verso l'alto o il basso per una brave distanza lungo
il \emph{tratto di Lissauer}, ma alla fine formano sinapsi con neuroni
di secondo ordine presenti nel corno dorsale.

Sono proprio questi neuroni di secondo ordine che, attraversando il
midollo spinale controlateralmente, ascendono nel quadrante
anterolaterale del midollo spinale verso il tronco encefalico e
terminano nel talamo.

Una volta nel talamo formano sinapsi con neuroni di terzo ordine che
terminano nella corteccia somatosensoriale.

I due stimoli (dorale e spinotalamico) arrivano nella stessa porzione
della corteccia ma in colonne differenti pecorrendo diverse vie.

\subsubsection{La percezine del dolore}\label{la-percezine-del-dolore}

Esistono due tipi di dolore, quello rapido e quello lento, che sono
trasmessi da differenti classi di neuroni afferenti.

Il \textbf{dolore rapido} è percepito come una netta sensazione di
puntura facilmente localizzabile; questo tipo di dolore è trasmesso da
\emph{fibre A\(\delta\)}, mieliniche e di piccolo diametro.

Il \textbf{dolore lento} è percepito in modo poco localizzato, dando
origine ad una sensazione che insorge lentamente; esso è trasmesso da
\emph{fibre C}, amieliniche e di piccolo diametro.

Entrambi questi tipi di fibre formano sinapsi con neuroni di secondo
ordine nel cordo dorsale del midollo spinale. La trasmissione sinaptica
a tale livello è resa possibile da differenti neurotrasmettitori. Tale
sostanza, rilasciata dai neuroni afferenti primari, si lega a recettori
sui neuroni di secondo ordine. Questi ultimi ascendono verso il talamo,
mediante il tratto spinotalamico.

La percezione del dolore non è limitata alla superficie corporea, esiste
anche un \emph{dolore viscerale}.

L'attivazione dei recettori viscerali dà origine ad un dolore chiamato
\textbf{dolore riferito} (poichè è stato riferito alla superficie
corporea).

Il dolore riferito è dovuto al fatto che i neuroni di secondo ordine che
ricevono impulsi da afferenze viscerali ricevono anche afferenze
somatiche.

La sensibilità nocicettiva ha un pesante influsso su tutte le attività
cerebrali e si ripercuote sui nuclei profondi, sull'amigdala e
sull'ipotalamo portando reazioni emotive ed endocrine rispettivamente.

I segnali riguardanti informazioni sensoriali possono essere modulati
durante la loro trasmissione lungo le vie sensoriali, attraverso
facilitazione o attenuazione di segnali che possono portare a
cambiamenti nella percezione finale dell'informazione. I segnali
sensoriali possono essere modulati in qualsiasi punto della via in cui
ci sia una sinapsi.

Secondo la \textbf{teoria del controllo a cancello} la percezione del
dolore può essere inibita a livello spinale attraverso afferenze
somatiche non dolorifiche. Questa teoria postula l'esistenza di
un'inibizione sinaptica ad opera di interneuroni spinali sui neuroni di
secondo ordine che trasportano le informazioni dolorifiche.

È come se ci fossero dei cancelli che tengono chiuse le vie del dolore e
che necessitano di un'apertura affinché lo stimolo attraversi il
cancello e percorra la via. I cancelli consistono di interneuroni
inibitori che inibiscono il neurone di secondo ordine impedendo la sua
stimolazione. L'inibizione deve essere vinta affinché il neurone di
secondo ordine venga stimolato.

I cancelli possono essere anche operati da vie che provengono dal
sistema nervoso centrale. Le endorfine, ad esempio, agiscono in questo
modo.

Durante lo stress fisico sono attive vie discendenti che raggiungono il
midollo spinale provenendo dalla corteccia cerebrale e attraversando la
zona periacquiduttale (attorno all'acquedotto di silvio) e portano
stimoli che raggiungono le giunzioni sinaptiche tra i neuroni di primo e
secondo ordine e attivano neuroni inibitori che rilasciano encefalina
(un oppioide endogeno) e diminuiscono le sensazioni dolorifiche.

Il neurotrasmettitore inibisce la sinapsi inibendo la sensazione di
dolore o riducendo la scarica della sinapsi stessa. Questo succede
quando il corpo è sotto stress. È un meccanismo sfruttato
farmacologicamente. La morfina del papavero da oppio fa lo stesso lavoro
ed è dunque usato come potente antidolorifico.

\section{La vista}\label{la-vista}

La vista è dovuta alla percezione di stimoli luminosi, e consiste nella
ricostruzione di tutti i punti luminosi che provengono dal campo visivo
con la stessa disposizione con cui raggiungono il sistema recettoriale.

L'organo in grado di ricostruire l'immagine è l'occhio.

\subsection{Anatomia dell'occhio}\label{anatomia-dellocchio}

L'occhio può essere diviso in 3 strati concentrici:

\begin{itemize}
\itemsep1pt\parskip0pt\parsep0pt
\item
  quello più esterno, formato da \textbf{sclera} e \textbf{cornea}. La
  \emph{sclera} è un tessuto connettivo consistente (forma la parte
  bianca dell'occhio). Nella parte anteriore la sclera dà origine alla
  \emph{cornea}, una struttura trasparente che consente alla lune di
  penetrare nell'occhio;
\item
  lo strato medio è costituito dalla \textbf{coroide}, dal \textbf{corpo
  ciliare} e dall'\textbf{iride}.

  \begin{itemize}
  \itemsep1pt\parskip0pt\parsep0pt
  \item
    La \emph{coroide} è uno strato di tessuto altamente pigmentato posto
    al di sotto della sclera, include i fotorecettori e vasi ematici che
    nutrono lo strato profondo dell'occhio;
  \item
    Il \emph{corpo ciliare} contiene \textbf{muscoli ciliari}, che sono
    attaccati ad una lente, il \emph{cristallino}, attraverso dei
    sottili tendini di tessuto connettivo chiamati \textbf{fibre
    zonulari}. Il \textbf{cristallino} focalizza la luce sulla
    \textbf{retina}, dove l'informazione visiva viene trasdotta. I
    \emph{muscoli ciliari} cambiano la forma del cristallino,
    permettendo la focalizzazione dei raggi luminosi.
  \item
    L'\emph{iride}, che è formata da due strati di cellule muscolari
    pigmentate, è localizzata davanti al cristallino e determina il
    colore degli occhi. La \textbf{pupilla} è un foro, posizionati
    alcentro dell'iride, che permette alla luce di penetrare nella parte
    posteriore dell'occhio (non è una struttura). L'iride regola il
    diametro della pupilla, variando in tal modo la quantità di luce che
    raggiunge la parte posteriore dell'occhio.
  \end{itemize}
\item
  lo strato più interno è rappresentato dalla \textbf{retina}, che è
  formata da tessuto nervoso contenente i fotorecettori (cellule
  sensibili ale onde luminose). I \emph{fotorecettori} sono di due tipi,
  i \textbf{coni} e i \textbf{bastoncelli}, che percepiscono
  rispettivamente la luce intensa e quella soffusa. La retina funzona
  come un \emph{fototrasduttore}, trasformando l'energia luminosa in
  energia elettrica. Nella parte esterna della retina e attaccato alla
  coroide si trova l'\textbf{epitelio pigmentato della retina}. Questa
  struttura ha un'alta concentrazione del pigmento \emph{nero melanina},
  che assorbe la luce che arriva alla parte posteriore dell'occhio,
  impedendo così la riflessione attraverso la retina e la distorsione
  dell'immagine. Due aree della retina sono molto importanti:

  \begin{itemize}
  \itemsep1pt\parskip0pt\parsep0pt
  \item
    la \textbf{fovea}, che rappresenta il punto centrale della retina,
    dove si dirige la luce proveniente dal centro del campo visivo. È
    l'area della retina con la maggiore acuità visiva;
  \item
    il \textbf{disco ottivo}, cioè la porzione della retina attraversata
    dal nervo ottivo e dai vasi ematici che irrorano l'occhio. Poichè
    questa regione è sprovvista di fotorecettori, essa costituisce un
    \textbf{punto cieco} della retina, dove la luce non può generare
    impulsi elettrici e quindi essere percepita.
  \end{itemize}
\end{itemize}

(immagine 10.19 p271)

Il cristallino e il corpo ciliare suddividono l'occhio in due camere
piene di liquido:

\begin{itemize}
\itemsep1pt\parskip0pt\parsep0pt
\item
  davanti al cristallino e al corpo ciliare si trova il \textbf{segmento
  anteriore}. Questo contiene un liquido limpido e acquoso, definito
  \textbf{umor acqueo}, che fornisce nutrienti alla cornea e al
  cristallino. Poichè la cornea e il cristallino sono strutture
  trasparenti che devono essere facilemnte attraversate dalla luce, se
  dipendessero dall'apporto ematico di nutrienti, la presenza dei vasi
  ostruirebbe parzialmente il passaggio della luce. Il segmento
  anteriore è a sua volta diviso in:

  \begin{itemize}
  \itemsep1pt\parskip0pt\parsep0pt
  \item
    \textbf{camera anteriore} (tra cornea e iride);
  \item
    \textbf{camera posteriore} (tra l'iride e il cristallino).
  \end{itemize}
\item
  posterioremente al cristallino e al corpo ciliare vi è una camera
  trasparente (\textbf{camera vitrea} o \textbf{segmento posteriore})
  contenente una sotanza gelatinosa, definita \textbf{umor vitreo}, che
  contribuisce a mantenere la struttura sferica dell'occhio.
\end{itemize}

L'occhi è un sistema sensibile a fotoni compresi tra i 350 e 750 nm
perché a queste lunghezze le particelle si comportano come onde.

La luce possiede tutte le caratteristiche delle onde e può quandi essere
riflessa e rifratta. La \textbf{riflessione} è un fenomeno per il quale
le onde luminose urtano e rimbalzano su una superficie. La
\textbf{rifrazione} rappresenta il fenomeno per il quale le onde
luminose cambiano direzione nel passare attraverso materiali trasparenti
di densità differenti

Quando i raggi luminosi sono perpendicolari alla superficie da
attraversare, non modificano la propria direzione; tuttavia, s ei raggi
attraversano le superfici con angolazioni diverse da quella
perpendicolare, le lenti concave li fanno divergere, mentre quelle
convesse li fanno convergere verso un punto definito \emph{punto
focale}. La distanza tra l'asse maggiore della lente convessa ed il
punto focale è definita \emph{distanza focale}.

Sia la cornea che il cristallino hanno superfici convesse, che
funzionano facendo convergere le onde luminose che penetrano nell'occhio
a livello retinico; in tal modo, l'immagine che si forma nella retina è
a fuoco.

Per vedere l'immagine a fuoco, la luce proveniente da un determinato
punto del \emph{campo visivo} deve convergere in un singolo punto della
retina. Sebbena la cornea abbia un potere di rifrazione maggiore del
cristallino, a causa del maggior raggio di curvatura, il potere di
rifrazione del cristallino può essere variato per permettere la
focalizzazione della luce sulla retina. La capacità del cristallino di
modificare il suo potere di rifrazione nella visione da vicino e da
lontano è definita \textbf{accomodazione}.

La rifrazione dei raggi luminosi quando passano attraverso la cornea e
il cristallino fa sì che l'immagine venga proiettata sulla retina
invertita e capovolta.

Il punto in cui si focalizza l'oggetto dipende anche dalla distanza
dell'oggetto dalla lente. L'occhio non riesce a mettere a fuoco
qualsiasi distanza, ma solo gli oggetti che si trovano entro una certa
distanza; a lunghezze minori si ha un'immagine sfuocata.

L'occhio umano mette a fuoco solo un piano del campo visivo, però può
variare la distanza della messa a fuoco e mettere a fuoco tutti i piani
del campo visivo che sta osservando ma non contemporaneamente. Questa
funzione è chiamata accomodazione.

La forma del cristallino è controllata dal muscolo ciliare, che presenta
fibre disposte concentricamente, mediante la tensione che esos esercita
sulle fibre zonulari che collegano il muscolo ciliare al cristallino.

Maggiore è la concentrazione di un muscolo circolare e minore sarà il
diametro interno del cercio (muscolo ciliare), cui corrisponderà una
minore tensione delle fibre zonulari e una maggiore curvatura del
cristallino.

Per la visione di oggetti distanti il muscolo ciliare è rilasciato, il
che aumenta il diametro del muscolo stesso, tende le fibre zonulari e
riduce la curvatura del cristallino, in modo tale che questo assuma una
forma più schiacciata (minore convessità).

L'accomodazione è sotto il controllo del sistema nervoso parasimpatico,
che attiva la contrazione del muscolo ciliare per la visione da vicino.
In assenza di attività parasimpatica, il muscolo ciliare si rilascia.

Se le onde non sono adeguatamente focalizzate sulla retina, la visione è
distorta.

I difetti visivi più comuni sono:

\begin{itemize}
\itemsep1pt\parskip0pt\parsep0pt
\item
  \textbf{miopia}. La persona può vedere chiaramente gli oggetti vicini,
  ma non quelli distanti, in quanto il cristallino o la cornea
  rifrangono in maniera eccessiva i raggi luminosi; per tale motivo, gli
  oggetti vicini all'occhio possono essere messi a fuoco senza
  accomodazione, ma quelli posti a distanza vengono focalizzati davanti
  alla retina, con conseguente distorsione dell'immagine;
\item
  \textbf{ipermetropia}. Il cristallino o la cornea sono inadeguati in
  relazione alla lunghezza del bulbo oculare; pertanto, gli oggetti a
  distanza possono essere focalizzati sulla retina solo mediante
  accomodazione, il che significa che il cristallino non riesce a
  ottimizzare l'accomodazione in maniera sufficiente nella visione da
  vicino. La luce porveniente da un oggetto vicino all'occhio viene così
  messa a fuoco oltre la retina, generando una distorsione
  dell'immagine;
\item
  \textbf{astigmatismo}. Le irregolarità della superficie della cornea o
  del cristallino alterano la direzione delle onde luminose;
\item
  \textbf{presiopia}. È un indurimento del cristallino che si verifica
  con il passare degli anni. Questo causa una perdita di elasticità del
  cristallino, che riduce la sua capacità di diventare sferico e rende
  difficile l'accomodazione per la visione da vicino;
\item
  \textbf{cataratta}. È un'altra alterazione clinica correlata all'età,
  che provoca un'opacizzazione del cristallino e ne riduce la
  trasparenza;
\item
  \textbf{glaucoma}. È un aumento del volume dell'umor acqueo che
  determina un incremento della pressione nella cavità anteriore del
  bulbo oculare, alterando la forma della cornea e modificando la
  posizione del cristallino. Il cambiamento di posizione del cristallino
  può trasmettere un'aumentata pressione al corpo vitreo, comprimendo i
  vasi ematici che irrorano la retina e generando cecità permanente.
\end{itemize}

Nella \textbf{emmetriopia} o visione normale, invece, una persona vede
bene sia oggetti lontani che vicini.

Gli occhi sono capaci di regolare il quantitativo di luce che penetra in
essi variando il diametro delle pupille.

Nella luce intensa, le pupille sono di diametro ridotto, o
\emph{costrette}, in modo tale che i fotorecettori non vengano
``accecati'' dalla luce troppo intensa. In presenza di luce fioca, al
contrario, le pupille sono \emph{dilatate}, in modo tale da permettere
un maggior passaggio di luce; la dimensione della pupilla è controllata
dall'iride.

L'iride è formata da due trati di cellule muscolari lisce che circondano
la pupilla:

\begin{itemize}
\itemsep1pt\parskip0pt\parsep0pt
\item
  uno strato interno di \textbf{muscolatura circolare}, detto
  \emph{muscolo costrittore}. I muscoli circolari formano cerchi
  concentrici attorno alla pupilla e, quando si contraggono, il diametro
  della pupilla diminuisce;
\item
  uno strato esterno di \textbf{muscolatura radiale}, detto
  \emph{muscolo dilatatore}. Questi muscoli sono organizzati a raggio e,
  quando si contraggono, il diametro della pupilla aumenta.
\end{itemize}

L'iride è sotto controllo del sistema nervoso autonomo. I neuroni
parasimpatici innervano lo strato di cellule muscolari circolari, mentre
i neuroni simpatici innervano le cellule muscolari radiali.

\subsection{La retina}\label{la-retina}

Nella retina, che è costituita da tessuto nervoso, sono localizzati i
fotorecettori: coni e bastoncelli.

Sulla retina i fotorecettori sono condensati principalmente nella fovea.

I \textbf{bastoncelli} permettono la visione in bianco e nero in
condizioni di luce poco intensa o crepuscolare. I \textbf{coni}
forniscono la visione a colori, ma sono attivi soltanto quando la luce è
intensa (visione diurna).

La retina consta di 3 strati distinti:

\begin{enumerate}
\def\labelenumi{\arabic{enumi}.}
\itemsep1pt\parskip0pt\parsep0pt
\item
  uno strato interno formato da \textbf{cellule gangliari}, che inviano
  il loro assone nel nervo ottico;
\item
  uno strato intermedio formato da \textbf{cellule bipolari}, che unisce
  i recettori e le cellule gangliari;
\item
  uno strato esterno contenente i \textbf{fotoreccetori}, coni e
  bastoncelli.
\end{enumerate}

(immagine 10.29 p277)

Nella retina sono presenti anche altre cellule:

\begin{itemize}
\itemsep1pt\parskip0pt\parsep0pt
\item
  le \textbf{cellule orizzontali}, che presentano protuberanze e sono
  disposte orizzontalmente rispetto alle altre, unendo più fotorecettori
  e cellule bipolari tra loro;
\item
  le \textbf{cellule amacrine}, disposte tra le varie cellule gangliari.
\end{itemize}

Tutto quanto è circondato da uno strato fortemente pigmentato che
impedisce l'entrata di luce lateralmente.

I coni e i bastoncelli, essendo posizionati nello strato esterno della
retina, vengono eccitati dalla luce dopo che questa ha attraversato gli
strati retinici interno e medio. Inoltre, i vasi ematici che perfondono
la retina si trovano lungo il percorso dei raggi luminosi, per cui, al
fine di migliorare la trasmissione della luce alla fovea, le cellule
bipolari e quelle gangliari sono disposte lateralmente al centro della
retina. Si crea così una depressione al centro della retina, definita
\textbf{macula lutea}, che circonda la fovea ed è gialla per la presenza
di carotenoidi.

La fovea contiene solo coni; il rapporto tra bastoncelli e coni aumenta
all'aumentare della distanza dalla fovea, fino alla parte periferica
della retina, dove sono presenti solo bastoncelli.

La fovea è la regione centrale di massima acuità visiva. La direzione
del campo visivo corrisponde alla localizzazione della fovea sulla
retina. Il punto in cui si vede meglio del campo visivo è quello verso
cui viene focalizzato lo sguardo.

\subsection{Fototrasduzione}\label{fototrasduzione}

La fototrasduzione rappresenta il fenomeno mediante il quale l'energia
luminosa viene convertita in segnali elettrici. Questo fenomeno si
realizza nei coni e nei bastoncelli.

L'aspetto morfologico di questi due fotorecettori è simile, in quanto
ciascuno è formato da due parti rilevanti, definite \emph{segmento
interno} e \emph{segmento esterno}.

Il segmento esterno contiene invaginazioni della membrana che formano
strati simili a dischi membranosi, contenenti molecole che, assorbendo
l'energia luminosa, permettono ai fotorecettori di eccitarsi. Il
segmento interno contiene il nucleo cellulare e vari organuli; esso
termina con un bottone sinaptico dove sono presenti le vescicole
contenenti il neurotrasmettitore.

L'assorbimento della luce da parte di molecole chiamate
\textbf{fotopigmenti}, contenute nel segmento esterno, rappresenta il
primo evento della fototrasduzione. Nei recettori sono presenti 4 tipi
differenti di fotopigmenti. Ciascuna molecola di fotopigmento contiene
un componente chiamato \textbf{retinale} ed una proteina chiamata
\textbf{opsina}.

Il retinale è comune a tutti i fotopigmenti, mentre il tipo di opsina
presente determina quali lunghezze d'onda sono assorbite da un
determinato pigmento.

L'attivazione di tutte queste varie opsine permette all'occhio la
visione dei colori.

La \textbf{rodopsina} è una proteina di membrana con 7 domini
transmembrana a \(\alpha\)-elica che si trova principalmente nelle
cellule a bastoncello della retina umana che permettono la vista in
bianco e nero.

Queste cellule hanno una forma allungata e nella loro parte apicale
hanno numerosi dischi di membrana con molte rodopsine, costituite da un
pigmento, l'\emph{11-cis-retinale}, sensibile alla luce, legato
all'\emph{opsina}, una proteina della retina.

Il retinale è legato ad un glutammato dell'opsina forma una base di
Schiff, cioè il pigmento rodopsina.

Quando la rodopsina viene a contatto con un fotone di luce, subisce una
fotodecomposizione, o imbianchimento, che porta alla dissociazione della
molecola con formazione di \textbf{retinale tutto-trans}.

L'opsina varia la sua conformazione e diventa \textbf{metarodopsina II},
mentre il retinale tutto trans si stacca finendo nel citosl e uscendo
dalla cellula dove viene captato dall'epitelio pigmentato.

La metarodopsina II non è più sensibile alla luce e viene chiamata
\textbf{opsina scolorita}. Il fotorecettore è meno sensibile alla luce.
Può essere ricostituita la sensibilità rifornendo 11 cis-retinale per
ricostruire il complesso fotosensibile 11-cis-opsina.

\begin{center}\rule{0.5\linewidth}{\linethickness}\end{center}

\textbf{Registrazione}

Fototrasduzione ..

Al buio non arriva luce e \ldots{}

Al buio il forecettore diventa sinapticamente attivo mentre alla luce è
inattivo.

Al buio stimola la cellula bipolare alla luce smette di farlo.
\_\_\_\_\_\_\_\_\_\_\_\_\_\_\_\_\_\_\_\_\_\_\_\_\_\_\_\_\_\_\_\_\_\_\_\_\_\_\_\_\_\_\_\_\_\_\_\_\_\_\_\_\_\_\_\_\_\_\_\_\_\_\_\_\_\_\_\_\_\_\_\_\_\_\_\_\_\_\_\_\_\_\_\_\_\_\_\_\_\_\_\_\_\_\_\_\_\_\_\_\_\_\_\_\_\_\_\_\_\_\_\_\_\_\_\_\_\_\_\_\_\_\_\_\_\_\_\_\_\_\_\_\_\_\_\_\_\_\_\_\_\_\_\_\_\_

Bastoncelli e coni si comportano in maniera diversa. I bastoncelli sono
specializzati per luce debole e sono molto numerosi nella retina. Questi
non sono in grado di percepire il colore perché assorbono solo una
lunghezza d'onda.

Sono in grado di rispondere anche ad un singolo fotone e se sottoposti a
luce intensa sono completamente scoloriti.

In un ambiente poco illuminato i bastoncelli funzionano e i coni no.
Passando ad un ambiente illuminato i bastoncelli sono tutti in funzione
e vengono stimolati intensamente dalla grande quantità di luce che
arriva e il nervo ottico molto stimolato dà la sensazione di abbaglio ma
gradualmente i bastoncelli si scoloriscono e si inizia a vedere
attraverso i coni che meno sensibili alla luce e danno meno l'effetto di
abbagliamento.

La retina è un sistema nervoso con un'organizzazione relativamente
semplice da descrivere ma abbastanza complessa a livello di interazioni
tra neuroni.

Si è capito come si ricostruisce l'immagine del campo visivo e la
determinazione del contrasto.

La retina compie una prima rielaborazione delle informazioni provenienti
dal campo visivo in quanto è un sistema nervoso. Migliaia di batoncelli
convergono su una singola cellula bipolare, mentre di coni ve ne sono
meno.

Anche qui dei abbiamo campi recettoriali che corrispondono a zone della
retina. L'acuità visiva dipende dal fatto di distinguere come separati
due diversi stimoli che agiscono contemporaneamente sulla retina, quindi
percepire due sorgenti luminose come distinte.

Nel caso dei bastoncelli siccome si hanno molte cellule recettoriali su
una cellula nervosa singola si avrà minore acuità, mentre i coni
consentono di vedere con un acuità visiva maggiore.

La retina rappresenta la porzione nervosa dell'occhio, anche se contiene
i fotorecettori che non sono cellule nervose.

È la luce a indurre i processi che determinano l'attività sinaptica di
queste cellule.

I fotorecettori formano sinapsi con cellule bipolari che possono essere
inibitorie o eccitatorie.

Le cellule bipolari sono capaci di trasmettere potenziali graduati, ma
non potenziali d'azione. In alcune sinapsi tra fotorecettore e cellule
bipolari, l'azione del neurotrasmettitore è eccitatoria mentre in altre
è inibitoria.

Nelle cellule bipolari abbiamo potenziali graduati che influenzano
l'attività delle cellule gangliari.

Le cellule gangliari rappresentano i neuroni sensoriali di primo ordine
nelle vie nervose coinvolte nella visione, capaci di trasmettere
potenziali d'azione. Gli assoni delle cellule gangliari formano il
\textbf{nervo ottico} e sono quindi neuroni di uscita dalla retina.

La retina con i suoi fotorecettori sta a monte nel neurone sensoriale
primario. Questa è una caratteristica specifica di questo organo.

I neuroni sensoriali di primordine hanno i loro campi recettivi. Il
\textbf{campo recettivo} di una cellula gangliare può essere definito
come l'area del campo visivo nella quale uno stimolo luminoso determina
un incremento o un decremento della frequenza di potenziali d'azione
nella cellula.

Sono state caratterizzate con due modalità sensoriali:

\begin{itemize}
\itemsep1pt\parskip0pt\parsep0pt
\item
  il primo tipo di cellula gangliare è definito a \emph{centro ON} e
  \emph{periferia OFF}. In queste cellule gangliari la luce nel centro
  del campo recettivo eccita la cellula (la ``accende''), mentre
  l'applicazione della luce nell'area circostante del campo recettivo la
  inibisce (la ``spegne'');
\item
  il secondo tipo di cellula gangliare è definita a \emph{centro OFF} e
  \emph{periferia ON}. In queste cellule gangliari, l'applicazione della
  luce nel centro del campo recettivo inibisce la cellula, mentre
  l'applicazione della luce nelle zone circostanti la eccita.
\end{itemize}

Caratteristica peculiare sempre della vista è che i nuclei delle cellule
gangliari si trovano nella retina.

Ogni cellula gangliare ha un campo recettoriale e c'è una zona della
retina che convoglia gli stimoli su una cellula gangliare.

C'è una pre-elaborazione del campo recettivo che stimola il neurone
fornendo adesso un'attività decrivibile con un'attività centro ON/
periferia OFF e viceversa. La retina è un insieme di campi recettoriali.
Grazie a questa attività si ha l'effetto contrasto; nel punto dove c'è
una variazione tra due situazioni si ha un'esaltazione della variazione
stessa data dal fenomeno di contrasto.

Il sistema ottico porta sulla retina un'immagine focalizzata in cui
l'immagine è ricostruita fedelmente. Il sistema nervoso della retina
prende i punti del campo visivo e realizza un'accentuazione dei
contrasti esaltando i confini tra situazioni diverse, vale a dire gli
oggetti presenti nel campo visivo. Questo è fondamentale perché la
visione serve proprio a identificare oggetti nel campo visivo.

La visione della retina ha un corrispettivo nella corteccia cerebrale
dove i centri superiori sono in grado di distinguere nettamente gli
oggetti rispetto allo sfondo nel campo visivo. L'oggetto quindi può
essere classificato.

L'attività fisiologica che consente questi processi è un'attività
elettrica che si va a identificare come sinapsi. Il recettore viene
sempre stimolato dalla luce allo stesso modo ma può fare sinapsi
eccitatorie o inibitorie. Se colpito dalla luce si iperpolarizza.

Il cono riceve la luce e ricevendo la luce si iperpolarizza.
L'iperpolarizzazione ha due effetti sulla cellula bipolare a seconda che
ci troviamo in un centro ON o in un centro OFF.

Se ci troviano in un centro ON la cellula bipolare si depolarizza
producendo un potenziale graduato depolarizzato, a sua volta la cellula
gangliare si depolarizza, e con il suo assone genera un potenziale di
azione. Se invece ci trovassimo in un centro OFF la iperpolarizzazione
del recettore darebbe ipolarizzazione della cellula bipolare e gangliare
che andrebbe a spegnere la frequenza del potenziale d'azione fino a
cessare quasi del tutto.

Se la luce cade alla periferia i neuroni del centro ON vengono spenti
mentre quelli del centro OFF vengono attivati.

I centri ON e OFF funzionano anche per i coni necessari per la visione a
colori. L'individuazione dell'oggetto avviene tramite il suo grado di
illuminazione ma anche tramite il suo colore.

Le cellule gangliari inviano ai loro assoni lungo le vie visive. Gli
assoni si concentrano in un punto della retina dove c'è l'emergenza del
nervo ottico, la zona del \textbf{disco ottico}. Il nervo ottico è il
secondo nervo cranico. Molte fibre del nervo incrociano nel
\textbf{chiasma ottico}. La suddivisione delle fibre è molto precisa e
dipende dallo loro provenienza.

I neuroni gangliari costituiscono i neuroni che danno origine ai
potenziali d'azione che vengono trasportati al SNC. Gli assoni delle
cellule gangliari formano il nervo ottivo (II paio di nervi cranici). I
due nervi ottici fuoriescono dagli occhi a livello del \textbf{disco
ottico} e confluiscono alla base dell'encefalo davanti al tronco
encefalico, per formare il \textbf{chiasma ottico}.

A tale livello, metà degli assoni provenienti da ciascun occhio incrocia
per dirigersi verso l'altro lato dell'encefalo. I segnali relativi
all'emicampo visivo di sinistra giungono alla porzione nasale della
retina dell'occhio sinistro e a quella temporale dell'occhio destro.
Allo stesso modo, i segnali provenienti dall'emicampo visivo di destra
giungono all'emiretina nasale dell'occhio destro e a quella temporale
dell'occhio sinistro.

Nel chiasma ottico, gli assoni provenienti dalle cellule gangliari
nasali incrociano, mentre quelli provenienti dal lato temporale
decorrono ipsilateralmente.

Sebbene gli assoni siano sempre quelli delle cellule gangliari, dopo il
chiasma ottico essi viaggiano in un fascio di fibre chiamato
\textbf{tratto ottico}. Le cellule gangliari terminano in un nucleo del
talamo, noto come \textbf{corpo genicolato laterale}, dove formano
sinapsi con neuroni che ascendono fino alla corteccia visiva primaria
nel lobo occipitale, formando le cosiddette \textbf{radiazioni ottiche}.

Sulla retina temporale si forma l'immagine della parte centrale del
campo visivo. È la parte vista da entrambi gli occhi contemporaneamente.
La retina nasale vede invece una parte del campo marginale e vista solo
dall'occhio che sta da quella data parte. La retina temporale invia gli
assoni dal talamo dallo stesso lato mentre quella nasale al lato
opposto. Sono dunque gli assoni della retina nasale ad incrociare nel
chiasma ottico. Quelli della retina temporale no. Una parte del campo
visivo è vista da entrambi gli occhi in modo da garantire la visione
tridimensionale e dare l'effetto di profondità del campo visivo.

\subsection{L'orecchio e l'udito}\label{lorecchio-e-ludito}

\subsubsection{Anatomia dell'orecchio}\label{anatomia-dellorecchio}

L'orecchio può essere diviso in 3 parti:

\begin{enumerate}
\def\labelenumi{\arabic{enumi}.}
\itemsep1pt\parskip0pt\parsep0pt
\item
  \emph{orecchio esterno};
\item
  \emph{orecchio medio};
\item
  \emph{orecchio interno}.
\end{enumerate}

L'orecchio esterno e quello medio sono cavità piene d'aria, mentre
l'orecchio interno è una cavità piena di liquido.

L'orecchio esterno comprende il \emph{padiglione auricolare} ed il
\textbf{meato uditivo esterno}, o canali acustico. La sua funzione
primaria è quella di convogliare e trasmettere le onde sonore verso la
\textbf{membrana timpanica}, che separa l'orecchio esterno da quello
medio.

La funzione dell'orecchio medio è quella di amplificare le onde sonore
per la successiva trasmissione di queste dall'aria ad un ambiente
fluido. All'interno dell'orecchio medio vi sono 3 ossicini, il
\textbf{martello}, l'\textbf{incudine} e la \textbf{staffa} che
collegano la membrana timpanica ad un'altra membrana sottile, la
\textbf{finestra ovale}, che separa l'orecchio medioo da quello interno.

Vi è poi anche un'altra membrana, la \textbf{finestra rotonda}, a
separare l'orecchio medio da quello interno.

La \textbf{tuna di Eustachio}, che connette l'orecchio medio con la
\emph{faringe}, contribuisce a mantenere normale la pressione
nell'orecchio medio.

L'orecchio interno contiene le strutture associate sia all'udito che
all'equilibrio. La \textbf{coclea} è una struttura a spirale che
contiene le cellule recettoriali respondabili della sensibilità uditiva.

Il nervo contenente le fibre nervose afferenti responsabili dell'udito e
dell'equilibrio è noto come \textbf{nervo vestibolare}.

(immagine 10.41 p285)

L'orecchio registra onde sonore. Queste sono delle sequenze di
compressione e dilatazione che si verificano in un mezzo. Se il mezzo è
rigido vengono chiamate vibrazioni, se invece il mezzo è fluido vengono
dette compressioni e dilatazioni.

Queste compressioni e dilatazioni non avvengono a scatti ma con
variazioni progressive e graduali rappresentabili come un'onda. Le onde
sonore hanno una loro scala di intensità, i \emph{decibel}. Questa è una
scala logaritmica.

La soglia del danno auricolare è di 100 dB, mentre la soglia del dolore
è di 130-140 dB.

La frequenza delle onde viene misurata in \emph{Hertz} (cicli/secondo).
Esiste un intervallo di sensibilità anche per la frequenza delle onde
sonore che va dai 20 ai 20000 Hz (l'intervallo di maggior sensibilità è
1000-4000 Hz) Intervalli con maggio frequenza sono gli ultrasuoni. Le
basse frequenze corrispondono a suoni bassi, mentre le frequenze alte
corrispondono a suoni acuti.

\subsubsection{Amplificazione delle onde sonore
nell'orecchio}\label{amplificazione-delle-onde-sonore-nellorecchio}

Quando le onde sonore stimolano l'orecchio, comprimono la membrana
timpanica, causandone un'oscillazione che segue la stessa frequenza
dell'onda sonora. Poichè il timpano è connesso al martello, le
oscillaizoni della membrana determineranno un'oscillazione del martello
alla stessa frequenza e con un'ampiezza che rifletterà quella delle
oscillazioni che fanno vibrare il timpano. Il movimento del martello
amplifica a sua volta quello dell'incudine che, a sua volta, causa un
ulteriore aumento del movimento della staffa. L'effetto netto risultante
è un'amplificazione delle onde sonore.

Questi tre ossicini sono articolati tra loro (come tutte le ossa dello
scheletro) e derivano da ossa mandibolari presenti nei pesci.

Poichè la stassa è disposta sopra la finestra ovale, che a sua volta è
in contatto con la coclea piena di fluido, le oscillazioni della staffa
generano onde nel liquido cocleare.

L'amplificazione delle onde sonore avviene grazie alla trasmissione
delle onde da una membrana più grande, quella timpanica, a una membrana
più piccola, quella della finestra ovale, in quanto una certa forza che
si esercita su una superficie più piccola produce una pressione
maggiore.

\subsubsection{Trasduzione del segnale
sonoro}\label{trasduzione-del-segnale-sonoro}

La coclea è l'organo nel quale avviene la straduzione sonora.

Dall'esterno, la coclea appare come una conchiglia a forma di spirale;
il punto terminale della spirale è chiamato \textbf{elicotrema}.

La coclea è separata da due membrane in 3 compartimenti pieni di fluido:
la \textbf{membrana vestibolare} e la \textbf{membrana basilare}
suddividono la coclea in \textbf{scala vestibolare}, \textbf{scala
timpanica} e \textbf{scala media}. Le membrane vestibolare e basilare si
uniscono a livello dell'elicotrema.

Il liquido presente nella scala vestibolare e in quella timpanica è
chiamato \textbf{perilinfa}; esso differisce da quello che si trova
nella scala media, chiamato \textbf{endolinfa}.

La composizione della perilinfa è simile a quella del liquido
cerebrospinale, mentre l'endolinfa ha una composizione simile a quella
del liquido intracellulare, con un'elevata concentrazione di ioni
potassio e una bassa concentrazione di ioni sodio.

La coclea è una struttura chiusa piena di liquido separata dall'orecchio
medio tramite la finestra ovale e la finestra rotonda. Poichè il liquido
è incomprimibile, per generare onde è richiesto un movimento nel sistema
senza cambiamento di volume.

La scala timpanica è in contatto con il timpano e in fondo all'elica c'è
una fessura che mette in comunicazione la scala timpanica e vestibolare,
la quale arriva alla finestra rotonda.

L'\textbf{organo del Corti}, che rappresenta la struttura sensoriali in
grado di trasdurre lo stimolo sonoro, è localizzato sulla superficie
della membrana basilare. L'organo del Corti contiene \textbf{cellule
ciliate}, cellule di supporto ed una membrana sovrastante le cellule
ciliate, chiamata \textbf{membrana tettoria}.

(immagine 10.45 p289)

I liquidi sono incomprimibili ma i suoni si propagano anche in essi. Se
in un liquido applichiamo un colpo di pressione, all'altra estremità
dopo un certo tempo avremo l'eco del colpo di pressione. Tutto il
liquido è attraversato dall'onda di compressione che si riverbera anche
sulle pareti del contenitore del liquido. L'onda di compressione che
arriva sulla finestra ovale si ripercuote come vibrazione sulle membrane
della coclea. Le membrane della coclea vanno in vibrazione con la stessa
frequenza della finestra ovale, quindi del timpano. Il tutto si scarica
sulla finestra rotonda dove viene dissipata l'energia dell'onda di
compressione.

Una ``lingua'' della membrana tettoria va a lambiere le cellule
recettoriali del corti. C'è una continuità tra le ciclia dell'organo del
corti e questa porzione di membrana. Le cilia entrano in oscillazione
come conseguenza dell'oscillazione delle membrane.

Le ciclia dell'organo del corti sono \textbf{stereocilia} in quando non
hanno uno scheletro tubulinico ma sono sono ricche di canali del
potassio sensibili allo stiramento.

Le punte delle stereociglia sono collegate tra loro da ponti proteici,
così che esse si muovono tutte insieme. In base alla direzione della
deflessione delle stereociglia, possono aprirsi o chiudersi canali per
il potassio sulla membrana delle cellule ciliate. Poichè l'endolinfa ha
una concentrazione di ioni K più elevata di quella presente all'interno
delle cellule ciliate, l'apertura dei canali per il potassio determina
l'ingresso dello ione nelle cellule ciliate, determinandone la
depolarizzazione; al contrario, la chiusura dei canali per il K riduce
il flusso entrante di K, determinando l'iperpolarizzazione delle cellule
ciliate.

Quando i canali per il potassio si aprono indocono l'entrata di ioni K
depolarizzando la cellula. Questa parziale depolarizzazione determina
l'apertura di alcuni canali per il calcio, con conseguente ingresso di
tale ione e rilasco per esocitosi del neurotrasmettitore a livello delle
sinapsi tra la cellula ciliata e la fibra afferente.

Quando le stereociglia sono piegate nella direzione dello stereociglio
più lungo, si esercita una maggiore tensione sui filamenti elastici e i
canali per il potassio risulteranno maggiormente aperti. Di conseguenza,
un maggior quantitativo di K penetra nelle cellule ciliate, aumentando
lo stato di depolarizzazione, che in ultima analisi determina
l'incremento della frequenza di scarica nel neurone afferente.

Quando le stereociglia sono piegate in direzione opposta a quella dello
stereociglio più lungo, i filamenti elastici non sono messi in tensione
ed i canali per il potassio risultano chiusi. La chiusura dei canali per
il K impedisc eil flusso entrante di K e determina una
iperpolarizzazione delle cellule ciliate rispetto alla condiizone di
riposo. In queste condizione si determina la chiusura dei canali per il
calcio, a cui consegue un minor rilascio di neurotrasmettitore e, in
ultima analisi, la diminuzione drastica della frequenza di scarica nel
neurone afferente.

Un'ampiezza maggiore dell'onda comporta un'oscillazione maggiore.

La codificazione della frequenza dei suoni si basa sulla localizzazione
delle cellule ciliate sulla membrana basilare. Onde sonore di differente
frequenza determinano la deflessione della membrana basilare in regioni
differenti, in quanto la struttura di questa membrana varia lungo la sua
lunghezza. In prossimità delle finestre ovale e rotonda, la membrana
basilare ha una maggiore rigidità ed è più stretta; in prossimità
dell'elicotrema essa è più ampia e flessibile. Pertanto, le onde sonore
ad alta frequenza e di tono elevato, determinano una maggiore
deflessione della membrana basilare nella regione più prossima alle
finestre ovale e rotonda, con conseguente attivazione delle cellule
ciliate localizzate in questa regione. Le onde sonore di bassa
frequenza, al contrario, determinano una maggiore deflessione della
membrana basilare nella regione più prossima all'elicotrema, con
consegunete attivazione delle cellule ciliate localizzate in questa
regione.

La frequenza del tono è codificata in termini di topografia delle
cellule ciliate che sono maggiormente attivate.

\textbf{23.11.2015 A}

I suoni puri sono formati da una sola frequenza e sono difficilmente
ascoltabili, vengono prodotti da strumenti e in genere risultano
sgradevoli. Al pianoforte quando si parla del La centrale, il suono
prodotto non è un tono puro ma come tutti i suoni degli strumenti
musicali è composto da frequenze ben precise che si dicono armoniche
(frequenze ad intervalli definiti). Più regioni della coclea sono
coinvolte contemporaneamente nella percezione di questo suono.

Il rumore è dato da un'accozzaglia di frequenze con rapporti non
regolari. I suoni sono facilemente interpretabili come codici rispetto
ai rumori perché gli ultimi non possono essere facilmente discriminati
su un fondo di rumore dell'ambiente. Per questo possono essere usati dai
cetacei per comunicare.

Normalmente tutti i suoni che arrivano alla coclea, siano essi rumori o
suoni musicali, coinvolgono sempre regioni diverse della coclea a meno
che non siano suoni puri prodotti da strumenti.

Il neurotrasmettitore rilasciato dalle cellule ciliate si lega ai
recettori di neuroni afferenti del \textbf{nervo cocleare}, parte
dell'VIII paio di nervi cranici. I neuroni afferenti (di primo ordine,
il cui corpo cellulare si trova nel ganglio del Corti) terminano nei
\textbf{nuclei cocleari} del bulbo, dove comunicano con neuorni di
secondo ordine che terminano in un nucleo talamico chiamato
\textbf{corpo genicolato mediale}.

A tale livello, i neuroni di secondo ordine formano sinapsi con neuroni
di terzo ordine che trasmettono le informazioni acustiche alla
\textbf{corteccia uditiva}, situata nel lobo temporale.

I neuroni di secondo ordine arrivano in parte al \textbf{nucleo olivare}
sia dello stesso lato che del lato opposto incrociando e arrivando al
follicolo inferiore del tronco encefalico dove formano sinapsi
nell'oliva dello stesso lato o del lato opposto o nel collicono
inferiore. Qui formano sinapsi per arrivare al \textbf{corpo genicolato
mediale}. Ci sono dei collaterali che vanno anche al cervelletto. Dal
corpo genicolato mediale i neuroni del quarto ordine inviano la fibra
assonale alla corteccia.

\subsection{L'orecchio e l'equilibrio}\label{lorecchio-e-lequilibrio}

Alla percezione sensoriale acustica è allegata una percezione sensoriale
organizzata in maniera molto simile. Nei pesci la linea laterale
funziona allo stesso modo percependo vibrazioni del mezzo idrico.

L'organo dell'equilibrio nei mammiferi è l'orecchio interno ossia
l'organo vestibolare. Anche questo è costituito da un insieme di
membrane che racchiudono spazi sacciformi.

L'apparato vestibolare è localizzato in una cavità dell'osso temporale
nota come \emph{labirinto osseo}. Poichè l'apparato vestibolare è
costituito da strutture membranose all'interno del labirinto osseo,
queste strutture sono chiamate anche \emph{labirinto membranoso} (di cui
fa parte anche la coclea).

Il labirinto membranoso è pieno di endolinfa, mentre lo spazio compreso
tra il labirinto membranoso e quello osseo contiene perilinfa (gli
stessi presenti nella coclea).

L'\textbf{apparato vestibolare} è formato dai \textbf{canali
semicircolari}, dall'\textbf{otricolo} e dal \textbf{sacculo}. Questi 3
canali sono disposti più o meno a 90° tra loro. La disposizione di
questi canali ci dimostra in modo lampante che il modello di
disposizione tridimensionale dello spazio ha una sua validità.

Il sistema non comunica funzionalmente con la coclea ma in comune hanno
la perilinfa e l'endolinfa, anatomicamente inoltre derivano dagli stessi
elementi morfogenetici. La disposizione ortogonale è utile per percepire
informazioni lungo tutti gli assi spaziali ed è legata ai movimenti del
capo.

Alla base di ciascun canale semicircolare vi è un'area dilatta chiamata
\textbf{ampolla}, che contiene cellule recettoriali cicliate.
All'interno di ciascuna ampolla vi è la \textbf{cupola}, una struttura
gelatinosa separata dall'endolinfa mediante una membrana. Alla base
della cupola, tra le cellule di supporto, vi sono cellule ciliate simili
a quelle cocleari che proiettano le loro ciglia verso la cupola
gelatinosa; una di queste stereociglia, più grande delle altre, prende
il nome di \textbf{chinociglio}.

Come per il sistema uditivo, la deflessione delle ciglia determina
l'apertura o la chiusura di canali ionici.

Quando la testa è a riposo, nessuna forza agisce sulle ciglia, che
pertanto sono in posizione verticale. In questo caso le cellule sono
parzialmente depolarizzate, per cui la frequenza di scarica delle
afferenze nervose è bassa.

Quando la testa inizia a ruotare, il labirinto osseo ruota con essa;
tuttavia il movimento dell'endolinfa avviene successivamente a quello
del labirinto osseo, a causa dell'inerzia del liquido, che esercita una
forza sulla cupola in seguito allo spostamento dell'endolinfa nella
direzione opposta a quella della rotazione.

Il piegamento delle cilia porta a chiusura dei canali potassio e
diminuendo la frequenza del potenziale di membrana, viceversa se la
testa si muove nell'altra direzione. A testa ferma si ha frequenza
intermedia del potenziale di azione. Funziona come nell'organo del
Corti.

Sacculo e otricolo funzionano in modo simile ai canali semicircolari, in
quanto contengono cellule ciliate con stereociglia che si piegano.

L'otricolo e il sacculo sono strutture relativamente ampie, localizzate
tra i canali semicircolari e la coclea dell'orecchio interno.
All'interno di tali strutture si trovano cellule ciliate con
stereociglia che si estendono all'interno di una sostanza gelatinosa.
Sulla sommità delle cellule ciliate, adagiati sulla sostanza gelatinosa,
si trovano piccoli cristalli di carbonato di calcio, gli
\textbf{otoliti}, che incrementano la massa della sostanza gelatinosa.

Le cellule ciliate dell'otricolo sono disposte orizzontalmente rispetto
all'asse principale della testa, con le stereociglia orientate
verticalmente; le cellule ciliate del sacculo sono invece disposte
verticalmente, con le stereociglia orientate orizzontalmente. A causa di
tale orientamento, l'otricolo è in grado di percepire l'accelerazione
lineare avanti e indietro, mentre le cellule ciliate del sacculo si
eccitano in seguito all'accelerazione lineare verso l'alto o il basso.

Questi movimenti sono meno percepibili di quelli rotatori e sono la
percezione per esempio dell'accelerazione e della frenata. Si avvertono
solo le variazioni velocità e non i moti lineari uniformi.

Le afferenze vestibolari giungono al tronco encefalico come nervo
vestibolare. La maggior parte di esse termina nei \textbf{nuclei
vestibolari}, sebbene alcune procedano direttamente verso il cervelletto
per fornire immediate informazioni per poter regolare l'equilibrio e il
bilanciamento del baricentro del corpo, che è di fondamentale importanza
nella coordinazione dell'attività motoria.

Le vie vestibolari sono adiacenti ai sistemi per l'udito e non stupisce
che ci sia un unico nervo che porta sia le fibre uditive che quelle
vestibolari.

Anche queste percezioni sensoriali viaggiano all'interno dell'ottavo
nervo cranico e giungono ai nuclei vestibolari per poi andare in parte
al cervelletto (dal ganglio cocleo vestibolare) e in parte alla
corteccia. La parte che va al cervelletto è importante per la
coordinazione motoria e l'organo dell'equilibrio è fondamentale alla
coordinazione motoria. Legata alle vie vestibolari e alla coordinazione
delle vie vestibolari e visive è la nausea da movimento. Dalla corteccia
si hanno ripercussioni ai nuclei profondi e da lì ripercussioni
endocrine dirette agli organi viscerali. Queste vie sono fondamentali
per la coordinazione ed il controllo dei movimenti.

\begin{center}\rule{0.5\linewidth}{\linethickness}\end{center}

\textbf{CONTROLLARE} Il neurone sensoriali primario invia una fibra a
formare sinapsi con il recettore e l'altra con il nucleo vestibolare nel
tronco encefalico o direttamente al cervelletto (via in cui non c'è
neurone di secondo ordine). Dal nucleo vestibolare al tronco encefalico
invece c'è il nucleo di secondo ordine. Le vie acustiche vestibolari
sono comuni fino all'ingresso nel snc dove vanno nel cervelletto o
\ldots{}
\_\_\_\_\_\_\_\_\_\_\_\_\_\_\_\_\_\_\_\_\_\_\_\_\_\_\_\_\_\_\_\_\_\_\_\_\_\_\_\_\_\_\_\_\_\_\_\_\_\_\_\_\_\_\_\_\_\_\_\_\_\_\_\_\_\_\_\_\_\_\_\_\_\_\_\_\_\_\_\_\_\_\_\_\_\_\_\_\_\_\_\_\_\_\_\_\_\_\_\_\_\_\_\_\_\_\_\_\_\_\_\_\_\_\_\_\_\_\_\_\_\_\_\_\_\_\_\_\_\_\_\_\_\_\_\_\_\_\_\_\_\_\_\_\_

\subsection{Il gusto}\label{il-gusto}

La possibilità di gustare il cibo è dovuta alla presenza di chemocettori
nel cavo orale che rispondono ad alcune sostanze chimiche presenti negli
alimenti.

I chemocettori gustativi sono localizzati in strutture chiamate
\textbf{bottoni gustativi} ciascuno dei quali contiene da 50 a 150
cellule recettoriali e numerose cellule di supporto.

All'estremità di ciascun bottone gustativo vi è un poro che permette
alle cellule recettoriali di venire direttamente in contatto con la
saliva e con le molecole in essa disciolte.

Ciascun individuo ha più di 10.000 bottoni gustativi, localizzati
prevalentemente sulla lingua e sul palato, oltre che nella faringe.

Le cellule recettoriali gustative sono cellule epiteliali modificate,
con microvilli che si propagano all'interno del poro. Sulla superficie
della membrana dei microvilli sono presenti recettori gustativi in grado
di legare selettivamente le differenti sostanze chimiche, chiamate
\textbf{gustanti}.

Nella corteccia cerebrale le sensazioni vengono fuse insieme per creare
la sensazione dei sapori.

La cellula recettoriale, reagendo agli stimoli, va incontro a una
depolarizzazione che viene chiamato ``potenziale di recettore''. Tanto
più forte è lo stimolo, tanto più ampia sarà la depolarizzazione.

La depolarizzazione della membrana dei recettori induce apertura dei
canali del calcio e conseguente rilascio del neurotrasmettitore tramite
rilascio delle vescicole sinaptiche.

Un neurone avverte l'effetto del neurotrasmettitore e se viene superata
la soglia di attivazione si genera potenziale d'azione.

Quanto maggiore è il rilascio del neurotrasmettitore tanto più frequente
diventa il potenziale d'azione.

Le papille gustative sono delle protuberanze della lingua in cui si
trovano molti bottoni gustativi, e ci permetto di percepire i gusti.

I gusti percepiti sono un numero molto limitato: salato, acido, dolce,
amaro e umami.

L'umami è una risposta al glutammato (un amminoacido) ed è associato al
sapore della carne, del formaggio e in generale ai cibi ricchi di
proteine.

Se i sapori venissero percepiti solo con questo tipo di sensibilità
sarebbe impossibile creare nuovi sapori. I gusti esistono perché
esistono recettori che rispondono a determinati stimoli.

Ad esempio, uno stimolo è legato a recettori che reagiscono agli ioni
H\(^+\) che bloccano i canali K\(^+\) depolarizzando la cellula, e in
questo caso parliamo del gusto acido.

Il sapore salato è determinato dalla presenza di ioni sodio nel cibo.
Quando la concentrazione di ioni sodio all'esterno della cellula aumenta
in seguito all'assunzione di cibi salati, aumenta la forza
elettrochimica che spinge verso l'intenro ntali ioni; ciò determina un
aumento del flusso di sodio nella cellula, che depolarizza la membrama,
determinando l'apertura di canali voltaggio-dipendenti del Ca ed il
rilascio del neurotrasmettitore.

Nel caso del sapore dolce il recettore risponde a molecole simili al
saccarosio. Il saccarosio ed altre sostanze chimiche dolci formano
legani con recettori di membrana che attivano una proteina G chiamata
\textbf{gustducina} che attiva l'adenilato ciclasi. Il cAMP, a sua
volta, attiva una proteina chinasi che catalizza la fosforilazione dei
canali per il potassio, che si chiudono. Ciò causa una diminuzione della
fuoriuscita di potassio dalla cellula ed una conseguente
depolarizzazione. Si aprono i canali del calcio e si verifica il
conseguente rilascio del neurotrasmettitore.

Esistono poi recettori che reagiscono a molecole contenenti azoto.
Poichè queste molecole spesso sono tossiche, l'amaro viene percepito
come gusto sgradevole. Queste sostanze bloccano i canali per il
potassio, determinando minor diffusione di tale ione fuori dalla
cellula, depolarizzazione della stessa ed apertura di canali per il
calcio voltaggio-dipendenti. Il calcio entrato nella cellula determina
il sonceguente rilascio di neurotrasmettitore.

Per il glutammato si stimolano recettori purinergici.

Ciascuna cellula recettoriale presenta tutti e quanttro i meccanismi di
trasduzione ed è quindi in grado di rispondere a tutti i sapori primari.
In ogni caso, una cellula recettoriale tende a rispondere maggiormente
ad un sapore primario specifico piuttosto che ad un altro.

Quando lo stimolo è il dolce la cellula aumenta molto la frequenza di
potenziale di azione, per gli altri gusti invece c'è molta meno
risposta. La codificazione avviene perché se c'è uno stimolo di un certo
tipo certe cellule si attivano di più, altre meno ed altre ancora per
nulla. Se c'è un altro stimolo il quadro della situazione varia ancora.

La percezione dei gusti è piuttosto schematica, ma la gamma dei sapori
che percepiamo è molto più grande rispetto a quelli elencati.

L'amplificazione del gusto è data dalla fusione cerebrale delle vie
olfattive e di quelle gustative.

Le vie nervose sono organizzate nei nervi cranici 7, 9 e 10 che
terminano nel \textbf{nucleo gustativo} del bulbo nel tronco encefalico
dove formano sinapsi con neuroni di secondo ordine. Tali neuroni
viaggiano verso il talamo controlaterale dove formano sinapsi con
neuroni di terzo ordine, che terminano nella \textbf{corteccia
gustativa}.

\subsection{L'olfatto}\label{lolfatto}

Il senso dell'olfatto dipende in gran parte dagli stessi meccanismi che
sono allabase del senso del gusto.

Nell'olfatto specifiche sostanze chimiche, chiamate \textbf{odoranti},
si devono sciogliere nel \emph{muco} per potersi legare a specifici
chemocettori ed essere percepite.

All'interno della cavità nasale è presente l'organo dell'olfatto,
l'\textbf{epitelio olfattivo}, che ha una superficie di circa 1 cm\(^2\)
in ciascun lato del naso.

(immagine 10.54 p297)

L'epitelio olfattivo è composto da due strati cellulari: la \emph{mucosa
olfattiva} e la \emph{lamina propria}.

La \textbf{mucosa olfattiva} è costituita da diversi tipi di cellule,
compresi:

\begin{itemize}
\itemsep1pt\parskip0pt\parsep0pt
\item
  i corpi cellulari delle \textbf{cellule recettoriali olfattive}
  (neuroni);
\item
  le \textbf{cellule basali}, che sono i precursori per lo sviluppo di
  nuove cellule recettoriali;
\item
  le \textbf{cellule di supporto} o \textbf{sostentacolari}, che
  mantengono l'ambiente extracellulare che circonda le cellule
  recettoriali.
\end{itemize}

Le cellule olfattive sono gli unici neuroni dell'organismo ad essere
rimpiazzati regolarmente.

Nella \textbf{lamina propria} ci sono:

\begin{itemize}
\itemsep1pt\parskip0pt\parsep0pt
\item
  le \textbf{ghiandole di Bowman}, che producono il muco che abbiamo
  nella cavità nasale;
\item
  gli assoni dei neuroni olfattivi.
\end{itemize}

Le cellule recettoriali sono neuroni bipolari ed hanno ciglia sulle
terminazioni dendritiche che si proiettano nel muco che riveste la
cavità nasale. Tali ciglia hanno dei recettori che si legano con
specifici odoranti. Queste ciglia non arrivano alla superficie aerea ma
alla superficie dell'epitelio che è rivestita da abbondante muco.

Gli assoni delle cellule recettoriali salgono verso il \textbf{bulbo
olfattivo} (è la punta più avanzata del SNC) ed entrano nel SNC
attraverso dei fori nella \textbf{lamina cribrosa} del cranio, che fa
parte delle ossa della cavità nasale.

La capacità di distinguere odori diversi è molto molto ampio in rapporto
a quella di distinguere i gusti.

Per farci sentire gli odori, gli odoranti devono attraversare lo strato
di muco e legarsi ai recettori olfattivi.

Il legame di un odorante ad un recettore di membrana attiva una proteina
G (chiamata G\(_o\)\(_l\)\(_f\)), che a sua volta attiva l'enzima
adenilato ciclasi, che catalizza la formazione di AMP ciclico. Qui il
cAMP si lega rirettamente a canali cationici, aprendoli e permettendo
l'ingresso nella cellula di ioni sodio e calcio. L'effetto risultante è
la depolarizzazione della cellula recettoriale.

L'ingresso del calcio nella cellula determina anche l'apertura di canali
per il cloruro, permettendo a questo ione di fuoriuscire dalla cellula e
di aumentarne la depolarizzazione che, se è di ampiezza sufficiente,
genera nell'assone della cellula recettoriale potenziali d'azione.

Abbiamo specifiche cellule recettoriali che rispondono a specifiche
molecole odorose. Le molecole odorose sono soprattutto molecole
organiche volatili (alcol, benzeni e terpenoidi dei fiori infatti si
sentono bene insieme all'odore della frutta) che arrivano ai recettori
dell'olfatto nel punto di comunicazione tra la cavità orale e quella
nasale.

Gli assoni dei neuroni afferenti formano il \textbf{nervo olfattivo} (I
paio di nervi cranici). Gli assoni terminano in un'area dell'encefalo
chiamata \textbf{bulbo olfattivo}, dove le afferenze nervose
stabiliscono sinapsi con neuroni di secondo ordine chiamati
\textbf{cellule mitrali}.

Numerosi neuroni olfattivi convergono su una singola cellula mitrale e
queste sinapsi formano dei grappoli (cluster) chiamati
\textbf{glomeruli}.

Sembrano dei gomitoli di protuberanze neuronali.

I glomeruli sono formati da dentriti di cellule mitrali e assoni di
neuroni sensoriali che convergono sulle cellule mitrali. Dove c'è
maggiore convergenza l'olfatto è molto più sviluppato, come succede, ad
esempio nei cani. Il glomeruli del cane sono più sviluppati di quelli
dell'uomo.

I neuroni di secondo ordine formano il \emph{tratto olfattivo}, che
termina nel \emph{tubercolo olfattivo}. Da tale livello, le vie
olfattive terminano in due aree della corteccia cerebrale: la corteccia
olfattiva, che trasmette informaizoni necessarie alla percezione e alla
discriminazione delle sostanze odorose, e il sistema limbico, che è alla
base dei comportamenti guidati dall'olfatto, come il comportamento
sessuale.

Ci sono vie che vanno al talamo, all'amigdala e all'ippocampo con
connessioni dirette dell'olfatto con i centri della memoria e
dell'emotività.

\section{Il sistema endocrino}\label{il-sistema-endocrino}

Le ghiandole endocrine possono essere suddivise in due tipologie:

\begin{itemize}
\itemsep1pt\parskip0pt\parsep0pt
\item
  \textbf{organi endocrini primari}, la cui principale funzione è la
  secrezione di ormoni;
\item
  \textbf{organi endocrini secondari}, per i quali la secrezione di
  ormoni è secondaria rispetto ad altre funzioni.
\end{itemize}

Tra gli organi endocrini primari troviamo:

\begin{itemize}
\itemsep1pt\parskip0pt\parsep0pt
\item
  l'\textbf{ipotalamo};
\item
  l'\textbf{ipofisi} o ghiandola pituitaria;
\item
  l'\textbf{epifisi} o ghiandola pineale;
\item
  la \textbf{tiroide};
\item
  le \textbf{paratiroidi};
\item
  il \textbf{timo};
\item
  le \textbf{ghiandole surrenali};
\item
  il \textbf{pancrea};
\item
  le \textbf{gonadi}.
\end{itemize}

Anche nel cuore c'è una produzione endocrina.

Il termine \emph{``endocrino''} si riferisce ad un organo che produce
una sostanza che, veicolata dal sangue, viene portata in varie parti del
corpo. Le molecole segnale prodotte dalle ghiandole generalmente sono
chiamate ormoni.

Siccome l'attuale teoria della biologia è basata su assi e non cicli,
anche l'endocrinologia è influenzata da questa impostazione ad assi.
L'asse degli schemi endocrinologici ha un punto di partenza principale:
l'ipotalamo.

\subsection{L'ipotalamo e l'ipofisi}\label{lipotalamo-e-lipofisi}

Insieme, l'ipotalamo e l'ipofisi regolano il funzionamento di quasi
tutti i sistemi del nostro corpo.

L'\textbf{ipotalamo} è una regione encefalica che svolge molte alre
funzioni oltre a quella di ghiandola endocrina.

L'ipotalamo è posizionato in posizione ventrale rispetto l'acquedotto e
le cavità del liquido cerebrospinale.

È considerato una ghiandola endocrina primaria perchè secerne molti
ormoni, la maggior parte dei quali agisce sull'ipofisi, struttura avente
la dimensione di un pisello collegata all'ipotalamo da un sottile
peduncolo tissutale, chiamato \emph{infundibolo}.

L'ipofisi è suddivisa anatomicamente e funzionalmente in due lobi
distinti, denominati \textbf{adenoipofisi} (anteriormente) e
\textbf{neuroipofisi} (posteriormente).

L'ipotalamo contiene neuroni neurosecernenti che arrivano all'ipofisi
con i loro assoni. Il peduncolo è attraversato da una massa di fibre che
mettono in comunicazione l'ipotalamo con la neuroipofisi (è una
protuberanza dell'encefalo in quanto costituita da cellule neurali).

Attaccata a questa massa neurale vi è una massa di cellule non neurali.
Le due parti dell'ipofisi hanno origini embrionali diverse: ectodermica
quella neurale e mesodermica quella non neurale.

Sono però entrambe coinvolte nella secrezione endocrina.

I nuclei dei neuroni dell'ipofisi sono chiamati \textbf{nucleo
sopraottico} e \textbf{nucleo paraventricolare}. Questi contengono
neuroni che mandano gli assoni alla neuroipofisi.

Il lobo posteriore dell'ipofisi contiene tessuto nervoso costituito
dalle terminazioni dei neuroni il cui corpo cellulare si trova
nell'ipotalamo. Queste terminazioni nervose secernono due ormoni
peptidici: l'\textbf{ormone antidiuretico} (ADH, o
\textbf{vasopressina}) nel nucleo paraventricolare, e
l'\textbf{ossitocina} nel nucleo sopraottico.

L'ipotalamo e il lobo anteriore dell'ipofisi (adnoipofisi) sono connessi
mediante il \textbf{sistema portale ipotalamo-ipofisario}. La
definizione di \textbf{sistema portale} indica una disposizione
particolare dei vasi sanguigni in cui due letti capillari sono disposti
in serie, uno dopo l'altro.

(immagine 6.4 pag153)

Dopo che l'ipotalamo ha secreto gli \textbf{ormoni trofici} (regolano la
secrezione di altri ormoni) nel letto capillare che lo perfonde, questi
ormoni attraversano l'infundibolo all'interno di una vena portale
raggiungendo un secondo letto capillare situato nell'adenoipofisi. Qui
gli ormoni trofici stimolano o inibiscono il rilascio degli ormoni
adenoipofisari.

Il sistema portale ha la funzione di portare direttamente i fattori
trofici ipotalamici alle loro cellule bersaglio situate
nell'adenoipofisi, evitando che questi ormoni vengano diluiti e
degradati dagli enzimi nella circolazione sistemica.

Per ogni ormone secreto dall'adenoipofisi vi è un ormone ipotalamico che
ne consente il rilascio specifico, per questo vengono detti
\emph{fattori di rilascio}. Esistono poi dei \emph{fattori inibenti il
rilascio}.

Per ogni ormone ipofisario vi è un fattore stimolante o inibente, e
questi agiscono sulle cellule degli organi corrispondenti. Non tutti
agiscono su tutte le cellule.

La secrezione dei fattori trofici ipotalamici è a sua volta regolata da
segnali nervosi che raggiungono i neuroni ipotalamici. Ad eccezione di
uno, tutti questi ormoni sono peptidi:

\begin{enumerate}
\def\labelenumi{\arabic{enumi}.}
\itemsep1pt\parskip0pt\parsep0pt
\item
  \emph{fattore stimolante il rilascio di prolattina} (PRH); stimola
  l'adenoipofisi a secernere la \emph{prolattina}, che a sua volta
  stimola lo sviluppo della ghiandola mammaria e la secrezione del latte
  nelle femmine;
\item
  \emph{fattore inibente il rilascio di prolattina} (PIH) o
  \emph{dopamina}, inibisce la secrezione adenoipofisaria di prolattina;
\item
  \emph{fattore stimolante il rilascio di tireotropina} (TRH); stimola
  l'adenoipofisi a rilasciare \emph{tireotropina} o \emph{ormone
  tiroide-stimolante} (TSH). Il TSH stimola la tiroide a secernere
  ormoni tiroidei;
\item
  \emph{fattore stimolante il rilascio di corticotropina} (CRH); stimola
  l'adenoipofisi a rilasciare corticotropina (ACTH). L'ACTH stimola la
  corticale del surrene a secernere altri ormoni;
\item
  \emph{fattore stimolante il rilascio dell'ormone della crescita}
  (GHRH); stimola l'adenoipofisi a secernere l'ormone della crescita
  (GH). Il GH regola la crescita e il metabolismo energetico, inoltre,
  agisce come ormone trofico stimolando il fegato a secernere fattori di
  crescita insulino-simili;
\item
  \emph{fattore inibente il rilascio dell'ormone della crescita} (GHIH)
  o \emph{somatostatina}; inibisce la secrezione adenoipofisaria
  dell'ormone della crescita;
\item
  \emph{fattore stimolante il rilascio di gonadotropine} (GnRH); stimola
  l'adenoipofisi a secernere sia l'ormone follicolo-stimolante (FSH) che
  l'ormone luteinizzante (LH). L'LH stimola l'ovulazione nelle femmine e
  la secrezione di ormoni sessuali da parte delle gonadi. L'FSH promuove
  lo sviluppo delle cellule uovo e degli spermatozio, ma stimola anche
  la secrezione di estrogeni nelle femmine e di \emph{inibina} in
  entrambi i sessi.
\end{enumerate}

(immagine 6.5 p153)

La produzione di fattori trofici ipotalamici ed ipofisari è regolata
mediante \textbf{circuiti a feedback}.

I fattori trofici prodotti dall'adenoipofisi possono agire tramite
circuiti a feedback negativo sull'ipotalamo per ridurre la loro stessa
secrezione.

L'inibizione della secrezione dei fattori trofici ipotalamici da parte
dei fattori trofici adenoipofisari è definita \textbf{feedback negativo
corto} e previene l'eccessiva sintesi di fattori trofici adenoipofisari.
Inoltre gli ormoni la cui secrezione venga stimolata da fattori trofici
in genere agiscono sull'ipotalamo per inibire la secrezione dei fattori
trofici, limitando così la loro stessa secrezione. Questo meccanismo è
definito \textbf{feedback negativo lungo}.

Il feedback negativo lungo inibisce la ghiandola, mentre quello corto
invece l'ipotalamo.

Nell'ipotalamo gli ormoni che producono il rilascio permettono il
rilascio di TSH. Questo nel sangue raggiunge la tiroide e induce la
produzione di ormoni tiroidei che inibiscono anche il rilascio degli
ormoni che producono il fattore di rilascio del TSH nell'ipotalamo e la
produzione di troppi ormoni tiroidei nella stessa tiroide.

Gli ormoni peptidici ipofisari seguono varie modificazioni per
raggiungere la forma finale ormonale.

Le cellule che producono ACTH producono in primis un peptide POMC che ha
vari punti di taglio per cui può venire scomposto in vari sottopeptidi
che possono essere ulteriormente scomposti in altri peptidi e da questo
si possono originare in tal modo tanti ormoni diversi se subiscono
modifiche post-traduzionali.

\subsection{L'epifisi}\label{lepifisi}

La ghiandola pineale si trova nell'encefalo ed è composta da tessuto
ghiandolare che secerne l'ormone \textbf{melatonina}, importante per la
regolazione del \textbf{ritmo circadiano} (un ritmo giornaliero che
sincronizza le attività dell'organismo rispetto al ciclo luce-buio).

Disturbi della produzione della melatonina producono disturbi
dell'alternanza sonno-veglia.

La secrezione di melatonina aumenta durante la notte e si riduce durante
il giorno.

La melatonina migliora la risposta immunitaria ed esercita un effetto
inibente sull'attività riproduttiva interferendo con l'attività di altri
ormoni.

La melatonina è secreta anche dalla retina e da altri tessuti.

La melatonina è un \emph{derivato del triptofano} così come il
neurotrasmettitore serotonina (la sua ammina) e dalla serotonina deriva
la melatonina per acetilazione.

La produzione di melatonina è regolata dalla retina attraverso la
percezione di stimoli luminosi recepiti da neuroni non coinvolti nella
formazione dell'immagine. Questi sono cellule gangliari retiniche
contenenti melanoxina che ha un picco di sensibilità nella luce blu
(quella crepuscolare) e che vanno a prendere contatto con i nuclei
soprachiasmatici situati sopra il chiasma ottico andando a stimolare le
cellule dell'epifisi.

\subsection{La tiroide}\label{la-tiroide}

La tiroide è una struttura a forma di farfalla che si trova sulla
superficie ventrale della trachea, più o meno all'altezza della laringe.

La ghiandola ha un'istologia tipica; vi è un parenchima connettivale e
una serie di corpi globoidi sferoidali immersi nel connettivo,
circondati dall'epitelio.

Questi corpi sono chiamati \textbf{follicoli tiroidei} e sono pieni di
materiale gelatinoso.

La tiroide secerne \textbf{tiroxina (T\(_4\))}, la
\textbf{triiodotironina (T\(_3\))} e la \textbf{calcitonina}.

Gli ormoni T\(_3\) e T\(_4\) regolano la velocità del metabolismo
corporeo e sono necessari per la crescita e il normale sviluppo
dell'organismo.

La calcitonina regola la concentrazione plasmatica di calcio.

La tiroide produce ormoni tiroidei derivando dalla tirosina (un aa
aromatico). Questi sono caratteristici perché contengono \emph{iodio}.
La tiroide funziona infatti come riserva di iodio. Lo iodio arriva dal
sangue e viene assimilato dal corpo con la dieta. Dentro i follicoli il
materiale gelatinose è chiamato \emph{colloide} e le cellule follicolari
accumulano iodio captandolo dal sangue tramite cotrasportatori
sodio-iodio (entrano insieme).

La tiroide contiene numerosi follicoli che producono gli ormoni
tiroidei. Ciascun follicolo è formato da un singolo strato di cellule
secretorie, definite \textbf{cellule follicolari}, che circondano la
colloide la quale è costituita da glicoproteine secrete dalle cellule
follicolari.

Negli spazi interstiziali fra i follicoli sono localizzate le cellule C,
che sintetizzano e secernono la calcitonina.

Le cellule tiroidee presentano la porzione baso-laterale rivolta verso
il connettivo e la porzione apicale rivolta verso il lume follicolare
pieno di colloide. Sul lato baso-laterale entra iodio per cotrasporto
con il sodio, mentre sul lato apicale vengono rilasciate molecole di
\textbf{tireoglobulina (TG)} (una proteina che agisce come precursore
degli ormoni tiroidei) nel colloide tramite fenomeni di esocitosi.

Nel colloide si trovano anche gli enzimi per la sintesi di T\(_3\) e
T\(_4\) e lo ione ioduro.

La tireoglobulina e gli enzimi sono sintetizzati nelle cellule
follicolari e secreti nella colloide per esocitosi, mentre lo ioduro è
trasportato attivamente ad opera delle cellule follicolari dal sangue
nella colloide.

I residui tirosinici della TG vengono iodati. L'aggiunta di uno ione
ioduro forma la \textbf{monoiodotirosina (MIT)}, mentre l'aggiunta di un
secondo ioduro allo stesso residuo tirosinico forma la
\textbf{diiodotirosina (DIT)}.

Due residui tirosinici iodinati vengono accoppiati sulla stessa molecola
di tireoglobulina e uniti mediante un legame covalente; se si uniscono
due molecole di DIT si avrà T\(_4\), mentre se si uniscono una molecola
di DIT e una di MIT si avrà T\(_3\).

Due molecole di MIT non possono unirsi tra loro.

L'ormone stimolante la tiroide (TSH), che arriva dal circolo sanguigno,
stimola il rilascio di ormoni tiroidei. Il TSH prima si lega ai
recettori sulla membrana delle cellule follicolari, attivando il secondo
messaggero cAMP che porta alla fosforilazione di una serie di proteine
delle cellule follicolari necessarie per la secrezione degli ormoni.

Le cellule follicolari assumono nella colloide, mediante fagocitosi, le
molecole di tireoglobulina iodata. Il fagosoma contenente la
tireoglobulina iodata si fonde con un lisosoma.

Gli enzimi lisosomali, demolendo le molecole di tireoglobulina,
determinano la liberazione di T\(_3\) e T\(_4\) nella cellula
follicolare. Poichè T\(_3\) e T\(_4\) sono molecole lipofile possono
passare in circolo diffondendo attraverso la membrana plasmatica.

Il T\(_4\) è prodotto e secreto più velocemente rispetto a T\(_3\), ma
T\(_3\) è più potente sugli organi bersaglio (per questo la maggio parte
di T\(_4\) viene convertito in T\(_3\)).

I livelli di ormoni tiroidei sono virtualmente sempre costanti in
condizioni normali, perchè il principale meccanismo di controllo della
loro secrezione è un feedback negativo. La secrezione dell'ormone
tiroideo è stimolata dal TSH proveniente dall'adenoipofisi. La
secrezione di TSH è a sua volta stimolata dal *fattore di rilascio di
\textbf{tireotropina (TRH)}, proveniente dall'ipotalamo.

Quando vengono rilasciati in circolo, gli ormoni tiroidei agiscono con
un feedback negativo sull'ipotalamo e sull'adenoipofisi per limitare la
secrezione di TRH e TSH.

Gli ormoni tiroidei sono liposolubili e reagiscono con recettori
specifici nel nucleo, dove innescano la trascrizione genica. Di solito
la trascrizione è repressa da un corepressore. Quando arriva l'ormone il
corepressore si toglie, si lega l'ormone e si forma un complesso di
trascrizione. La trascrizione è così attivata.

Il TSH messo in circolo va ad agire sulla tiroide dove accelera il
rilascio degli ormoni tiroide.

Viene stimolato soprattutto il rilascio di T\(_4\) che entra in circolo
e va a colpire molteplici organi bersaglio (fegato, muscolo, osso,
ecc\ldots{}). Sugli organi bersaglio agisce prevalentemente nella forma
convertita T\(_3\), poiché è una forma più attiva ossia con maggiore
affinità per il recettore.

Un'altro fattore importante nel sangue è il calcio. Nei vertebrati è
fondamentale perché lo scheletro dei vertebrati è composto da sali di
calcio e fosfato.

Il calcio è per lo più bloccato in un sale minerale in forma solida.
Solo una quota è in circolo ed è fondamentale nella buona salute delle
cellule dello scheletro oltre che in circolo nelle cellule.

Su questo movimento di notevoli quantità di calcio ci sono dei controlli
endocrini. Un ormone coinvolto in questi fenomeni è la calcitonina,
anch'essa prodotta dalla tiroide.

Nella tiroide si trovano cellule chiamate \textbf{parafollicolari}.

Tra la popolazione di cellule mesenchimatiche ci sono cellule secernenti
che producono un peptide. Il peptide è chiamto calcitonina e tende ad
abbassare i livelli di calcio ematico.

A carico della tiroide c'è anche questa secrezione. Si attua riducendo
il calcio ematico per mezzo di assorbimento del calcio dall'intestino,
inibendo l'attività di osteoclasti nell'osso, inibendo il riassorbimento
di fosfato nel rene e aumentando il riassorbimento di calcio e magnesio
nel rene.

La secrezione è stimolata da strumento di calcio nel plasma e dalla
gastrina.

\textbf{Lezione 26.11.2015}

Gli endocrinologi parlano di assi per controlli da sistema endocrino a
sistema endocrino.

Il punto di partenza è sempre l'ipotalamo.

Si parla di asse ipotalamo \(\rightarrow\) ipofisi \(\rightarrow\) altro
(ad esempio surrene).

L'ipotalamo produce \emph{CRH} (ormone di rilascio delle
corticotropine), un peptide in grado di agire sulle cellule
dell'adenoipofisi stimolandole a produrre \emph{ACTH} (ormone
adenocorticotropo) e quindi stimolando l'organismo a produrre
corticoidi.

In realtà il sistema ad assi non è corretto perché è meglio parlare di
cicli.

Si parla anche di asse ipotalamo \(\rightarrow\) ipofisi \(\rightarrow\)
tiroide.

L'ipotalamo secerne TRH che stimola l'ipofisi a produrre TSH che stimola
la tiroide a produrre ormoni tiroidei (T3, T4). Esistono anche altri
assi.

\subsubsection{Controllo del metabolismo del
calcio}\label{controllo-del-metabolismo-del-calcio}

I vertebrati sono organismi che manipolano grandi quantità di calcio (il
loro scheletro è formato da calcio).

Ne devono assumere molto e ne eliminano molto, ma ovviamente le quantità
in circolo devono rimanere stabili e non aumentare troppo.

Per questa ragione ci sono molti controlli sul suo metabolismo.

La \textbf{calcitonina} è un ormone prodotto dalla tiroide che tende ad
abbassare il livello del calcio nel sangue. Il sangue contiene il pool
di calcio in transito, ovvero quello che entra ed esce. Questo pool
contiene la riserva di calcio utilizzata per formare lo scheletro, e
deve avere un livello il più possibile costante.

La calcitonina interviene ripristinando il livello di calcio ematico.

La secrezione della calcitonina è stimolata dall'aumento del calcio
plasmatico e dalla gastrina. Dopo i pasti si ha un aumento del calcio
presente in circolo, e viene prodotta gastrina che stimola la secrexione
della calcitonina.

La gastrina è un ormone prodotto dallo stomaco presente in circolo nella
fase postprandiale, quando è in atto la digestione.

La calcitonina inibisce la demolizione del tessuto osseo ad opera degli
osteoclasti.

A dosi farmacologiche (più alte di quelle fisiologiche) la calcitonina
aumenta il riassorbimento di calcio e magnesio nel rene, mentre a dosi
fisiologiche inibisce il riassorbimento di calcio e fosfato (sempre nel
rene).

La calcitonina dunque agisce nelle fasi di eccessiva perdita di calcio
dallo scheletro e previene l'ipercalcemia postprandiale, ha effetto
dell'induzione del senso di sazietà (fa parte dell'insieme degli ormoni
postprandiali).

Ha meccanismo di azione legato a proteina G. il suo recettore è presente
in cellule del rene, degli osteoclasti e in cellule dell'intestino.

Sul lato posteriore della tiroide si trovano 4 piccole ghianfdole
chiamate \textbf{paratiroidi}.

Queste secernono l'\textbf{ormone paratiroideo (PTH)}, un importante
regolatore della concentrazione plasmatica di calcio.

Il PTH ha un effetto opposto rispetto alla calcitonina, e viene secreto
quando la concentrazione di calcio nel sangue si abbassa.

Questo è un ormone peptidico (è formato da una catena non lunga di aa)
che stimola gli osteoclasti a degradare il tessuto osseo per ottenerne
Ca. Induce un aumento nell'assorbimento di calcio nel rene e
nell'intestino agendo tramite la vitamina D.

Nel sangue il Ca ha una concentrazione intorno a 9-11 mg su 100 ml. Uno
scostamento da questi valori in aumento, stimola le cellule C della
tiroide a produrre calcitonina, la quale stimola la deposizione ossea
(inibisce la demolizione ossea), inibisce l'assorbimento intestinale
ecc\ldots{} in modo da riportare i livelli di Ca in equilibrio.

Se invece, c'è una riduzione del Ca nel sangue verrà stimolata la
produzione dell'ormone paratiroideo che stimolerà la degradazione ossea
di Ca riportando i valori ematici all'equilibrio. Questo effetto tenderà
ad attivare nuovamente la calcitonina che tenderà a stimolare la
deposizoone ossea, ecc\ldots{} Questo è un sistema ciclico.

Nel controllo del metabolismo del calcio interviene anche la vitamina D.
Questa per l'esattezza non è un'unica molecola ma una serie di molecole,
sostanze liposolubili cui la sostanza base è il \emph{calciferolo}.

La vitamina D in parte viene sintetizzata nel corpo e in parte viene
assunta con la dieta.

Il nostro organismo non è in grado di produrre tutta la vitamina D
necessaria, ma può modificare la vitamina D che viene assunta con la
dieta (abbiamo la vitamina D1, 2, 3, \ldots{})

La forma attiva della vitamina D è l'\textbf{1-25 diidrossicalciferolo}
o \textbf{calcitriolo}. Questa è una molecola complessa con unità
cicliche e un gruppo OH (-olo).

Il calcitriolo è una molecola liposolubile formata da carbonio e
idrogeno con un idrossile che da solo non riesce a conferire
idrofilicità a tutta la molecola.

La molecola realmente attiva, ossia il calcitriolo, aumenta i livelli
ematici di calcio tramite un'azione sinergica con l'ormone paratiroideo.

Ha un'azione in parte sovrapponile all'ormone paratiroideo e in parte
alla calcitonina: permette l'aumento dei livello di calcio nel sangue,
favorisce la mineralizzazione dell'osso, promuove l'assorbimento
nell'intestino e nel rene.

È un ormone fondamentale per lo sviluppo e il rimodellamento dello
scheletro; una sua carenza favorisce il rachitismo e l'osteoporosi. La
luce solare è essenziale per un passaggio di formazione del calcitriolo.

Il calcitriolo viene prodotto nell'organismo a partire da sostanze
assunte con la dieta.

Il punto di partenza è uno sterolo, una molecola affine agli steroidi
chiamata \textbf{7-deidrocolesterolo}, che nella pelle si trasforma,
mediante una reazione causata dalla luce UV, in un composto che è una
\textbf{pre-vitamina D3}. La pre-vitamina deriva dall'apertura di un
anello dello steroide e da una redistribuzione dei doppi legami.

La pelle è la principale parte del corpo esposta alla luce, per questo
la reazione avviene in questa zona. La radizione UV non è dannosa se non
vi si espone troppo a lungo, anzi è essenziale.

La pre-vitamina D3 va incontro a riorganizzazione e forma
\textbf{colecalciferolo} (sempre un precursore della vitamina D3).

\textbf{CERCARE sintesi vitamina D3}

La vitamina D3 può essere anche presente nella dieta e venire modificata
nel fegato dove viene trasformata in \textbf{calcidiolo}.

Il fegato compie processi di metabolizzazione di diverse sostanze
organiche. Il calcidiolo accumulato nel fegato e poi rilasciato nel
sangue, viene trasformato nel rene dall'enzima 1\(\alpha\)-idrossilasi
(attivato da PTH) in calcitriolo o calciferolo (forma attiva).

Vengono inseriti gruppi ossidrilici (ossidazioni svolte da ossigenasi ad
azione mista).

Dal carbonio 1 al carbonio 3 al carbonio 25 si ha ossidazione
progressiva.

Quando il calcitriolo torna nel sangue svolge una funzione ormonale.

Nel sangue non viaggia come molecola in soluzione libera ma essendo
sostanza lipofila viaggia nel sangue associata a una proteina, in
particolare alla \textbf{VDBP} \emph{(Vitamin D Binding Protein)}, in
grado di rilasciare il calcitriolo che entra nelle cellule ed agisce
tramite un recettore intracellulare.

Il calcitriolo entrato nella cellula si lega al \textbf{VDR}
\emph{(vitamin D receptor)} presente nel nucleo.

VDR è un fattore di trascrizione per un canale del calcio
(\textbf{TRPV6}, un vanilloide) e \textbf{calbindina} (una molecola che
lega calcio).

La calbindina è un buffer del calcio in quanto lega calcio nel citosol
aumentando la quantità di calcio all'interno della cellula, ma non
aumentandone la concentrazione citosolica.

A livello instinale la vitamina D stimola l'assunzione di calcio da
parte degli enterociti che, man mano che lo accumulano, lo espellono.

Gli enterociti però, essendo cellule epiteliali \textbf{(?
CONTROLALRE)}, hanno sempre l'inidirizzamento differenziato alla
superficie delle proteine di membrana che esprimono. Mandano le proteine
o sulla membrana apicale o su quella basolaterale in quanto hanno
indicatori che indirizzano la proteina nella parte giusta.

I trasportatori attivi del calcio si trovano nella zona basolaterale
mentre il canale del calcio andrà in posizione apicale. È sempre così
sia per il sodio che per il calcio.

Il calcio entra tramite canale, per gradiente, ed esce tramite trasporto
attivo.

Quando la cellula esprime maggiormente il canale in superficie,
automaticamente libera più calcio nella porzione basolaterale.

Così il livello citosolico del calcio rimane sempre basso. Per questo
viene aumentato il tamponamento del calcio citoplasmatico e per questo
la cellula esprime la calbindina che lega calcio nel citosol.

Il risultato per l'organismo è aumentare l'assorbimento di calcio dal
lume intestinale al sangue.

\subsection{Le ghiandole surrenali}\label{le-ghiandole-surrenali}

Le ghiandole surrenali sono due piccole formazioni ghiandolari
posizionate sulla porzione superiore di ciascun rene.

I reni sono organi a forma di fagiolo incapsulati in una guaina
connettivale e hanno superficie ben definita dove si addensa una massa
di tessuto ghiandolare. Queste ghiandole formano una sorta di cappuccio
in posizione anteriore.

Ciascuna ghiandola surrenale è formata da uno strato esterno definito
\textbf{corticale} (80\% di tutta la massa ghiandolare) e da uno strato
interno definito \textbf{midollare}. Le porzioni corticale e midollare
sono anatomicamente e funzionalmente distinte.

La \emph{zona corticale} è composta da 3 strati cellulari distinti:

\begin{enumerate}
\def\labelenumi{\arabic{enumi}.}
\itemsep1pt\parskip0pt\parsep0pt
\item
  uno strato più esterno definito \textbf{zona glomerulare} (strato di
  cellule organizzate a glomeruli);
\item
  uno strato intermedio definito \textbf{zona fascicolata} (cordoni
  paralleli);
\item
  uno strato interno definito \textbf{zona reticolare} (dove i cordoni
  assumono aspetto più reticolato).
\end{enumerate}

Le cellule nei 3 strati possiedono differenti gruppi di enzimi coinvolti
nella sintesi ormonale, producono ormoni differenti e in diverse
proporzioni.

La corticale del surrene secerne un gruppo di ormoni, definiti
\textbf{adrenocorticoidi}, ormoni di natura steroidea (derivati del
colesterolo).

Gli \emph{adrenocorticoidi} comprendono 3 tipi di ormoni:

\begin{enumerate}
\def\labelenumi{\arabic{enumi}.}
\itemsep1pt\parskip0pt\parsep0pt
\item
  \textbf{mineralcorticoidi} (principalmente \emph{aldosterone}),
  secreti solo nella zona glomerulare, che regolano il riassorbimento di
  Na e l'escrezione di K nei reni;
\item
  \textbf{glucocorticoidi} (principalmente \emph{cortisolo}), secreti
  dalla zona fascicolata e reticolare, che regolano le risposte
  dell'organismo allo stress, il metabolismo proteico, glucidico e
  lipidico in diversi tessuti e i livelli plasmatici di glucosio;
\item
  \textbf{ormoni sessuali} (principalmente \emph{androgeni}), secreti
  dalla zona fascicolata e reticolare (oltre che dalle gonadi) e
  regolano la funzione riproduttiva e altri processi.
\end{enumerate}

Il punto di partenza per la sintesi di questi ormoni è il colesterolo.
Il colesterolo viene trasformato in pregnenolone da cui derivano gli
ormoni per scomparsa della catena lineare del cortisone.

Le reazioni che avvengono sul reticolo endoplasmatico liscio o sulla
catena di trasporto di elettroni del reticolo endoplasmatico liscio
oppure ancora nel mitocondrio.

Essendo ormoni lipofili abbiamo recettori intracellulari. Sono stati
descritti recettori degli estrogeni, del progesterone e degli androgeni.

La zona midollare del surrene non ha nulla a che vedere con la regione
corticale se non per il fatto che sono localizzate l'una all'interno
dell'altra.

La \emph{midollare del surrene} contiene \textbf{cellule cromaffini} che
secernono \textbf{catecolamine} (così come i neuroni post gangliari del
sistema nervoso simpatico), di cui circa l'80\% è rappresentato
dall'\emph{epinefrina} (adrenalina), il 20\% \emph{norepinefrina}
(noradrenalina) e neno dell'1\%* dopamina.

La secrezione è controllata dai neuroni pregangliari del sistema
simpatico.

\subsection{Il pancreas}\label{il-pancreas}

Il pancreas svolge funzioni sia endocrine che esocrine.

Il \emph{pancreas esocrino} è costituito da cellule degli acini e
cellule dei dotti, che secernono enzimi e liquidi all'interno del canale
digerente; il \emph{pancreas endocrino} è formato da gruppi di cellule
definiti \emph{isolotti di Langerhans}, distribuiti in tutto il pancreas
negli spazi attorno ai dotti.

Negli isolotti di Langerhans vengono prodotti i due principali ormoni
pancreatici, ognuno die quali viene secreto da uno specifico tipo di
cellula: l'\textbf{insulina}, secreta dalle \emph{cellule beta}, e il
\textbf{glucagone}, secreto dalle \emph{cellule alfa}.

Entrambi questi ormoni sono importanti per la regolazione del
metabolismo energetico e della concentrazione plasmatica di glucosio.

Negli isolotti di Langerhans si trovano anche due tipi di cellule:

\begin{itemize}
\itemsep1pt\parskip0pt\parsep0pt
\item
  le \emph{cellule delta} secernenti la \textbf{somatostatina}, che
  contribuisce alla regolazione della digestione e dell'assorbimento di
  nutrienti e può regolare la secrezione di altri ormoni pancreatici (la
  somatostatina è anche il fattore trofico ipotalamico inibente la
  secrezione dell'ormone della crescita da parte dell'adenoipofisi);
\item
  le \emph{cellule F} secernono il \textbf{polipeptide pancreatico}
  (funzione non ancora nota).
\end{itemize}

\textbf{lezione 30.11.2015}

\subsection{Interazioni ormonali}\label{interazioni-ormonali}

Gli ormoni e le sostanze che si prendono in considerazione in
endocrinologia hanno interazioni e possiamo avere antagonismo quando una
sostanza si oppone agli effetti di un'altra con effetti vari. Ad
esempio, la calcitonina e l'ormone paratiroideo producono sul corpo
effetti antitetici: la calcitonina abbassa la concentrazione ematica del
calcio mentre l'ormone paratiroideo la innalza.

Un'altra possibilità è la sommazione degli effetti: due o più sostanze
concorrono alla produzione dello stesso effetto, eventualmente
potenziato.

Possiamo avere due tipologie per questa modalità di azione:

\begin{itemize}
\itemsep1pt\parskip0pt\parsep0pt
\item
  sinergia;
\item
  additività.
\end{itemize}

Si parla di \textbf{additività} quando l'effetto globale
dell'interazione di due o più sostanze è dato dalla somma degli effetti
delle singole sostanze.

Nella \textbf{sinergia}, invece, l'effetto totale è maggiore della somma
dei singoli effetti.

Un approccio utile è quello di valutare i meccanismi dose-risposta. La
regolazione tra dose e risposta è descritta da una \emph{funzione
logistica} (con il logaritmo della dose in \textbf{x} e la risposta in
\textbf{y}).

La funzione tende asintoticamente al 100\% della risposta man mano che
si aumenta la dose.

Il modo in cui aumenta la riposta (o diminuisce, dipende dai casi) può
essere descritto adeguatamente da una funzione logistica di tipo
esponenziale come questa:

\textbf{Y = Ke\^{}(h)}

La concentrazione efficace per dare il 50\% di risposta è detta
\textbf{EC50}.

Se abbiamo due sostanze ciascuna avrà la sua curva di dose-risposta e
ciascuna avrà un suo EC50 sull'asse x.

L'EC50 è proprio una dose (mg di sostanza per kg ecc..), se le due
sostanze sono additive su una certe risposta, significa che, se vengono
miscelate usando metà dose di una e metà dose dell'altra, l'EC50 trovato
sarà la somma degli EC50 singoli.

L'\textbf{additività} si scrive come:

\textbf{EC50\(_c\) = EC50\(_a\)/2 + EC50\(_b\)/2}

Se le sostanze invece sono sinergiche, la loro miscelazione dà una
risposta più forte della semplice additività. La curva dose-risposta che
si ottiene dalla combinazione avrà un EC50 spostato a sinistra rispetto
a quello atteso dall'additività.

L'EC50 totale (della sinergia) sarà minore della somma di EC50\(_a\)/2 +
EC50\(_b\)/2 perchè le due sostanze insieme si potenziano.

Si parla di \textbf{permissività} quando un ormone rende possibile
l'azione di un altro ormone.

Un esempio è quello degli ormoni tiroidei che inducono l'espressione di
recettori beta-adrenergici su cellule del musoclo liscio dei bronchioli
polmonari che rispondono all'adrenalina con la bronco-dilatazione.

\subsection{Il metabolismo del
glicogeno}\label{il-metabolismo-del-glicogeno}

Il glicogeno è un omopolimero del glucosio che funge da riserva
energetica glucidica per l'uomo. Esso è prevalentemente depositato nel
fegato e nel muscolo scheletrico.

Il glicogeno è un polisaccaride ramificato che presenta unità di
glucosio legate per lo più tramite legami \(\alpha\) (1-4), ma anche
\(\alpha\) (1-6).

Il glicogeno può essere sia sintetizzato che degradato dalle cellule (ma
non contemporaneamente). Se le via di glicogeno sintesi e glicogeno lisi
fossere attive contemporaneamente nella cellula si creerebbe un ciclo
futile con consumo di una molecola di ATP per ciclo.

La regolazione del metabolismo del glicogeno comporta sia un
\emph{controllo allosterico} che un controllo ormonale che si manifesta
con una \emph{regolazione covalente} degli enzimi regolatori della via
metabolica.

Esistono due ormoni principali per il controllo della concentrazione di
glucosio ematico:

\begin{itemize}
\itemsep1pt\parskip0pt\parsep0pt
\item
  l'\textbf{insulina}, per quando il glucosio ematico è alto;
\item
  il \textbf{glucagone}, per quando il livello di glucosio ematico è
  basso.
\end{itemize}

L'insulina è prodotta dalle cellule beta delle isole di Langerhans
all'interno del pancreas, mentre il glucagone è prodotto dalle cellule
alfa delle isole di Langherans, entrambi questi ormoni agiscono sulle
cellule del fegato e su quelle muscolari.

Questi ormoni sono stati molto studiati a causa di molte malattie
provocate da un mal funzonamento legato alla loro produzione o funzione
(ad esempio il diabete).

I principali enzimi coinvolti nella degradazione e nella sintesi di
glicogeno sono:

\begin{itemize}
\itemsep1pt\parskip0pt\parsep0pt
\item
  la \textbf{glicogeno fosforilasi};
\item
  la \textbf{glicogeno sintetasi}.
\end{itemize}

Questi enzimi rispondono in maniera opposta agli stessi segnali e sono
sotto il controllo allosterico di: ATP, G6P e AMP.

La \textbf{glicogeno fosforilasi} muscolare è attivata da AMP e inibita
da ATP e G6P, mentre la \textbf{glicogeno sintasi} è attivata dal G6P.

Un'elevata rischiesta di ATP significa un abbassamento del livello di
ATP e Gluc6P, ma un aumento di AMP. Questo significa attivazione della
glicogeno fosforilasi con conseguente demolizione di glicogeno, e
inibizione della glicogeno sintetasi.

Al contrario, elavati livelli di ATP e Gluc6P significano un'attivazione
della glicogeno sintasi con conseguente sintesi di glicogeno e
un'inibizione della glicogeno fosforilasi.

Le forme attive e inattive dei due enzimi differiscono fra loro per lo
stato di fosforilazione (regolazione covalente):

\begin{itemize}
\itemsep1pt\parskip0pt\parsep0pt
\item
  la \emph{glicogeno fosforilasi} è \textbf{attiva} se fosforilata;
\item
  la \emph{glicogeno sintetasi} è \textbf{inattiva} se fosforilata.
\end{itemize}

L'interconversione delle forse fosforilate e defosforilate dei due
enzimi è regolata da un sistema di reazioni enzimatiche a cascata
controllare per via ormonale.

Gli ormoni \textbf{glucagone} e \textbf{adrenalina} attivano i recettori
associati alle G-protein innescando una \emph{cascata di cAMP}.

Il glucagone stimola la cellula a degradare glicogeno producendo così
glucosio. Anche l'adrenalina produce questo effetto.

I mediatori cellulari di questo effetto sono l'\emph{AMPciclico} e la
\emph{PKA} (protein chinasi A). Il cAMP attiva un enzima, la
\emph{proteina chinasi-cAMP dipendente}, che a sua volta fosforila la
glicogeno sintasi, inattivandola. La proteina chinasi cAMP-dipendente
inoltre, fosforila la \emph{fosforilasi chinasi} attivandola, la quale
una volta attiva fosforial la glicogeno fosforilasi attivandola.

La glicogeno fosforilasi agisce fosforilando il glicogeno staccando da
esso molecole di \emph{glucosio 1-fosfato}.

Questi ormoni vengono prodotti in caso di stress, fame, digiuno,
situazioni in cui l'organismo ha bisogno di utilizzare le riserve.

La cascata fosforilativa oltre ad attivare la demolizione del glicogeno,
inibisce la sua sintesi mediante fosforilazione da parte della PKA che
produce una cascata di fosforilazioni che arriva alla glicogeno sintasi.
In questo modo l'enzima \emph{glicogeno sintasi} viene inattivato
mediante fosforilazione (al contrario dalla \emph{glicogeno fosforilasi}
che si attiva per fosforilazione).

Per la sintesi di glicogeno invece si avrà attivazione della sintasi e
inattivazione della fosforilasi. Questo è causato dall'insulina che
agisce da una \emph{macchinasi} (MAPK) attivando una proteina fosfatasi.

La fosfatasi ha come compito la defosforilazione delle proteine e quindi
ha un effetto inverso rispetto alla PKA.

L'insulina produce attivazione della protein fosfatasi 1 che con una
cascata di fosforilazioni defosforila la glicogeno fosforilasi
inibendola, e defosforila la glicogeno sintasi attivandola.

La proteina fosfatasi 1 è attivata dall'insulina ma può essere anche
inattivata. L'AMP ciclico attiva una chinasi AMP ciclico dipendente che
fosforila un inibitore della fosfatasi inibendo la fosfatasi stessa.

I livelli di AMP ciclico salgono quando la cellula epatica è sotto
l'azione dell'adrenalina o del glucagone e si abbassano quando entra in
gioco l'insulina. Quando arriva l'insulina ad attivarla dunque l'AMP
ciclico è sceso.

Le vie dell'insulina e del glucagone si intersecano e di volta in vola
prevale l'una o prevale l'altra.

L'insulina agisce medianti recettori protein-chinasi che si
autofosforilano quando legano il ligando. Possono poi fosforilare il
IRS1 un altro substrato che attiva le macchinasi. In fondo alla cascata
c'è ina S6chinasi che fosforila una proteina fosfatasi che è appunto la
proteina fosfatasi 1 attivandola. questa poi svolge il compito visto
prima. \textbf{(controllare)}

L'insulina induce un sacco di effetti nella cellula.

Un'altra via di trasduzione rilevante è l'inibizione di una glicogeno
sintasi chinasi. Sulla glicogeno sintasi agiscono vari regolatori. Una è
la \emph{glicogeno sintasi chinasi 3}. Questa, insiema alla PKA, la
inibisce.

Un altro effetto dell'insulina sulla cellula è l'attivazione del
trasporto di glucosio glu4-dipendente che fa entrare glucosio nella
cellula.

L'insulina stimola anche la sintesi proteica. Mentre il glucagone induce
il consumo di proteine.

La glicogeno sintasi va attivata quando agisce l'insulina e si vedono
tre vie:

\begin{enumerate}
\def\labelenumi{\arabic{enumi}.}
\itemsep1pt\parskip0pt\parsep0pt
\item
  attivazione della \emph{protein fosfatasi 1} e inibizione della
  \emph{glicogeno sintasi chinasi 3} e della \emph{PKA}. Se inibisco gli
  inibitori di conseguenza attivo la proteina colpita da inibizione.
  L'azione dell'insulina sulla \emph{fosfatidilinositolo-3-chinasi}
  attiva la \emph{protein chinasi B}. L'insulina agisce sempre tramite
  IRS1 che va a portare avanti la cascata delle macchinasi e recluta la
  \emph{PI3-K} (enzima fosfoinositide 3 chinasi) che ha un dominio
  \emph{SH\(_2\)} che si va a legare al fattore fosforilato.
\end{enumerate}

In questo modo il dominio SH\(_2\) si va a legare alla prima proteina
fosforilata reclutata sul recettore. La PI3-K fosforila
inositolodisfosfato e si produce un \emph{fosfatidilinositolo} che
recluta la \emph{protein chinasi B} attivandola.

La PKB attivata inattiva la glicogeno sintasi chinasi tramite
fosforilazione.

Lo stesso glucosio promuove l'attivazione della glicogeno sintasi
attivando anch'esso la proteina fosfatasi 1 che viene bloccata dalle
glicogeno fosforilasi attiva.

Si forma un complesso tra le due.

Il segnale innescato dall'insulina viene spento da delle proteine
tirosin fosfatasi. Il recettore dell'insulina è di tipo tirosina
chinasi. Il meccanismo dunque è spento da proteine tirosin fosfatasi che
defosforilano la tirosina. L'attivazione del segnale dell'insulina è
infatti regolata da attivazione mediante fosforilazione. La
defosforilazione è di conseguenza una inattivazione.

(come riassunto delle varie attività vedere schema utile che ha caricato
su internet)

\subsection{Le gonadi}\label{le-gonadi}

Le gonadi sono degli importanti organi endocrini (secernono
abbondantemente).

Nei maschio vengono prodotti \textbf{androgeni} (\emph{testosterone} e
\emph{androsterone}), mentre nelle femmine vengono prodotti
\emph{estradiolo} e \emph{progesterone}.

Tra le gonadi femmini e maschili non vi è omologia poichè essi insorgono
durante lo sviluppo da sistemi diversi.

In gravidanza anche la placente diventa un importante organo endocrino
(produce anch'essa estrogeni e progesterone).

Organi endocrini secondari (secernono meno) sono:

\begin{itemize}
\itemsep1pt\parskip0pt\parsep0pt
\item
  il cuore, che secerne il \emph{peptide natriuretico atriale} (ANP);
\item
  il rene, che secerne \emph{eritropoietina} (EPO, prodotta
  fisiologicamente per la stimolazione dell'eritropoiesi),
  \emph{renina}, \emph{calcitriolo}\ldots{};
\item
  il fegato, che secerne fattori di crescita insulino-simili;
\item
  lo stomaco, che secerne gastrina;
\item
  intestino tenue, che secerne \emph{colecistochinina}, \emph{secretina}
  e \emph{GIP}.
\end{itemize}

Per gli ormoni e per i farmaci è importante il tempo in cui la molecola
permane nel sangue. I tempi sono analizzati dal punto di vista
quantitativo mediante il tempo di dimezzamento. Tempo che intercorre tra
il picco ella sostanza nel sangue ed il tempo che impiega a ridursi del
50\%. La sostanza difficilmente scopare del tutto dal sangue perché
viene riprodotta e quindi la sua scomparsa totale non è un parametro
interessante per fare confronti. Alcune sostanze rallentano la
biodisponibilità dell'ormone rallentando anche i processi degradativi
che il corpo mette in atto. Fondamentale la presenza di proteine
trasportatrici per regolare la mancanza di sostanze liposolubili.
\textbf{(registrazione)}

\end{document}
