\documentclass[]{article}
\usepackage{lmodern}
\usepackage{amssymb,amsmath}
\usepackage{ifxetex,ifluatex}
\usepackage{fixltx2e} % provides \textsubscript
\ifnum 0\ifxetex 1\fi\ifluatex 1\fi=0 % if pdftex
  \usepackage[T1]{fontenc}
  \usepackage[utf8]{inputenc}
\else % if luatex or xelatex
  \ifxetex
    \usepackage{mathspec}
    \usepackage{xltxtra,xunicode}
  \else
    \usepackage{fontspec}
  \fi
  \defaultfontfeatures{Mapping=tex-text,Scale=MatchLowercase}
  \newcommand{\euro}{€}
\fi
% use upquote if available, for straight quotes in verbatim environments
\IfFileExists{upquote.sty}{\usepackage{upquote}}{}
% use microtype if available
\IfFileExists{microtype.sty}{\usepackage{microtype}}{}
\ifxetex
  \usepackage[setpagesize=false, % page size defined by xetex
              unicode=false, % unicode breaks when used with xetex
              xetex]{hyperref}
\else
  \usepackage[unicode=true]{hyperref}
\fi
\hypersetup{breaklinks=true,
            bookmarks=true,
            pdfauthor={},
            pdftitle={},
            colorlinks=true,
            citecolor=blue,
            urlcolor=blue,
            linkcolor=magenta,
            pdfborder={0 0 0}}
\urlstyle{same}  % don't use monospace font for urls
\setlength{\parindent}{0pt}
\setlength{\parskip}{6pt plus 2pt minus 1pt}
\setlength{\emergencystretch}{3em}  % prevent overfull lines
\setcounter{secnumdepth}{0}


\begin{document}

\section{Fisiologia generale}\label{fisiologia-generale}

5/10/2015\_01

\subsubsection{Che cos'è la
fisiologia?}\label{che-cosuxe8-la-fisiologia}

E' una disciplina che studia il funzionamento degli esseri viventi. E'
la definizione meccanicistica di ciò che accade nell'organismo di ogni
persona. Questa disciplina ti fa capire che non sei quel che pensi di
essere.

La composizione chimica delle materia vivente: H, O, C, N. La
composizione chimica della materia vivente è molto simile a quelle
dell'universo e delle stelle, piuttosto che a quella della terra e della
crosta terrestre.

Tra la fine del '700 e l'inizio del '800, dopo la diffusione del
microscopio, ci si rese conto che gli esseri viventi sono organizzati in
cellule. La materia vivente ha una sua unitarietà molto solida, ovvero
tutte le cellule contengono all'incirca gli stessi organelli e nelle
stesse quantità. L'acqua è sempre la sostanza più abbondante. Le
proteine sono la classe di composti maggiormente diversificata
nell'organismo.

\subsection{L'omeostasi}\label{lomeostasi}

Cosa significa omeostasi? Significa tendenza a mantenere costante il
sistema. L'omeostasi è un insieme di processi che mantengono costanti le
condizioni corporee (es: costanza nella composizione chimica del
plasma). Queste condizioni devono mantenersi al variare delle condizioni
esterne. L'omeostasi è garantita da meccanismi autoregolatori. Esistono
dei meccanismi di retroazione che viene detto di feedback.

Supponiamo ci siano due fattori che agiscono a vicenda con un feedback
negativo (al crescere di uno l'altro diminuisce). Un fattore agirà come
y= sen(t) e l'altro x= cos(t), dove t è il tempo (in grafico i valori
oscillano in maniera sfalsata).

Se con gli stessi fattori costruisco una funzione y=f(x) otterrò un
cerchio. Spostandosi nel tempo vengono ripercorsi gli stessi punti. La
distanza di questa funzione dall'origine degli assi è sempre 1.

Due fattori che si influenzano in modo da variano oscillando in maniera
sfalsata danno luogo a una serie di situazioni che si ripetono
ciclicamente passando sempre dagli stessi punti.

\subsubsection{Il principio fisiologico
fondamentale}\label{il-principio-fisiologico-fondamentale}

Cos'è l'agente biologico funzionale? Un qualsiasi agente biologico
ereditario in grado di operare una trasformazione fisica (es. enzima).
Per ogni agente biologico funzionale esiste almeno un agente biologico
funzionale che lo controlla a monte ed esiste almeno un agente biologico
funzionale che viene controllato da questo a valle. Il numero degli
agenti biologici funzionali presenti in una cellula è finito. Questo
significa che per ogni a.b.f esiste almeno un loop (ciclo). ``L'ultimo''
a.b.f. agirà sul primo a.b.f andando a chiudere il ciclo.

In questo discorso bisogna tenere conto anche dell'invecchiamento
dell'organismo. L'omeostasi varia dunque sul lungo termine. Esistono
anche altre trasformazioni come, per esempio, l'adattamento. L'organismo
può variare il proprio stato funzionale passando da uno stato di
omeostasi ad un nuovo punto di equilibrio. Un esempio è la preparazione
atletica: l'organismo che viene sottoposto ad esercizio fisico subisce
uno stress e reagisce allo stress con un rinforzo. Questo può progredire
mediante adattamento progressivo. Avvengono fisicamente trasformazioni
nell'organismo. Sono il raggiungimento progressivo di nuovi punti di
omeostasi.

Lo schema prima presentato è dunque plastico: non perde mai le proprie
caratteristiche di base, ma può essere modulato in risposta a
sollecitazioni. Questi cicli possono avere dei punti di rottura, se
eccessivamente sollecitati, che possono essere più' o meno mortali.

05/10/2015\_02 \#\# La membrana cellulare Nella materia vivente ci sono
due ambienti chimici completamenti diversi e agli antipodi: * l'ambiente
\emph{idrofilo}, la sostanza fondamentale è l'acqua e contiene composti
idrosolubili; * l'ambiente \emph{idrofobo} o \emph{lipofilo}, la
sostanza fondamentale è costituita da lipidi o grassi.

Questi due ambienti non possono compenetrarsi: perché? Sulle molecole si
trovano delle cariche elettriche che possono essere positive o negative.
Una molecola completa nella sua configurazione, se osservata
globalmente, è elettricamente neutra. Esistono sostanze in cui la
distribuzione di queste cariche può non essere omogenea. Questo può
originare molecole polari in cui la distribuzione degli elettroni lungo
gli orbitali di legame non è omogenea causando la formazione di poli
elettrici positivi e negativi. Questo è quello che succede nell'acqua.
Se una molecole polarizzata viene immersa in una soluzione d'acqua,
questa molecola viene circondata dalle molecole d'acqua il cui polo
negativo è attratto dal polo positivo del soluto, ed il cui polo
negativo è attratto dal polo positivo del soluto. Questo è ciò che
avviene quando mettiamo un cucchiaino di zucchero in un bicchiere
d'acqua: le molecole d'acqua ``strappano'' molecole polari di saccarosio
al cristallo di zucchero.

Se invece la molecola che mettiamo nel bicchiere d'acqua non è polare,
le molecole d'acqua sono attratte maggiormente dalle altre molecole
d'acqua che non dalla molecola apolare (es. olio) poiché non presenta
poli carichi elettricamente.

Negli esseri viventi la contemporanea presenza di molecole polari e
apolari è costante e viene applicata una massimizzazione
dell'interfaccia tra le due molecole. Questo avviene anche quando
laviamo ad esempio un vestito con il detersivo. I detersivi contengono
tensioattivi, ovvero molecole che presentano un estremità polare e una
non polare così da poter reagire sia con l'acqua che con lo sporco.

Anche nell'organismo esistono sostanze tensioattive che formano le
membrane. A questo gruppo appartengono i fosfolipidi: molecole che
presentano due catene di acidi grassi (coda idrofoba) esterificate su
una molecola di glicerolo al quale è esterificato anche un gruppo
fosfato (testa idrofila).

Come si dispongono i fosfolipidi in acqua? Se buttiamo dell'olio in
acqua si formano delle micelle. I fosfolipidi possono però anche
organizzarsi in un doppio strato che risulta stabile in acqua andando a
formare una membrana. Le membrane fosfolipidiche tenderanno a chiudersi
andando a formare delle vescicole poiché altrimenti, se le membrane
fossero piatte come dei fogli, l'interno dei bordi (che è idrofobico)
rimarrebbe a contatto con l'ambiente acquoso circostante.

Le membrane cellulari formano dunque un piccolo ambiente lipidico
all'interno del quale troviamo anche altre molecole. Una delle più
importanti è il colesterolo (composto steroideo fortemente apolare). A
cosa serve? La membrana ha una natura fluida.

Cos'è che permette ad un composto grasso d'essere liquido o solido? La
presenza di legami doppi (insaturo, liquido) o singoli (saturo, solido).
Il doppio legame introduce un angolo di piegatura diverso dagli altri
che fa sì che l'attrazione dovuta alle forze di VdW tra le molecole sia
ridotta rispetto alla forza presente tra molecole lineari (solo legami
singoli). Una forte attrazione dovuta alle presenze di VdW dovuta alla
prevalenze di legami singoli darà origine a sostanze solide come il
burro.

Il colesterolo è necessario per regolare la fluidità della membrana. Se
la T è intorno ai 37° il colesterolo impedisce la mobilità dei
fosfolipidi, diminuendo la fluidità della membrana. A T più basse
invece, il colesterolo impedisce l'impaccamento dei fosfolipidi che
diminuirebbe eccessivamente la fluidità della membrana (???)

I fosfolipidi sono mobili nella membrana, possono ruotare lungo i loro
assi e possono scorrere lateralmente.

Esperimento: vennero fatte fondere cellule umane e cellule di topo. Dopo
la fusione vennero marcate e si vide che si erano distribuite lungo
tutta la cellula fuso. Ovvero le membrane non erano rimaste separate ma
si erano fuse e dunque i materiali delle due membrane si erano
distribuiti lungo tutta la nuova membrana dimostrando la loro fluidità.

\subsubsection{Gli sfingolipidi}\label{gli-sfingolipidi}

I fosfolipidi sono i più abbondanti, ma non sono gli unici. I primi
contengono glecerolo, mentre gli sfingolipidi contengono un amminoalcol
(sfingosina con testa polare e coda apolare) al posto del glicerolo e un
acido grasso collegato con la parte amminica. A volte alla sfingolisi
può essere lengato un ulteriore gruppo -R. Se non è presente si parla di
ceramide.

Gli sfingolipidi sono maggiormente presenti in determinate zone della
membrana dette zattere lipidiche (zone più spesse perché le catene degli
sfingolipidi sono un po' più lunghe di quelle dei fosfolipidi) e
contengono proteine che non sono presenti nelle altre zone.

La membrana esterna forma una barriera tra l'ambiente intra- ed extra-
cellulare. Le membrane permettono il trasporto selettivo di nutrienti,
prodotti di rifiuto e metaboliti tra i vari compartimenti cellulari. Le
membrane servono a formare vescicole per catturare e secernere
macromolecole e altre particelle. (\ldots{})

Nelle membrane sono presenti molte proteine che possono essere integrali
(sono incorporate nella membrana mediante domini idrofobici) o di
superficie (sono associate o agganciate alla membrana ma non hanno parti
idrofobiche immerse nello strato fosfolipidico.

Le proteine di membrana possono avere un ruolo strutturale
(citoscheletrico), possono servire per attaccare la cellula ad un
substrato (strutture di adesione cellulare), possono far attaccare le
cellule l'una all'altra, possono essere dei trasportatori (trasporto di
membrana), possono essere recettori (capaci di legare molecole segnale
che inducono processi interni alla cellula), possono essere
enzimi\ldots{}

Le proteine di membrana se la attraversano, possiedono una parte
extracellulare che molto spesso porta legate delle molecole zuccherine
lineari o ramificate (proteine glicosilate). Questo fa si che la
superficie cellulare sia ricoperta di zuccheri che sono molecole molto
idrofiliche che rendono lo strato acquoso subito presente esternamente
alla cellula molto denso (=glicocalice). Il glicocalice è un ulteriore
strato che si oppone ad un contatto troppo diretto della cellula con il
mondo esterno (protezione). Il glicocalice rappresenta anche un fattore
di riconoscimento per le cellule.

\end{document}
