\documentclass[]{article}
\usepackage{lmodern}
\usepackage{amssymb,amsmath}
\usepackage{ifxetex,ifluatex}
\usepackage{fixltx2e} % provides \textsubscript
\ifnum 0\ifxetex 1\fi\ifluatex 1\fi=0 % if pdftex
  \usepackage[T1]{fontenc}
  \usepackage[utf8]{inputenc}
\else % if luatex or xelatex
  \ifxetex
    \usepackage{mathspec}
    \usepackage{xltxtra,xunicode}
  \else
    \usepackage{fontspec}
  \fi
  \defaultfontfeatures{Mapping=tex-text,Scale=MatchLowercase}
  \newcommand{\euro}{€}
\fi
% use upquote if available, for straight quotes in verbatim environments
\IfFileExists{upquote.sty}{\usepackage{upquote}}{}
% use microtype if available
\IfFileExists{microtype.sty}{\usepackage{microtype}}{}
\ifxetex
  \usepackage[setpagesize=false, % page size defined by xetex
              unicode=false, % unicode breaks when used with xetex
              xetex]{hyperref}
\else
  \usepackage[unicode=true]{hyperref}
\fi
\hypersetup{breaklinks=true,
            bookmarks=true,
            pdfauthor={},
            pdftitle={},
            colorlinks=true,
            citecolor=blue,
            urlcolor=blue,
            linkcolor=magenta,
            pdfborder={0 0 0}}
\urlstyle{same}  % don't use monospace font for urls
\setlength{\parindent}{0pt}
\setlength{\parskip}{6pt plus 2pt minus 1pt}
\setlength{\emergencystretch}{3em}  % prevent overfull lines
\setcounter{secnumdepth}{0}


\begin{document}

\section{Fisiologia generale}\label{fisiologia-generale}

Che cos'è la fisiologia? E' una disciplina che studia il funzionamento
degli esseri viventi e consiste nella definizione meccanicistica di ciò
che accade nell'organismo di ogni persona. Questa disciplina ti fa
capire che non sei quel che pensi di essere.

La materia vivente e' principalmente composta di: H, O, C, N. La
composizione chimica della materia vivente è molto piu' simile a quella
dell'universo e delle stelle, piuttosto che a quella della terra e della
crosta terrestre.

Tra la fine del '700 e l'inizio del '800, dopo la diffusione del
microscopio, ci si rese conto che gli esseri viventi sono organizzati in
cellule. La materia vivente ha una sua unitarietà molto solida, ovvero
tutte le cellule contengono all'incirca gli stessi organelli e nelle
stesse quantità. L'acqua è sempre la sostanza più abbondante all'interno
della cellula e le proteine sono la classe di composti maggiormente
diversificata nell'organismo.

\subsection{L'importanza dell'acqua}\label{limportanza-dellacqua}

L'acqua non è un inerte riempitivo delle strutture organiche, ma le sue
molecole sono molto reattive. Le fondamentali caratteristiche dell'acqua
sono:

\begin{itemize}
\itemsep1pt\parskip0pt\parsep0pt
\item
  possiede una \emph{capacità solvente} molto elevata;
\item
  ha un'elevata \emph{capacità termica} ed un \emph{elevato calore
  latente di evaporazione};
\item
  presenta un'\emph{elevata tensione superficiale} che facilita i
  fenomeni di capillarità;
\item
  promuove la formaizone di \emph{legami idrofobici} tra le molecole non
  idrosolubili che vi si trovano immerse.
\end{itemize}

Tutte queste proprietà sono presenti quando l'acqua, alla pressione
atmosferica, si trova allo stato liquido.

Nell'H2O, un atomo di O e due atomi di H sono legati da legami covalenti
polari, ovvero: la densità della carica elettrica nell'intorno dei tre
nuclei atomici non è uniforme poichè l'atomo di O contiene un numero
maggiore di protoni di quello dei due H, per cui esso attrae gli
elettroni di legame più fortemente. Questo fa sì che la molecola di H2O,
sebbene sia neutra, presenti una distribuzione asimmetrica delle cariche
comportandosi come un \emph{dipolo} che tende ad orientarsi quando si
trova in un campo elettrico.

Questa proprietà conferisce all'acqua la capacità di funzionare da
\emph{solvente} soprattutto per quei composti che nell'acqua si
dissociano in ioni (\emph{elettroliti}).

Cosa succede quando poniamo un elettrolita in acqua? Quando un sale come
NaCl viene posto in acqua, i dipoli idrici subiscono una forte
attrazione elettrostatica da parte degli ioni si positivi che negativi e
vengono spinti a penetrare nella struttura cristallina del sale; essi si
disporranno attorno a ciascuno ione in modo da formare un involucro
detto \textbf{alone di solvatazione}, in cui i dipoli sono orientati con
polarità opposta a della quello ione.

L'acqua è un ottimo solvengte non solo per gli elettroliti, ma anche per
un'ampia gamma di sostanze organiche \emph{polari}. Queste molecole sono
solubili in acqua e perciò dette \textbf{idrofiliche}.

L'acqua è invece un pessimo solvente per quei composti organici (es.
oli) le cui molecole sono \emph{non polari} e che perciò non attraggono
in alcun modo i dipoli idrici. Queste molecole sono insolubili in acqua
e perciò dette \textbf{idrofobiche}.

E' poi possibile trovare composti organici le cui molecole possiedono,
disposti in regioni diverse, sia gruppi chimici dissociabili o polari
che gruppi apolari. Questi gruppi vengono detti \textbf{anfipatici}. Un
esempio di composto anfipatico sono i \emph{saponi}: se posti in acqua
costituiscono delle particelle submicroscopiche sferiche dette
\emph{micelle}. Nelle micelle le molecole anfipatiche sono disposte
ordinatamente in modo radiale con le code idrofobiche dirette verso il
centro della micelle e le teste idrofiliche verso l'esterno. Un
conmportamento analogo a quello dei saponi è presentato dai fosfolipidi.

\subsection{L'omeostasi}\label{lomeostasi}

Cosa significa omeostasi? Significa \emph{tendenza a mantenere costante
il sistema}. L'omeostasi è un insieme di processi che mantengono
costanti le condizioni corporee (es: costanza nella composizione chimica
del plasma), ed è garantita da meccanismi autoregolatori. Queste
condizioni devono mantenersi anche al variare delle condizioni esterne.

Si def9inisce \emph{omeostato} un insieme di strutture che permettono di
far sì che il comportamento di un sistema segua un andamento
prestabilito, determinato dalla grandezza del segnale applicato al suo
ingresso. I sistemi di controllo possono essere di due tipi:

\begin{enumerate}
\def\labelenumi{\arabic{enumi}.}
\itemsep1pt\parskip0pt\parsep0pt
\item
  ad \emph{anello aperto};
\item
  ad \emph{anello chiuso} o a \emph{retroazione}.
\end{enumerate}

Il \textbf{controllo ad anello aperto (feedforward)} è più semplice,
perchè la grandezza del sognale in ingresso è indipendente da quella in
uscita. Per applicare questo tipo di controllo occorre avere una buona
conoscenza della dinamica del sistema da controllare: quanto più è
esatta la rappresentazione interna del sistema, tanto più questo tipo di
controllo sarà affidabile. Il segnale in ingresso non entra direttamente
nel sistema da controllare ma nel sistema di controllo e solo
successivamente nel sistema da controllare.

(immagine)

Il \textbf{controllo ad anello chiuso}, invece, funziona diversamente:
immaginiamo che un generico segnale, agendo sull'ingresso \textbf{(I)}
di un sistema, causi un effetto alla sua uscita \textbf{(O)}. Il
\textbf{feedback} consiste nel ``trasferimento all'indietro'' (su I) di
un segnale di retroazione \textbf{(Sr)} ricavato da O, in modo che esso
agisca sulla causa stessa che lo produce.

(immagine)

A seconda che l'anello di retroazione sia ``invertente'' o ``non
invertente'' il segnale in uscita, esistono due tipi di retroazione:

\begin{enumerate}
\def\labelenumi{\arabic{enumi}.}
\itemsep1pt\parskip0pt\parsep0pt
\item
  \emph{negative feedback} quando Sr ha un segno opposto a quello di O;
\item
  \emph{positive feedback}, quando Sr ed O hanno lo stesso segno.
\end{enumerate}

Il \textbf{feedback positivo} è un meccanismo che in natura si presenta
solo in particolari occasioni poichè è \emph{destabilizzante} (allontana
qualunque sistema da un regime stazionario).

Nel \textbf{feedback negativo} o \textbf{controreazione} invece,
possiamo immaginare che il segnale in uscita dal sistema (O) rappresenti
il parametro che deve essere stabilizzato, e che un fattore perturbatore
(I) tenda ad innalzarlo dal suo valore normale. Aumenterà
conseguentemente il segnale di retroazione negativa il quale,
opponendosi ad I, tenderà a riportare O al suo valore normale. In questo
modo il sistema tende a \emph{mantenere stabile} la sua uscita.

Da un punto di vista grafico, supponendo che ci siano due fattori che
agiscono a vicenda con un \emph{feedback negativo} dove al crescere di
un fattore l'altro diminuisce, un fattore agirà come \textbf{y=sen(t)} e
l'altro \textbf{x=cos(t)}, dove \emph{t} è il tempo. Se posti in grafico
questi valori oscillano in maniera sfalsata. Se con gli stessi fattori
costruisco una funzione \textbf{y=f(x)} otterrò un cerchio che ci indica
che, postandoci nel tempo, vengono ripercorsi gli stessi punti. La
distanza di questa funzione dall'origine degli assi è sempre 1.

\subsubsection{Il principio fisiologico
fondamentale}\label{il-principio-fisiologico-fondamentale}

Cos'è un \textbf{agente biologico funzionale}? Un qualsiasi \emph{agente
biologico ereditario} capace di operare una \emph{trasformazione fisica}
(es. enzima). Per ogni agente biologico funzionale esiste almeno un
a.b.f. che lo controlla a monte ed esiste almeno un agente biologico
funzionale che viene controllato da questo a valle. Il numero degli
agenti biologici funzionali presenti in una cellula è finito. Questo
significa che per ogni a.b.f esiste almeno un loop dove l'ultimo a.b.f.
sarà in grado di agire sul primo a.b.f andando a chiudere il ciclo.

L'importanza dell'omeostasi, ovvero la capacità degli esseri viventi di
mantenere costanti i parametri del loro ambiente corporeo, non deve far
pensare che l'ampio spettro delle funzioni vitali non sia in grado di
subire delle piccole variazioni: l'omeostasi varia dunque sul lungo
termine (bisogna tenere conto anche dell'invecchiamento dell'organismo).
Gli organismi viventi si modificano in funzione delle informazioni che
ricevono dall'ambiente in cui vivono e queste modificazioni possono
persistere nel tempo. Questa proprietà prende il nome di
\emph{plasticità} e consiste nella capacità di \textbf{adattamento}
degli organismi: l'organismo può variare il proprio stato funzionale
passando da un iniziale stato di omeostasi ad un nuovo punto di
equilibrio. Un esempio è la preparazione atletica: l'organismo che viene
sottoposto ad esercizio fisico subisce uno stress e reagisce allo stress
con un rinforzo che può progredire mediante adattamento progressivo.
Durante il processo di adattamento avvengono fisicamente delle
trasformazioni nell'organismo che rappresentano il raggiungimento
progressivo di nuovi punti di omeostasi.

Lo schema degli agenti biologici funzionali è dunque plastico: non perde
mai le proprie caratteristiche di base, ma può essere modulato in
risposta a sollecitazioni. Questi cicli possono avere dei punti di
rottura, se eccessivamente sollecitati, che possono essere più o meno
mortali.

\subsection{La membrana cellulare}\label{la-membrana-cellulare}

Nella materia vivente ci sono due ambienti chimici completamenti diversi
e agli antipodi:

\begin{itemize}
\itemsep1pt\parskip0pt\parsep0pt
\item
  l'ambiente \textbf{idrofilo}, la sostanza fondamentale è l'acqua e
  contiene composti idrosolubili;
\item
  l'ambiente \textbf{idrofobo} o \emph{lipofilo}, la sostanza
  fondamentale è costituita da lipidi o grassi.
\end{itemize}

Nell'organismo esistono sostanze tensioattive che formano le membrane
cellulari. A questo gruppo appartengono i \textbf{fosfolipidi}: molecole
che presentano due catene di \emph{acidi grassi} (coda idrofoba)
esterificate su una molecola di \emph{glicerolo} al quale è
ulteriormente esterificato un \emph{gruppo fosfato} (testa idrofila).

Come si dispongono i fosfolipidi in acqua? Normalmente se poniamo dei
lipidi in soluzione acquosa questi formano delle micelle. I fosfolipidi
fanno al stessa cosa, ma possono anche organizzarsi in un \emph{doppio
strato} che risulta stabile in acqua andando a formare una struttura
chiamata \emph{membrana}. Le membrane fosfolipidiche tenderanno a
chiudersi andando a formare delle vescicole poiché altrimenti, se le
membrane fossero piatte come dei fogli, l'interno dei bordi (che è
idrofobico) rimarrebbe a contatto con l'ambiente acquoso circostante.

Le membrane cellulari formano dunque un piccolo ambiente lipidico
all'interno del quale troviamo anche altre molecole: una delle più
importanti è il \emph{colesterolo}, un composto steroideo fortemente
apolare. A cosa serve la presenza del colesterolo? Per mantenere fluida
la membrana.

Cos'è che permette ad un composto grasso d'essere liquido o solido? La
presenza di legami doppi (grassi insaturi, liquidi) o singoli (grassi
saturi, solidi). Il doppio legame introduce un angolo di piegatura nella
catena diverso dagli altri facendo sì che l'attrazione dovuta alle forze
di VdW tra le molecole sia ridotta rispetto alla forza presente tra
molecole lineari (solo legami singoli). Una forte attrazione data dalle
forze di VdW, dovuta alla prevalenze di legami singoli, darà origine a
sostanze solide (come il burro).

Il colesterolo (grasso saturo) è necessario per regolare la fluidità
della membrana. Se la T è intorno ai 37° il colesterolo impedisce la
mobilità dei fosfolipidi, diminuendo la fluidità della membrana. A T più
basse invece, il colesterolo impedisce l'impaccamento dei fosfolipidi
che diminuirebbe eccessivamente la fluidità della membrana.

I fosfolipidi sono mobili nella membrana, possono ruotare lungo i loro
assi e possono scorrere lateralmente.

Esperimento: vennero fatte fondere cellule umane e cellule di topo. Dopo
la fusione vennero marcate e si vide che si erano distribuite lungo
tutta la cellula fuso. Ovvero le membrane non erano rimaste separate ma
si erano fuse e dunque i materiali delle due membrane si erano
distribuiti lungo tutta la nuova membrana dimostrando la loro fluidità.

\subsubsection{Gli sfingolipidi}\label{gli-sfingolipidi}

I fosfolipidi sono i più abbondanti, ma non sono gli unici. I primi
contengono glecerolo, mentre gli sfingolipidi contengono un amminoalcol
(sfingosina con testa polare e coda apolare) al posto del glicerolo e un
acido grasso collegato con la parte amminica. A volte alla sfingolisi
può essere lengato un ulteriore gruppo -R. Se non è presente si parla di
ceramide.

Gli sfingolipidi sono maggiormente presenti in determinate zone della
membrana dette zattere lipidiche (zone più spesse perché le catene degli
sfingolipidi sono un po' più lunghe di quelle dei fosfolipidi) e
contengono proteine che non sono presenti nelle altre zone.

La membrana esterna forma una barriera tra l'ambiente intra- ed extra-
cellulare. Le membrane permettono il trasporto selettivo di nutrienti,
prodotti di rifiuto e metaboliti tra i vari compartimenti cellulari. Le
membrane servono a formare vescicole per catturare e secernere
macromolecole e altre particelle. (\ldots{})

Nelle membrane sono presenti molte proteine che possono essere integrali
(sono incorporate nella membrana mediante domini idrofobici) o di
superficie (sono associate o agganciate alla membrana ma non hanno parti
idrofobiche immerse nello strato fosfolipidico.

Le proteine di membrana possono avere un ruolo strutturale
(citoscheletrico), possono servire per attaccare la cellula ad un
substrato (strutture di adesione cellulare), possono far attaccare le
cellule l'una all'altra, possono essere dei trasportatori (trasporto di
membrana), possono essere recettori (capaci di legare molecole segnale
che inducono processi interni alla cellula), possono essere
enzimi\ldots{}

Le proteine di membrana se la attraversano, possiedono una parte
extracellulare che molto spesso porta legate delle molecole zuccherine
lineari o ramificate (proteine glicosilate). Questo fa si che la
superficie cellulare sia ricoperta di zuccheri che sono molecole molto
idrofiliche che rendono lo strato acquoso subito presente esternamente
alla cellula molto denso (=glicocalice). Il glicocalice è un ulteriore
strato che si oppone ad un contatto troppo diretto della cellula con il
mondo esterno (protezione). Il glicocalice rappresenta anche un fattore
di riconoscimento per le cellule.

\section{DIFFUSIONE e OSMOSI}\label{diffusione-e-osmosi}

una parte della fisiologia si occupa di capire se attraverso le membrane
biologica passa qualcosa e se passa, come passa. Se non passasse nulle e
la cellula fosse uno spazio totalmente isolato dal resto dell'universo,
dovrebbe usare il materiale che contiene all'interno per compiere i
processi che la tengono in vita e riciclare tutti gli scarti dei
processi per ricreare i cicli. Le cellule nella realtà scambiano
continuamente materiali con l'esterno. Ma la cellula è anche separata,
isolata, dall'esterno. Questo ci fa capire che attraverso le membrane
possono passare delle molecole.

La fisiologia applica a questi fenomeni le leggi della fisica.

Alcuni di questi fenomeni avvengono spontaneamente senza l'intervento
del metabolismo: diffusione e osmosi. I doppi strati lipidici (bilayer)
sono impermeabili a molte molecole e ioni essenziali. Poiché la membrana
cellulare ha una natura lipidica tutte le molecole apolari possono
attraversarla tramite diffusione semplice. Le sostanze polari e quelle
cariche (ioni), invece, non riescono a passare. Maggiore è la carica
dello ione, maggiore è la resistenza della membrana. Nelle soluzioni
acquose vi possono essere ioni in soluzione. Nella cellula, sia dentro
che fuori, sono presenti degli ioni (sia minerali che organici) che non
possono passare tramite diffusione spontanea attraverso la membrana.
Sostanze inorganiche apolari di enorme importanza per la cellula e
capaci di attraversare la membrana sono i gas atmosferici (O2 e N2).

Cos'è la diffusione? La materia, così come la si conosce è costituita da
cellule che hanno un loro dinamismo (non sono statiche). Tanto più le
cose sono dinamiche tanto più emergono fenomeni come quelli di
diffusione. I più mobili sono i gas. Poi vi sono i liquidi ed infine i
solidi. Questro dinamismo fa sì che nelle sostanza in cui le molecole
hanno la possibilità di muoversi si crea il fenomeno della diffusione:
se c'è un'abbondanza in una certa regione di una sostanza o di un gas,
questa tenderà a spostarsi diffondendo. Quando c'è una diffusione non
omogenea si parla di gradiente.

Se non sono distribuite uniformemente sui due lati di una membrana le
loro diverse concentrazioni formano un gradiente, forma di energia
potenziale. Energia libera di una soluzione: deltaG= RT ln{[}S{]}

Energia del gradiente: deltaG= enrgia esterna -- energia interna deltaG=
RT ln{[}S{]}o -- RT ln{[}S{]}i = RT ln{[}S{]}o/RT ln{[}S{]}i = legge di
van't Hoff

la membrana cellulare permette alla cellula di avere al suo interno una
concentrazione di una molecola X diversa dalla concentrazione della
stessa molecole X all'esterno della cellula. Avremo quindi due soluzioni
omogenee dalle due parti della membrana che creeranno un potenziale che
premerà sulla membrana.

(\ldots{})

Il processo osmotico è spontaneo, non richiede energia. Questo è
impossibile per la seconda legge della termodinamica (l'energia non si
crea dal nulla), abbiamo un apparente paradosso.

Legge di Fick o legga di diffusione: se consideriamo due compartimenti a
e b separai da parete con pori di lunghezza delta x superficie totale F
e differenza di concentrazione del soluto Ca-Cb= deltaC Il soluto
diffonde da a verso b.

Il tasso di diffusione è direttamente proporzionale alla differenza di
concentrazione, alla temperatura, alla superficie di scamb9io, e
inversamente proporzionale alla distanza, alle dimensioni delle molecole
e alla viscosità del solvente.

Se abbiamo una membrana lipidica abbiamo un ulteriore elemento da
considerare: il coefficiente di partizione, ovvero la capacità della
sostanza di passare da uno spazio acquoso ad uno lipidico.

Se la permeabilità è 0, il flusso attraverso la membrana sarà 0
indipendentemente dal gradiente.

Se il coefficiente di permeabilità invece è 1, la velocità di diffusione
dipenderà solo dal gradiente.

DIFFUSIONE FACILITATA nella membrana cellulare ci sono delle proteine
che creano dei pori che possono consentire il passaggio a sostanze che
altrimenti non passerebbero.

È sempre un processo spontaneo che non richiede l'aggiunta di energia,
ma che richiede la presenza di elementi aggiuntivi nella membrana. La
cinetica di trasporto è diversa rispetto a quella della diffusione
semplice. Diff. Semplice: la velocità di diffusione è in relazione
lineare con ilgradiente di concentrazione della molecola che diffonde.
Diff. Facilitata: la velocità di diffusione dipende dalla disponibilità
delle molecole che aiutano la diffusione. Quando queste sono sature
l'aumento di concentrazione della molecola che diffonde non incrementa
ulteriormente il tasso di diffusione. Questo significa che avrò per un
po' una crescita e poi arriverò ad un plateau oltre il quale l'aumento
del solvente che diffonde non servirà più per aumentare la velocità di
diffusione.

OSMOSI cosa serve: due soluzioni a concentrazione diversa separate da
una membrana permeabile al solvente ma non al soluto.

L'acqua tenderà a passare dalla zona meno concentrata a quella più
concentrata. Il flusso d'acqua continuerà anche qui fin quando non
avremo un equilibrio. L'acqua qui crea una pressione idrostatica guidata
dalla pressione osmotica.

20151012

Le cellule risentono della pressione osmotica poiché presentano una
membrana plasmatica che lascia passare sempre l'acqua. Questo passaggio
può essere controllato.

Esistono soluti che non possono assolutamente passare, creando un
gradiente osmotico tra l'ambiente esterno ed interno della cellula.
L'effetto osmotico è dato dalla somma di tutti i soluti (osmoliti) che
non passano attraverso la membrana, è dato solo dalle concentrazioni e
non dal tipo di molecola. Se le concentrazioni degli osmoliti dentro e
fuori sono uguali ovviamente non si ha l'effetto osmotico. Normalmente
le cellule animali sono in condizioni di equilibrio osmotico. Se non c'è
equilibrio possono rigonfiarsi o avvizzire.

Nel plasma e nei globuli rossi abbiamo una concentrazione di 300
mOsmoles/L (isotonico). Soluzione ipotonica (es 150 mOsmoles/L), la
cellula rigonfia fino a scoppiare. Soluzione ipertonica (es 500
mOsmoles/L) la cellula raggrinzisce.

IL TRASPORTO DI MEMBRANA Le cellule scambiano sostanze con l'esterno.
Non sono un sistema chiuso. Questo scambio viene fatto in diversi modi
che complessivamente vengono definiti come ``trasporto di membrana''.
Poiché la membrana ricopre interamente la cellula e non presenta
``buchi'', deve esistere un modo per le molecole di attraversarla. I
fisiologi catalogano il trasporto di membrana creando in primo luogo due
grandi divisioni: 1. trasporto attivo, tutto ciò che passa attraverso la
membrana con consumo di energia; 2. trasporto passivo, tutto ciò che
passa attraverso la membrana senza che la cellula debba consumare
energia.

Possiamo avere trasporto attivo o passivo sia in ingresso che in uscita
dalla cellula.

Il trasporto passivo può essere: 1. semplice, non servono intermediari
perché la sostanza diffonda attraverso la membrana. 2. facilitato, la
sostanza segue il gradiente di concentrazione ma non può attraversare
spontaneamente la membrana ma utilizza dei trasportatori di membrana.

Per il trasporto semplice serve una sostanza che può diffondere
attraverso la membrana ed un gradiente osmotico tra l'interno e
l'esterno della cellula.

Gli esempi della diffusione semplice è ad esempio il passaggio
dell'acqua (a volte possono sfruttare anche il trasporto facilitato
grazie alle acquaporine), i gas atmosferici, le sostanze lipidiche. Per
diffusione facilitata passano sostanze come i principali zuccheri del
metabolismo (glucosio e fruttosio), i principali ioni minerali (Na, K,
Cl, Ca).

Il trasporto attivo viene suddiviso in primario e secondario. Il
trasporto attivo primario crea i gradienti e viene generato da speciali
trasportatori detti ``pompe'' che consumano energia sotto forma di ATP
(dette anche pompe atipasiche o ATP dipendenti). Nel trasporto attivo le
sostanze vengono spostate contro gradiente (da dove ce n'è poca a dove
ce n'è tanta).

In quello secondario il trasportatore sfrutta l'energia di un gradiente
trasportando una sostanza per trasportare un'altra sostanza contro
gradiente. Nel cotrasporto vi sono delle proteine di membrana dette
cotrasportatori (a due o 3 sostanze) che trasportano una sostanza contro
gradiente e una secondo gradiente. Non c'è consumo diretto di ATP ma
ch'è dissipazione di un gradiente. Il movimento delle sostanze può
essere nella stessa direzione (simporto) o in direzioni opposte
(antiporto). I cotrasportatori non sono in grado di creare gradienti ma
possono sfruttare un gradiente già esistente per trasportare un'altra
sostanza contro gradiente.

Possono poi esserci fenomeni di esocitosi e endocitosi. Le sostanze non
vengono trasportate molecola per molecola ma come massa di materia
grazie a movimenti della membrana plasmatica che ingloba una certa
quantità di materiale. L'endocitosi e l'esocitosi implicano sempre la
formazione di vescicole.

LE POMPE Sono proteine di membrana: la parte esterna è glicosilata. La
parte citosolica è molto più grande di quella esterna e c'è una parte
interna nella membrana. Il suo substrato enzimatico è l'ATP. L'idrolisi
dell'ATP libera un'energia che il trasportatore può utilizzare facendo
avvenire il passaggio da una sostanza da un lato all'altro della
membrana. Trasportano numerosissime sostanze, noi vedremo soprattutto
ioni minerali.

I trasportatori di membrana sono presenti anche all'interno della
cellula in alcuni organuli cellulari (es: lisosomi, mitocondri\ldots{}).
Le ATPasi possono essere F-ATPasi dei mitocondri sono coinvolti nella
formazione dell'ATP (?); V-ATPasi dei vacuoli che acidificano il lume
interno dei vacuoli portando al loro interno ioni H+; P-ATPasi pompano
ioni Na+, K+, Ca2+\ldots{} hanno un intermedio fosforilato. La molecola
del trasportatore viene fosforilata e trasportata dal compartimento
extra a quello intracellulare.

Il cotrasporto non genera gradiente ma lo sfrutta (questo gradiente sarà
stato prodotto da u trasportatore attivo primario), lascia diffondere la
sostanza lungo il suo gradiente e questa diffusione crea i presupposti
affinché lo stesso trasportatore possa far attraversare la membrana ad
un'altra sostanza contro gradiente. Questo trasporto può avvenire nella
stessa direzione o in direzioni opposte. Le sostanze trasportate possono
essere anche tre.

I(n una normale condizione fisiologica esistono diversi gradienti di
membrana (indispensabili per la vita delle cellule). La fisiologia duna
cellula ruota attorno ad alcuni ioni minerali che sono quelli più
abbondanti nei liquidi biologici. I più abbondanti in assoluto sono il
Na e il K, Cl e Ca. Questi ioni hanno sempre concentrazioni diverse
dentro e fuori la cellula. Na \textgreater{} nell'ambiente esterno 140,
dentro 5 K \textgreater{} nel citosol 5-15, dentro 145 Cl \textgreater{}
maggiore nell'ambiente esterno 110, dentro 4 Ca \textgreater{}
nell'ambiente esterno 2,5-5, dentro 0,0001

Se uno di questi gradienti cambia la cellula muore, dunque la cellula
mantiene questi gradienti costanti sempre. Questi gradienti sono
mantenuti dal trasporto attivo primario che li crea ed in parte da
quello secondario che sfruttando questi gradienti fondamentali nel tiene
stabili altri.

Il sodio è lo ione più importante extracellularmente, mentre il K è il
più importante intracellularmente. Esiste una Na-K ATP, o ATPasi sodio
potassio dipendente o pompa sodio potassio. E' una pompa di tipo P,
localizzata nella membrana plasmatica, che sfrutta un intermedio
fosforilato. Questo trasportatore crea il gradiente del K e del Na (più
importanti dal punto di vista quantitativo). Movimenta la maggior
quantità di materiale attraverso la membrana. Trasporta ioni Na fuori
dalla cellula e contemporaneamente ioni K all'interno, entrambe contro
il loro gradiente elettrochimico. Questa pompa è essenziale per
l'eccitabilità dei neuroni e del muscolo. E' essenziale per l'equilibrio
elettrico e osmotico della cellula.

(riguardare elettromagnetismo)

questa proteina è formata da una subunità alfa (10 domini transmembrna e
sporge in prevalenza nel citosol, porta i siti di legame per l'ATP e per
i cationi) e da una subunità beta più piccola, ha un solo dominio
transmembrana, sporge verso l'esterno con una porzione glicosilata.
Questa proteina è un enzima, è un trasportatore con un ciclo di
attività.

La proteina viene fosforilata (idrolisi di ATP sulla proteina) in uno
stato in cui la proteina lega il Na sul lato citosolico (dove sodio ce
n'è ``poco''). Questa fosforilazione produce una variazione
conformazionale della proteina. A questo punto la zona che lega il Na
(può ospitare fino a 3 ioni alla volta) si apre all'interno della
cellula dove il Na è molto concentrato. Questa variazione
conformazionale rende i siti di legame per il sodio molto meno affini
allo ione permettendo che lo ione si stacchi all'interno. L'affinità per
il ligando si presenta con un valore simile a quello della
concentrazione dello ione (????). Quando la proteina varia la sua
conformazione rivolgendo i siti di legame verso l'esterno il
coefficiente di dissociazione deve salire a 100/150 mmol.

La fosforilazione della proteina non dura molto a lungo (millisecondi).
La proteina viene defosforilata e quello che è avvenuto per il sodio
avviene per il K. Ogni 3 ioni Na che escono che ne entrano 2 di K.
Questo trasporto elettrogenico poiché crea una differenza elettrica.

C'è in altro trasportatore molto espresso che fa un lavoro simile ma con
lo ione Ca2+. PMCA o Ca2+ ATPasi della plasma membrana. Espelle ioni
calcio all'esterno della cellula. Ne esistono 4 isoforme.

SERCA Ca2+ ATPasi del reticolo sarco-endoplasmatico. E' il principale
sistema di omeostasi del muscolo scheletrico e cardiaco. Questa proteina
movimenta lo ione sulla membrana del reticolo endoplasmatico, mentre le
altre due viste movimentano gli ioni sulla membrana plasmatica.

Troviamo poi l'antiporto Na/Ca (cotrasporto). Questo trasportatore
sfrutta il gradiente del sodio facendolo entrare nella cellula e
contemporaneamente porta Ca fuori dalla cellula. Tende a dissipare il
gradiente del Na che viene mantenuto dalla Na--K ATPasi. Nel miocardio
la stechiometria è di 3 Na per ogni Ca nei bastoncelli è di

V-ATPasi tipica dei lisosomi e di tutte le vescicole a pH basso. Porta
ioni H dall'esterno all'interno del vacuolo.

Un altro tipo di trasporto attivo è tipico dello stomaco.

L'acidità dei succhi gastrici è dovuta alla pompa H-K, ATPasi dei tipo
P. le estremità carbossiliche e amminiche sono esterne alla cellula. Fa
uscire H e entrare K.

Scambiatore Na-H (non è una pompa), molto espresso nel rene.

Nel rene avviene la filtrazione del sangue (liquido di partenza=
plasma), il liquido finale è l'urina (molto diversa dal plasma). Tutte
le differenze tra il plasma e l'urina sono dovute all'attività di
trasportatori che agiscono nei tubicini dove passa il plasma.

Fagocitosi, endocitosi, pinocitosi, esocitosi. (\ldots{})

Trasporto transepiteliale. Gli epiteli funzionano come dispositivi che
trasportano sostanze da un compartimento all'altro.

\end{document}
