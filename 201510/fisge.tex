\documentclass[]{article}
\usepackage{lmodern}
\usepackage{amssymb,amsmath}
\usepackage{ifxetex,ifluatex}
\usepackage{fixltx2e} % provides \textsubscript
\ifnum 0\ifxetex 1\fi\ifluatex 1\fi=0 % if pdftex
  \usepackage[T1]{fontenc}
  \usepackage[utf8]{inputenc}
\else % if luatex or xelatex
  \ifxetex
    \usepackage{mathspec}
    \usepackage{xltxtra,xunicode}
  \else
    \usepackage{fontspec}
  \fi
  \defaultfontfeatures{Mapping=tex-text,Scale=MatchLowercase}
  \newcommand{\euro}{€}
\fi
% use upquote if available, for straight quotes in verbatim environments
\IfFileExists{upquote.sty}{\usepackage{upquote}}{}
% use microtype if available
\IfFileExists{microtype.sty}{\usepackage{microtype}}{}
\ifxetex
  \usepackage[setpagesize=false, % page size defined by xetex
              unicode=false, % unicode breaks when used with xetex
              xetex]{hyperref}
\else
  \usepackage[unicode=true]{hyperref}
\fi
\hypersetup{breaklinks=true,
            bookmarks=true,
            pdfauthor={},
            pdftitle={},
            colorlinks=true,
            citecolor=blue,
            urlcolor=blue,
            linkcolor=magenta,
            pdfborder={0 0 0}}
\urlstyle{same}  % don't use monospace font for urls
\setlength{\parindent}{0pt}
\setlength{\parskip}{6pt plus 2pt minus 1pt}
\setlength{\emergencystretch}{3em}  % prevent overfull lines
\setcounter{secnumdepth}{0}


\begin{document}

\section{Fisiologia generale}\label{fisiologia-generale}

Che cos'è la fisiologia? E' una disciplina che studia il funzionamento
degli esseri viventi e consiste nella definizione meccanicistica di ciò
che accade nell'organismo di ogni persona. Questa disciplina ti fa
capire che non sei quel che pensi di essere.

La materia vivente e' principalmente composta di: H, O, C, N. La
composizione chimica della materia vivente è molto piu' simile a quella
dell'universo e delle stelle, piuttosto che a quella della terra e della
crosta terrestre.

Tra la fine del '700 e l'inizio del '800, dopo la diffusione del
microscopio, ci si rese conto che gli esseri viventi sono organizzati in
cellule. La materia vivente ha una sua unitarietà molto solida, ovvero
tutte le cellule contengono all'incirca gli stessi organelli e nelle
stesse quantità. L'acqua è sempre la sostanza più abbondante all'interno
della cellula e le proteine sono la classe di composti maggiormente
diversificata nell'organismo.

\subsection{L'importanza dell'acqua}\label{limportanza-dellacqua}

L'acqua non è un inerte riempitivo delle strutture organiche, ma le sue
molecole sono molto reattive. Le fondamentali caratteristiche dell'acqua
sono:

\begin{itemize}
\itemsep1pt\parskip0pt\parsep0pt
\item
  possiede una \emph{capacità solvente} molto elevata;
\item
  ha un'elevata \emph{capacità termica} ed un \emph{elevato calore
  latente di evaporazione};
\item
  presenta un'\emph{elevata tensione superficiale} che facilita i
  fenomeni di capillarità;
\item
  promuove la formaizone di \emph{legami idrofobici} tra le molecole non
  idrosolubili che vi si trovano immerse.
\end{itemize}

Tutte queste proprietà sono presenti quando l'acqua, alla pressione
atmosferica, si trova allo stato liquido.

Nell'H2O, un atomo di O e due atomi di H sono legati da legami covalenti
polari, ovvero: la densità della carica elettrica nell'intorno dei tre
nuclei atomici non è uniforme poichè l'atomo di O contiene un numero
maggiore di protoni di quello dei due H, per cui esso attrae gli
elettroni di legame più fortemente. Questo fa sì che la molecola di H2O,
sebbene sia neutra, presenti una distribuzione asimmetrica delle cariche
comportandosi come un \emph{dipolo} che tende ad orientarsi quando si
trova in un campo elettrico.

Questa proprietà conferisce all'acqua la capacità di funzionare da
\emph{solvente} soprattutto per quei composti che nell'acqua si
dissociano in ioni (\emph{elettroliti}).

Cosa succede quando poniamo un elettrolita in acqua? Quando un sale come
NaCl viene posto in acqua, i dipoli idrici subiscono una forte
attrazione elettrostatica da parte degli ioni si positivi che negativi e
vengono spinti a penetrare nella struttura cristallina del sale; essi si
disporranno attorno a ciascuno ione in modo da formare un involucro
detto \textbf{alone di solvatazione}, in cui i dipoli sono orientati con
polarità opposta a della quello ione.

L'acqua è un ottimo solvengte non solo per gli elettroliti, ma anche per
un'ampia gamma di sostanze organiche \emph{polari}. Queste molecole sono
solubili in acqua e perciò dette \textbf{idrofiliche}.

L'acqua è invece un pessimo solvente per quei composti organici (es.
oli) le cui molecole sono \emph{non polari} e che perciò non attraggono
in alcun modo i dipoli idrici. Queste molecole sono insolubili in acqua
e perciò dette \textbf{idrofobiche}.

E' poi possibile trovare composti organici le cui molecole possiedono,
disposti in regioni diverse, sia gruppi chimici dissociabili o polari
che gruppi apolari. Questi gruppi vengono detti \textbf{anfipatici}. Un
esempio di composto anfipatico sono i \emph{saponi}: se posti in acqua
costituiscono delle particelle submicroscopiche sferiche dette
\emph{micelle}. Nelle micelle le molecole anfipatiche sono disposte
ordinatamente in modo radiale con le code idrofobiche dirette verso il
centro della micelle e le teste idrofiliche verso l'esterno. Un
conmportamento analogo a quello dei saponi è presentato dai fosfolipidi.

\subsection{L'omeostasi}\label{lomeostasi}

Cosa significa omeostasi? Significa \emph{tendenza a mantenere costante
il sistema}. L'omeostasi è un insieme di processi che mantengono
costanti le condizioni corporee (es: costanza nella composizione chimica
del plasma), ed è garantita da meccanismi autoregolatori. Queste
condizioni devono mantenersi anche al variare delle condizioni esterne.

Si def9inisce \emph{omeostato} un insieme di strutture che permettono di
far sì che il comportamento di un sistema segua un andamento
prestabilito, determinato dalla grandezza del segnale applicato al suo
ingresso. I sistemi di controllo possono essere di due tipi:

\begin{enumerate}
\def\labelenumi{\arabic{enumi}.}
\itemsep1pt\parskip0pt\parsep0pt
\item
  ad \emph{anello aperto};
\item
  ad \emph{anello chiuso} o a \emph{retroazione}.
\end{enumerate}

Il \textbf{controllo ad anello aperto (feedforward)} è più semplice,
perchè la grandezza del sognale in ingresso è indipendente da quella in
uscita. Per applicare questo tipo di controllo occorre avere una buona
conoscenza della dinamica del sistema da controllare: quanto più è
esatta la rappresentazione interna del sistema, tanto più questo tipo di
controllo sarà affidabile. Il segnale in ingresso non entra direttamente
nel sistema da controllare ma nel sistema di controllo e solo
successivamente nel sistema da controllare.

(immagine)

Il \textbf{controllo ad anello chiuso}, invece, funziona diversamente:
immaginiamo che un generico segnale, agendo sull'ingresso \textbf{(I)}
di un sistema, causi un effetto alla sua uscita \textbf{(O)}. Il
\textbf{feedback} consiste nel ``trasferimento all'indietro'' (su I) di
un segnale di retroazione \textbf{(Sr)} ricavato da O, in modo che esso
agisca sulla causa stessa che lo produce.

(immagine)

A seconda che l'anello di retroazione sia ``invertente'' o ``non
invertente'' il segnale in uscita, esistono due tipi di retroazione:

\begin{enumerate}
\def\labelenumi{\arabic{enumi}.}
\itemsep1pt\parskip0pt\parsep0pt
\item
  \emph{negative feedback} quando Sr ha un segno opposto a quello di O;
\item
  \emph{positive feedback}, quando Sr ed O hanno lo stesso segno.
\end{enumerate}

Il \textbf{feedback positivo} è un meccanismo che in natura si presenta
solo in particolari occasioni poichè è \emph{destabilizzante} (allontana
qualunque sistema da un regime stazionario).

Nel \textbf{feedback negativo} o \textbf{controreazione} invece,
possiamo immaginare che il segnale in uscita dal sistema (O) rappresenti
il parametro che deve essere stabilizzato, e che un fattore perturbatore
(I) tenda ad innalzarlo dal suo valore normale. Aumenterà
conseguentemente il segnale di retroazione negativa il quale,
opponendosi ad I, tenderà a riportare O al suo valore normale. In questo
modo il sistema tende a \emph{mantenere stabile} la sua uscita.

Da un punto di vista grafico, supponendo che ci siano due fattori che
agiscono a vicenda con un \emph{feedback negativo} dove al crescere di
un fattore l'altro diminuisce, un fattore agirà come \textbf{y=sen(t)} e
l'altro \textbf{x=cos(t)}, dove \emph{t} è il tempo. Se posti in grafico
questi valori oscillano in maniera sfalsata. Se con gli stessi fattori
costruisco una funzione \textbf{y=f(x)} otterrò un cerchio che ci indica
che, postandoci nel tempo, vengono ripercorsi gli stessi punti. La
distanza di questa funzione dall'origine degli assi è sempre 1.

\subsubsection{Il principio fisiologico
fondamentale}\label{il-principio-fisiologico-fondamentale}

Cos'è un \textbf{agente biologico funzionale}? Un qualsiasi \emph{agente
biologico ereditario} capace di operare una \emph{trasformazione fisica}
(es. enzima). Per ogni agente biologico funzionale esiste almeno un
a.b.f. che lo controlla a monte ed esiste almeno un agente biologico
funzionale che viene controllato da questo a valle. Il numero degli
agenti biologici funzionali presenti in una cellula è finito. Questo
significa che per ogni a.b.f esiste almeno un loop dove l'ultimo a.b.f.
sarà in grado di agire sul primo a.b.f andando a chiudere il ciclo.

L'importanza dell'omeostasi, ovvero la capacità degli esseri viventi di
mantenere costanti i parametri del loro ambiente corporeo, non deve far
pensare che l'ampio spettro delle funzioni vitali non sia in grado di
subire delle piccole variazioni: l'omeostasi varia dunque sul lungo
termine (bisogna tenere conto anche dell'invecchiamento dell'organismo).
Gli organismi viventi si modificano in funzione delle informazioni che
ricevono dall'ambiente in cui vivono e queste modificazioni possono
persistere nel tempo. Questa proprietà prende il nome di
\emph{plasticità} e consiste nella capacità di \textbf{adattamento}
degli organismi: l'organismo può variare il proprio stato funzionale
passando da un iniziale stato di omeostasi ad un nuovo punto di
equilibrio. Un esempio è la preparazione atletica: l'organismo che viene
sottoposto ad esercizio fisico subisce uno stress e reagisce allo stress
con un rinforzo che può progredire mediante adattamento progressivo.
Durante il processo di adattamento avvengono fisicamente delle
trasformazioni nell'organismo che rappresentano il raggiungimento
progressivo di nuovi punti di omeostasi.

Lo schema degli agenti biologici funzionali è dunque plastico: non perde
mai le proprie caratteristiche di base, ma può essere modulato in
risposta a sollecitazioni. Questi cicli possono avere dei punti di
rottura, se eccessivamente sollecitati, che possono essere più o meno
mortali.

\subsection{La membrana cellulare}\label{la-membrana-cellulare}

Nella materia vivente ci sono due ambienti chimici completamenti diversi
e agli antipodi:

\begin{itemize}
\itemsep1pt\parskip0pt\parsep0pt
\item
  l'ambiente \textbf{idrofilo}, la sostanza fondamentale è l'acqua e
  contiene composti idrosolubili;
\item
  l'ambiente \textbf{idrofobo} o \emph{lipofilo}, la sostanza
  fondamentale è costituita da lipidi o grassi.
\end{itemize}

Nell'organismo esistono sostanze tensioattive che formano le membrane
cellulari. A questo gruppo appartengono i \textbf{fosfolipidi}: molecole
che presentano due catene di \emph{acidi grassi} (coda idrofoba)
esterificate su una molecola di \emph{glicerolo} al quale è
ulteriormente esterificato un \emph{gruppo fosfato} (testa idrofila).

Come si dispongono i fosfolipidi in acqua? Normalmente se poniamo dei
lipidi in soluzione acquosa questi formano delle micelle. I fosfolipidi
fanno al stessa cosa, ma possono anche organizzarsi in un \emph{doppio
strato} che risulta stabile in acqua andando a formare una struttura
chiamata \emph{membrana}. Queste membrane presentano due porzioni
idrofile esterne (una a contatto con il citosol e l'altra a contatto con
l'ambiene acquoso esterno alla cellula) e una porzione lipofila interna.
Le membrane fosfolipidiche tenderanno a chiudersi andando a formare
delle vescicole poiché altrimenti, se le membrane fossero piatte come
dei fogli, la porzione interna dei bordi rimarrebbe a contatto con
l'ambiente acquoso circostante.

Le membrane cellulari formano dunque un piccolo ambiente lipidico
all'interno del quale troviamo anche altre molecole: una delle più
importanti è il \emph{colesterolo}, un composto steroideo fortemente
apolare. A cosa serve la presenza del colesterolo? Per mantenere fluida
la membrana.

Cos'è che permette ad un composto grasso d'essere liquido o solido? La
presenza di legami doppi (grassi insaturi, liquidi) o singoli (grassi
saturi, solidi). Il doppio legame introduce un angolo di piegatura nella
catena idrocarburica diverso dagli altri facendo sì che l'attrazione
dovuta alle forze di Van der Waals tra le molecole sia ridotta rispetto
alla forza presente tra molecole lineari (solo legami singoli). Una
forte attrazione data dalle forze di VdW, dovuta alla prevalenze di
legami singoli, darà origine a sostanze solide (come il burro).

Il colesterolo, un grasso saturo, è necessario per regolare la fluidità
della membrana. Se la T è intorno ai 37° il colesterolo impedisce la
mobilità dei fosfolipidi, diminuendo la fluidità della membrana. A T più
basse invece, il colesterolo impedisce l'impaccamento dei fosfolipidi
che diminuirebbe eccessivamente la fluidità della membrana.

I fosfolipidi sono mobili nella membrana, possono ruotare lungo i loro
assi e possono scorrere lateralmente.

Per provare la fluidità delle membrane venne fatto un esperimento:
vennero fatte fondere cellule umane e cellule di topo. Dopo la fusione
vennero marcate e si vide che si erano distribuite lungo tutta la nuova
membrana cellulare, ovvero le membrane non erano rimaste separate ma si
erano fuse e dunque i materiali delle due membrane si erano distribuiti
lungo tutta la nuova membrana dimostrando la loro fluidità.

Nelle membrane cellulari possiamo trovare anche gli
\textbf{sfingolipidi} I fosfolipidi sono i più abbondanti, ma non sono
gli unici. I primi contengono glicerolo, mentre gli sfingolipidi
contengono un \emph{amminoalcol} (sfingosina con testa polare e coda
apolare) al posto del glicerolo e un acido grasso collegato con la parte
amminica. A volte alla sfingolisina può essere lengato un ulteriore
gruppo -R; se non è presente si parla di \emph{ceramide}.

Gli sfingolipidi sono maggiormente presenti in determinate zone della
membrana dette zattere lipidiche (zone più spesse perché le catene degli
sfingolipidi sono un po' più lunghe di quelle dei fosfolipidi) e
contengono proteine che non sono presenti nelle altre zone.

La membrana esterna forma una barriera tra l'ambiente intra- ed extra-
cellulare. Le membrane permettono il trasporto selettivo di nutrienti,
prodotti di rifiuto e metaboliti tra i vari compartimenti cellulari. Le
membrane servono a formare vescicole per catturare e secernere
macromolecole e altre particelle.

Nelle membrane sono presenti molte proteine che possono essere:

\begin{itemize}
\itemsep1pt\parskip0pt\parsep0pt
\item
  proteine integrali di membrana, se sono incorporate nella membrana
  mediante domini idrofobici;
\item
  proteine di superficie, se sono associate o agganciate alla membrana
  ma non hanno parti idrofobiche immerse nello strato fosfolipidico.
\end{itemize}

Le proteine di membrana possono svolgere diversi ruoli:

\begin{itemize}
\itemsep1pt\parskip0pt\parsep0pt
\item
  possono avere un ruolo strutturale (citoscheletrico);
\item
  possono servire per attaccare la cellula ad un substrato (strutture di
  adesione cellulare);
\item
  possono far attaccare le cellule l'una all'altra;
\item
  possono essere dei trasportatori (trasporto di membrana);
\item
  possono essere dei recettori (capaci di legare molecole segnale che
  inducono processi interni alla cellula)\ldots{}
\end{itemize}

Le proteine di membrana, se la attraversano, possiedono una parte
extracellulare che molto spesso porta legate delle molecole zuccherine
lineari o ramificate, ed in questo caso sono dette \emph{proteine
glicosilate}. Questo fa si che la superficie cellulare sia ricoperta di
zuccheri, molecole molto idrofiliche che rendono lo strato acquoso
subito presente esternamente alla cellula molto denso formando una
struttura chiamata \textbf{glicocalice}. Il glicocalice si oppone
ulteriormente ad un contatto troppo diretto della cellula con il mondo
esterno svolgendo dunque una funzione protettiva, e rappresenta inoltre
un fattore di riconoscimento per le cellule.

\section{DIFFUSIONE e OSMOSI}\label{diffusione-e-osmosi}

Una parte della fisiologia si occupa di capire se attraverso le membrane
biologiche possono passare delle sostanze e, se passano, come passano.
Se non passasse nulla e la cellula fosse uno spazio totalmente isolato
dal resto dell'universo, dovrebbe usare il materiale che contiene
all'interno per compiere i processi che la tengono in vita e riciclare
tutti gli scarti dei suoi processi per ricreare i cicli. Le cellule
nella realtà scambiano continuamente diversi tipi di molecole con
l'esterno. Ma la cellula è anche separata dall'esterno, e questo ci
permette di capire che le membrane possono essere penetrate e
oltrepassate da diversi tipi molecolari.

La fisiologia applica a questi fenomeni le leggi della fisica.

Alcuni di questi fenomeni avvengono spontaneamente, senza l'intervento
del metabolismo, e sono i fenomeni di diffusione e osmosi. I doppi
strati lipidici (bilayer) sono impermeabili a molte molecole e ioni
essenziali. Poiché la membrana cellulare ha una natura lipidica tutte le
molecole apolari possono attraversarla tramite diffusione semplice,
mentre le sostanze polari e gli ioni non riescono a oltrepassarla.

Sostanze inorganiche apolari di enorme importanza per la cellula e
capaci di attraversare la membrana sono invece i gas atmosferici (O2 e
N2).

Cos'è la diffusione? La materia, così come la si conosce è costituita da
cellule che hanno un loro dinamismo (non sono statiche). Tanto più le
cose sono dinamiche tanto più emergono fenomeni come quelli di
diffusione: le molecole più mobili sono i gas, poi vi sono i liquidi ed
infine i solidi.

Questro dinamismo fa sì che nelle sostanze in cui le molecole hanno la
possibilità di muoversi si crei il fenomeno della \emph{diffusione}: se
in una certa regione c'è abbondanza di una certa sostanza o gas, questa
tenderà a spostarsi diffondendo. Quando c'è una diffusione non omogenea
si parla di \emph{gradiente}.

Se le molecole non sono distribuite uniformemente sui due lati di una
membrana le loro diverse concentrazioni formano un gradiente, ovvero una
forma di energia potenziale. Energia libera della soluzione viene
definita con: deltaG= RT ln{[}S{]}

Energia del gradiente, invece, viene definita con: deltaG = energia
esterna -- energia interna

ovvero, deltaG= RT ln{[}S{]}o -- RT ln{[}S{]}i = RT ln{[}S{]}o/RT
ln{[}S{]}i

Questa viene detta legge di Van't Hoff

In una soluzione acquosa si possono avere 3 modalità di flusso che si
differenziano per la natura della driving force che determina il moto: +
flusso di massa in cui la driving force è generata da una differenza di
pressione idraulica; + la \textbf{diffusione}, generata da una
differenza di concentrazione delle particelle; + la migrazione in campo
elettrico di particelle elettricamente cariche (ioni), in cui la driving
force è generata da una differenza di potenziale elettrico.

La \emph{diffusione} si verifica quando tra due regioni di una soluzione
esiste una differenza di concentraizone; si ha allora un flusso
particellare dalla regione a concentrazione magiore verso quelle a
concentrazione minore, finchè le particelle non sono distribuite in modo
omogeneo in tutto lo spazio disponibile.

La membrana cellulare permette alla cellula di avere una concentrazione
di una molecola X al suo interno, diversa dalla concentrazione della
stessa molecole X all'esterno della cellula. La presenza di due
soluzioni omogenee, ma con una diversa concentrazione di soluti, dalle
due parti della membrana crea un potenziale che premere sulla membrana
stessa.

Il processo osmotico è spontaneo, non richiede energia aggiunta. Questo
è impossibile per il \emph{primo principio della termodinamica}
(l'energia di un sistema termodinamico isolato non si crea né si
distrugge, ma si trasforma, passando da una forma a un'altra), abbiamo
un apparente paradosso. \textbf{(guardare se Gaia l'ha segnato)}

Se consideriamo due ambienti 1 e 2 separati da una sezione ideale S, ci
attendiamo che il numero di molecole di soluto che nell'unità di tempo
escono da 1 (e entrano in 2) sia \emph{proporzionale} al numero di
molecole che sono presenti per unità di volume, cioè alla
\emph{concentrazione C1} (e viceversa).

Questo può essere espresso tramite l'\textbf{equazione di Fick} o
\emph{legge di diffusione}:

\textbf{Fd=DA (C1-C2)}

(Fd sta per flusso diffusionale netto) dove \textbf{D} è il
\emph{coefficiente di diffusione} (dipende dalla natura dei partecipanti
al processo e dalla temperatura) e \textbf{A} è l'\emph{area della
sezione interessata al processo diffusivo}.

Se consideriamo due compartimenti 1 e 2 separai da una parete con pori
di lunghezza delta per una superficie totale F e differenza di
concentrazione del soluto deltaC=C1-C2, il soluto diffonde da 1 verso 2.

Il tasso di diffusione è dunque \emph{direttamente proporzionale} alla
differenza di concentrazione, alla temperatura e alla superficie di
scambio, mentre è \emph{inversamente proporzionale} alla distanza, alle
dimensioni delle molecole e alla viscosità del solvente.

Questa equazione è valida nel caso di una membrana ideale. Supponiamo
ora che i due compartimenti siano separati da una barriera reale,
costituita da una membrana (M) di spessore deltax e composizion propria,
e che le particelle siano spinte solo da un gradiente di concentrazione.
In queste condizioni, il passaggio di una generica sestanza (i) può
essere descritto dalla semplice equazione di Fick purchè il coefficiente
di diffusione venga sostituito dal coefficiente di permeabilità P.

\textbf{Fi=Pi (C1-C2)}

Il coefficiente di permeabilità di una data sostanza in una membrana
omogenea comprende tutti i fattori che ne condizionano il passaggio:

\textbf{Pi= betai (Dm - deltax)}

dove \textbf{Dm} è il \emph{coefficiente di diffusione della sostanza}
nel materiale costitutivo della membrana e \textbf{deltax} lo
\emph{spessore della membrana}. \textbf{Betai} invece, esprime
\emph{capacità della sostanza di passare da uno spazio acquoso ad uno
lipidico}, ovvero si potersi disciogliere nel materiale che costituisce
la membrana.

Il coefficiente di permeabilità \textbf{Pi} esprime la \emph{velocità
con cui una sostanza può attraversare la membrana}.

Se la permeabilità è 0, il flusso attraverso la membrana sarà 0 (nessun
flusso) indipendentemente dal gradiente.

Se il coefficiente di permeabilità invece è 1, la velocità di diffusione
dipenderà solo dal gradiente.

\subsection{DIFFUSIONE FACILITATA}\label{diffusione-facilitata}

Nella membrana cellulare ci sono delle proteine che creano dei pori che
permettono il passaggio di sostanze che altrimenti non passerebbero.

La diffusione facilitata è sempre un processo spontaneo che non richiede
l'aggiunta di energia, ma che richiede la presenza di elementi
aggiuntivi nella membrana (proteine). La cinetica di trasporto è diversa
rispetto a quella della diffusione semplice: nella \emph{diff. semplice}
la velocità di diffusione è in relazione lineare con il gradiente di
concentrazione della molecola che diffonde, mentre nella \emph{diff.
facilitata} la velocità di diffusione dipende dalla disponibilità delle
molecole che aiutano la diffusione. Quando queste molecole sono sature
l'aumento della concentrazione della molecola che diffonde non
incrementa ulteriormente il tasso di diffusione. Questo significa che
avrò per un po' una crescita lineare e poi arriverò ad un plateau oltre
il quale l'aumento del soluto che diffonde non servirà più per aumentare
la velocità di diffusione.

\subsection{OSMOSI}\label{osmosi}

Quando due soluzioni diversamente concentrate vengono in contatto, non
si osserva solo la diffusione delle molecole del soluto dalla soluzione
più concentrata verso quella meno concentrata, ma anche la diffusione
delle molecole d'acqua nella \emph{direzione opposta}.

Questo flusso diffusionale dell'acqua prende il nome di \textbf{osmosi},
e la forza che determina il flusso diffusionale dell'acqua riferita
all'unità di superficie prende il nome di \textbf{pressione osmotica}.

Affinchè avvenga il fenomeno osmotico è necessario che le due soluzioni
a concentrazione diversa siano separate da una \emph{membrana permeabile
al solvente ma non al soluto}.

La forza che spinge le molecole idriche a passare attraverso la membrana
semipermeabile si traduce in un aumento della pressione idrostatica
all'interno del recipiente con più soluto al suo interno, pressione che
aumenta fino ad equilibrare esattamente, a livello di membrana, la forza
che spinge le molecole idriche ad attraversarla (\emph{pressione
osmotica}).

L'acqua tenderà a passare dalla zona meno concentrata a quella più
concentrata.

la relazione quantitativa tra la differenza di pressione osmotica (delta
pigreco) è data dalla \textbf{legge di van't Hoff}:

\textbf{delta pi greco= RT (C1-C2)}

Una delle conseguenze della legge di van't Hoff della pressione osmotica
è che \textbf{NON} dipende dalla natura delle particelle di soluto, ma
solo dal loro numero per unità di volume.

Le cellule risentono della pressione osmotica poiché presentano una
membrana plasmatica che lascia sempre passare l'acqua (questo passaggio
può essere controllato).

Il gradiente osmotico tra l'ambiente esterno ed interno della cellula è
creato da tutti quei soluti, detti \textbf{osmoliti}, che non possono
assolutamente attraversare la membrana.

Se le concentrazioni degli osmoliti dentro e fuori sono uguali
ovviamente non si ha l'effetto osmotico. Normalmente le cellule animali
sono in condizioni di equilibrio osmotico, mentre se non c'è equilibrio
possono rigonfiarsi o avvizzire.

Nel plasma e nei globuli rossi abbiamo una concentrazione di 300
mOsmoles/L (ambiente \emph{isotonico}). Se i globuli rossi vengono posti
in una \emph{soluzione ipotonica} (es 150 mOsmoles/L), la cellula
rigonfia fino a scoppiare, mentre se i globuli rossi vengono posti in
una \emph{soluzione ipertonica} (es 500 mOsmoles/L) la cellula
raggrinzisce.

\section{IL TRASPORTO DI MEMBRANA}\label{il-trasporto-di-membrana}

Le cellule non rappresentano un sistema chiuso, ma scambiano
costantemente sostanze con l'esterno. Questo scambio viene fatto in
diversi modi che complessivamente vengono definiti come ``trasporto di
membrana''. Poiché la membrana ricopre interamente la cellula e non
presenta ``buchi'', deve esistere un modo per le molecole di
attraversarla. I fisiologi catalogano il trasporto di membrana creando
in primo luogo due grandi divisioni:

\begin{enumerate}
\def\labelenumi{\arabic{enumi}.}
\itemsep1pt\parskip0pt\parsep0pt
\item
  \textbf{trasporto attivo}, le particelle possono essere trasportate
  anche \emph{contro gradiente} e l'energia necessaria al trasporto
  proviene dal metabolismo cellulare;
\item
  \textbf{trasporto passivo}, le particelle attraversano la membrana
  cellulare solo \emph{in favore di un gradiente} e l'energia necessaria
  al trasporto proviene dalla dissipazione del gradiente che muove le
  particelle.
\end{enumerate}

Possiamo avere trasporto attivo o passivo sia in ingresso che in uscita
dalla cellula.

Il \emph{trasporto passivo} può essere:

\begin{enumerate}
\def\labelenumi{\arabic{enumi}.}
\itemsep1pt\parskip0pt\parsep0pt
\item
  \textbf{semplice}, se non servono intermediari affinchè la sostanza
  diffonda attraverso la membrana;
\item
  \textbf{facilitato}, se la sostanza non può attraversare
  spontaneamente la membrana ma utilizza delle proteine trasportatrici.
\end{enumerate}

Per il \emph{trasporto semplice} serve una sostanza che possa diffondere
attraverso la membrana ed un gradiente osmotico tra l'interno e
l'esterno della cellula.

Un esempio di diffusione semplice è il passaggio dell'acqua (a volte può
sfruttare anche il trasporto facilitato grazie alle \emph{acquaporine}),
dei gas atmosferici o le sostanze lipidiche. Per diffusione facilitata
inoltre, passano sostanze come i principali zuccheri del metabolismo
(glucosio e fruttosio) e i principali ioni minerali (Na, K, Cl, Ca)
necessari per la vita della cellula.

Il \emph{trasporto attivo} invece viene suddiviso in:

\begin{itemize}
\itemsep1pt\parskip0pt\parsep0pt
\item
  \emph{primario}, crea i gradienti di concentrazione e viene generato
  da speciali trasportatori detti \textbf{pompe} che consumano energia
  sotto forma di ATP (dette anche pompe ATPasiche o ATP dipendenti);
\item
  \emph{secondario}, non viene speso direttamente ATP ma viene sfruttata
  la differenza di potenziale elettrochimico creata dai trasportatori
  primari che pompano ioni al di fuori della cellula, il trasportatore
  secondario sfrutta l'energia di questo gradiente per trasportare
  un'altra sostanza contro gradiente.
\end{itemize}

Il trasporto attivo secondario viene poi diviso in:

\begin{itemize}
\itemsep1pt\parskip0pt\parsep0pt
\item
  \textbf{uniporto}, è il trasporto secondario di una sola sostanza che
  si muove sfruttando la differenza di potenziale elettrochimico creato
  da trasportatori primari;
\item
  \textbf{costrasporto}, è il trasporto contemporaneo di due specie
  ioniche o di altri soluti.
\end{itemize}

Il cotrasporto avviene grazie alla presenza di proteine di membrana
dette \textbf{cotrasportatori}: queste proteine possono trasportare 2 o
3 molecole diverse di cui una contro gradiente e l'altra secondo
gradiente. Come già detto non c'è consumo diretto di ATP ma c'è
dissipazione di un gradiente che sarà stato precedentemente creato da un
trasportatore attivo primario. Nel cotrasporto si ha la diffusione di
una sostanza lungo il suo gradiente ed è proprio questa diffusione a
creare i presupposti affinché lo stesso trasportatore possa far
attraversare la membrana ad un'altra sostanza contro gradiente (le
sostanze trasportate possono essere anche tre).

Il cotrasporto si divide poi in:

\begin{itemize}
\itemsep1pt\parskip0pt\parsep0pt
\item
  \textbf{antiporto}, è il trasporto contemporaneo di due specie ioniche
  o di altri soluti che si muovono in \emph{direzioni diverse}
  attraverso la membrana. Una delle due sostanze viene lasciata fluire
  secondo gradiente, da un compartimento ad alta concentrazione ad uno a
  bassa concentrazione. Questo genera l'energia entropica necessaria per
  guidare il trasporto dell'altro soluto contro gradiente, da bassa ad
  alta concentrazione;
\item
  \textbf{simporto}, usa il flusso di un soluto secondo gradiente per
  muovere un'altra molecola contro gradiente ma il movimento avviene in
  questo caso attraversando la membrana nella \emph{stessa direzione}.
\end{itemize}

Nei tipi di trasporto possiamo poi includere fenomeni di
\textbf{esocitosi} e \textbf{endocitosi}. In questo caso le sostanze non
vengono trasportate molecola per molecola ma come massa di materia
grazie a movimenti della membrana plasmatica che ingloba una certa
quantità di materiale. L'endocitosi e l'esocitosi implicano sempre la
formazione di vescicole.

\section{Trasporto attivo primario: le
pompe}\label{trasporto-attivo-primario-le-pompe}

Le pompe sono trasportatori di membrana, ovvero proteine di membrana
formate da 3 porzioni: una parte esterna e glicosilata, una parte
citosolica (molto più grande di quella esterna) ed una parte immersa
nella membrana.

Queste proteine sono capaci di accoppiare il trasporto contro gradiente
di diversi substrati con la defosforilazione di una molecola di ATP.
L'idrolisi dell'ATP libera energia che il trasportatore può utilizzare
per far avvenire il passaggio di una sostanza da un lato all'altro della
membrana.

I trasportatori di membrana sono presenti anche all'interno della
cellula sulle membrana di alcuni organuli cellulari (es: lisosomi,
mitocondri\ldots{}). Trasportano numerosi tipi di composti (noi vedremo
soprattutto gli ioni minerali)

Le proteine responsabili dei trasporti attivi primari si possono
suddividere in 3 gruppi:

\begin{itemize}
\itemsep1pt\parskip0pt\parsep0pt
\item
  le ABC ATPasi;
\item
  le H-ATPasi (comprende le F,V e A ATP-asi);
\item
  le P-ATPasi.
\end{itemize}

\textbf{(IMMAGINE p87)}

Le \textbf{F-ATPasi} (o ATP sintasi) si trovano nei mitocondri e sono
coinvolte nella sintetizzazione dell'ATP; queste ATPasi funzionano in
modo opposto rispetto alle altre, ovvero sinsetizzano ATP sfruttando
l'energia di un trasporto ionico secondo gradiente invece di trasportare
contro gradiente ioni sfruttando l'energia prodotta dall'idrolisi
dell'ATP.

Le \textbf{V-ATPasi} acidificano il lume interno dei vacuoli e di altri
organuli endocellulari portando al loro interno ioni H+.

Le F e le V-ATPasi sono costituite da due gruppi di subunità
(\emph{complessi}) denominati F1 e F0 o V1 e V0. I gruppi F1 e V1 sono
citoplasmatici e presentano i siti catalitici per l'ATP, mentre le
porzioni F0 e V0 sono transmembrana e sono la sede del trasporto degli
idrogenioni.

\textbf{(IMMAGINE P89)}

Le \textbf{A-ATPasi} si trovano esclusivamente negli Archea (funzione
delle F-ATPasi e struttura simile alle V-ATPasi).

Le \textbf{P-ATPasi} pompano ioni Na+, K+, Ca2+\ldots{} hanno un
intermedio fosforilato. La molecola del trasportatore viene fosforilata
e trasportata dal compartimento extracellulare a quello intracellulare.

In una normale condizione fisiologica esistono diversi gradienti di
membrana (indispensabili per la vita delle cellule). La fisiologia di
una cellula ruota attorno ad alcuni ioni minerali che sono quelli più
abbondanti nei liquidi biologici.

I più abbondanti in assoluto sono il Na e il K, seguiti da Cl e Ca.
Questi ioni hanno sempre concentrazioni diverse dentro e fuori la
cellula:

\begin{itemize}
\itemsep1pt\parskip0pt\parsep0pt
\item
  \textbf{Na+} \textgreater{} nell'ambiente esterno (140 mmol
  nell'ambiente esterno, 5 mmol nel citosol);
\item
  \textbf{K+} \textgreater{} nel citosol (5-15 mmol nell'ambiente
  esterno, 145 mmol nel citosol);
\item
  \textbf{Cl-} \textgreater{} nell'ambiente esterno (110 mmol
  nell'ambiente esterno, 4 mmol nel citosol);
\item
  \textbf{Ca2+} \textgreater{} nell'ambiente esterno (2,5-5 mmol
  nell'ambiente esterno, 0,0001 mmol nel citosol).
\end{itemize}

Nel momento in cui uno di questi gradienti cambia la cellula muore,
dunque la cellula tende a mantenere questi gradienti sempre costanti.

Questi gradienti sono mantenuti dal trasporto attivo primario che li
crea ed in parte da quello secondario che sfruttando questi gradienti
fondamentali ne tiene stabili altri.

\subsection{Le P-ATPasi}\label{le-p-atpasi}

Queste proteine presentano un meccanismo di trasporto comune a tutte,
anche se è stato individuato con precisione solo nelle pompe SERCA e
nella Na+/K+-ATPasi.

Il loro modello di funzionamento è chiamato ``Post-Albers'' e assume che
la proteina trasportatrice possa assumere \emph{due conformazioni}: +
nella conformaizone \textbf{E1} i siti di legame sono esposti dal lato
citoplasmatico ed hanno alta affinità per i substrati che devono essere
trasportati dall'altro lato della membrana e bassa affinità per i
substrati che sono trasportati in senso opposto; + nella conformazione
\textbf{E2} gli stessi siti sono esposti al lato extracellulare della
membrana ed hanno bassa affinità per i substrati che sono stati legati
in precedenza al lato intracellulare ed alta affinità per i substrati
che saranno importati nel citoplasma.

Come avviene il cambiamento conformazionale?

Il ciclo parte con la proteina nella conformazione E1 in cui presenta
sia un sito di legame ad alta affinità per l'ATP (che vi si lega) che i
siti di legame per i substrati ad alta affinità.

\begin{enumerate}
\def\labelenumi{\arabic{enumi}.}
\itemsep1pt\parskip0pt\parsep0pt
\item
  Si forma il legame tra i substrati presenti nel citosol e i siti di
  legame ad alta affinità. L'energia liberata da tale legame provoca un
  primo riarrangiamento della proteina;
\item
  il cambiamento conformazionale provoca la chiusura delle vie di
  accesso ai siti di legame verso il citoplasma intrappolando il
  substrato che si era legato. Questo viene detto stato di
  fosforilazione E1-P;
\item
  si ha un cambiamento nella proteina che riduce l'affinità dei siti di
  legame per i substrati che aveva legato precedentemente, mentre
  aumenta l'affinità per i substrati da legare nell'ambiente
  extracellulare. Contemporaneamente si apre un varco di uscita al lato
  extracellulare della membrana dove, per la ridotta affinità dei siti
  di legame, vengono rilasciati i substrati precedentemente legati. A
  questi siti di legame vengono poi legati i nuovi substrati. Durante
  questo passaggio la proteina passa dalla conformazione E1-P a quella
  E2-P;
\item
  il legame dei nuovi substrati ai siti di legame causa la chiusura di
  questo varco ed il conseguente intrappolamento dei substrati.
\end{enumerate}

Questo processo avviene in maniera ciclica permettendo il continuo
legame e trasporto di molecole da una parte all'altra della cellula.

\subsubsection{Pompa Na+/K+}\label{pompa-nak}

Il Na+ è lo ione più importante extracellularmente, mentre il K+ è il
più importante intracellularmente. Esiste una Na+/K+ ATP, o ATPasi
sodio-potassio dipendente o pompa sodio-potassio. E' una pompa di tipo
P, localizzata nella membrana plasmatica, che sfrutta un intermedio
fosforilato.

Questo trasportatore crea il gradiente del Na+ e del K+, che sono i più
importanti dal punto di vista quantitativo. Movimenta la maggior
quantità di materiale attraverso la membrana.

Trasporta contemporaneamente ioni Na+ fuori dalla cellula e ioni K+
all'interno, entrambi contro il loro gradiente elettrochimico. Questa
pompa è essenziale per l'eccitabilità dei neuroni e del muscolo poichè è
essenziale per l'equilibrio elettrico e osmotico della cellula.

\textbf{(riguardare ELETTROMAGNETISMO su appunti fisica)}

Questa proteina è formata da due subunità:

\begin{itemize}
\itemsep1pt\parskip0pt\parsep0pt
\item
  una \textbf{subunità alfa} formata da 10 domini transmembrna che
  sporge in prevalenza nel citosol. Questa subunità porta i siti di
  legame per l'ATP e per i cationi;
\item
  una \textbf{subunità beta} più piccola, formata da un solo dominio
  transmembrana, che sporge verso l'esterno con una porzione
  glicosilata.
\end{itemize}

La proteina viene fosforilata, tramite l'idrolisi di una molecola di
ATP, mentre si trova in uno stato in cui la proteina presenta elevata
affinità per il legame di Na+ sul lato citosolico (dove sodio ce n'è
``poco''). Questa fosforilazione produce una variazione conformazionale
della proteina. A questo punto la zona che lega il Na+ (può ospitare
fino a 3 ioni alla volta) si apre all'interno della cellula dove il Na+
è molto concentrato.

Questa variazione conformazionale rende i siti di legame per il Na+
molto meno affini allo ione permettendogli di venire rilasciato
all'interno della cellula. L'affinità per il ligando si presenta con un
valore simile a quello della concentrazione dello ione. Quando la
proteina varia la sua conformazione rivolgendo i siti di legame verso
l'esterno il coefficiente di dissociazione deve salire a 100/150 mmol.

La fosforilazione della proteina non dura molto a lungo (parliamo di
millisecondi) e quando viene defosforilata l'affinità per il Na+
diminuisce ma aumenta quella per il K+.

Ogni \emph{3 ioni Na+} che escono ne entrano \emph{2 di K+}. Questo
trasporto viene definito \textbf{elettrogenico} poiché crea una
differenza di potenziale elettrico tra l'esterno e l'interno della
cellula a causa del differente numero di cariche positive trasportate.

\subsubsection{Pompa protonica o H+/K+
ATPasi}\label{pompa-protonica-o-hk-atpasi}

Questo trasportatore opera un \emph{antiporto} molto simile alla
Na+/K+-ATPasi (è un'ATPasi dei tipo P) ma, a differenza di quest'ultima
è \emph{elettroneutra}. Questo trasportatore estrude 2 ioni H+ ed
intrude 2 ioni K+. E' tipico delle cellule ossintiche delle ghiandole
gastriche della mucosa dello stomaco per l'acidificazione del chimo.

\subsubsection{Pompa Ca2+-ATPasi}\label{pompa-ca2-atpasi}

Questi trasportatori attivi primari dovrebbero essere più propriamente
denominati Ca2+/H+-ATPasi e sono anch'essi molto espressi.

Ne possiamo riconoscere almeno due tipi distinti:

\begin{enumerate}
\def\labelenumi{\arabic{enumi}.}
\itemsep1pt\parskip0pt\parsep0pt
\item
  Le \textbf{pompe PMCA} o Ca2+ ATPasi della plasma membrana,
  trasportano \emph{uno ione Ca2+} all'esterno della cellula e \emph{uno
  ione H+} all'interno della cellula per ogni molecola di ATP
  idrolizzata. Ne esistono 4 isoforme.
\item
  le \textbf{SERCA} o Ca2+ ATPasi del reticolo sarco-endoplasmatico,
  trasportano verso il lume del reticolo endoplasmatico \emph{due ioni
  Ca2+} ed un certo numero di ioni H+ (per ogni coppia di Ca2+ vengono
  traslocati un po' meno di 4 ioni H+). E' il principale sistema di
  omeostasi del muscolo scheletrico e cardiaco.
\end{enumerate}

Questa proteina movimenta gli ioni sulla membrana del reticolo
endoplasmatico, mentre le altre due viste movimentano gli ioni sulla
membrana plasmatica.

\subsubsection{Cotrasporto 3Na+/2Ca2+}\label{cotrasporto-3na2ca2}

Questo è un antiporto.

Questo trasportatore sfrutta il gradiente del Na+ facendolo entrare
nella cellula e contemporaneamente porta Ca2+ fuori dalla cellula.

E' un processo che tende ad innalzare il potenziale eletttrico
transmembranario a causa del diverso numero di cariche spostate.

\subsubsection{Controtrasporto Na+/H+}\label{controtrasporto-nah}

Questi trasportatori provvedono all'espulsione nello spazio
extracellulare degli ioni H+ liberati nel citosol dalla deidrogenazione
dei substrati. L'ingresso di ioni Na+ comporta un'equivalente uscita di
H+ in direzione opposta.

Questi trasportatori sono molto espressi nel rene dove rappresentano il
principale meccanismo di acidificazione dell'urina.

Nel rene avviene la filtrazione del sangue (liquido di partenza=
plasma), il liquido finale è l'urina (molto diversa dal plasma). Tutte
le differenze tra il plasma e l'urina sono dovute all'attività di
trasportatori che agiscono nei tubicini dove passa il plasma.

\subsection{Trasportatori mediati da
vescicole}\label{trasportatori-mediati-da-vescicole}

Molecole ed aggregati sovramolecolari di varie dimensioni possono
entrare o uscire dalle cellule con un meccanismo completamente diverso
da quelli considerati finora: superando la emmbrana plasmatica
\emph{racchiusi in vescicole}.

\textbf{Endocitosi.} Quando il carico delle vescicole è una goccia
fluida, la vescicola è di minor diametro e si parla di
\textbf{pinocitosi}. Solitamente la pinocitosi è \emph{mediata da
recettori}, essendo innescata dal contatto delle molecole che devono
essere introdotte nella cellula con specifici recettori della membrana
plasmatica.

Quando invece viene unglobato un corpuscolo solido, la vescicola che lo
include ha diametro maggiore e il processo è detto \textbf{fagocitosi}.
La fagocitosi richiede l'\emph{intervento del citoscheletro} ed in
particolare, una volta che il materiale da fagocitare ha attivato
specifici recettori di membrana, mette in moto i microfilamenti di
actina ad esso sottostanti, assumendo il chiaro carattere di azione
motoria.

\textbf{Esocitosi.} Qui la sequenza degli eventi è inversa a quella
dell'endocitosi: essa inizia con l'adesione alla superficie interna
della membrana plasmatica di una vescicola in arrivo dal citoplasma,
carica del materiale che dev'essere espulso dalla cellula. Seguono la
fusione e l'apertura della vescicola, che rende libero il suo contenuto
nell'ambiente.

Le vescicole destinate a fondersi con la membrana plasmatica provengono
dal complesso di Golgi.

\subsection{Il trasporto
transepiteliale}\label{il-trasporto-transepiteliale}

Negli organismi superiori, ogni scambio di materia con l'ambiente
avviene attraverso particolari tessuti denominati \emph{epiteli}.

Mentre gli epiteli pluristratificati delimitano e proteggono
l'organismo, quelli monostratificati assolvono al compito di regolare
gli scambi materiali tra il suo \emph{ambiente interno} e l'ambiente in
cui esso vive, svolgendo una funzione analoga alla membrana plasmatica
della singola cellula.

L'alterazione del trasporto transepiteliale causa malattie come la
fibrosi cistica, malattia dovuta ad un trasporto anomalo del cloro.

\section{La comunicazione cellulare}\label{la-comunicazione-cellulare}

Le cellule possono comunicare tra loro sia per contatto che a distanza
grazie all'utilizzo di molecole che funzionano come messaggeri chimici.

Per la comunicazione cellulare serve:

\begin{itemize}
\itemsep1pt\parskip0pt\parsep0pt
\item
  un messaggero chimico;
\item
  un recettore, ovvero una molecola in grado di recepire la molecola
  segnale;
\item
  un marchingegno cellulare che converta il segnale ricevuto dal
  recettore in un'attività della cellula.
\end{itemize}

Quando un recettore cellulare lega una molecola segnale ad esso
indirizzata modifica lo stato funzionale della una cellula (= la cellula
stava facendo qualcosa e comincia a fare qualcos'altro). Ci sono
sostanze che hanno una capacità enorme di modificare lo stato funzionale
di una certa cellula, dette ``ormoni'' (ne basta una concentrazione
molto bassa perché si modifichi lo stato funzionale della cellula
interessata).

\subsection{I recettori cellulari}\label{i-recettori-cellulari}

I recettori chimici sono i sensori dei messaggeri extracellulari. Essi
sono sempre \textbf{molecole proteiche} foggiate in modo da legare con
\emph{altissima affinità} quelle dei messaggeri (chiamati genericamente
\emph{ligandi}).

Si tratta di legami relativamente \emph{labili} (ponti idrogeno, forze
di van der Waals\ldots{}) e quindi facilmente reversibili.

La molecola segnale solitamente arriva da un'altra cellula, entra in
contatto con la cellula che possiede il recettore e si lega al recettore
attivandolo.

Il recettore, dopo essere entrato in contatto con la molecola segnale,
modifica la sua conformazione legando altre proteine della cellula, che
a loro volta legano altre proteine, fino ad attivare degli enzimi che
cambieranno lo stato funzionale della cellula. Questo processo viene
chiamato \textbf{``trasduzione del segnale''}.

Quando avvengono le comunicazioni cellulari? Sempre. Non esistono
cellule vive che non siano coinvolte in fenomeni di comunicazione
cellulare.

Nelle colture cellulari in vitro non tutti i tipi di cellule possono
essere coltivate, ma per quelle che lo sono è sempre essenziale che ci
siano dei fattori di crescita nel terreno, altrimenti la cellula andrà
in apoptosi. Solitamente viene utilizzato il siero fetale bovino poichè
contiene molti fattori di crescita.

I recettori possono essere:

\begin{itemize}
\itemsep1pt\parskip0pt\parsep0pt
\item
  \textbf{intracellulari} (o endocellulari), ubicati nel citoplasma o
  nel nucleo, accessibili soltanto ai messaggeri extracellulari
  \textbf{liposolubili} che possono superare la membrana della cellula
  bersaglio. Generalmente esplicano al loro azione regolando
  direttamente svariati processi di trascrizione genica che presiedono
  alla sintesi di nuove proteine;
\item
  \textbf{di superficie} (o membranali), ubicati sulla membrana
  plasmatica, riservati ai ligandi \textbf{idrosolubili} i quali, per le
  dimensioni ed altre caratteristiche delle loro molecole, non possono
  superare la membrana della cellula bersaglio.
\end{itemize}

Gli \emph{endocrinologi} studiano da dove arriva a dove va la molecola
segnale. Si distingue un segnale \textbf{endocrino} quando le due
cellule sono situate a distanza, ed il messaggio deve viaggiare nel
torrente sanguigno.

Un altro segnale è quello \textbf{paracrino}, qui non serve il passaggio
della molecola segnale nel circolo sanguigno poichè il punto di origine
e di arrivo della molecola sono molto vicini (cellule vicine), dunque la
moleocla devve attraversare solo lo spazio intracellulare.

Possiamo distinguere anche un segnale \textbf{autocrino} se la molecola
segnale rilasciata da una cellula trova recettori sulla propria membrana
(es. fattori di crescita).

Il \textbf{ligando} consiste nella la molecola segnale (ormone,
feromone, ione, neurotrasmettitore, farmaco, etc\ldots{}) che si lega in
maniera specifica a un sito sulla molecola del recettore (situato sulla
superficie o all'intenro della cellula bersaglio).

Esistono metodi raffinati (es. HPLC cromatogragfia liquida ad alta
prestazione) con i quali si è possibile individuare queste molecole
quando presenti nel sangue; tramite questi metodi è stato osservato che
la maggior parte di queste molecole, quando stanno agendo, sono a
concentrazioni di 10\^{}-9/-10 molare.

Per definire l'affinità del recettore per una molecola possiamo fare un
grafico dove l'\textbf{asse y} indica la \textbf{percentuale di molecole
recettore legate} e l'\textbf{asse x} indica la \textbf{concentrazione
del ligando nel sangue}. E' possibile stimolare la cellula bersaglio con
una concentrazione crescente di molecole segnale e valutare quante
molecole vengono legate dai recettori. Il grafico mostrerà una crescita
inizialmente esponenziale e poi evidenzierà un valore di plateau che
indica che tutti i recettori sono stati legati.

Le cellule spesso hanno una frazione di \emph{recettori ``di riserva''},
così che la concentrazione del ligando che fornisce la massima risposta
è minore della concentrazione del ligando che saturerebbe i recettori.

L'intensità di una risposta biologica a un ligando è generalmente
proporzionale al numero di recettori occupati: {[}RL{]} =
{[}R{]}{[}L{]}/k

Per una data concentrazione di ligando, cellule con più recettori
avranno più recettori occupati. L'interazione tra la molecola segnale e
il recettore è sempre basata su legami deboli, perciò il tempo che il
ligando rimane attaccato al recettore è molto breve.

Più il ligando è affine al recettore, più le cellule saranno sensibili
al ligando.

\subsection{Molecole segnale di tipo
gassoso}\label{molecole-segnale-di-tipo-gassoso}

\subsubsection{L'ossido nitrico (NO)}\label{lossido-nitrico-no}

Il monossido di azoto (molecola inquinante dal punto di vista
ambientale) viene prodotta dalle cellule ed è dunque presente
all'interno dei tessuti. L'ossido nitrico è prodotto nelle cellule dagli
enzimi \textbf{NO sintasi (NOS)} a partire da \emph{arginina} che viene
convertita in \emph{citrullina} liberando monossido di azoto.

Il primo effetto di questo enzima che fu scoperto riguardava la sua
azione come \emph{vasodilatatore}: il monossido di azoto agisce sulla
\textbf{guanosima monofosfato ciclico (cGMP)} che inibisce la
contrazione della muscolatura liscia causando il rilassamento dei vasi e
il loro ``ingrandimento''.

Il monossido di azoto viene prodotto dall'endotelio dei vasi ed essendo
un gas può diffondere bene nelle cellule vicine.

Un'altro gas è il \textbf{monossido ci carbonio (CO)}, prodotto
dall'organismo tramite l'attività della \textbf{eme-ossigenasi} e che
può agire come \emph{molecola segnale delle macchinasi}.

L'attuale teoria sulla quale si basa tutta la biologia è la
\textbf{``Teoria del codice genetico''}: la materia vivente viene
prodotta secondo più processi partendo dalla decodificazione delle
informazione contenute in certe molecole (DNA). Attraverso questo
processo l'informazione può essere trasmessa alla progenie. Questa
teoria spiega tutti i fenomeni che vengono osservati nella materia
vivente.

Solo piccole parti di DNA vengono trascritte in RNA, e solo piccole
porzioni di RNA vengono tradotte in proteine. Le proteine sono poi
responsabili di quasi tutto quello che avviene negli esseri viventi.

Questa teoria è focalizzata a spiegare come nasce la materia vivente ma
pochissimo su come questa possa rimanere in vita.

Le basi di questo sistema sono state individuate nel 1870 circa, e
all'incirca negli anni '50 è stato individuato un modo per poter
manipolare il DNA così da poter mutare le proteine che desideriamo.

\subsection{Molecole segnale di tipo
organico}\label{molecole-segnale-di-tipo-organico}

Molte molecole segnale sono state chiamate ``ormoni''.

Le molecole segnale vengono suddivise in lipofile e idrofile ed
intervengono in maniera diversa sulla cellula bersaglio (quella che
contiene i recettori).

\subsubsection{I recettori
intracellulari}\label{i-recettori-intracellulari}

I ligandi liposolubili sono veicolati fino alle cellule bersaglio dai
liquidi circolanti (sangue e plasma), legati a speciali \emph{proteine
vettrici} per formare complessi idrofili. Per raggiungere i recettori
intracellulari, le molecole liposolubili vengono separate dalla proteina
vettrice ed attraversano la membrana della cellula bersaglio in forma
idrofobica.

I recettori intracellulari, una volta attivati, accedono al DNA
nucleare, sul quale operano come \emph{fattori di trascrizione}. Essi
infatti facilitano o inibiscono la trascrizione di particolari geni
bersaglio, quindi la sintesi di nuove proteine (sia strutturali che
enzimatiche).

Tra le molecole lipofile troviamo:

\begin{itemize}
\itemsep1pt\parskip0pt\parsep0pt
\item
  gli \textbf{ormoni steroidei} e gli \textbf{steroidi}. Questi
  viaggiano nel sangue combinati con \emph{proteine carrier} e non in
  maniera libera (non sono solubili nel siero), diffondono passivamente
  attraverso la plasmamembrana e legano recettori intracellulari. Hanno
  una \emph{struttura comune tetraciclica} che è propria anche del
  colesterolo e di un idrocarburo chiamato sterano. L'estradiolo ha un
  anello aromatico che gli altri ormoni steroidei non hanno. Es.
  steroidi sessuali, corticosteroidi\ldots{};
\item
  gli \textbf{ormoni tiroidei}, la famiglia delle molecole della
  \textbf{vitamina D}, l'\textbf{acido retinoico}\ldots{}
\end{itemize}

Un'altra grande categoria di molecole segnale sono \textbf{molecole di
natura peptidica (polimeri di amminoacidi)}. Di solito non hanno grandi
dimensioni ma in alcusi casi, come l'insulina, possono essere grandi
come una proteina. L'insulina è formata da due catene di amminoacidi
unite da vari ponti disolfuro, viaggia nel sangue e può influenzare
l'attività in vari tessuti.

Vi sono poi i \textbf{neurotrasmettitori} che agiscono sul sistema
nervoso o neuromuscolare. Questa è una categoria molto eterogenea di
composti dove tutti condividono una caratteristica: \emph{possiedono
sempre l'azoto (composti organo-azotati)}. Queste molecole hanno la
capacità di condizionare l'attività dei neuroni e delle cellule
muscolari.

Vi sono poi gli \textbf{eicosanoidi}, ovvero molecole lipidiche capaci
di agire su recettori di superficie come le prostaglandine e le
prostacicline. Queste molecole agiscono come \emph{segnali paracrini} o
\emph{autocrini} stimolando una varietà di risposte fra cui
l'aggregazione delle piastrine, la risposta infiammatoria e la
contrazione della muscolatura liscia. Hanno una notevole importanza
nelle situazioni al confine tra fisiologia e patologia.

La molecola capostipite di tutte queste sostanze è l'\textbf{acido
arachidonico} (tetrainsaturo); questa molecola si può facilmente
ripiegare su se stessa e offrire l'azione a degli enzimi, in particolare
le \textbf{cicloossigenasi (COX)}. La COX è l'enzima chiave per la
produzione di tutte queste molecole e siccome alcune sono mediatori
dell'infiammazione (scatenano l'infiammazione), la ciclo ossigenasi è un
bersaglio anche dei farmaci antinfiammatori. Un farmaco antinfiammatorio
di largo uso è l'aspirina (il cui principio attivo è l'acido salicilico
o acetil-salicilico per far durare di più l'effetto della molecola) che
è un inibitore della cicloossigenasi. Le prostaglandine scatenano
l'infiammazione con annesso dolore perché vengono attivate anche le
terminazioni nervose del dolore, si formano il rossore e il gonfiore a
causa della vasocostrizione.

L'aspirina inibisce anche l'aggregazione delle piastrine e la
coagulazione del sangue (inibizione dei tromboxani).

Come mai queste molecole agiscono solo su alcuni tipi cellulari e non su
tutti? Perchè hanno bisogno di recettori che vengono espressi solo da
alcuni tipi cellulari.

\textbf{NOTA BENE:} il termine recettore è utilizzato sia per indicare
le molecole che interagiscono con molecole segnale, sia per indicare
degli organi o delle cellule presenti negli organi sensoriali (es. coni
e bastoncelli sono recettori visivi).

\subsubsection{I recettori di membrana}\label{i-recettori-di-membrana}

Molti ligandi esplicano la loro azione legandosi a recettori chimici che
sono \emph{glicoproteine intrinseche} della membrana plasmatica delle
cellule bersaglio.

Per poter funzionare un recettore membranale deve comprendere almeno due
domini:

\begin{itemize}
\itemsep1pt\parskip0pt\parsep0pt
\item
  un \textbf{dominio recettoriale} che presenta uno o più siti di legame
  per la molecola segnale;
\item
  un \textbf{dominio effettore} che, dopo essere stato attivato dalla
  formazione del complesso ligando-recettore, innesca la risposta
  cellulare.
\end{itemize}

Esistono 3 classi di recettori membranali:

\begin{itemize}
\itemsep1pt\parskip0pt\parsep0pt
\item
  recettori \textbf{ionotropici} o recettori legati a \textbf{canali
  ionici};
\item
  recettori \textbf{metabotripici}. Questi si dividono in due gruppi:
\end{itemize}

\begin{enumerate}
\def\labelenumi{\arabic{enumi}.}
\itemsep1pt\parskip0pt\parsep0pt
\item
  recettori legati alle \textbf{proteina G} (o a 7 domini
  transmembranari);
\item
  recettori operanti per \textbf{via enzimatica};
\end{enumerate}

\begin{itemize}
\itemsep1pt\parskip0pt\parsep0pt
\item
  la classe delle \textbf{proteine adesive}.
\end{itemize}

Per ognuna di queste categorie abbiamo moltissimi recettori diversi.

\paragraph{Recettori-canale}\label{recettori-canale}

Questi recettori, quando sono attivati dal ligando extracellulare,
aprono nella loro molecola un condotto transmembranario che consente il
transito di ioni, determinando una pronta \textbf{variazione del
potenziale di membrana}. La molecola di questi recettori presenta una
\textbf{porzione recettrice} esposta al lato extracellulare della
membrana e dotata di uno o più \emph{siti di legame} per la molecola del
ligando, ed una \textbf{porzione effettrice}, costituita da un
\emph{canale ionico} che attraversa tutto lo spessore della membrana.

Il canale possiede almeno un \emph{gate} che ne controlla lo stato di
apertura/chiusura provvisto di un \emph{filtro di selettività} che lo
rende permeabile solo a determinate specie ioniche.

Questi canali agisocno facendo passare una corrente ionica poichè,
quando attivati, permettono il passaggio di ioni attraverso la membrana
(diffusione facilitata).

I tre tipi di recettori-canale meglio conosciuti sono:

\begin{itemize}
\itemsep1pt\parskip0pt\parsep0pt
\item
  i recettori per l'\textbf{acetilcolina}, alla cui attivazione è
  affidata la trasmissione sinaptica. Il canale ionico che essi
  costituiscono è relativamente poco selettivo, perchè percorribile
  dalla maggior parte dei cationi (Na+, K+, Ca2+, Mg2+) presenti nei
  liquidi fisiologici;
\item
  i recettori per l'\textbf{acido glutammico};
\item
  i recettori per l'\textbf{acido gamma-amminobutirrico}.
\end{itemize}

\paragraph{Recettori accoppiati a proteine
G}\label{recettori-accoppiati-a-proteine-g}

Queste sono \emph{proteine intrinseche di membrana}. La loro larga
diffusione nella membrana di tutte le cellule è in accordo con la
capacità che essi hanno di attivare/inibire una grande varietà di
processi intracellulari.

La sequenza di eventi intracellulari che fa seguito all'attivazione di
questi recettori si svolge in questo modo:

\begin{enumerate}
\def\labelenumi{\arabic{enumi}.}
\itemsep1pt\parskip0pt\parsep0pt
\item
  il recettore, attivato dal ligando extracellulare, comunica il segnale
  ad una proteina G trimerica che trasferisce l'attivazione all'enzima
  produttore del secondo messaggero. Nel più comune dei casi l'enzima su
  cui agisce la proteina G trimerica è l'\emph{adenilato ciclasi} ed il
  secondo messaggero che viene prodotto è l'\emph{adenosin-monofosfato
  ciclico};
\item
  in alcuni casi i secondi messaggeri sono capaci di attivare
  direttamente canali ionici o di aumentare la concentrazione
  intracellulare degli ioni Ca2+, ma più frequentemente attivano una
  \textbf{protein-chinasi}, che diviene capace di rendere operative (per
  \emph{via fosforilativa}) le proteine-bersaglio.
\end{enumerate}

\textbf{IMMAGINE p137 (5-6)}

Le \textbf{protein-chinasi} costituiscono un'ampia famiglia di
\emph{proteine fosforilanti}, destinate primariamente a regolare
l'attività delle proteine-bersaglio attuatrici delle varie risposte
cellulari.

Tra i ligandi che attivano i recettori accoppiati a proteine G
trimeriche troviamo ormoni, neuritrasmettitori\ldots{}

Le proteine G sono formate da 7 segmenti transmembranari dei recettori
``serpentini'' sono connessi a 6 anse: 3 extracellulari e 3
intracellulari. La terza ansa intracellulare contiene il \emph{dominio
di interazione con la proteina G trimerica}.

L'estremità N'terminale della catena polipeptidic si sviluppa
nellàambiente extracellulare e contiene, oltre a diversi siti di
possibile glicosilazione, il \emph{dominio recettoriale} contenente i
\emph{siti di legame} per il ligando. L'estremità C-terminale si estende
nel citosol e contiene diversi \emph{siti regolatori}.

\textbf{CONTROLLARE GAIA} \textbf{IMMAGINE p137 (5-7)}

\paragraph{Recettori operanti per via
enzimatica}\label{recettori-operanti-per-via-enzimatica}

Poi ci sono recettori con attività enzimatica: la categoria più numerosa
è quella dei recettori tirosin chinasici. Sono recettori capaci di
fosforilare la tirosina presente sulla proteina (gli altri sono serina
e. fosforilano la (\ldots{})

possiamo avere anche delle moleocle che inibiscono i recettori.

Una molecola che si lega ad un recettore e attiva il sistema di
trasduzione del segnale viene detta agonista. Una molecola che si lega
ad un recettore ma n0on lo attiva, cioè non genera un segnale nella
cellula è definita antagonista. Essa blocca (\ldots{})

Nel morbo di Parkinson la cura consiste nella somministrazione di
agonisti della dopamina (recettori presenti nei neuroni che legano la
dopamina che è un neurotrasmettitore tra i più abbondanti nel sistema
nervoso. La malattia è caratterizzata da una deficienza nei sistemi
dopaminergici. Gli agonisti sopperiscono (\ldots{})

I recettori endocellulari: anche loro sono proteine, ma citosoliche (o
endonucleari). Possono essere informa inattiva extracellulari, legari il
ligando entrare nel nucleo e costruire dei fattori di attivazione della
trascrizione genica. A questi recettori intracellulari agiscono sempre
(gaia)

(\ldots{})

LA TRASDUZIONE DEL SEGNALE

Un tipico meccanismo della trasduzione del segnale è l'attivazione di
cascate di fosforilazione. Sono mediate da enzimi (protein-chinasi). In
queste cascate si attivano in serie delle proteine chinasi che
fosforilano altre proteine una dopo l'altra.

Segnale mediato da proteine G Sono capaci di legare molecole che
arrivano dallo spazio extracellulari e diventano in grado di interagire
con una proteina (proteina G). Le proteine associate ai recettori sono
trimeriche . Questa proteina è in grado di legare il (GTP) nucleotide.
Il nucleotide si lega alla subunità alfa. La proteina è legata alla
membrana grazie a delle molecole di acido grasso a lunga catena. Questi
acidi rimangono immersi nella membrana ancorando la proteina, Gli acidi
grassi sono legati alle subunità alfa e gamma.

Il recettore attivandosi cambia conformazione e lega la proteina G
tramite la subunità alfa. Allo stato inattivato lega GDP, ma quando lega
il recettore il GDP si stacca e lega il GTP. Contemporaneamente si
staccano anche gamma e beta lasciando alfa separata ma sempre attaccata
alla membrana.

Il terzo gruppo fosfato del GTP provoca una variazione conformazionale
della subunità alfa che coinvolge due amminoacidi `switch', una glicina
e una treonina. La modificazione conformazionale degli `switch'
favorisce il distacco delle subunità alfa dal complesso beta-gamma.
Quando la subunità alfa si attiva si innescano una serie di eventi:

Dopo che la subunità alfa ha attivato la proteina bersaglio si
auto-disattiva idrolizzando il GTP. La subunità alfa si dissocia dalla
proteina bersaglio e si ricombina col complesso beta-gamma,
ricostituendo la proteina G inattiva.

Capire bene: in questo ciclo non esiste un punto di partenza e di
arrivo. Quanlsiasi situaizone non è altro che un momento che darà luogo
ad un altro momento che alla fine sarà un momento che avevamo già
incontrato.

Il recettore attivato viene disattivato tramite:

\begin{itemize}
\itemsep1pt\parskip0pt\parsep0pt
\item
  fosforilazione da parte della chinasi del recettore legato a proteina
  G (GRK);
\item
  legame di una molecola arrestina;
\item
  endocitosi del recettore
\end{itemize}

Una proteina con cui la proteina G attivata interagisce è l'adenilato
ciclasi, che ha molti passaggi transmembrana e ha le estremità terminali
dalla parte citosolica. La proteina G è in grado di attivare l'adenilato
ciclasi che produce cAMP (adenosin monofosfato ciclico).

Il cAMP ha un'influenza notevole sull'attivazione delle cellule. Induce
le cellule a fare cose in più oltre al metabolismo basale. La
fosfodiesterasi degrada l'cAMP così che non raggiunga concentrazioni
troppo elevate che porterebbero la cellula fuori dall'omeostasi.

Si chiama ciclo poiché l'unico fosfato forma un ponte con un idrossile
del ribosio. Quindi si forma un anello dovuto al fosfato che ciclizza
con l'idrossile. La fosfodiesterasi rompe questo anello. Una forma
equivalente si trova in presenza di guanina invece che adenina.

La produzione di cAMP all'interno di una cellula dovuta a serotonina e
mediata da proteina G.

Com'è che l'cAMP attiva le cellule? Attiva la protein chinasi A (PKA).
La chinasi forma un compleso con un'altra proteina, l'cAMP ciclico si
attacca a quest'altra proteina staccandosi dalla chinasi che una volta
libera può attivarsi.

La PKA può attivare una cascata di fosforilazione. Es. adrenalina. Si
attacca al recettore beta-adrenergico che lega una proteina G la quale
attiva l'adenilato ciclasi che forma cAMP la quale attiva la fosforilasi
chinasi che a sua volta attiva la fosforilasi ({[}passa dalla
conformazione b alla a) (?) Questi processi devono essere veloci.

Una molecola, rispetto ad una cellula, è molto piccola, quindi i
recettori che reagiscono con i ligandi sono anch'essi molto piccoli.
Ogni cellula possiede molte molecole di un certo recettore localizzate
sulla superficie. Questi eventi avvengono tutti sulla superficie. Per
far sì che una cellula si metta in movimento occorre influenzare il
volume della cellula.

La cascata di fosforilazione può essere interpretato come
un'amplificazione del segnale.

Funzioni della PKA: attivazione di fattori di trascrizione.

\begin{itemize}
\itemsep1pt\parskip0pt\parsep0pt
\item
  la subunità catalitica attiva della PKA migra nel nucleotide
\item
  nel nucleo P (\ldots{})
\end{itemize}

(\ldots{}molta roba\ldots{})

RECETTORI LEGATI AD ATTIVITA' ENZIMATICA

hanno un aspetto comune fatto da un ampia porzione extracellulare (lega
la molecola segnale), un dominio transmembrana e una porzione
intracellulare con attività tirosina chinasica.

Legano fattori di crescita. E' un po' diverso il recettore per
l'insulina.

I recettori tirosina chinasici:

\begin{itemize}
\itemsep1pt\parskip0pt\parsep0pt
\item
  legano ormoni peptidici idrosolubili il ligando stimola l'attività
  enzimatica del recettore che a sua volta induce una cascata di
  fosforilazione
\item
  questi recettori sono implicati nella proliferazione e nel
  differenziamento cellulare, promuovo la sopravvivenza delle cellule
\end{itemize}

quando il recettore lega il ligando esso di dimerizza e si
fosforilizzano a vicenda e a questi dimeri si aggregano altre proteine
il cui nucleo di partenze è formato dall'insieme di ras, grb2 e sos e
produce la cascata di fosforilazione. Il punto finale della cascata è
l'attivazione di geni per un fattore di trascrizione.

Il metabolismo del cGMP

cGMP = guanosin monofosfato ciclico

derivato da GTP ad opera di due enzimi:

\begin{itemize}
\itemsep1pt\parskip0pt\parsep0pt
\item
  guanilato ciclasi solubile (sGC)
\item
  guanilato ciclasi di mimbrana (pGC)
\end{itemize}

Viene degradato a GMP da fosfodiesterasi (è una famiglia di enzimi che
degradata abbastanza indifferentemente cAMP e cGMP). Il cGMP attiva il
metabolismo. Ne troviamo sia forme di membrana che forme solubili.
Agisce da secondo messaggero, es. attivando la proteina

Via di segnale dipendente da NO/cGMP Gli esnzimi NO sintasi (NOS)
producono NO (monossido d'azoto). Il NO è in grado di stimolare la sGC
(guanilato ciclasi) ??????????????????

La guanilato ciclasi è formata da 2 subunità e contiene un gruppo eme
collegato ad un amminoacido istidina della subunità beta e a 4 atomi di
azoto. Il NO lega un atomo di Fe del gruppo eme inserito tra le subunità
alfa e beta.

L'ossigeno del No attira il Fe del gruppo eme carico positivamente
apportando una modifica della struttura dell'eme che induce una modifica
sulla proteina tramite l'istidina. Modifica conformazionale nell'enzima
che produce attivazione nel sito attivo dell'enzima e quinid produce
cGMP da guanosin trifosfato.

Funzione della via NO/cGMP Il No è il più potente vasodilatatore
conosciuto

Il meccanismo di vasodilatazione:

\begin{itemize}
\itemsep1pt\parskip0pt\parsep0pt
\item
  Il NO viene prodotto da cellule endoteliali in risposta ad
  acetilcolina, GABA, adrenalina; I vasi sono formati da cellule
  endoteliali che li circondano e da fibre muscolari che ne causano la
  dilatazione e la contrazione;
\item
  diffonde dalle cellule endoteliali al muscolo liscio;
\item
  nelle cellule muscolari attiva la guanilato ciclasi
\item
  la GC produce xGMP che attiva una pompa SERCA
\item
  si abbassa la Ca2+
\item
  le fibre del muscolo liscio si rilassano e quindi c'è vasodilatazione.
  (sara)
\end{itemize}

farmaci storici utilizzati come vasodilatatori il NO è un componente
della nitroglicerina che veniva usata contro 'angina e l'infarto (la
niutroglicerinavenendo metabolizzata accentua la produzione di monossido
di azoto permettendo un miglioramento della pressione sanguigna
coronarica).

La nitroglicerana è trinitrossidata.

\end{document}
